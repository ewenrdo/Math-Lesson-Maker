\documentclass{article}

\usepackage[a4paper,landscape,margin=2cm]{geometry}
\usepackage{../../../components/components}
\usepackage{colortbl}
\usepackage{array}

\usepackage{fancyhdr}
\pagestyle{fancy}
\fancyhf{}
\fancyhead[L]{DL2 Math-Info}
\fancyhead[C]{Distances, normes et topologie}
\fancyhead[R]{2025-2026}
\fancyfoot[L]{Ewen Rodrigues de Oliveira}
\fancyfoot[R]{\thepage}

\usepackage{paracol}

\begin{document}
\setlength{\columnsep}{2cm}

\begin{paracol}{2}

\renewcommand{\arraystretch}{1.3}
\noindent
\begin{tabular}{|p{0.6\columnwidth}|>{\centering\arraybackslash}p{0.35\columnwidth}|}
\hline
\rowcolor[RGB]{230,237,247}
\textbf{Savoir-faire} & \textbf{Exercices} \\
\hline
Manipuler un ensemble, ses parties et l'opérateur d'inclusion. & ? \\
\hline
Déterminer le cardinal d'une partie d'un ensemble. & ? \\
\hline
Utiliser les formules de combinaisons. & ? \\
\hline
Utiliser la formule de combinaison à k éléments. & ? \\
\hline
\end{tabular}

\vspace{1em}

\exercise{1}{}{
    On considère l'ensemble $E = \{1,2,3\}$.
    \begin{enumerate}
        \item Dire si les ensembles suivants sont des parties de E :
        \begin{itemize}
            \item $\{1 ; 3\}$
            \item $\{2 ; 4\}$
            \item $\varnothing$
        \end{itemize}
        \item Donner deux parties différentes de $E$ ayant exactement un élément.
    \end{enumerate}
}

\exercise{2}{}{
    Soit $E$ un ensemble à $7$ éléments.
    \begin{enumerate}
        \item Donner le nombre de parties de $E$.
        \item Combien $E$ possède-t-il de parties non vides ?
        \item Combien $E$ possède-t-il de parties à $3$ éléments ?
    \end{enumerate}
}

\exercise{3}{}{
    Dans un groupe de $12$ élèves, on souhaite former un jury de $4$ élèves.
    \begin{enumerate}
        \item Expliquer pourquoi l'ordre n'intervient pas dans le choix des élèves.
        \item Donner le nombre de juries possibles.
    \end{enumerate}
}   
\switchcolumn
\exercise{4}{}{
    Un club comporte $8$ filles et $5$ garçons. On forme une équipe de $3$ élèves.
    \begin{enumerate}
        \item Combien d'équipes différentes peut-on former ?
        \item Combien d'équipes comportent uniquement des filles ?
        \item Combien d'équipes peut-on composer avec exactement $1$ garçon ?
    \end{enumerate} 
}


\exercise{5}{}{
    Calculer :
    \begin{enumerate}
        \item $\binom{6}{2}$, $\binom{6}{4}$, $\binom{6}{0}$
        \item Vérifier la symétrie des coefficients binomiaux pour $n = 6$.
        \item Vérifier la formule de Pascal :
        \[
            \binom{5}{2} + \binom{5}{3} = \binom{6}{3}
        \]
    \end{enumerate}
}


\exercise{9}{}{
    On considère l’ensemble $E$ à $10$ éléments.
    \begin{enumerate}
        \item Combien de parties de $E$ ont un nombre pair d’éléments ?
        \item Combien de parties ont un nombre impair d’éléments ?
        \item Justifier que ces deux nombres sont égaux.
    \end{enumerate}
}

\switchcolumn
\newpage

\exercise{10}{Type-bac}{
    On considère une classe de $20$ élèves.
    
    \textbf{Partie A}
    \begin{enumerate}
        \item Combien de groupes de $5$ élèves peut-on former ?
        \item Combien de groupes contiennent un élève donné ?
    \end{enumerate}
    
    \textbf{Partie B}
    On considère l’ensemble $E$ des $20$ élèves.
    \begin{enumerate}
        \item Donner le nombre de parties de $E$.
        \item Combien de parties contiennent exactement $5$ élèves ?
        \item Retrouver le résultat de la partie A à l’aide des combinaisons.
    \end{enumerate}
    
    \textbf{Partie C}
    On choisit au hasard une partie de $E$.
    \begin{enumerate}
        \item Quelle est la probabilité que cette partie contienne exactement $5$ élèves ?
        \item Quelle est la probabilité qu’elle soit non vide ?
    \end{enumerate}
}

\switchcolumn

\end{paracol}

\end{document}
