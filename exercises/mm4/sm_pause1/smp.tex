\documentclass{article}

\usepackage[a4paper,landscape,margin=2cm]{geometry}
\usepackage{../../../components/components}
\usepackage{colortbl}
\usepackage{array}

\usepackage{fancyhdr}
\usepackage{stmaryrd}

\pagestyle{fancy}
\fancyhf{}
\fancyhead[L]{DL2 Math-Info}
\fancyhead[C]{Révision et remise à niveau (semaine de pause)}
\fancyhead[R]{2025-2026}
\fancyfoot[L]{Ewen Rodrigues de Oliveira}
\fancyfoot[R]{\thepage}

\usepackage{paracol}

\begin{document}
\setlength{\columnsep}{2cm}

\begin{paracol}{2}

\renewcommand{\arraystretch}{1.3}
\noindent
\begin{tabular}{|p{0.6\columnwidth}|>{\centering\arraybackslash}p{0.35\columnwidth}|}
\hline
\rowcolor[RGB]{230,237,247}
\textbf{Savoir-faire} & \textbf{Exercices} \\
\hline
Rattraper son retard et consolider les acquis & $\infty$\\
\hline
\end{tabular}

\vspace{1em}

\textbf{ANALYSE}\\

\exercise{1}{Convergence uniforme d'une suite de fonctions}{
Étudier la convergence simple et la convergence uniforme des suites de fonctions \((f_n)_{n \ge 0}\) dans les cas suivants :

\begin{enumerate}
    \item \(f_n(x) = e^{-nx} \sin(2nx)\) sur \(\mathbb{R}_+\) puis sur \([a, +\infty[\), avec \(a > 0\).
    \item \(f_n(x) = \frac{1}{(1+x^2)^n}\) sur \(\mathbb{R}\), puis sur \([a, +\infty[\), avec \(a > 0\).
\end{enumerate}
}

\exercise{2}{Convergence de séries de fonctions}{
Étudier les convergences simple, uniforme et normale de la série de fonctions \(\sum f_n\) dans chacun des cas suivants :

\begin{enumerate}
    \item \(f_n(x) = n x^2 e^{-x\sqrt{n}}\) sur \(\mathbb{R}_+\)
    \item \(f_n(x) = \frac{1}{n + n^3 x^2}\) sur \(\mathbb{R}_+^*\)
    \item \(f_n(x) = \frac{(-1)^n x}{(1+x^2)^n}\) sur \(\mathbb{R}\)
    \item \(f_n(x) = \frac{\sin(n^3 x^2)}{1 + 3n x^2 + 2 n^2}\) pour \(x \in \mathbb{R}\)
    \item \(f_n(x) = n x \exp(-nx)\) pour \(x \in [0, +\infty[\)
\end{enumerate}
}

\exercise{3}{}{
Soit \(a \in ]-1, 1[\). Pour \(n \in \mathbb{N}\) et \(t \in [0, \pi/2]\), on pose :
\[
u_n(t) = a^n \cos^n(t).
\]

\begin{enumerate}
    \item Montrer que la série de fonctions \(\sum u_n\) converge uniformément sur \([0, \pi/2]\). Déterminer la somme de cette série.
    \item En déduire l'égalité suivante :
    \[
    \int_0^{\pi/2} \frac{dt}{1 - a \cos t} = \sum_{n=0}^{\infty} \left( \int_0^{\pi/2} \cos^n(t) \, dt \right) a^n.
    \]
\end{enumerate}
}
\switchcolumn
\exercise{4}{Convergence normale locale de la série des dérivées}{
    Pour tout entier \(n \ge 1\), soit  
    \(f_n : \mathbb{R} \to \mathbb{R}\) la fonction définie par :  
    \[
    f_n(x) = \frac{1}{n^2 + x^2}.
    \]

    \begin{enumerate}
        \item Montrer que la série \(\sum f_n\) converge simplement sur \(\mathbb{R}\). On note \(s\) la somme de cette série.
        \item Montrer que \(s\) est continue sur \(\mathbb{R}\).
        \item Montrer que pour tout réel \(a > 0\), la série \(\sum f_n'\) converge normalement sur \([-a, a]\).
        \item Montrer que \(s\) est dérivable sur \(\mathbb{R}\).
    \end{enumerate}
}
\exercise{5}{}{
    Soit \(f : \, ]-1, 1[ \to \mathbb{R}\) la fonction définie par :  
\[
f(x) = \sum_{n=1}^{\infty} \frac{x^n \sin(nx)}{n}.
\]

\begin{enumerate}
    \item Montrer que \(f\) est de classe \(C^1\) sur \(]-1, 1[\).
    \item Calculer \(f'(x)\) et en déduire que :  
    \[
    f(x) = \arctan\left( \frac{x \sin x}{1 - x \cos x} \right).
    \]
\end{enumerate}
}

\exercise{6}{Rayons de convergence - Exemples}{
Calculer le rayon de convergence des séries entières de la forme \(\sum a_n z^n\) lorsque la suite \((a_n)\) est donnée par :

\begin{enumerate}
    \item \(a_n = \frac{1}{n \pi^n}\)
    \item \(a_n = \frac{n^2 \log n}{2^{3n}}\)
    \item \(a_n = \frac{1}{n^\alpha}, \ \alpha \in \mathbb{R}\)
    \item \(a_n = \frac{n^n}{n!}\)
   \end{enumerate}
}

\switchcolumn

\exercise{7}{Rayons de convergence - Exemples}{
Calculer le rayon de convergence des séries entières de la forme \(\sum a_n z^n\) lorsque la suite \((a_n)\) est donnée par :

\begin{enumerate}
    \item \(a_n = n^{\sqrt{n}}\)
    \item \(a_n = n^n\)
    \item \(a_n = n^{\ln n}\)
    \item \(a_n = \begin{cases} 2^p & \text{si } n = 2p \\ 0 & \text{si n impair} \end{cases}\)
\end{enumerate}
}

\exercise{8}{Rayons de convergence – Séries données}{
Calculer le rayon de convergence des séries entières suivantes :

\begin{enumerate}
    \item \(\sum_{n \ge 0} \frac{z^n}{9^n}\)
    \item \(\sum_{n \ge 0} \frac{z^{2n}}{9^n}\)
    \item \(\sum_{n \ge 0} \frac{z^{n^2}}{9^n}\)
\end{enumerate}
}

\exercise{9}{Rayons et sommes par séries connues}{
En se ramenant à des séries entières bien connues, calculer le rayon de convergence et la somme des séries suivantes :

\begin{enumerate}
    \item \(\sum_{n \ge 1} \frac{(n+2)^2}{(n+2)!} x^n\)
    \item \(\sum_{n \ge 0} \frac{\sin(n \alpha)}{n!} x^n\)
\end{enumerate}
}
\exercise{10}{Développement en série entière}{
À l'aide des théorèmes d'interversion, donner le développement en série entière de :

\begin{enumerate}
    \item \(\arctan\left(\frac{1 - x^2}{1 + x^2}\right)\) en \(0\)
    \item \(\ln(x + a)\) en \(0\), avec \(a > 0\)
\end{enumerate}
}
\switchcolumn
\exercise{11}{Développement en série de \(f(x) = \frac{e^x}{1-x}\)}{
Pour tout \(x \neq 1\), posons \(f(x) = \frac{e^x}{1-x}\).

\begin{enumerate}
    \item Montrer que \(f\) est développable en série entière au voisinage de 0. \\
          \textit{On note \(f(x) = \sum_{n=0}^{\infty} a_n x^n\) ce développement et \(R\) le rayon de convergence.}
    \item Montrer que \(R \ge 1\) et que \(f(x) = \sum_{n=0}^{\infty} a_n x^n\) pour tout \(x \in ]-1, 1[\).
    \item En calculant un produit de séries, montrer que \(a_n = \sum_{k=0}^{n} \frac{1}{k!}\).
    \item En raisonnant par l'absurde, montrer que \(R = 1\).
\end{enumerate}
}

\exercise{12}{Rayons de convergence et sommes – Applications}{
À l'aide des théorèmes d'interversion, calculer le rayon de convergence et la somme des séries entières suivantes :

\begin{enumerate}
    \item \(\sum_{n \ge 0} n^2 x^n\)
    \item \(\sum_{n \ge 1} \frac{(-1)^n}{ 2^{2n-1}} x^{2n}\)
    \item \(\sum_{n \ge 1} \frac{x^n}{1 + 2 + \dots + n}\)
\end{enumerate}
}

\textbf{ALGÉBRE LINÉAIRE}\\\\

\exercise{13}{Permutation dans \(S_8\)}{
Soit 
\[
\sigma =
\begin{pmatrix}
1 & 2 & 3 & 4 & 5 & 6 & 7 \\
3 & 5 & 6 & 7 & 1 & 2 & 4
\end{pmatrix}
\in S_8.
\]

\begin{enumerate}
    \item Décomposer \(\sigma\) en produit de cycles à supports disjoints.
    \item Donner la signature de \(\sigma\).
    \item Décomposer \(\sigma\) en produit de transpositions.
    \item Calculer \(\sigma^{2021}\).
\end{enumerate}
}
\switchcolumn
\exercise{14}{Permutation dans \(S_{10}\)}{
Soit \(\sigma\) la permutation de \(S_{10}\) donnée par :
\[
\sigma =
\begin{pmatrix}
1 & 2 & 3 & 4 & 5 & 6 & 7 & 8 & 9 & 10 \\
5 & 6 & 7 & 4 & 8 & 9 & 3 & 10 & 2 & 1
\end{pmatrix}.
\]

\begin{enumerate}
    \item Écrire l'inverse de \(\sigma\).
    \item Décomposer \(\sigma\) en cycles.
    \item Calculer sa signature.
    \item Quel est le plus petit entier non nul \(n\) tel que \(\sigma^n = \mathrm{id}\) ?
    \item Calculer \(\sigma^{147}\).
\end{enumerate}
}

\exercise{15}{Permutation dans \(S_9\)}{
Soit \(\sigma\) la permutation de \(S_9\) donnée par :
\[
\sigma =
\begin{pmatrix}
1 & 2 & 3 & 4 & 5 & 6 & 7 & 8 & 9 \\
3 & 7 & 8 & 9 & 4 & 5 & 2 & 1 & 6
\end{pmatrix}.
\]

\begin{enumerate}
    \item Déterminer le nombre d'inversions de \(\sigma\) et sa signature.
    \item Décomposer \(\sigma\) en produit de cycles disjoints.
    \item Décomposer \(\sigma\) en produit de transpositions.
    \item On dit qu'une transposition est simple si elle est de la forme \((i\ i+1)\). Décomposer \(\sigma\) en produit de transpositions simples.
    \item Quel est l'ordre de \(\sigma\) dans \(S_9\) ? Calculer \(\sigma^{1000}\).
\end{enumerate}
}

\switchcolumn
\textbf{ALGÉBRE BILINÉAIRE}\\\\

\exercise{17}{Gram-Schmidt dans \(\mathbb{R}^3\)}{
Soit \(\mathbb{R}^3\) muni du produit scalaire canonique. Construire une base orthonormée en utilisant le procédé de Gram-Schmidt à partir de la famille \((u, v, w)\) définie par :
\[
u = (1, 0, 1), \quad v = (1, 1, 1), \quad w = (-1, 1, 0).
\]
}

\exercise{17}{Projection orthogonale dans \(\mathbb{R}^4\)}{
Soit \(\mathbb{R}^4\) muni du produit scalaire usuel. On se donne :
\[
u = (2, 1, 1, -1), \quad v = (1, 1, 3, -1),
\]
et on pose \(F = \mathrm{Vect}(u, v)\).

\begin{enumerate}
    \item Déterminer une base orthonormale de \(F\)
    \item Donner un système d'équations cartésiennes de \(F^\perp\)
    \item Donner la projection de \(w = (1, 2, -2, 2)\) sur \(F\) et sur \(F^\perp\)
    \item En déduire la distance de \(w\) à \(F\)
\end{enumerate}
}

\exercise{18}{Supplémentaire orthogonal et distance à un plan dans \(\mathbb{R}^3\)}{
On munit \(\mathbb{R}^3\) du produit scalaire usuel. Soit \(P\) le plan d'équation :
\[
2x + y - z + 2 = 0.
\]
Déterminer le supplémentaire orthogonal de \(P\) et donner la distance de 
\[
u = (3, 4, 5)
\]
à \(P\).
}

\switchcolumn
\newpage

\exercise{19}{Projection orthogonale sur un plan dans \(\mathbb{R}^3\)}{
On munit \(\mathbb{R}^3\) de son produit scalaire canonique. Soit \(H\) le plan d'équation :
\[
2x + y - z = 0.
\]

\begin{enumerate}
    \item Donner les coordonnées d'un vecteur non nul orthogonal à \(H\)
    \item Déterminer les coordonnées
    \(\begin{pmatrix} x \\ y \\ z \end{pmatrix}\)
    du projeté orthogonal sur \(H\) d'un vecteur 
    \(\begin{pmatrix} a \\ b \\ c \end{pmatrix}\)
    \item Donner la matrice dans la base canonique de la projection orthogonale sur \(H\)
    \item Déterminer une base orthonormée de \(H\)
\end{enumerate}
}

\exercise{20}{Produit scalaire sur \(C^0([a,b],\mathbb{R})\)}{
On considère l'espace \(E = C^0([a, b], \mathbb{R})\) muni du produit scalaire suivant :
\[
\langle f \mid g \rangle = \int_a^b f(t) g(t) \, dt.
\]

\begin{enumerate}
    \item Justifier qu'il s'agit bien d'un produit scalaire.
    \item Soit \(f\) une fonction strictement positive sur \([a, b]\). On pose :
    \[
    l(f) = \left(\int_a^b f(t) \, dt\right) \left( \int_a^b \frac{dt}{f(t)} \right).
    \]
    Montrer que \(l(f) \ge (b-a)^2\). Étudier les cas d'égalité.
\end{enumerate}
}

\exercise{21}{Identité de Polarisation}{
Soit \(E\) un espace euclidien, et \(x, y \in E\). Interpréter et montrer l'égalité :
\[
\|x + y\|^2 + \|x - y\|^2 = 2(\|x\|^2 + \|y\|^2).
\]
}
\switchcolumn

\exercise{22}{Projection orthogonale}{
Soit \(E\) un espace euclidien et \(F\) un sous-espace vectoriel de \(E\). Montrer que l'ensemble 
\(\{\|x - y\| \mid y \in F\}\) admet un minimum et que ce minimum est atteint en un unique vecteur \(v \in F\), donné par 
\( v = p_F(x), \)
la projection orthogonale de \(x\) sur \(F\).
}

\exercise{23}{Base duale dans \(\mathbb{R}_n[X]\)}{
Soit \(E = \mathbb{R}_n[X]\). Soit \(B = (1, X, \dots, X^n)\) la base canonique de \(E\), et soit \(B^* = (f_0, \dots, f_n)\) la base duale de \(B\).

\begin{enumerate}
    \item Soit \(P = a_0 + \dots + a_n X^n\). Pour tout \(i \in \{0, \dots, n\}\), exprimer \(f_i(P)\) en fonction de \(a_0, \dots, a_n\).
    \item Soient \(\phi\) et \(\psi\) les deux éléments de \(E^*\) définis par :
    \[
    \forall P \in E, \quad \phi(P) = P(1), \quad \psi(P) = P'(0).
    \]
    Déterminer les coordonnées de \(\phi\) et \(\psi\) dans la base \(B^*\).
\end{enumerate}
}

\exercise{24}{Critère pour une base de \(E^*\)}{
Soit \(E\) un espace vectoriel de dimension finie \(n\) et soient \(\ell_1, \ell_2, \dots, \ell_n\) des formes linéaires sur \(E\). On suppose que :
\[
\bigcap_{1 \le k \le n} \ker \ell_k = \{0\}.
\]
On considère l'application linéaire \(L : E \to \mathbb{R}^n\) définie pour \(x \in E\) par :
\[
L(x) = 
\begin{pmatrix}
\ell_1(x) \\
\ell_2(x) \\
\vdots \\
\ell_n(x)
\end{pmatrix}.
\]

\begin{enumerate}
    \item Montrer que \(L\) est un isomorphisme d'espaces vectoriels.
    \item En déduire l'existence d'une base \((e_1, \dots, e_n)\) de \(E\) telle que :
    \[
    \forall j,k \in \{1, \dots, n\}, \quad \ell_k(e_j) = \delta_{j,k}.
    \]
    \item En déduire que \((\ell_1, \dots, \ell_n)\) est une base de \(E^*\).
\end{enumerate}
}
\exercise{25}{Supplémentaire d'un hyperplan}{
Soit \(E\) un \(\mathbb{R}\)-espace vectoriel de dimension finie et soit \(\varphi\) une forme linéaire non nulle sur \(E\). Montrer que pour tout \(u \in E \setminus \ker(\varphi)\), \(\ker(\varphi)\) et \(\mathrm{Vect}(u)\) sont supplémentaires dans \(E\).
}
\switchcolumn

\textbf{PROBABILITÉS}\\\\
\exercise{26}{Deux boîtes}{
On considère deux boîtes : l'une contient une bille noire et une blanche, et l'autre deux noires et une blanche.  
On choisit une boîte au hasard, puis on tire une bille dans cette boîte.

\begin{enumerate}
    \item Quelle est la probabilité que la bille tirée soit noire ?
    \item Sachant que la bille est blanche, quelle est la probabilité que ce soit la première boîte qui ait été choisie ?
\end{enumerate}
}

\exercise{27}{Jeu de cartes}{
A propose à B le jeu suivant : tirer \(r\) cartes parmi 52 ; si l'as de pique figure parmi les \(r\) cartes, B a gagné.

\begin{enumerate}
    \item Quelle est la probabilité que B gagne ?
    \item A envisage de tricher de la façon suivante : il subtilise \(k\) cartes (\(k \le 52 - r\)) avant que B ne tire ses \(r\) cartes (ces \(k\) cartes étant prises au hasard).  
    Quelle est alors la probabilité que B gagne ?
\end{enumerate}
}

\exercise{28}{Formule des probabilités composées}{
Montrer que si \(n\) événements \((E_i)_{1 \le i \le n}\) vérifient
\[
P\left(\bigcap_{i=1}^{n-1} E_i\right) > 0,
\]
alors
\[
P(E_1 \cap \cdots \cap E_n)
= P(E_1)\, P(E_2 \mid E_1)\, P(E_3 \mid E_1 \cap E_2)\, \cdots \, P(E_n \mid E_1 \cap \cdots \cap E_{n-1}).
\]
}
\exercise{29}{Somme et produit de variables uniformes}{
Soient \(X\) et \(Y\) deux variables aléatoires indépendantes et uniformes sur l'ensemble \(\{0,1,2\}\).  

Calculer les lois de \(X + Y\) et de \(XY\).
}

\switchcolumn
\newpage
\exercise{30}{Indépendance}{
\begin{enumerate}
    \item On munit \(\Omega := \{0,1\}^2\) de la probabilité uniforme et l'on considère les événements :
    \[
    A = \{0\} \times \{0,1\}, \quad
    B = \{0,1\} \times \{0\}, \quad
    C = \{(0,1),(1,0)\}.
    \]
    Montrer que les événements \(A, B, C\) sont deux-à-deux indépendants mais non mutuellement indépendants.

    \item On munit \(\Omega := \{0,1\}^3\) de la probabilité uniforme et l'on considère les événements :
    \[
    A = \{0\} \times \{0,1\}^2,
    \]
    \[
    B = \{0,1\}^2 \times \{0\},
    \]
    \[
    C = \{(0,0,0),(0,0,1),(0,1,1),(1,1,1)\}.
    \]
    Montrer que
    \[
    P(A \cap B \cap C) = P(A)P(B)P(C).
    \]
    Les événements \(A, B, C\) sont-ils mutuellement indépendants ?
\end{enumerate}
}

\exercise{31}{Suite de lancers d'une pièce}{
On effectue une suite de \(n\) lancers d'une pièce à pile ou face. On suppose les lancers indépendants et on note \(p\) la probabilité d'obtenir pile à un lancer donné (\(0 \le p \le 1\)).

\begin{enumerate}
    \item Décrire l'espace probabilisé associé à cette expérience.
    \item Calculer la probabilité :
    \begin{enumerate}
        \item d'obtenir au moins un pile au cours des \(n\) lancers (\(n \ge 1\)),
        \item d'obtenir exactement \(k\) pile (\(0 \le k \le n\)).
    \end{enumerate}
    \item Décrire le nouvel espace obtenu lorsqu'on conditionne par l'événement « obtenir exactement \(k\) pile au cours des \(n\) lancers ».
    \item On prend \(n = 6\). Calculer la probabilité, sachant qu'on a obtenu exactement deux piles, que ces deux piles soient consécutifs.
\end{enumerate}
}
\switchcolumn
\newpage
\exercise{32}{Deux dés distinguables}{
On lance deux dés distinguables.

\begin{enumerate}
    \item Définir un espace probabilisé décrivant l'expérience aléatoire.
    \item Soit \(S\) la variable aléatoire représentant la somme des deux nombres obtenus.  
    Donner l'expression de \(S\) et déterminer sa loi.
    \item Soit \(Z\) la variable aléatoire donnant la valeur absolue de la différence entre les valeurs des deux dés.  
    Donner l'expression de \(Z\) et déterminer sa loi.
\end{enumerate}
}

\exercise{33}{Marche aléatoire à trois pas}{
On lance une pièce équilibrée trois fois de suite. Pour chaque pile obtenu, on fait un pas en avant, et pour chaque face obtenu, on fait un pas en arrière.  

Soit \(D\) la distance entre le point d'arrivée et le point de départ (en nombre de pas).

\begin{enumerate}
    \item Définir un espace probabilisé décrivant l'expérience.
    \item Donner une expression de la variable aléatoire \(D\) (on pourra introduire la variable aléatoire \(X\) donnant le nombre de piles obtenus).
    \item Déterminer la loi de \(D\).
\end{enumerate}
}

\exercise{34}{Lois d'images et loi conjointe}{
On considère \(\Omega = \{a,b,c,d,e\}\) muni de la probabilité uniforme.  

Soient \(X\) et \(Y\) deux variables aléatoires définies par :
\[
X(a)=X(b)=X(c)=1, \quad X(d)=2, \quad X(e)=3,
\]
\[
Y(b)=Y(d)=1, \quad Y(a)=Y(c)=Y(e)=2.
\]

Calculer les lois de \(X\), \(Y\), \(X+Y\) et du couple \((X,Y)\).
}
\switchcolumn
\newpage
\exercise{35}{Loi géométrique et absence de mémoire}{
Soit \(X\) une variable aléatoire géométrique de paramètre \(p\), c'est-à-dire
\[
P(X=n) = p(1-p)^n \quad \text{pour } n \ge 0.
\]

\begin{enumerate}
    \item Calculer \(P(X \ge n)\) pour tout entier \(n\).
    \item En déduire la propriété d'absence de mémoire :
    \[
    P(X \ge n+m \mid X \ge n) = P(X \ge m)
    \quad \text{pour tous } n,m.
    \]
\end{enumerate}
}

\exercise{36}{Somme d'indicatrices}{
Soient \(A\), \(B\), \(C\) trois événements tels que :
\[
P(A)=\frac{1}{2}, \quad
P(B)=P(C)=\frac{5}{12},
\]
\[
P(A \cap B)=P(B \cap C)=\frac{4}{12}, \quad
P(A \cap C)=\frac{3}{12}, \quad
P(A \cap B \cap C)=\frac{3}{12}.
\]

Chercher la loi de la variable aléatoire
\[
X = \mathbf{1}_A + \mathbf{1}_B + \mathbf{1}_C.
\]
}

\textbf{LANGAGE C}\\\\

\exercise{37}{Apprentissage du cours}{Voir et s'entraîner sur l'allocation mémoire dynamique et les pointeurs.\\}

\textbf{ÉLÉMENTS D'ALGORITHMIQUE}\\\\

\exercise{38}{Apprentissage du cours}{
    Revoir les notions suivantes des algorithmes :
    \begin{enumerate}
        \item Complexité des additions/multiplications en fonction de la taille du nombre
        \item Algorithme de multiplication naïf et de Karatsuba
        \item Algorithme de Fibonacci naïf et de Fibonacci rapide
        \item Algorithmes de tri : tri par insertion, tri par sélection, tri fusion
    \end{enumerate}
}

\switchcolumn
\newpage

\exercise{39}{Problème de l'élément le plus original}{
    On s'intéresse au problème suivant : étant donné une liste $L$ de nombres \textit{(non nécessairement entiers)} de longueur $n \geq 2$, déterminer l'élément le plus original de $L$, \textit{i.e.} celui qui y apparaît le moins de fois (ou l'un quelconque d'entre eux, en cas d'égalité).\\
    \begin{enumerate}
        \item Décrire un algorithme naïf permettant de résoudre ce problème sans modifier la liste $L$, et avec mémoire auxiliaire constante.
        \item Quel est l'ordre de grandeur de sa complexité en temps dans le pire cas ? Justifier.
        \item Comment résoudre ce problème avec une complexité en temps strictement meilleure ? Laquelle ?
    \end{enumerate}
}

\exercise{40}{Algorithme de Karatsuba revisité}{
    On rappelle l'identité remarquable $2ab = (a + b)^2 - a^2 - b^2$. En déduire une expression de $(2p \cdot a + b)^2$ faisant intervenir uniquement des puissances de 2 et les trois carrés $a^2$, $b^2$ et $(a + b)^2$.
    \begin{enumerate}
        \item En déduire un algorithme de type "diviser pour régner" inspiré de l'algorithme de Karatsuba pour
        \item Calculer le carré d'un entier de $n$ bits en temps $\Theta(n^{\log_2 3})$.
        \item Justifier la complexité de cet algorithme.
    \end{enumerate}
}
\switchcolumn
\exercise{41}{Montagne}{
On dit qu'un tableau \(T\) de \(n\) entiers est une \emph{montagne} s'il est constitué d'une première partie strictement croissante, suivie d'une deuxième strictement décroissante, chacune pouvant éventuellement être vide.  

Autrement dit, \(T\) est une montagne s'il est strictement croissant ou strictement décroissant, ou s'il existe un indice \(m \in \llbracket 1, n-2 \rrbracket\) tel que :
\[
T[0] < T[1] < \dots < T[m]
\quad \text{et} \quad
T[m] > T[m+1] > \dots > T[n-1].
\]

\begin{enumerate}
    \item Proposer un algorithme \texttt{est\_une\_montagne(T)} de complexité en temps optimale\footnote{C'est-à-dire l'algorithme qui vous semble le plus efficace ; il ne vous est pas demandé de prouver son optimalité.} qui teste si \(T\) est une montagne.  
    Justifier rapidement sa correction et sa complexité.

    \item \textbf{On suppose maintenant que \(T\) est une montagne.}\\  
    Proposer un algorithme \texttt{pied(T)} de complexité en temps optimale\footnotemark[1] qui renvoie le plus petit élément de \(T\).  
    Justifier sa correction et sa complexité.

    \item Étant donné un indice \(i\), comment tester en temps constant si \(i < m\), où \(m\) est l'indice (inconnu a priori) du maximum de \(T\) ?

    \item En déduire un algorithme \texttt{sommet(T)} de complexité en temps optimale\footnotemark[1] qui renvoie le plus grand élément de \(T\).  
    Justifier rapidement sa correction et sa complexité.

    \item Proposer un algorithme \texttt{nivelle(T)} de complexité en temps optimale\footnotemark[1] qui renvoie un tableau trié contenant les mêmes éléments que \(T\).  
    Justifier rapidement sa correction et sa complexité.
\end{enumerate}
}

\end{paracol}
\end{document}
