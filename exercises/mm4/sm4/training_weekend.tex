\documentclass{article}

\usepackage[a4paper,landscape,margin=2cm]{geometry}
\usepackage{../../../components/components}
\usepackage{colortbl}
\usepackage{array}

\usepackage{fancyhdr}
\pagestyle{fancy}
\fancyhf{}
\fancyhead[L]{DL2 Math-Info}
\fancyhead[C]{Séance révision semaine 4 (bis)}
\fancyhead[R]{2025-2026}
\fancyfoot[L]{Ewen Rodrigues de Oliveira}
\fancyfoot[R]{\thepage}

\usepackage{paracol}

\begin{document}
\setlength{\columnsep}{2cm}

\begin{paracol}{2}

\renewcommand{\arraystretch}{1.3}
\noindent
\begin{tabular}{|p{0.6\columnwidth}|>{\centering\arraybackslash}p{0.35\columnwidth}|}
\hline
\rowcolor[RGB]{230,237,247}
\textbf{Savoir-faire} & \textbf{Exercices} \\
\hline
Calculer des probabilités conditionnelles & 2,3,4,5 \\
\hline
Appliquer le procédé de Gram-Schmidt & 6,7,8,9 \\
\hline
\end{tabular}

\vspace{1em}

\attention{Ces exercices n'ont pour la plupart pas été traités en cours. Si vous avez besoin d'une correction et que vous n'avez pas accès à ChatGPT, n'hésitez pas à m'envoyer un message ! L'exercice 3 est à faire pour lundi.}

\exercise{1}{Une virée en enfer}{
    Pour chaque programme, calculer rigoureusement le nombre $\mathcal{A}(n)$ d'opérations $(+, \times)$ en fonction de $n$.
}
\begin{lstlisting}[language=Python]
def foo_1(T) : 
    return 4 if len(T) == 0 else 6 * foo_1(T[3:]) + 9

def foo_2(T) : 
	m = len(T) // 3
	return 1 if m == 0 else 9 * foo_2(T[:m]) + 6 * foo_2(T[m:2*m]) + 4 * foo_2(T[2*m:])

def foo_3(T) : 
    m = len(T) // 4
	return 1 if m == 0 else 9 * foo_3(T[2*m:]) + foo_3(T[:2*m]) + 2 * foo_3(T[m:3*m]) * sum(3 for e in T)
\end{lstlisting}

\exercise{2}{Un petit modèle du Covid}{
    On considère que dans une population donnée, une personne sur 10000 est atteinte d'une certaine maladie.
    On dispose d'un test médical qui, si le patient est malade, sera positif avec probabilité $\frac{99}{100}$ et négatif avec probabilité
    $\frac{1}{100}$. Si le patient est sain, le test est positif avec probabilité $\frac{2}{100}$ et négatif avec probabilité $\frac{98}{100}$. On effectue le test
    sur une personne choisie au hasard. Calculer :
    \begin{enumerate}
        \item La probabilité que le test soit positif.
        \item La probabilité que le test donne un diagnostic correct.
        \item La probabilité que le patient soit malade sachant que le résultat du test est positif
        \item La probabilité que le patient soit sain sachant que le résultat du test est négatif
    \end{enumerate}
}

\switchcolumn

\exercise{3}{Condition sur l'indépendance}{
    Un groupe comporte 4 garçons et 6 filles de première année, 6 garçons de seconde année et n filles de seconde année. Combien doit valoir $n$ si l'on veut que, dans le choix au hasard d'un étudiant, les événements : "être un garçon" et "être en 1ère année" soient indépendants ? 
}

\exercise{4}{Raclette party}{
Une classe de $2^{e}$ année organise une grande raclette de fin de semestre. Les élèves sont répartis en trois catégories selon leur niveau d'appétit : 
\textit{petit appétit}, \textit{appétit moyen} et \textit{gros appétit}.\\
Les statistiques des années précédentes indiquent que la probabilité qu'un élève reprenne au moins une portion supplémentaire de raclette est respectivement de 5\,\%, 15\,\% et 30\,\% selon la catégorie.
On estime que dans la classe, 20\,\% des élèves ont un petit appétit, 50\,\% un appétit moyen et 30\,\% un gros appétit.
On choisit un élève au hasard dans cette classe.

    \begin{enumerate}
        \item Quelle est la probabilité que l'élève choisi reprenne au moins une portion supplémentaire de raclette ?

        \item Sachant que l'élève n'a pas repris de portion supplémentaire, quelle est la probabilité qu'il ait un petit appétit ?  
        Quelle est la probabilité qu'il ait un appétit moyen ?

        \item On apprend qu'un élève a repris au moins une portion supplémentaire.  
        Quelle est la probabilité qu'il appartienne à la catégorie ``gros appétit'' ?  
        Comparer cette probabilité avec la proportion initiale d'élèves ``gros appétit'' dans la classe et interpréter la différence.
    \end{enumerate}
}

\exercise{5}{Urne de Polya}{
    Une urne contient $b$ boules bleues et $r$ boules rouges. Une boule est tirée au hasard ; on la replace dans l'urne en ajoutant $d$ boules de la même couleur, puis on tire une seconde boule. Quelle est la probabilité :
    \begin{enumerate}
    \item que la seconde boule tirée soit rouge ?
    \item que la première boule soit rouge sachant que la seconde est rouge ?
    \item que les deux boules soient de la même couleur ?
    \end{enumerate}
}

\switchcolumn
\newpage

\exercise{6}{Espaces préhilbertiens (TD4.7)}{
Un espace préhilbertien est un espace vectoriel muni d’un produit scalaire, mais pas forcément de dimension finie.  

Soit $(e_1, e_2, \dots, e_n)$ une famille de vecteurs unitaires d’un espace préhilbertien réel $E$ telle que
\[
\forall x \in E, \quad \|x\|^2 = \sum_{i=1}^{n} (e_i \mid x)^2.
\]

Montrer que $(e_1, e_2, \dots, e_n)$ est une base orthonormée de $E$.
}


\exercise{7}{Procédé de Gram-Schmidt (TD4.9)}{
Soit $\mathbb{R}^3$ muni du produit scalaire canonique.  

Construire une base orthonormée en utilisant le procédé de Gram--Schmidt à partir de la famille $(u, v, w)$ définie par :
\[
u = (1,0,1), 
\qquad
v = (1,1,1), 
\qquad
w = (-1,1,0).
\]
}


\exercise{8}{Gram-Schmidt dans $\mathbb{R}^3$}{
Soit $\mathbb{R}^3$ muni du produit scalaire canonique.  

Appliquer le procédé de Gram--Schmidt à la famille $(u_1,u_2,u_3)$ définie par :
\[
u_1 = (1,1,0), 
\qquad
u_2 = (1,0,1), 
\qquad
u_3 = (0,1,1).
\]

On demandera :
\begin{enumerate}
    \item de vérifier que la famille est libre ;
    \item de construire une base orthogonale ;
    \item de normaliser cette base.
\end{enumerate}
}
\switchcolumn
\newpage


\exercise{9}{Gram-Schmidt dans un espace de polynômes}{
On considère l’espace $E = \mathbb{R}_2[X]$ muni du produit scalaire
\[
\langle P,Q\rangle = \int_0^1 P(t)Q(t)\,dt.
\]

À partir de la famille $(1, X, X^2)$, construire une base orthonormée de $E$ en utilisant le procédé de Gram--Schmidt.
}

\end{paracol}
\end{document}
