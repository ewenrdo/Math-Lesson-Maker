\documentclass{article}

\usepackage[a4paper,landscape,margin=2cm]{geometry}
\usepackage{../../../components/components}
\usepackage{colortbl}
\usepackage{array}

\usepackage{fancyhdr}
\pagestyle{fancy}
\fancyhf{}
\fancyhead[L]{DL2 Math-Info}
\fancyhead[C]{Séance révision semaine 4}
\fancyhead[R]{2025-2026}
\fancyfoot[L]{Ewen Rodrigues de Oliveira}
\fancyfoot[R]{\thepage}

\usepackage{paracol}

\begin{document}
\setlength{\columnsep}{2cm}

\begin{paracol}{2}

\renewcommand{\arraystretch}{1.3}
\noindent
\begin{tabular}{|p{0.6\columnwidth}|>{\centering\arraybackslash}p{0.35\columnwidth}|}
\hline
\rowcolor[RGB]{230,237,247}
\textbf{Savoir-faire} & \textbf{Exercices} \\
\hline
Utiliser les théorèmes sur la convergence des séries des dérivées & 2,3 \\
\hline
Dénombrer et calculer des probabilités & 4,5,6,7 \\
\hline
\end{tabular}

\vspace{1em}


\exercise{1}{TD3.5}{
    Pour tout entier \(n \ge 1\), soit  
    \(f_n : \mathbb{R} \to \mathbb{R}\) la fonction définie par :  
    \[
    f_n(x) = \frac{1}{n^2 + x^2}.
    \]

    \begin{enumerate}
        \item Montrer que la série \(\sum f_n\) converge simplement sur \(\mathbb{R}\). On note \(s\) la somme de cette série.
        \item Montrer que \(s\) est continue sur \(\mathbb{R}\).
        \item Montrer que pour tout réel \(a > 0\), la série \(\sum f_n'\) converge normalement sur \([-a, a]\).
        \item Montrer que \(s\) est dérivable sur \(\mathbb{R}\).
    \end{enumerate}
}


\exercise{2}{TD3.7}{
    Soit \(f : \, ]-1, 1[ \to \mathbb{R}\) la fonction définie par :  
\[
f(x) = \sum_{n=1}^{\infty} \frac{x^n \sin(nx)}{n}.
\]

\begin{enumerate}
    \item Montrer que \(f\) est de classe \(C^1\) sur \(]-1, 1[\).
    \item Calculer \(f'(x)\) et en déduire que :  
    \[
    f(x) = \arctan\left( \frac{x \sin x}{1 - x \cos x} \right).
    \]
\end{enumerate}
}

\exercise{3}{TD3.2b}{
    Étudier la convergence de $\sum_{n \geq 1} \frac{1}{n + n^3 x^2}$ sur $\mathbb{R}_{+}^{*}$. 
}

\switchcolumn

\exercise{4}{TD2.11 : Supplémentaire d'hyperplans}{
    Soit $E$ un $\mathbb{R}$-espace vectoriel de dimension finie et soit $\phi$ une forme linéaire non nulle sur $E$. Montrer que pour tout $u \in E \setminus \ker(\phi)$, $\ker(\phi)$ et $\mathrm{Vect}(u)$ sont supplémentaires dans $E$.
}

\exercise{5}{TD1.10}{
    On choisit un nombre au hasard parmi les entiers entre $1$ et $100$. Quel est l'espace probabilisé suggéré par cet énoncé ?
}

\exercise{6}{TD1.19}{
    Une urne contient 4 boules blanches et 6 boules noires. On tire au hasard successivement 3 boules sans remise.
    \begin{enumerate}
        \item Définir un espace de probabilité $(\Omega, \mathbb{P})$ correspondant à cette expérience.
        \item Quelle est la probabilité d'obtenir les 3 boules blanches ?
        \item Quelle est la probabilité d'obtenir une boule noire au 2ème tirage ?
        \end{enumerate}
}

\exercise{7}{Paradoxe des anniversaires}{
    On considère un groupe de n personnes prises au hasard dans la population. Quelle est la probabilité qu’au moins deux d’entre elles aient leur anniversaire le même jour ? Pour simplifier on pourra considérer que toutes les années ont 365 jours.
}

\exercise{8}{TD1.14}{
    Pour chacune des expériences aléatoires suivantes, proposer un espace probabilisé adéquat :
    \begin{enumerate}
    \item On jette une pièce de monnaie 10 fois de suite en s’intéressant à l’apparition à chaque jet de pile (P) ou face (F).
    \item On distribue à un joueur 13 cartes d’un jeu de 52 cartes correctement battues.
    \item On joue à pile ou face jusqu’à obtenir face
    \end{enumerate}
}

\noindent\textit{Il conviendra également d'apprendre le reste du cours, en particulier : 
\begin{itemize}
    \item le lemme d'Abel (et le théorème qui s'ensuit)
    \item le cours sur les espaces euclidiens
\end{itemize}    
}

\switchcolumn
\newpage

\exercise{9}{Logique propositionnelle (CAPES sujet 0)}{
    On désigne par $\mathcal{F}$ un ensemble de fonctions définies sur $\mathbb{R}$. On donne deux assertions $P$ et $Q$ :
    \[ (P) : \forall f \in \mathcal{F}, \exists x \in \mathbb{R}, f (x) = 0 \]
    \[ (Q) : \exists x \in \mathbb{R}, \forall f \in \mathcal{F}, f (x) = 0 \]
Ces deux assertions sont-elles équivalentes ?
}

\exercise{10}{Polynômes (CAPES sujet 0)}{
    On désigne par $\mathbb{R}[X]$ l’ensemble des polynômes d'une variable $X$ à coefficients réels. Les affirmations suivantes sont-elles vraies ou fausses ? Justifier votre réponse.
    \begin{enumerate}
        \item Tout polynôme $P \in \mathbb{R}[X]$ sans racine réelle est irréductible.
        \item Le polynôme $P = X^3 + 2X^2 + X$ est le polynôme caractéristique d'une matrice carrée inversible à coefficients réels.
    \end{enumerate}
}

\exercise{11}{Calcul intégral de Bibm@th}{
    Calculer les intégrales suivantes :
    \[
    I=\int_1^2 \frac{\ln(1+t)}{t^2} dt \quad J=\int_0^1 x(\arctan x)^2 dx \quad K=\int_0^1 \frac{x\ln x}{(x^2+1)^2} dx
    \]
}


\end{paracol}
\end{document}
