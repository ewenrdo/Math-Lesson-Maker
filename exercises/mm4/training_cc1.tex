\documentclass{article}

\usepackage[a4paper,landscape,margin=2cm]{geometry}
\usepackage{../../components/components}
\usepackage{colortbl}
\usepackage{array}

\usepackage{fancyhdr}
\pagestyle{fancy}
\fancyhf{}
\fancyhead[L]{DL2 Math-Info}
\fancyhead[C]{Séance révision semaine 4}
\fancyhead[R]{2025-2026}
\fancyfoot[L]{Ewen Rodrigues de Oliveira}
\fancyfoot[R]{\thepage}

\usepackage{paracol}

\begin{document}
\setlength{\columnsep}{2cm}

\begin{paracol}{2}

\renewcommand{\arraystretch}{1.3}
\noindent
\begin{tabular}{|p{0.6\columnwidth}|>{\centering\arraybackslash}p{0.35\columnwidth}|}
\hline
\rowcolor[RGB]{230,237,247}
\textbf{Savoir-faire} & \textbf{Exercices} \\
\hline
Dualité & 1\\
\hline
Utiliser les théorèmes sur la convergence uniforme & 2,3,4\\
\hline
Dénombrer et calculer des probabilités & 4,5,6,7\\ \hline
\end{tabular}

\vspace{1em}

\exercise{1}{Annales 2023-2024 - Anthony F.}{
    Soit $E = \mathbb{R}_2[X]$ et soit $\mathcal{B} = (1, X, X^2)$ la base canonique de $E$.
    \begin{enumerate}
        \item Soit $\mathcal{B}^* = (\mu_1, \mu_2, \mu_3)$ la base duale de $\mathcal{B}$ de $E$. Calculer $\mu_1(X)$.
        \item Montrer que la matrice dans la base $\mathcal{B}$ de l'application linéaire $\phi : P(X) \mapsto P(X+1)$ est : \[ A = \begin{pmatrix} 1 & 1 & 1 \\ 0 & 1 & 2 \\ 0 & 0 & 1 \end{pmatrix} \]
        \item Déterminer la matrice de $\phi^*$ dans la base duale $\mathcal{B}^*$. 
        \item L'application $P \mapsto P(0) \cdot P(1)$ est-elle une forme linéaire sur $E$ ?
        \item Indiquer $\lambda_1$, une combinaison linéaire non nulle de $\mathcal{B^*}$, qui s'annule sur le polynôme $1+X+X^2$.
    \end{enumerate}
}


\exercise{2}{TD3.5}{
    Pour tout entier \(n \ge 1\), soit  
    \(f_n : \mathbb{R} \to \mathbb{R}\) la fonction définie par :  
    \[
    f_n(x) = \frac{1}{n^2 + x^2}.
    \]

    \begin{enumerate}
        \item Montrer que la série \(\sum f_n\) converge simplement sur \(\mathbb{R}\). On note \(s\) la somme de cette série.
        \item Montrer que \(s\) est continue sur \(\mathbb{R}\).
        \item Montrer que pour tout réel \(a > 0\), la série \(\sum f_n'\) converge normalement sur \([-a, a]\).
        \item Montrer que \(s\) est dérivable sur \(\mathbb{R}\).
    \end{enumerate}
}

\switchcolumn

\exercise{2}{TD3.6}{
   Pour tout entier \(n > 0\) et tout réel \(x\), on pose :  
\[
f_n(x) = \frac{\arctan(nx)}{n^2},
\]  
et on considère  $f(x) = \sum_{n \ge 1} f_n(x)$ définie sur $\mathbb{R}$. On précise de $f$ est impaire.

\begin{enumerate}
    \item Étudier la convergence normale de \(f\) sur \(\mathbb{R}\), préciser si \(f\) est continue et quelles sont ses limites en \(+\infty\) et en \(-\infty\).
    \item Montrer que \(f\) est de classe \(C^1\) sur \(\mathbb{R}^*\). Montrer que \(f'\) est décroissante sur \(]0, +\infty[\).
    \item Montrer que \(f\) n'est pas dérivable en 0.
\end{enumerate}
}

\exercise{3}{TD3.7}{
    Soit \(f : \, ]-1, 1[ \to \mathbb{R}\) la fonction définie par :  
\[
f(x) = \sum_{n=1}^{\infty} \frac{x^n \sin(nx)}{n}.
\]

\begin{enumerate}
    \item Montrer que \(f\) est de classe \(C^1\) sur \(]-1, 1[\).
    \item Calculer \(f'(x)\) et en déduire que :  
    \[
    f(x) = \arctan\left( \frac{x \sin x}{1 - x \cos x} \right).
    \]
\end{enumerate}

}

\exercise{4}{Le banc sexiste}{
    \begin{enumerate}
    \item De combien de façons peut-on asseoir en rang 3 garçons et 3 filles (qui sont des individus discernables !) ?
    \item Même question si les garçons doivent rester ensemble et les filles aussi.
    \item Même question si seuls les garçons doivent rester ensemble.
    \item Même question si deux personnes de même sexe ne doivent jamais être voisins
    \end{enumerate}
}
\switchcolumn
\exercise{5}{Au loto}{
On range \(p\) boules dans \(n\) cases.  
Dans chacun des cas suivants, dire combien il y a de rangements possibles, et indiquer les contraintes éventuelles sur \(p\) et \(n\) (par exemple \(p \le n\)) :  

\begin{enumerate}
    \item Les boules et les cases sont discernables, l'ordre dans chaque case n'est pas pris en compte, chaque case pouvant recevoir un nombre quelconque de boules.
    \item Les boules et les cases sont discernables, chaque case ne pouvant recevoir qu'une boule au maximum.
    \item Les boules sont indiscernables, les cases sont discernables, chaque case ne pouvant recevoir qu'une seule boule.
    \item Les boules sont indiscernables, les cases sont discernables, chaque case pouvant recevoir un nombre quelconque de boules.
\end{enumerate}
}

\exercise{6}{Et si elle tombe droite ?}{
    On joue 4 fois à pile ou face.
\begin{enumerate}
\item Donner un espace de probabilité associé à cette expérience.
\item Quelle est la probabilité d’obtenir au moins 3 faces ou 3 piles consécutivement ?
\end{enumerate}
}

\exercise{7}{Secrétaire distrait}{
    Un secrétaire un peu distrait a tapé \(N\) lettres et préparé \(N\) enveloppes portant les adresses des destinataires, mais il répartit au hasard les lettres dans les enveloppes.  
Pour \(1 \le j \le N\), on note \(A_j\) l’événement « la \(j\)-ème lettre se trouve dans la bonne enveloppe ».  

\begin{enumerate}
    \item Définir un espace de probabilité \((\Omega_N, P_N)\) associé à cette expérience aléatoire.
    \item Calculer \(P_N(A_j)\).
    \item On fixe \(k\) entiers \(i_1 < i_2 < \dots < i_k\) entre \(1\) et \(N\).  
    Dénombrer toutes les permutations \(\sigma\) sur \(\{1, \dots, N\}\) telles que \(\sigma(i_1) = i_1, \dots, \sigma(i_k) = i_k\).  
    En déduire \(P_N(A_{i_1} \cap A_{i_2} \cap \dots \cap A_{i_k})\).
    \switchcolumn
    \item On note \(B\) l’événement « au moins une des lettres est dans la bonne enveloppe ».  
    Exprimer \(B\) à l’aide des \(A_j\).
    \item Utiliser la formule de Poincaré pour calculer \(P_N(B)\) et sa limite quand \(N \to +\infty\).
\end{enumerate}

}

\textit{Il conviendra également d'apprendre le reste du cours, en particulier : le théorème III.B.1 et le lemme d'Abel (et le théorème qui s'ensuit)}


\switchcolumn

\end{paracol}

\end{document}
