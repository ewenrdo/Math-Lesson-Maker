\documentclass{article}

\usepackage[a4paper,landscape,margin=2cm]{geometry}
\usepackage{../../components/components}
\usepackage{colortbl}
\usepackage{array}

\usepackage{fancyhdr}
\pagestyle{fancy}
\fancyhf{}
\fancyhead[L]{DL1 Math-Info}
\fancyhead[C]{Applications linéaires}
\fancyhead[R]{2024-2025}
\fancyfoot[L]{Ewen Rodrigues de Oliveira}
\fancyfoot[R]{\thepage}

\usepackage{paracol}

\begin{document}
\setlength{\columnsep}{2cm}

% Début des colonnes
\begin{paracol}{2}

\renewcommand{\arraystretch}{1.3}
\noindent
\begin{tabular}{|p{0.6\columnwidth}|>{\centering\arraybackslash}p{0.35\columnwidth}|}
\hline
\rowcolor[RGB]{230,237,247}
\textbf{Savoir-faire} & \textbf{Exercices} \\
\hline
Reprehenderit amet mollit incididunt non quis eiusmod.& 1, 4 \\
\hline
Amet ex aute nulla esse laboris. & \\
\hline
\end{tabular}

\vspace{1em}

\exercise{1}{Titre de l'exercice}{Voici le texte de l'exercice qui commence sur la ligne d'après.}

\exercise{2}{Deuxième exercice}{Texte de l'exercice 2.}

\exercise{3}{Troisième exercice}{Texte de l'exercice 3 qui sera dans la colonne 2 si la colonne 1 est pleine.}

\switchcolumn % Changer de colonne

\exercise{4}{Titre de l'exercice}{Voici le texte de l'exercice qui commence sur la ligne d'après.}

\exercise{5}{Deuxième exercice}{Texte de l'exercice 2.}

\exercise{6}{Troisième exercice}{Texte de l'exercice 3 qui sera dans la colonne 2 si la colonne 1 est pleine.}


% Passer à la page suivante
\switchcolumn
\newpage
\exercise{7}{Titre de l'exercice}{Voici le texte de l'exercice qui commence sur la ligne d'après.}

\end{paracol}

\end{document}
