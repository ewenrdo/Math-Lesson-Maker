\documentclass{article}

\usepackage[a4paper, left=1.5cm, right=1.5cm, top=2cm, bottom=2cm]{geometry}
\usepackage{../../../components/components}

\usepackage{fancyhdr}

% Configuration des en-têtes et pieds de page
\pagestyle{fancy}
\fancyhf{} % reset tout

\fancyhead[L]{Terminale Spé Maths}
\fancyhead[C]{Dénombrement}
\fancyhead[R]{2025-2026}

\fancyfoot[L]{Ewen Rodrigues de Oliveira}
\fancyfoot[R]{\thepage}

\begin{document}

\docTitle{Dénombrement}

\section{Partie d'un ensemble}

\definition{
Soit $E$ un ensemble.

\begin{itemize}
  \item On dit qu'un ensemble $A$ est une \textbf{partie} de $E$ si tous les éléments de $A$ sont des éléments de $E$. On écrit dans ce cas $A \subseteq E$.
  \item L'ensemble de toutes les parties de $E$ est noté $\mathcal{P}(E)$.
\end{itemize}
}

\example{Notons $E$ l'ensemble des élèves du groupe de spécialité maths 4.}
\noindent\carreaux{6}

\example{
    \begin{enumerate}
        \item Si $E = \{0;1;2;3\}$ alors $\varnothing$, $\{0;1\}$, $\{0\}$, $\{0;2\}$ sont des parties de $E$.
        \item $\{0; 2024\}$ et $\mathbb{Z}$ sont des parties de $\mathbb{R}$.
    \end{enumerate}
}

\vocabulary{
Une partie à 1 élément s'appelle un \textbf{singleton}. Une partie à 2 éléments s'appelle une \textbf{paire}.
}

\remark{
Tout ensemble $E$ contient au moins deux parties : l'ensemble vide et lui-même. Autrement dit, nous avons toujours $\varnothing \subseteq E$ et $E \subseteq E$.
}

\theorem{Propriété}{Nombre de parties d'un ensemble fini}{}%
{
Soit $E$ un ensemble fini à $n$ éléments. Le nombre de parties de $E$ est égal à $2^n$. Autrement dit :
\[
\mathrm{Card}(\mathcal{P}(E)) = 2^n.
\]
}


\noindent{\textbf{Démonstration :}\\
    Lorsque l'on cherche à construire une partie de $E$, pour chaque élément de $E$ il y a deux choix possibles : soit il est dans cette partie, soit il n'y est pas.\\
    Si un élément est dans cette partie, on peut considérer qu'on lui affecte la valeur 1, et la valeur 0 sinon.\\
    Ainsi, cela revient à déterminer le nombre de $n$-uplets de $\{0;1\}$, ou encore le cardinal de l'ensemble $\{0;1\}^n$.\\
    On sait que $\mathrm{Card}(\{0;1\}) = 2$, et par conséquent, d'après la propriété sur le cardinal du produit cartésien, on a :
    \[
    \mathrm{Card}(\{0;1\}^n) = 2^n.
    \]
}

\newpage

\training{On pose $E = \{\ln(3);\, e^\pi;\, 2024\}$.\\
1. Déterminer le nombre de parties de $E$.\\
\noindent{\carreaux{2}}\\\\
2. Donner toutes les parties de $E$.\\
\carreaux{4}
}

\section{Combinaisons}

\subsection{Définition}

\definition{
Soit $E$ un ensemble fini de cardinal $n$ et $k$ un entier naturel tel que $k \le n$.

\begin{itemize}
  \item Une \textbf{combinaison} de $k$ éléments de $E$ est une partie de $E$ de cardinal $k$.
  \item Le nombre de combinaisons de $k$ éléments de $E$ est $\binom{n}{k}$.
\end{itemize}
}

\attention{
Dans une combinaison, l'ordre dans lequel on énumère les éléments ne compte pas. Par exemple, si $a$ et $b$ sont deux éléments d'un ensemble $E = \{a, b\} = \{b, a\}$ (en tant qu'éléments de $\mathcal{P}(E)$) alors que $(a ; b) \ne (b ; a)$ (en tant qu'éléments de $E \times E$).
}

\theorem{Propriété}{}{true}{
Soient $n$ et $k$ deux entiers naturels tels que $k \le n$. Alors :

\begin{enumerate}
  \item $\binom{n}{1} = n$ ; \qquad $\binom{n}{n} = 1$ ; \qquad et \qquad $\binom{n}{0} = 1$ ;
  \item \textbf{Symétrie des coefficients binomiaux :} \quad $\displaystyle \binom{n}{n-k} = \binom{n}{k}$ ;
  \item \textbf{Formule de Pascal :} \quad $\displaystyle \binom{n}{k} + \binom{n}{k+1} = \binom{n+1}{k+1}$.
\end{enumerate}
}

\ndlr{Démonstration laissée à l'appréciation du lecteur.}

\theorem{Propriété}{Somme des coefficients binomiaux}{}{
    Soit $n \in \mathbb{N}$. Alors :
    \[
    \sum_{k=0}^{n} \binom{n}{k} = 2^n.
    \]
}

\subsection{Nombre de combinaisons à $k$ éléments}

\theorem{Propriété}{Nombre de combinaisons à $k$ éléments}{true}
{
Soient $n$ et $k$ deux entiers naturels tels que $0 \le k \le n$. Alors :
\[
\binom{n}{k} = \frac{n!}{k!(n-k)!}
\]
}


\end{document}
