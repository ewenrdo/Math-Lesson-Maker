\documentclass{article}

\usepackage[a4paper, left=1.5cm, right=1.5cm, top=2cm, bottom=2cm]{geometry}

\usepackage{../../../../components/components}

\usepackage{fancyhdr}


% Configuration des en-têtes et pieds de page
\pagestyle{fancy}
\fancyhf{} % reset tout

\fancyhead[L]{DL2 Math-Info PI4}
\fancyhead[C]{Projet informatique}
\fancyhead[R]{2025-2026}

\fancyfoot[L]{Ewen Rodrigues de Oliveira}
\fancyfoot[R]{\thepage}

\begin{document}

\docTitle{Explication du Projet Java}

\section{Organisation générale}
\subsection{Déroulement}
4 à 5 étudiants
6 sujets à se répartir (3 équipes par sujet)
Evaluation continue
C'est l'étudiant qui a proposé le sujet qui nous gère, pendant des réunions de 15 à 20 minutes.
On doit avoir des tests unitaires, CI/CD, etc.
Utiliser une "difficulté bibliographique" (algo, domaine culturel, librairie, ...)

\subsection{Crédits}
3 ECTS.
On doit voir le travail. Donc communiquer sur GitLab, valider, etc.

Evaluation : une seule session.
Note individuelle. Une soutenance finale.
Tout le monde doit avoir une connaissance globale du projet.

\subsection{Travail en équipe}
Communication par mail avec le tuteur :
- horaire précis des rdv, présence obligatoire (absence à justifier au secrétariat)

Rendu : ce qui est sur main au moment du rendu. 
Noté sur tout ce qu'il y a sur gitlab (code, collaboration, etc.)

Contraintes :
- Compilation sans intermédiaire (javac, on peut utiliser maven/gradle mais en discuter d'abord avec le tuteur)
- Swing et AWT 

On créera un ticket "Journal" où on chaque étudiant fera un bilan de son travail chaque semaine en commentaire.
+ un comment avec la TDL de la semaine suivante.
--> Suivre la méthode Agile.

\subsection{Soutenance et critères d'évaluation}

Sujet est vague, et interprétable. Il convient de le préciser en équipe et de le cadrer.
D'ici mi-semestre, on doit avoir une version avec les fonctionnalités minimales.

On doit avoir une bonne portabilité

On veut de la documentation.
Programme facile à installer.
Ergonomie, qualité du code.
Compétences, initiatives, etc.
La soutenance prend une place importance, car les profs ont conscience qu'on va sans doute utiliser de l'IA.

Soutenance :
- Jury
- 15 minutes d'exposé, 20/30 minutes de questions.
Il y a une harmonisation des notes entre les groupes.

Emails des tuteurs :
Aldric Degorre aldric.degorre@irif.fr
Vincent Padovani padovani@irif.fr
Vlady Ravelomanana vlad@irif.fr
RdV en 538C (et coworking en 548C + 532C)

\section{Sujets}
\subsection{Aldric Degorre}
\textbf{Adresse e-mail :} aldric.degorre@irif.fr

1/ Course sans fin. Style Alto's Odyssey. (on avance sans fin, jusqu'à mourir)
-> Génération procédurale de niveaux, algo de génération de terrain
-> Physique
-> Doit être fluide et agréable.

2/ The Incredible Machine. Résolution de casse-têtes, utilisation d'éléments logico-physiques à placer dans le niveau qui doivent intéragir (eg. balles, tapis roulants, etc.).
Recoder le jeu original, et aller plus loin après. Beaucoup de physique.
\subsection{Vincent Padovani}
\textbf{Adresse e-mail :} padovani@irif.fr

1/ Robbie le robot. Jeu de déplacement avec des cases spéciales (déplacement, actions, etc.).
Travail de conception important, doit être esthétique et satisfaisant

2/ Star Stuff. Mélange du déplacement avec des robots programmables (scratch-like).
(faire un jeu qui va dans la même direction que ce jeu).

\subsection{Vlady Ravelomanana}
\textbf{Adresse e-mail :} vlad@irif.fr

1/ Symplexis: simulation sur la diplomatie. Jeu de plateau avec des guerres hybrides, des niveaux de pays, etc.
2/ Middle-Earth Civilisation (adaptable en Dune, SW, etc.). Conflits stratégiques, entretien, diplomatie, recherche, actions (bâtir, explorer, espionner, attaquer, etc.).
--> Accent à mettre sur les aspects stratégiques et de combat.
\end{document}