\documentclass{article}

\usepackage[a4paper, left=1.5cm, right=1.5cm, top=2cm, bottom=2cm]{geometry}

\usepackage{../../../components/components}

\usepackage{fancyhdr}


% Configuration des en-têtes et pieds de page
\pagestyle{fancy}
\fancyhf{} % reset tout

\fancyhead[L]{DL2 Math-Info EA4}
\fancyhead[C]{Éléments d'algorithmique}
\fancyhead[R]{2025-2026}

\fancyfoot[L]{Ewen Rodrigues de Oliveira}
\fancyfoot[R]{\thepage}

\begin{document}

\docTitle{Cours 2 : Introduction à Python}

Pour exécuter du code Python, il faut installer Python (version 3 de préférence) sur son ordinateur.\\
Et pour le lancer depuis le terminal, il faut préciser dans l'en-tête du fichier le chemin vers l'interpréteur Python (shebang) :
\begin{lstlisting}[language=Python]
#! /usr/bin/env python3 : indique au système que ce fichier doit etre exécuté avec l'interpréteur python3

print("Hello, World!") # affiche le texte entre guillemets
\end{lstlisting}

On peut aussi exécuter \texttt{python3 -i mon\_fichier.py} pour lancer un script et rester en mode interactif après son exécution.
\end{document}