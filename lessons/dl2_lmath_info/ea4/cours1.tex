\documentclass{article}

\usepackage[a4paper, left=1.5cm, right=1.5cm, top=2cm, bottom=2cm]{geometry}

\usepackage{../../../components/components}

\usepackage{fancyhdr}


% Configuration des en-têtes et pieds de page
\pagestyle{fancy}
\fancyhf{} % reset tout

\fancyhead[L]{DL2 Math-Info EA4}
\fancyhead[C]{Éléments d'algorithmique}
\fancyhead[R]{2025-2026}

\fancyfoot[L]{Ewen Rodrigues de Oliveira}
\fancyfoot[R]{\thepage}

\begin{document}

\docTitle{Cours 1}

\section{Thème du cours}

\definition{
    Par \textbf{algorithmique}, on entend la conception et l'analyse des algorithmes.
}
\definition{
    Un \textbf{algorithme} est une méthode systématique de résolution d'un problème.\\
    Ce concept n'est pas limité à l'informatique (de nombreux algorithmes ont été écrits avant l'invention des ordinateurs).\\\\
    La notion est devenue centrale en informatique avec l'apparition de machines capables d'exécuter \textbf{fidèlement} et \textbf{rapidement} une suite d'opérations prédéfinie.
}
\example{
    Plusieurs exemples d'algorithmes célèbres :
    \begin{itemize}
        \item Des algorithmes de calcul : \textit{opérations arithmétiques, approximation de $pi, \sqrt{2}, ...$}
        \item Des constructions géométriques : \textit{milieu d'un segment, triangle équilatéral, pentagone régulier, centre d'un cercle, ...}
        \item Des recettes de cuisine
        \item Des manuels de construction : \textit{IKEA, lego, ...}
    \end{itemize}
}

Trois axes d'études dans ce cours :
\begin{enumerate}
    \item Conception d'algorithmes\\
    \textit{y'a-t-il des techniques générales ?}
    \item Preuve de correction (ie. l'algorithme fait bien ce qu'on attend de lui).\\
    Un algorithme est \textbf{correct} si pour chaque entrée, il \textbf{termine (1)} en produisant la \textbf{bonne sortie (2)}.\\
    \textit{(on peut donc distinguer la preuve de terminison et la preuve de correction proprement dite (appelée aussi correction partielle))}
    \item Étude de l'efficacité.\\
    \textit{les ressources nécessaires (temps, mémoire) sont-elles raisonnables ? Est-il possible de faire mieux ?}
\end{enumerate}

\example{
    Addition de deux entiers : 1357 + 2468.
    On dessine ça :
    \begin{center}
        \begin{tabular}{c@{\hspace{1cm}}c@{\hspace{1cm}}c@{\hspace{1cm}}c@{\hspace{1cm}}c}
              & 1 & 3 & 5 & 7 \\
            + & 2 & 4 & 6 & 8 \\
            \hline
        \end{tabular}
    \end{center}
    Et on utilise des retenues et on additionne les colonnes de droite à gauche.\\
    C'est un algorithme correct (on peut le prouver) et efficace (il est linéaire en la taille des entiers : 4 opérations au lieu de 3825 pour une addition naïve).

    En Python, ça donne ça :
}
\begin{lstlisting}[language=Python]
def addition(nb1, nb2):                        # deux entiers sous forme de tableau de taille égale en commançant par les unités
    res = []
    retenue = 0
    for(chiffre1, chiffre2) in zip(nb1, nb2) : # parcours en parallèle les tableaux
        tmp = chiffre1 + chiffre2 + retenue
        retenue = tmp // 10;                   # division euclidienne
        res.append(tmp % 10)                   # ajout à la fin du tableau
    return res + [retenue]                     # concaténation de deux tableaux

print(addition([7,5,3,1], [8,6,4,2]))  # affiche [5,2,8,3] pour 1357 + 2468 = 3825
\end{lstlisting}

\end{document}