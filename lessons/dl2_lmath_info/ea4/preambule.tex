\documentclass{article}

\usepackage[a4paper, left=1.5cm, right=1.5cm, top=2cm, bottom=2cm]{geometry}

\usepackage{../../../components/components}

\usepackage{fancyhdr}


% Configuration des en-têtes et pieds de page
\pagestyle{fancy}
\fancyhf{} % reset tout

\fancyhead[L]{DL2 Math-Info EA4}
\fancyhead[C]{Éléments d'algorithmique}
\fancyhead[R]{2025-2026}

\fancyfoot[L]{Ewen Rodrigues de Oliveira}
\fancyfoot[R]{\thepage}

\begin{document}

\docTitle{Préambule}

\section{Assiduité}

Appel en TD et TP pour s'assurer que tout le monde soit présent.\\
Rendu en TP systématique en fin de séance, modifiable pendant quelques jours.\\\\

\noindent En TP, il faut aller toujours dans la même salle (pour avoir un suivi régulier).

\section{Contact}
Charlé de TD TP : Monika Csikos\\
Écrire toujours "[EA4]" dans l'objet des mails aux enseignants.

\section{MCC}

Session 1 :
\begin{itemize}
    \item Assiduité et sérieux
    \item Contrôles continus  en TD (env. 3, portant sur le cours)
    \item Partiel mi-semestre
    \item Examen, qui comptera pour 50\% de la note finale (l'examen est dûr et long, le barème est adapté et il n'est pas nécessaire de tout faire pour avoir une bonne note)
\end{itemize}

\noindent Session 2 : examen final comptant pour 100\% de la note finale.\\
Des autoévaluations seront proposées sur Moodle pour se préparer aux contrôles continus.

\section{Langage de programmation}
On utilisera Python pour coder, pour sa simplicité, sa lisibilité et sa proximité avec le pseudo-code.

\end{document}