\documentclass{article}

\usepackage[a4paper, left=1.5cm, right=1.5cm, top=2cm, bottom=2cm]{geometry}

\usepackage{../../../../components/components}

\usepackage{fancyhdr}


% Configuration des en-têtes et pieds de page
\pagestyle{fancy}
\fancyhf{} % reset tout

\fancyhead[L]{DL2 Math-Info MM4}
\fancyhead[C]{Analyse}
\fancyhead[R]{2025-2026}

\fancyfoot[L]{Ewen Rodrigues de Oliveira}
\fancyfoot[R]{\thepage}

\begin{document}

\docTitle{Chapitre 3 : Suites et séries de fonctions}

\section{Suites de fonctions}

\definition{Soit $I$ un intervalle ouvert, et soit $E = \mathbb{R}$ ou $\mathbb{C}$\\
Une \textbf{suite de fonctions} est une application de $\mathbb{N}$ dans $\mathcal{F}(I, E) = E^I$, l'ensemble des fonctions de $I$ dans $E$. On la note $(f_n)_{n \in \mathbb{N}}$, où $f_n$ est la fonction associée à l'entier naturel $n$.
}

\remark{Cette notion est généralisable si $I$ devient un disque de $\mathbb{C}$ et $E$ un espace vectoriel normé.}

\example{
    $f_n : \left[ 0, +\infty \right[ \to \mathbb{R}$ définie par $f_n(x) = x^n$, pour tout $n \in \mathbb{N}$.
    On a aussi : $f_n : \mathbb{R} \to \mathbb{R}$ définie par $f_n(x) = \cos(nx)$, pour tout $n \in \mathbb{N}$. On peut tracer les graphes de ces suites de fonctions : $f_0, f_1, f_2, \ldots$
}

\subsection{Convergence simple}

\definition{
    Soit $(f_n)$ une série de fonctions de $I$ dans $E$.\\
    On dit que $(f_n)$ est \textbf{simplement convergente} sur $I$ s'il existe une fonction $f : I \to E$ telle que, pour tout $x \in I$, la suite $(f_n(x))$ converge dans $E$.\\
    On appelle dans ce cas la \textbf{limite simple} de la suite $(f_n)$ la fonction $f$ définie par : \function{f}{I \to E}{x \mapsto \lim_{n \to +\infty} f_n(x)}.
}

\example{
    On pose \function{f_n}{\left[0,1\right] \to \mathbb{R}}{x \mapsto x^n.}{}\\
    Soit $x \in \left[0,1\right]$.\\
    $lim_{n \to +\infty} x^n = \begin{cases}
        0 & \text{si } x < 1\\
        1 & \text{si } x = 1
    \end{cases}$\\
    Donc la suite $(f_n)$ converge simplement sur $\left[0,1\right]$ et sa limite simple est la fonction $f$ définie par : \function{f}{\left[0,1\right] \to \mathbb{R}}{x \mapsto \begin{cases}
        0 & \text{si } x < 1\\
        1 & \text{si } x = 1
    \end{cases}}
}
\attention{On atteint un premier problème : la limite de $f_n$ est continue mais la limite simple $f$ ne l'est pas.}

\example{
    On pose $f_n = \begin{cases}
        n^2 x & \text{si } x \in \left[0, \frac{1}{n}\right]\\
        n^2 \left( \frac{2}{n} - x \right) & \text{si } x \in \left[\frac{1}{n}, \frac{2}{n}\right]\\
        0 & \text{si } x > \frac{2}{n}
    \end{cases}$ définie de $\left[0,1\right]$ dans $\mathbb{R}$.\\
    On a que $f_n$ converge simplement vers la fonction nulle sur $\left[0,1\right]$. En effet, $f_n(0) = 0 \longrightarrow 0$ et si $x \in \left]0,1\right]$, alors il existe $N \in \mathbb{N}$ tel que $n \geq N \implies x > \frac{2}{n} \implies f_n(x) = 0$ (axiome d'Archimède). Donc $f_n(x) \longrightarrow 0$, d'où le fait que la limite simple est la fonction nulle.\\
    Cependant, $max_{[0,1]} f_n = f_n\left(\frac{1}{n}\right) = n$, qui diverge vers $+\infty$. Donc la "convergence n'est pas uniforme".\\
    Autre chose étrange, $\int_0^1 f_n(x) dx = 1$, alors qu'on s'attendrait à ce que l'intégrale de la limite soit nulle.
}
\begin{center}
    \includegraphics[width=0.25\textwidth]{./images/figure1.png}
    \captionof{figure}{Graphique de $f_n$}
\end{center}
\newpage
\subsection{Convergence uniforme}

\definition{
    On dit que $(f_n)$ converge \textbf{uniformément} sur $I$ vers $f$ s'il existe une fonction $f : I \to E$ telle que : \\
    $\forall \varepsilon > 0, \exists n_0 \in \mathbb{N}, \forall n \geq n_0, \forall x \in I, |f_n(x) - f(x)| < \varepsilon$.
}

\remark{
    $(f_n)$ converge simplement vers $f$ si $\forall x \in I, \underbrace{\forall \varepsilon > 0, \exists n_0 \in \mathbb{N}, \forall n \geq n_0, |f_n(x) - f(x)| < \varepsilon}_{(f_n(x))_{n\in\mathbb{N}} \text{ converge vers } f(x)}$. Ce n'est donc pas la même chose que la convergence uniforme, où $n_0$ ne dépend pas de $x$.
}

\example{
    Dans l'example précédent (avec les $\frac{1}{n}$), $f_n(\frac{1}{n}) \longrightarrow +\infty$, donc la convergence n'est pas uniforme sur $\left[0,1\right]$ vers $0$. En effet, $\forall n \geq 1, |f_n(\frac{1}{n}) - 0| = n \geq 1$.\\
    
}

\theorem{Proposition}{}{false}{
    \begin{enumerate}
    \item Si $(f_n)$ converge uniformément vers $f$ sur $I$, alors elle converge simplement vers $f$ sur $I$.
    \item La limite simple de $(f_n)$ est unique. Si $f$ est limite simple de $(f_n)$  par définition : $f(x) = \lim_{n \to +\infty} f_n(x)$ (qui est unique).
    \item La limite uniforme de $(f_n)$ est unique (si elle existe).
\end{enumerate}
}

\remark{Si on a la limite simple de $(f_n)$, alors c'est la seule candidate possible pour être la limite uniforme. Donc pour calculer une limite uniforme, on commence par chercher la limite simple, puis on prouve la convergence uniforme vers cette limite simple.}

\example{
    Soit $a < 1$.\\
    $f_n = x^n$ sur $\left[0,a\right] \to \mathbb{R}$.\\
    $(f_n)$ converge simplement vers $0$.
    \begin{enumerate}
        \item Soit $x \in \left[0,a\right]$, $x < 1 \Rightarrow x^n \longrightarrow 0$.\\
        Donc $0$ est la limite simple de $(f_n)$.
        \item Soit $\varepsilon > 0$.\\
        Soit $n_0 \in \mathbb{N}$ tel que soit $n \geq n_0, \forall x \in \left[0,a\right], |x^n - 0| < a^n \leq a^{n_0} < \varepsilon$ (car $a < 1$).
    \end{enumerate}
}

\cexample{Par contre, montrons que $f_n = x^n$ sur $\left[0,1\right[$ ne converge pas uniformément vers $0$ (remarquons qu'on a pas pris $1$).\\
\begin{enumerate}
    \item La limite simple est toujours $0$ (même justification que tout à l'heure).
    \item $(1-\frac{1}{n})^n \longrightarrow \frac{1}{e} \neq 0$. Donc il n'y a pas de convergence uniforme.
\end{enumerate}
}

\remark{Pour montrer que $f_n$ converge uniformément vers $f$, il suffit de trouver une majoration $M_n$ indépendante de $x$ telle que $|f_n(x) - f(x)| \leq M_n$ et $M_n \longrightarrow 0$.}

\definition{
    Soit $f$ une fonction de $I$ dans $E$.\\
    La \textbf{norme uniforme (ou norme infinie)} de $f$ est définie par : $\|f\|_{I,\;\infty} = sup_{x \in I} |f(x)|$.
}

\theorem{Proposition}{}{false}{
    Soit $(f_n)$ une suite de fonctions de $I$ dans $E$.\\
    Alors $(f_n)$ converge uniformément vers $f$ sur $I$ si et seulement si \[\begin{cases}
        f_n - f \text{ est bornée sur } I \text{ à partir d'un certain rang}\\
        \|f_n - f\|_{I,\;\infty} \longrightarrow 0
    \end{cases}\]
}

\remark{La première condition est là pour que la norme infinie soit bien définie et réelle.}

\reminder{On dit que $f \colon I \to E$ est bornée sur $I$ s'il existe $M \geq 0$ tel que $\forall x \in I, |f(x)| \leq M$. Autrement dit, $\{f(x) \mid x \in I\} = f(I)$ est une partie bornée de $E$, $\| f \|_{\infty} < +\infty$.}

\noindent{\textbf{Démonstration :}\\
    Si $\|f_n - f\|_{I,\;\infty} \longrightarrow 0$, alors $\forall x \in I, |f_n(x) - f(x)| \leq \|f_n - f\|_{I,\;\infty} \longrightarrow 0$. Donc $(f_n)$ converge uniformément vers $f$ par la remarque.\\\\
    Réciproquement, si $\| f_n - f\|_{I,\;\infty} \nrightarrow 0$, alors il existe $\varepsilon > 0$ et une sous-suite $\varphi$ strictement croissante telle que $\| f_{\varphi(n)} - f\|_{I,\;\infty} \geq \varepsilon$.\\
    Donc il existe $x_n \in I$ tel que $|f_{\varphi(n)}(x_n) - f(x_n)| \geq \varepsilon$ par définition du sup. $\Box$
}

\example{\function{f_n}{[0,1[ \to \mathbb{R}}{x \mapsto x^n}\\
\begin{enumerate}
    \item $f_n$ converge simplement vers la fonction nulle. (CVS)
    \item $\|f_n - 0\|_{[0,1[,\;\infty} = sup_{x \in [0,1[} |x^n - 0| = 1$. En effet, le prolongement par continuité de $f_n$ est $x^n$ sur $[0,1]$ et est continue $1$.\\
    donc $lim_{x \to 1^-} f_n(x) = 1$. Donc $\| f_n \|_{\infty} \geq 1$.
\end{enumerate}
}
\ndlr{cf. Laurent pour la démo de $\|f_n\|_{\infty} \geq 1$}

\remark{On sait de plus que $\| f_n \|_{[0,1[,\;\infty} = 1$ car $\forall x \in [0,1[, |f_n(x)| = x^n < 1$. Donc $\| f_n \|_{\infty} \leq 1$. Donc $\| f_n \|_{\infty} = 1$.}
\example{(suite)
$f_n : [0,a] \to \mathbb{R}, x \mapsto x^n$ avec $a < 1$.\\
Alors : 
\begin{enumerate}
    \item $f_n$ converge simplement vers la fonction nulle sur $[0,a]$.
    \item $\| f_n \|_{[0,a]\;\infty} = f_n(a) = a^n$ car $f_n$ est continue $[0,a]$ et $f_n$ est croissante sur $[0,a]$.\\
    Donc $lim_{n \to +\infty} \| f_n \|_{[0,a],\;\infty}$ (car $= a^n$).\\
    Donc $f_n$ converge uniformément vers la fonction nulle sur $[0,a]$. 
\end{enumerate}
}

\remark{L'hypothèse "continue sur $[0,a]$" permet d'appliquer le théorème des bornes atteintes : $\| f_n \| = max_{x \in [0,a]} f_n(x)$.\\
et comme $f_n$ est croissante, le maximum est atteint en $a$.}

\remark{En général, pour calculer des bornes supérieures (extrema en général) d'une fonction, il faut faire le tableau de variations.}

\definition{
    Soit $(f_n)$ une suite de fonctions de $I$ dans $E$. On dit que $(f_n)$ est une \textbf{suite de Cauchy uniforme} sur $I$ s'il :
    \[\forall \varepsilon > 0, \exists n_0 \in \mathbb{N}, \forall p,q \geq n_0, \forall x \in I, |f_p(x) - f_q(x)| < \varepsilon.\]
}

\theorem{Théorème}{}{false}{
    Soit $(f_n)$ une suite de fonctions de $I$ dans $E$.\\
    Alors $(f_n)$ est uniformément de Cauchy si et seulement si elle converge uniformément sur $I$.
}

\noindent{\textbf{Démonstration :}\\
    Supposons $(f_n)$ est uniformément de Cauchy, donc elle est simplement de Cauchy. Donc $\forall x \in I$, la suite $(f_n(x))$ converge dans $E$ (car $E$ est complet).\\
    On note $f(x) = \lim_{n \to +\infty} f_n(x)$.\\
    Donc $f_n \to f$ (simplement).
    Montrons que $f_n \to f$ uniformément.\\
    A-t-on $\| f_n - f \|_{I,\;\infty} \longrightarrow 0$ ?\\
    Soit $\varepsilon > 0$.\\
    Soit $n_0 \in \mathbb{N}$ tel que $\forall p,q \geq n_0, \|f_p - f_q\|_{\infty} < \varepsilon$.\\
    Soit $n \geq n_0$.\\
    Soit $x \in I$.\\
    Soit $p_x \in \mathbb{N}$ tel que $|f_{p_x}(x) - f(x)| < \varepsilon$ (car $f_n(x) \to f(x)$).\\
    On a pour $p \geq max(p_x, n_0) \; |f_n(x) - f(x)| \leq |f_n(x) - f_p(x)| + |f_p(x) - f(x)| < \|f_n - f_p\|_{I,\;\infty} + \varepsilon < 2\varepsilon$ car $p \geq n_0$.\\
    En passant à la borne supérieur, $\| f_n - f\| \leq 2 \varepsilon$.\\
    Quitte à remplacer $\varepsilon$ par $\frac{\varepsilon}{2}$, on a bien $\| f_n - f\|_{I,\;\infty} < \varepsilon$. Donc $f_n \to f$ uniformément. $\Box$
}

\section{Propriétés de la limite uniforme}

\subsection{Bornitude et continuité de la limite uniforme}

\theorem{Proposition}{}{true}{
    Soit $(f_n)$ une suite de fonctions de $I \to E$. On suppose $f_n \to f$ uniformément.\\
    Alors, si $f_n$ est bornée à partir d'un certain rang, alors $f$ est bornée. De plus, $lim_{n \to +\infty} \| f_n \|_{I,\;\infty} = \| f \|_{I,\;\infty}$.
}

\ndlr{cf. Alex}

\theorem{Théorème d'interversion de limite}{}{true}{
    Soit $(f_n)$ une suite de fonctions de $I$ dans $E$. On suppose que $f_n \to f$ uniformément sur $I$.\\
    Soit $x_0 \in \overline{I}$. On suppose que $lim_{x \to x_0} f_k(x) = l_k$. Alors $(l_n)_{n \in \mathbb{N}}$ converge.\\
    Et $lim_{x \to x_0}f(x) = lim_{x \to x_0} lim_{n \to +\infty} f_n(x) = lim_{n \to +\infty} lim_{x \to x_0} f_n(x)$.
}

\ndlr{cf. Yaïr $\times$ Alex}

\theorem{Corollaire}{}{false}{
    Soit $(f_n)$ une suite de fonctions de $I$ dans $E$, et supposons que $f_n \to f$ uniformément sur $I$.\\
    Si $f_n$ continue sur $I$ à partir d'un certain rang, alors $f$ est continue sur $I$.
}

\subsection{Intégrabilité de la limite uniforme}

\theorem{Théorème}{}{false}{
    Soit $(f_n)$ une suite de fonctions de $[a,b]$ $(a < b$, finis) dans $\mathbb{R}$. On suppose que $f_n \to f$ uniformément sur $[a,b]$ et $f_n$ continue par morceaux.\\
    Alors on a $\int_a^b f_n(x) dx \longrightarrow_{n \to +\infty} \int_a^b f(x) dx$.
}

\ndlr{cf. Laurent}

\subsection{Dérivabilité de la limite uniforme}
\vocabulary{
    Soit $A \subset I$.\\
    On note pour $x \in I$, $1_A(x) = \begin{cases}
        1 & \text{si } x \in A\\
        0 & \text{sinon}
    \end{cases}$.
}

\theorem{Théorème}{}{false}{
    Soit $(f_n)$ une suite de fonctions de $I$ dans $E$ telle que :
    \begin{itemize}
        \item $\forall n \in \mathbb{N}$, $f_n \in \mathcal{C}^1(I,E)$.
        \item $\exists x_0 \in I$ tel que la suite $(f_n(x_0))$ converge dans $E$.
        \item La suite $(f_n')$ converge uniformément sur $I$ vers une fonction $g$.
    \end{itemize}
    Alors, $(f_n)$ converge uniformément sur tout segment $[a,b] \subset I$ vers une fonction $f : I \to E$, et de plus $f \in \mathcal{C}^1(I,E)$ et $f' = g$.
}

\noindent{\textbf{Démonstration :}\\
    Soit $x \in I$.\\
    $f_n(x) = f_n(x_0) + \int_{x_0}^x f_n'(t) dt$ (Théorème fondamental de l'analyse).\\
    On a $f_n(x_0)$ converge par hypothèse.\\
    Et $\int_{x_0}^x f_n'(t) dt \longrightarrow \int_{x_0}^x g(t) dt$ par le théorème d'intégrabilité de la limite uniforme.\\
    Donc $f_n(x)$ converge simplement vers $f$ où $f(x) = lim_{n \to +\infty} f_n(x_0) + \int_{x_0}^x g(t) dt$.\\
    On remarque que $f$ est de classe $\mathcal{C}^1(I,E)$ et $f' = g$.\\\\
    Il reste à montrer que $f_n \to f$ uniformément sur tout segment $[a,b] \subset I$.\\
    Soit $x \in I$.\\
    $|f_n(x) - f(x)| = |f_n'(t)dt + \int_{x_0}^x f_n'(t)dt - lim_{n \to +\infty} f_n(x_0) - \int_{x_0}^x g(t) dt|$\\
    $\leq |f_n(x_0) - lim_{n \to +\infty} f_n(x_0)| + |\int_{x_0}^x (f_n'(t) - g(t)) dt|$\\
    D'où $|f_n(x) - f(x)| \leq |f_n(x_0) - lim_{n \to +\infty} f_n(x_0)| + |x - x_0| \| f_n' - g \|_{I,\;\infty}$.\\
    $\leq |f_n(x_0) - lim_{n \to +\infty} f_n(x_0)| + (max(|a|,|b|)+|x_0|) \| f_n' - g \|_{I,\;\infty} \longrightarrow 0$ indépendamment de $x$.\\
    D'où la convergence uniforme sur $[a,b]$ de $(f_n)$ vers $f$.  $\Box$
}

\example{
    $f_n \to f$ uniformément et $f_n \in \mathcal{C}^1(I,E)$ ne suffit pas à garantir que $f \in \mathcal{C}^1(I,E)$.\\
    Par exemple, $f_n : [0,1] \to \mathbb{R}, x \mapsto \frac{x^{n+1}}{n+1}$.\\
    On a $\| f_n \|_{\infty} = \frac{1}{n+1} \longrightarrow 0$. Donc $f_n \to 0$ uniformément vers 0.\\
    Cependant, $f_n' : [0,1] \to \mathbb{R}, x \mapsto x^n$ qui converge simplement vers la fonction nulle, mais $f_n' \to 1_{1}$ (simplement) mais $f_n' \in \mathcal{C}^0([0,1], \mathbb{R})$ et $1_{\{1\}} \notin \mathcal{C}^0([0,1], \mathbb{R})$.\\
    Donc $(f_n')$ ne converge pas uniformément. (on ne satisfait pas les hypothèses du théorème).\\
    Donc la limite simple $1_{\{1\}} \neq  0 = f'$.
}

\theorem{Théorème}{}{false}{
    Soit $(f_n)$ une suite de fonctions de $I$ dans $E$ telle que :
    \begin{itemize}
        \item $\forall n \in \mathbb{N}$, $f_n$ est dérivable
        \item $\exists x_0 \in I$ tel que $(f_n(x_0))$ converge.
        \item $(f_n')$ converge uniformément vers $g$ sur $I$.
    \end{itemize}
    Alors, $(f_n)$ converge uniformément sur tout segment $[a,b] \subset I$ vers une fonction $f$ dérivable et $f' = g$.
}

\noindent{\textbf{Démonstration :}\\
    Montrons que $(f_n)$ est uniformément de Cauchy sur $[a,b]$.\\
    Sans erte de généralité, on suppose que $x_0 \in [a,b]$.\\
    Soit $x \in [a,b]$.\\
    $|f_p(x) - f_q(x)| \leq |(f_p - fq)(x) - (f_p - f_q)(x_0)| + |(f_p - f_q)(x_0)|$\\
    On utilise le théorème des accroissements finis pour $f_p - f_q$ :\\
    $\exists \zeta \in ]x,x_0[ ou ]x_0,x[$ tel que $(f_p - f_q)(x) - (f_p - f_q)(x_0) = (f_p' - f_q')(\zeta) \cdot (x - x_0)$.\\
    D'où $|f_p(x) - f_q(x)| \leq |x - x_0| \| f_p' - f_q' \|_{I,\;\infty} + |f_p(x_0) - f_q(x_0)|$.\\
    $\leq |b-a| \| f_p' - f_q' \|_{I,\;\infty} + |f_p(x_0) - f_q(x_0)|$.\\
    Comme $(f_n(x_0))$ converge, $(f_n(x_0))$ est de Cauchy.\\
    Et comme $(f_n')$ converge uniformément, $(f_n')$ est uniformément de Cauchy.\\
    On en déduit que $(f_n)$ est uniformément de Cauchy sur $[a,b]$.\\
    Donc $(f_n)$ converge uniformément sur $[a,b]$ vers une fonction $f$ sa limite uniforme.\\*
    Il reste à montrer que $f$ est dérivable et que $f' = g$.\\
    Soit $x_1 \in I$. (on choisit $a,b$ tels que $x_1 \in [a,b] \subset I$)\\
    Montrons que $f$ est dérivable en $x_1$ et que $f'(x_1) = g(x_1)$.\\
    On pose pour $x \in [a,b]$.\\
    \[
        \varrho(x) = \begin{cases}
            \frac{f(x) - f(x_1)}{x - x_1} & \text{si } x \neq x_1\\
            f_n'(x_1) & \text{si } x = x_1
        \end{cases}
    \]
    Comme $f_n$ et $f_n'$ convergent uniformément, on déduit que $(\varrho_n)$ converge simplement vers :
    \[
        \varrho(x) = \begin{cases}
            \frac{f(x) - f(x_1)}{x - x_1} & \text{si } x \neq x_1\\
            g(x_1) & \text{si } x = x_1
        \end{cases}
    \]
    Il suffit de montrer que $\varrho$ est continue en $x_1$.\\
    On observe que $f_n$ est continue car sur $[a,b] \setminus \{x_1\}$, $f_n$ est continue en $x_1$, $f_n$ est dérivable d'où :\\
    $\lim_{x \to x_1} \varrho{x} = \lim_{x \to x_1} \frac{f_n(x) - f_n(x_1)}{x - x_1} = f_n'(x_1)$ = $\varrho_n(x_1)$.\\
    Montrons que $(\varrho_n)$ converge uniformément vers $\varrho$ sur $[a,b]$.\\
    On en déduira que $\varrho$ est continue sur $[a,b]$ et en particulier en $x_1$.\\
    Soit $x \in [a,b] \setminus \{x_1\}$.\\
    Par le théorème des accroissements finis, $\exists \zeta_1 \in ]x,x_1[ ou ]x_1,x[$ tel que $(f_p - f_q)(x) - (f_p - f_q)(x_1) = (f_p' - f_q')(\zeta_1) \cdot (x - x_1)$.\\
    On obtient : $|\frac{(f_p - f_q)(x) - (f_p - f_q)(x_1)}{x - x_1}| = |(f_p' - f_q')(\zeta_1)|$.\\
    D'où $|\varrho_p(x) - \varrho_q(x)| \leq \| f_p' - f_q' \|_{I,\;\infty}$ si $x \neq x_1$.\\
    et $|\varrho_p(x_1) - \varrho_q(x_1)| = |f_p'(x_1) - f_q'(x_1)| \leq \| f_p' - f_q' \|_{I,\;\infty}$ si $x = x_1$.\\
    Donc en passant au sup : $\| \varrho_p - \varrho_q \|_{I,\;\infty} \leq \| f_p' - f_q' \|_{I,\;\infty}$.\\
    Comme $(f_n')$ converge uniformément, $(f_n')$ est uniformément de Cauchy.\\
    Donc $(\varrho_n)$ est uniformément de Cauchy.\\
    Donc $(\varrho_n)$ converge uniformément vers $\varrho$.\\
    Donc $\varrho$ est continue sur $[a,b]$ et en particulier en $x_1$.\\
    Donc $f$ est dérivable en $x_1$ et $f'(x_1) = g(x_1)$. $\Box$
}

\cexample{
    $f_n(x) = \frac{x^{n+1}}{n+1}$.\\
    $f_n \longrightarrow 0$ sur $[0,1]$ uniformément.\\
    $f_n' \longrightarrow 1_{\{1\}}$ simplement sur $[0,1]$ (et pas uniformément).\\
    et $0' = 0 \neq 1_{\{1\}}$.\\
}

\cexample{
    $f_n : [0,1] \to \mathbb{R}, x \mapsto \sqrt{x^2 + \frac{1}{n}}$\\
    \begin{itemize}
        \item $f_n \longrightarrow x$ (simplement).
        $|f_n(x) - x| = |\sqrt{x^2 + \frac{1}{n}} - \sqrt{x^2}| = \frac{\frac{1}{n}}{\sqrt{x^2 + \frac{1}{n}} + \sqrt{x^2}}$.\\
        On obtient : $|f_n(x) - x| \leq \frac{\frac{1}{n}}{\sqrt{\frac{1}{n}}} = \frac{1}{\sqrt{n}} \longrightarrow 0$ (indépendamment de $x$).\\
        Donc $f_n \to id$ uniformément sur $[0,1]$.
        \item $f_n' : [0,1] \to \mathbb{R}, x \mapsto \frac{x}{\sqrt{x^2 + \frac{1}{n}}}$\\
        $f_n' \longrightarrow \begin{cases}
            0 & \text{si } x = 0\\
            1 & \text{si } x > 0
        \end{cases} = 1_{]0,1]}$ simplement sur $[0,1]$.\\
        Comme $1_{]0,1]}$ n'est pas continue sur $[0,1]$, $f_n'$ ne converge pas uniformément vers $1_{]0,1]}$.
    \end{itemize}
}

\remark{Ce théorème est plus général que le précédent. En effet, on ne demande pas que $f_n$ soit de classe $\mathcal{C}^1(I,E)$, mais juste dérivable.}

\ndlr{cf. Laurent pour la preuve}
\ndlr{Le prof. n'est pas sûr des hypothèses du théorème.}

\section{Séries de fonctions}

\subsection{Convergence d'une série}

\subsubsection{Définitions générales}

\definition{
    On appelle \textbf{série de fonctions} une suite $(S_n)$ de fonctions de $I$ dans $E$ de la forme : \[S_n = \sum_{k=0}^n f_k\] où $f_k$ est son \textbf{terme général}, une fonction de $I$ dans $E$. On appelle \textbf{somme partielle} de la série la fonction $S_n$.
}


\definition{
    On dit qu'une série est \textbf{convergente} au point $x \in I$ si la suite des sommes partielles $(S_n(x))$ converge dans $E$.\\\\
    En particulier, on dit que la série de fonctions $\sum f_n$ converge \textbf{simplement} sur $I$ s'il existe une fonction $S : I \to E$ telle que, pour tout $x \in I$, la série numérique $\sum f_n(x)$ converge dans $E$ et $S(x) = \sum_{n=0}^{+\infty} f_n(x)$.
}

\definition{
    On dit qu'une série de fonctions $\sum_{n=0}^{+\infty} f_n$ converge \textbf{uniformément} si $(S_n)$ converge uniformément sur $I$.
}

\remark{
    De manière équivalente, on dit qu'une série $\sum_{n\geq0} f_n$ converge uniformément si $\sum_{k=0}^{+\infty} f_k$ existe et $\| \sum_{k=0}^n f_k - \sum_{k=0}^{+\infty} f_k \|_{I,\;\infty} \longrightarrow 0$.\\\\
    Autrement dit, $\sum_{n=0}^{+\infty} f_n$ converge uniformément $\Leftrightarrow$ $\sum_{n=0}^{+\infty} f_n$ existe et $(\sum_{k=n+1}^{+\infty} f_k)$ converge uniformément vers la fonction nulle.
}
\subsubsection{Convergence normale}
\definition{On dit que $\sum_{n=0}^{+\infty} f_n$ converge \textbf{normalement} sur $I$ si la série numérique $\sum_{n\geq0} \| f_n \|_{I,\;\infty}$ converge.}

\theorem{Théorème}{}{false}{
    Si $\sum_{n=0}^{+\infty} f_n$ converge normalement sur $I$, alors :
    \begin{itemize}
        \item $\sum_{n=0}^{+\infty} f_n$ converge uniformément sur $I$.
        \item $\forall x \in I, \; \sum_{n=0}^{+\infty} f_n(x)$ converge absolument dans $E$.
        \item La limite $\sum_{n=0}^{+\infty} f_n$ est une fonction bornée sur $I$.
    \end{itemize}
}

\ndlr{cf. Laurent pour la preuve.}

\example{
    On considère la série $\sum_{n \geq 0} x^n$.
    \begin{itemize}
        \item Si $|x| < 1$, $\sum_{n \geq 0} x^n$ converge absolument (série géométrique) donc la limite existe sur $]-1,1[$.\\
        Si $|x| \geq 1$, $\sum_{n \geq 0} x^n$ diverge grossièrement.
        \item Soit $0 \leq a < 1$, alors $\sum_{n \geq 0} x^n$ converge normalement sur $[-a, a]$.\\
        En effet, $\| x^n \|_{[-a,a],\;\infty} = a^n$ et $\sum_{n \geq 0} a^n$ converge (série géométrique). Donc $\sum_{n \geq 0} x^n$ converge uniformément sur $[-a,a]$.
        \item Exercice : montrer que $\sum_{n \geq 0} x^n$ ne converge pas uniformément sur $[0,1]$.
    \end{itemize}
}

\training{Montrer que la série $\sum_{n \geq 0} (-1)^n x^n$ converge uniformément sur $[0,1]$, mais pas normalement.}
\noindent\carreaux{10}

\subsection{Propriétés de la limite uniforme d'une série de fonctions}

\theorem{Théorème}{}{false}{
    On suppose que $\sum_{n \geq 0} f_n$ converge uniformément sur $I$.\\
    \begin{enumerate}
        \item Si $\forall n, \, f_n$ est bornée, alors $\sum_{n \geq 0} f_n$ est bornée.
        \item $x_0 \in \overline{I}$. On suppose que $\forall n, \lim_{x \to x_0} f_n(x) = l_n$. Alors la série numérique $\sum_{n\geq0} l_n$ converge et \[ \lim_{x \to x_0} \sum_{n=0}^{+\infty} f_n(x) = \sum_{n=0}^{+\infty} l_n.\]
        \item Si $\forall n, f_n$ est continue en $x_0 \in I$, alors $\sum_{n \geq 0} f_n$ est continue en $x_0$.
        \item Si $f_n$ est continue par morceau, alors $\int_a^b \sum_{n=0}^{+\infty} f_n(x) dx = \sum_{n=0}^{+\infty} \int_a^b f_n(x) dx$.
    \end{enumerate}
}

\theorem{Théorème}{}{false}{
    Soit $(f_n)$ une suite de fonctions de $I$ (intervalle) dans $E$ telle que :
    \begin{itemize}
        \item $\forall n \in \mathbb{N}, \, f_n$ est dérivable sur $I$.
        \item Il existe $x_0 \in I$ tel que $\sum_{n \geq 0} f_n(x_0)$ converge.
        \item La série $\sum_{n \geq 0} f_n'$ converge uniformément.
    \end{itemize}
    Alors, la série $\sum_{n \geq 0} f_n$ converge uniformément sur tout segment $[a,b] \subset I$ vers une fonction $f$ dérivable et $(\sum_{n=0}^{+\infty} f_n)' = \sum_{n=0}^{+\infty} f_n'$.
}

\subsection{Lemme d'Abel}

\theorem{Proposition}{}{false}{
    Soit $I$ une partie de $\mathbb{R}$.\\
    Soient $(u_n)$ une suite numérique réelle et $(v_n)$ une suite de fonctions de $I$ dans $E$.\\
    On pose : $U_n = \sum_{k=0}^n u_k$ et $V_n = \sum_{k=0}^n v_k$.\\
    Alors :\\
    $\sum_{n=q+1}^p u_n \cdot v_n = - \sum_{n=q+1}^{p-1} (u_{n+1} - u_n) \cdot V_n + u_p \cdot V_p - u_{q+1} \cdot V_q$. 
}

\training{Montrer la proposition précédente.}
\noindent\carreaux{10}

\remark{C'est une formule d'intégration par partie discrète.}

\theorem{Théorème}{}{false}{
    Soit $I \subset \mathbb{R}$.\\
    $(u_n)$ une suite numérique réelle décroissante tendant vers $0$.\\
    $(v_n)$ une suite de fonctions $v_n : I \to E$.\\
    On suppose que $\| \sum_{k=0}^n v_n \|_{\infty} < +\infty$.\\
    Alors la série de fonctions $\sum u_n v_n$ converge uniformément.
}

\remark{On peut remplacer la suite numérique $(u_n)$ par une suite de fonctions $(f_n)$ de $I$ dans $\mathbb{R}$ telle que $\forall x \in I, (f_n(x))$ est décroissante et tend vers $0$.}
\ndlr{cf. Laurent pour la preuve.}

\example{
    $\sum u_n v_n$ où $(u_n)$ décroissant vers 0 et $v_n(t) = sin(nt)$ pour $t \in [0, 2\pi]$.\\
    Si $\alpha \in ]0, \pi[$, alors $\sum u_n v_n$ converge uniformément sur $[\alpha, 2\pi - \alpha]$.\\
    Il suffit de montrer que $V_n(t) = \sum_{k=0}^n sin(kt)$ est uniformément bornée sur $[\alpha, 2\pi - \alpha]$ lorsque $n \to +\infty$.\\
    $\sum_{k=0}^n sin(kt) = Im(\sum_{k=0}^n e^{ikt}) = Im(\frac{1 - e^{i(n+1)t}}{1 - e^{it}}) = \ldots =  \frac{sin(\frac{nt}{2} \cdot sin(\frac{(n+1)t}{2})}{sin(\frac{t}{2})}$\\
    Si $t \in [\alpha, 2\pi - \alpha]$, on obtient $\frac{t}{2} \in [\frac{\alpha}{2}, \pi - \frac{\alpha}{2}]$.\\
    Donc $\sum_{k=0}^n sin(kt) \leq \frac{1}{sin(\frac{\alpha}{2})}$ pour tout $n \in \mathbb{N}$ et $t \in [\alpha, 2\pi - \alpha]$.\\
    Et puis on applique le théorème d'Abel pour conclure que $\sum u_n v_n$ converge uniformément sur $[\alpha, 2\pi - \alpha]$.
}
\ndlr{cf. Laurent pour le détail, c'était horrible à écrire en LaTeX.}

\newpage
\tableofcontents


\end{document}  