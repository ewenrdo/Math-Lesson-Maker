\documentclass{article}

\usepackage[a4paper, left=1.5cm, right=1.5cm, top=2cm, bottom=2cm]{geometry}

\usepackage{../../../../components/components}

\usepackage{fancyhdr}


% Configuration des en-têtes et pieds de page
\pagestyle{fancy}
\fancyhf{} % reset tout

\fancyhead[L]{DL2 Math-Info MM4}
\fancyhead[C]{Analyse}
\fancyhead[R]{2025-2026}

\fancyfoot[L]{Ewen Rodrigues de Oliveira}
\fancyfoot[R]{\thepage}

\begin{document}

\docTitle{Chapitre 3 : Suites et séries de fonctions}

\section{Suites de fonctions}

\definition{Soit $I$ un intervalle ouvert, et soit $E = \mathbb{R}$ ou $\mathbb{C}$\\
Une \textbf{suite de fonctions} est une application de $\mathbb{N}$ dans $\mathcal{F}(I, E)$, l'ensemble des fonctions de $I$ dans $E$. On la note $(f_n)_{n \in \mathbb{N}}$, où $f_n$ est la fonction associée à l'entier naturel $n$.
}

\vocabulary{$\mathcal{F}(I, E)$ : ensemble des fonctions de $I$ dans $E$. On le note aussi $E^I$.}
\remark{Cette notion est généralisable si $I$ devient un disque de $\mathbb{C}$ et $E$ un espace vectoriel normé.}

\example{
    $f_n : \left[ 0, +\infty \right[ \to \mathbb{R}$ définie par $f_n(x) = x^n$, pour tout $n \in \mathbb{N}$.
    On a aussi : $f_n : \mathbb{R} \to \mathbb{R}$ définie par $f_n(x) = \cos(nx)$, pour tout $n \in \mathbb{N}$.\\
    On peut tracer les graphes de ces suites de fonctions : $f_0, f_1, f_2, \ldots$
}

\newpage
\tableofcontents


\end{document}