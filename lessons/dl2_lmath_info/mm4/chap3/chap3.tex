\documentclass{article}

\usepackage[a4paper, left=1.5cm, right=1.5cm, top=2cm, bottom=2cm]{geometry}

\usepackage{../../../../components/components}

\usepackage{fancyhdr}


% Configuration des en-têtes et pieds de page
\pagestyle{fancy}
\fancyhf{} % reset tout

\fancyhead[L]{DL2 Math-Info MM4}
\fancyhead[C]{Analyse}
\fancyhead[R]{2025-2026}

\fancyfoot[L]{Ewen Rodrigues de Oliveira}
\fancyfoot[R]{\thepage}

\begin{document}

\docTitle{Chapitre 3 : Suites et séries de fonctions}

\section{Suites de fonctions}

\definition{Soit $I$ un intervalle ouvert, et soit $E = \mathbb{R}$ ou $\mathbb{C}$\\
Une \textbf{suite de fonctions} est une application de $\mathbb{N}$ dans $\mathcal{F}(I, E)$, l'ensemble des fonctions de $I$ dans $E$. On la note $(f_n)_{n \in \mathbb{N}}$, où $f_n$ est la fonction associée à l'entier naturel $n$.
}

\vocabulary{$\mathcal{F}(I, E)$ : ensemble des fonctions de $I$ dans $E$. On le note aussi $E^I$.}
\remark{Cette notion est généralisable si $I$ devient un disque de $\mathbb{C}$ et $E$ un espace vectoriel normé.}

\example{
    $f_n : \left[ 0, +\infty \right[ \to \mathbb{R}$ définie par $f_n(x) = x^n$, pour tout $n \in \mathbb{N}$.
    On a aussi : $f_n : \mathbb{R} \to \mathbb{R}$ définie par $f_n(x) = \cos(nx)$, pour tout $n \in \mathbb{N}$.\\
    On peut tracer les graphes de ces suites de fonctions : $f_0, f_1, f_2, \ldots$
}

\subsection{Convergence simple}

\definition{
    Soit $(f_n)$ une série de fonctions de $I$ dans $E$.\\
    On dit que $(f_n)$ est \textbf{simplement convergente} sur $I$ s'il existe une fonction $f : I \to E$ telle que, pour tout $x \in I$, la suite $(f_n(x))$ converge dans $E$.\\
    On appelle dans ce cas la \textbf{limite simple} de la suite $(f_n)$ la fonction $f$ définie par : \function{f}{I \to E}{x \mapsto \lim_{n \to +\infty} f_n(x)}.
}

\example{
    On pose \function{f_n}{\left[0,1\right] \to \mathbb{R}}{x \mapsto x^n.}{}\\
    Soit $x \in \left[0,1\right]$.\\
    $lim_{n \to +\infty} x^n = \begin{cases}
        0 & \text{si } x < 1\\
        1 & \text{si } x = 1
    \end{cases}$\\
    Donc la suite $(f_n)$ converge simplement sur $\left[0,1\right]$ et sa limite simple est la fonction $f$ définie par : \function{f}{\left[0,1\right] \to \mathbb{R}}{x \mapsto \begin{cases}
        0 & \text{si } x < 1\\
        1 & \text{si } x = 1
    \end{cases}}.
}
\attention{On atteint un premier problème : la limite de $f_n$ est continue mais la limite simple $f$ ne l'est pas.}

\example{
    On pose $f_n = \begin{cases}
        n^2 x & \text{si } x \in \left[0, \frac{1}{n}\right]\\
        n^2 \left( \frac{2}{n} - x \right) & \text{si } x \in \left[\frac{1}{n}, \frac{2}{n}\right]\\
        0 & \text{si } x > \frac{2}{n}
    \end{cases}$ définie de $\left[0,1\right]$ dans $\mathbb{R}$.\\
    On a que $f_n$ converge simplement vers la fonction nulle sur $\left[0,1\right]$.\\
    En effet, $f_n(0) = 0 \longrightarrow 0$ et si $x \in \left]0,1\right]$, alors il existe $N \in \mathbb{N}$ tel que $n \geq N \implies x > \frac{2}{n} \implies f_n(x) = 0$. Donc $f_n(x) \longrightarrow 0$.\\
    D'où la limite simple est la fonction nulle.\\
    Cependant, $max_{[0,1]} f_n = f_n\left(\frac{1}{n}\right) = n$, qui diverge vers $+\infty$. Donc la "convergence n'est pas uniforme".\\
    Autre chose étrange, $\int_0^1 f_n(x) dx = 1$, alors qu'on s'attendrait à ce que l'intégrale de la limite soit nulle.
}

\newpage
\tableofcontents


\end{document}