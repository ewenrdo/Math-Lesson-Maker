\documentclass{article}

\usepackage[a4paper, left=1.5cm, right=1.5cm, top=2cm, bottom=2cm]{geometry}

\usepackage{../../../../components/components}

\usepackage{fancyhdr}


% Configuration des en-têtes et pieds de page
\pagestyle{fancy}
\fancyhf{} % reset tout

\fancyhead[L]{DL2 Math-Info MM4}
\fancyhead[C]{Analyse}
\fancyhead[R]{2025-2026}

\fancyfoot[L]{Ewen Rodrigues de Oliveira}
\fancyfoot[R]{\thepage}

\begin{document}

\docTitle{Chapitre 3 : Suites et séries de fonctions}

\section{Suites de fonctions}

\definition{Soit $I$ un intervalle ouvert, et soit $E = \mathbb{R}$ ou $\mathbb{C}$\\
Une \textbf{suite de fonctions} est une application de $\mathbb{N}$ dans $\mathcal{F}(I, E) = E^I$, l'ensemble des fonctions de $I$ dans $E$. On la note $(f_n)_{n \in \mathbb{N}}$, où $f_n$ est la fonction associée à l'entier naturel $n$.
}

\remark{Cette notion est généralisable si $I$ devient un disque de $\mathbb{C}$ et $E$ un espace vectoriel normé.}

\example{
    $f_n : \left[ 0, +\infty \right[ \to \mathbb{R}$ définie par $f_n(x) = x^n$, pour tout $n \in \mathbb{N}$.
    On a aussi : $f_n : \mathbb{R} \to \mathbb{R}$ définie par $f_n(x) = \cos(nx)$, pour tout $n \in \mathbb{N}$. On peut tracer les graphes de ces suites de fonctions : $f_0, f_1, f_2, \ldots$
}

\subsection{Convergence simple}

\definition{
    Soit $(f_n)$ une série de fonctions de $I$ dans $E$.\\
    On dit que $(f_n)$ est \textbf{simplement convergente} sur $I$ s'il existe une fonction $f : I \to E$ telle que, pour tout $x \in I$, la suite $(f_n(x))$ converge dans $E$.\\
    On appelle dans ce cas la \textbf{limite simple} de la suite $(f_n)$ la fonction $f$ définie par : \function{f}{I \to E}{x \mapsto \lim_{n \to +\infty} f_n(x)}.
}

\example{
    On pose \function{f_n}{\left[0,1\right] \to \mathbb{R}}{x \mapsto x^n.}{}\\
    Soit $x \in \left[0,1\right]$.\\
    $lim_{n \to +\infty} x^n = \begin{cases}
        0 & \text{si } x < 1\\
        1 & \text{si } x = 1
    \end{cases}$\\
    Donc la suite $(f_n)$ converge simplement sur $\left[0,1\right]$ et sa limite simple est la fonction $f$ définie par : \function{f}{\left[0,1\right] \to \mathbb{R}}{x \mapsto \begin{cases}
        0 & \text{si } x < 1\\
        1 & \text{si } x = 1
    \end{cases}}
}
\attention{On atteint un premier problème : la limite de $f_n$ est continue mais la limite simple $f$ ne l'est pas.}

\example{
    On pose $f_n = \begin{cases}
        n^2 x & \text{si } x \in \left[0, \frac{1}{n}\right]\\
        n^2 \left( \frac{2}{n} - x \right) & \text{si } x \in \left[\frac{1}{n}, \frac{2}{n}\right]\\
        0 & \text{si } x > \frac{2}{n}
    \end{cases}$ définie de $\left[0,1\right]$ dans $\mathbb{R}$.\\
    On a que $f_n$ converge simplement vers la fonction nulle sur $\left[0,1\right]$. En effet, $f_n(0) = 0 \longrightarrow 0$ et si $x \in \left]0,1\right]$, alors il existe $N \in \mathbb{N}$ tel que $n \geq N \implies x > \frac{2}{n} \implies f_n(x) = 0$ (axiome d'Archimède). Donc $f_n(x) \longrightarrow 0$, d'où le fait que la limite simple est la fonction nulle.\\
    Cependant, $max_{[0,1]} f_n = f_n\left(\frac{1}{n}\right) = n$, qui diverge vers $+\infty$. Donc la "convergence n'est pas uniforme".\\
    Autre chose étrange, $\int_0^1 f_n(x) dx = 1$, alors qu'on s'attendrait à ce que l'intégrale de la limite soit nulle.
}
\begin{center}
    \includegraphics[width=0.25\textwidth]{./images/figure1.png}
    \captionof{figure}{Graphique de $f_n$}
\end{center}
\newpage
\subsection{Convergence uniforme}

\definition{
    On dit que $(f_n)$ converge \textbf{uniformément} sur $I$ vers $f$ s'il existe une fonction $f : I \to E$ telle que : \\
    $\forall \varepsilon > 0, \exists n_0 \in \mathbb{N}, \forall n \geq n_0, \forall x \in I, |f_n(x) - f(x)| < \varepsilon$.
}

\remark{
    $(f_n)$ converge simplement vers $f$ si $\forall x \in I, \underbrace{\forall \varepsilon > 0, \exists n_0 \in \mathbb{N}, \forall n \geq n_0, |f_n(x) - f(x)| < \varepsilon}_{(f_n(x))_{n\in\mathbb{N}} \text{ converge vers } f(x)}$. Ce n'est donc pas la même chose que la convergence uniforme, où $n_0$ ne dépend pas de $x$.
}

\example{
    Dans l'example précédent (avec les $\frac{1}{n}$), $f_n(\frac{1}{n}) \longrightarrow +\infty$, donc la convergence n'est pas uniforme sur $\left[0,1\right]$ vers $0$. En effet, $\forall n \geq 1, |f_n(\frac{1}{n}) - 0| = n \geq 1$.\\
    
}

\theorem{Proposition}{}{false}{
    \begin{enumerate}
    \item Si $(f_n)$ converge uniformément vers $f$ sur $I$, alors elle converge simplement vers $f$ sur $I$.
    \item La limite simple de $(f_n)$ est unique. Si $f$ est limite simple de $(f_n)$  par définition : $f(x) = \lim_{n \to +\infty} f_n(x)$ (qui est unique).
    \item La limite uniforme de $(f_n)$ est unique (si elle existe).
\end{enumerate}
}

\remark{Si on a la limite simple de $(f_n)$, alors c'est la seule candidate possible pour être la limite uniforme. Donc pour calculer une limite uniforme, on commence par chercher la limite simple, puis on prouve la convergence uniforme vers cette limite simple.}

\example{
    Soit $a < 1$.\\
    $f_n = x^n$ sur $\left[0,a\right] \to \mathbb{R}$.\\
    $(f_n)$ converge simplement vers $0$.
    \begin{enumerate}
        \item Soit $x \in \left[0,a\right]$, $x < 1 \Rightarrow x^n \longrightarrow 0$.\\
        Donc $0$ est la limite simple de $(f_n)$.
        \item Soit $\varepsilon > 0$.\\
        Soit $n_0 \in \mathbb{N}$ tel que soit $n \geq n_0, \forall x \in \left[0,a\right], |x^n - 0| < a^n \leq a^{n_0} < \varepsilon$ (car $a < 1$).
    \end{enumerate}
}

\cexample{Par contre, montrons que $f_n = x^n$ sur $\left[0,1\right[$ ne converge pas uniformément vers $0$ (remarquons qu'on a pas pris $1$).\\
\begin{enumerate}
    \item La limite simple est toujours $0$ (même justification que tout à l'heure).
    \item $(1-\frac{1}{n})^n \longrightarrow \frac{1}{e} \neq 0$. Donc il n'y a pas de convergence uniforme.
\end{enumerate}
}

\remark{Pour montrer que $f_n$ converge uniformément vers $f$, il suffit de trouver une majoration $M_n$ indépendante de $x$ telle que $|f_n(x) - f(x)| \leq M_n$ et $M_n \longrightarrow 0$.}

\definition{
    Soit $f$ une fonction de $I$ dans $E$.\\
    La \textbf{norme uniforme (ou norme infinie)} de $f$ est définie par : $\|f\|_{I,\;\infty} = sup_{x \in I} |f(x)|$.
}

\theorem{Proposition}{}{false}{
    Soit $(f_n)$ une suite de fonctions de $I$ dans $E$.\\
    Alors $(f_n)$ converge uniformément vers $f$ sur $I$ si et seulement si \[\begin{cases}
        f_n - f \text{ est bornée sur } I \text{ à partir d'un certain rang}\\
        \|f_n - f\|_{I,\;\infty} \longrightarrow 0
    \end{cases}\]
}

\remark{La première condition est là pour que la norme infinie soit bien définie et réelle.}

\reminder{On dit que $f \colon I \to E$ est bornée sur $I$ s'il existe $M \geq 0$ tel que $\forall x \in I, |f(x)| \leq M$. Autrement dit, $\{f(x) \mid x \in I\} = f(I)$ est une partie bornée de $E$, $\| f \|_{\infty} < +\infty$.}

\noindent{\textbf{Démonstration :}\\
    Si $\|f_n - f\|_{I,\;\infty} \longrightarrow 0$, alors $\forall x \in I, |f_n(x) - f(x)| \leq \|f_n - f\|_{I,\;\infty} \longrightarrow 0$. Donc $(f_n)$ converge uniformément vers $f$ par la remarque.\\\\
    Réciproquement, si $\| f_n - f\|_{I,\;\infty} \nrightarrow 0$, alors il existe $\varepsilon > 0$ et une sous-suite $\varphi$ strictement croissante telle que $\| f_{\varphi(n)} - f\|_{I,\;\infty} \geq \varepsilon$.\\
    Donc il existe $x_n \in I$ tel que $|f_{\varphi(n)}(x_n) - f(x_n)| \geq \varepsilon$ par définition du sup. $\Box$
}

\example{\function{f_n}{[0,1[ \to \mathbb{R}}{x \mapsto x^n}\\
\begin{enumerate}
    \item $f_n$ converge simplement vers la fonction nulle. (CVS)
    \item $\|f_n - 0\|_{[0,1[,\;\infty} = sup_{x \in [0,1[} |x^n - 0| = 1$. En effet, le prolongement par continuité de $f_n$ est $x^n$ sur $[0,1]$ et est continue $1$.\\
    donc $lim_{x \to 1^-} f_n(x) = 1$. Donc $\| f_n \|_{\infty} \geq 1$.
\end{enumerate}
}
\ndlr{cf. Laurent pour la démo de $\|f_n\|_{\infty} \geq 1$}

\remark{On sait de plus que $\| f_n \|_{[0,1[,\;\infty} = 1$ car $\forall x \in [0,1[, |f_n(x)| = x^n < 1$. Donc $\| f_n \|_{\infty} \leq 1$. Donc $\| f_n \|_{\infty} = 1$.}
\example{(suite)
$f_n : [0,a] \to \mathbb{R}, x \mapsto x^n$ avec $a < 1$.\\
Alors : 
\begin{enumerate}
    \item $f_n$ converge simplement vers la fonction nulle sur $[0,a]$.
    \item $\| f_n \|_{[0,a]\;\infty} = f_n(a) = a^n$ car $f_n$ est continue $[0,a]$ et $f_n$ est croissante sur $[0,a]$.\\
    Donc $lim_{n \to +\infty} \| f_n \|_{[0,a],\;\infty}$ (car $= a^n$).\\
    Donc $f_n$ converge uniformément vers la fonction nulle sur $[0,a]$. 
\end{enumerate}
}

\remark{L'hypothèse "continue sur $[0,a]$" permet d'appliquer le théorème des bornes atteintes : $\| f_n \| = max_{x \in [0,a]} f_n(x)$.\\
et comme $f_n$ est croissante, le maximum est atteint en $a$.}

\remark{En général, pour calculer des bornes supérieures (extrema en général) d'une fonction, il faut faire le tableau de variations.}

\definition{
    Soit $(f_n)$ une suite de fonctions de $I$ dans $E$. On dit que $(f_n)$ est une \textbf{suite de Cauchy uniforme} sur $I$ s'il :
    \[\forall \varepsilon > 0, \exists n_0 \in \mathbb{N}, \forall p,q \geq n_0, \forall x \in I, |f_p(x) - f_q(x)| < \varepsilon.\]
}

\theorem{Théorème}{}{false}{
    Soit $(f_n)$ une suite de fonctions de $I$ dans $E$.\\
    Alors $(f_n)$ est uniformément de Cauchy si et seulement si elle converge uniformément sur $I$.
}

\noindent{\textbf{Démonstration :}\\
    Supposons $(f_n)$ est uniformément de Cauchy, donc elle est simplement de Cauchy. Donc $\forall x \in I$, la suite $(f_n(x))$ converge dans $E$ (car $E$ est complet).\\
    On note $f(x) = \lim_{n \to +\infty} f_n(x)$.\\
    Donc $f_n \to f$ (simplement).
    Montrons que $f_n \to f$ uniformément.\\
    A-t-on $\| f_n - f \|_{I,\;\infty} \longrightarrow 0$ ?\\
    Soit $\varepsilon > 0$.\\
    Soit $n_0 \in \mathbb{N}$ tel que $\forall p,q \geq n_0, \|f_p - f_q\|_{\infty} < \varepsilon$.\\
    Soit $n \geq n_0$.\\
    Soit $x \in I$.\\
    Soit $p_x \in \mathbb{N}$ tel que $|f_{p_x}(x) - f(x)| < \varepsilon$ (car $f_n(x) \to f(x)$).\\
    On a pour $p \geq max(p_x, n_0) \; |f_n(x) - f(x)| \leq |f_n(x) - f_p(x)| + |f_p(x) - f(x)| < \|f_n - f_p\|_{I,\;\infty} + \varepsilon < 2\varepsilon$ car $p \geq n_0$.\\
    En passant à la borne supérieur, $\| f_n - f\| \leq 2 \varepsilon$.\\
    Quitte à remplacer $\varepsilon$ par $\frac{\varepsilon}{2}$, on a bien $\| f_n - f\|_{I,\;\infty} < \varepsilon$. Donc $f_n \to f$ uniformément. $\Box$
}

\section{Propriétés de la limite uniforme}

\theorem{Proposition}{}{true}{
    Soit $(f_n)$ une suite de fonctions de $I \to E$. On suppose $f_n \to f$ uniformément.\\
    Alors, si $f_n$ est bornée à partir d'un certain rang, alors $f$ est bornée. De plus, $lim_{n \to +\infty} \| f_n \|_{I,\;\infty} = \| f \|_{I,\;\infty}$.
}

\ndlr{cf. Alex}

\theorem{Théorème d'interversion de limite}{}{true}{
    Soit $(f_n)$ une suite de fonctions de $I$ dans $E$. On suppose que $f_n \to f$ uniformément sur $I$.\\
    Soit $x_0 \in \overline{I}$. On suppose que $lim_{x \to x_0} f_k(x) = l_k$. Alors $(l_n)_{n \in \mathbb{N}}$ converge.\\
    Et $lim_{x \to x_0}f(x) = lim_{x \to x_0} lim_{n \to +\infty} f_n(x) = lim_{n \to +\infty} lim_{x \to x_0} f_n(x)$.
}

\ndlr{cf. Yaïr $\times$ Alex}

\theorem{Corollaire}{}{false}{
    Soit $(f_n)$ une suite de fonctions de $I$ dans $E$, et supposons que $f_n \to f$ uniformément sur $I$.\\
    Si $f_n$ continue sur $I$ à partir d'un certain rang, alors $f$ est continue sur $I$.
}

\subsection{Intégrabilité de la limite uniforme}

\theorem{Théorème}{}{false}{
    Soit $(f_n)$ une suite de fonctions de $[a,b]$ $(a < b$, finis) dans $\mathbb{R}$. On suppose que $f_n \to f$ uniformément sur $[a,b]$ et $f_n$ continue par morceaux.\\
    Alors on a $\int_a^b f_n(x) dx \longrightarrow_{n \to +\infty} \int_a^b f(x) dx$.
}


\newpage
\tableofcontents


\end{document}