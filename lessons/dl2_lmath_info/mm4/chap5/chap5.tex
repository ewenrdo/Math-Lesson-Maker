\documentclass{article}

\usepackage[a4paper, left=1.5cm, right=1.5cm, top=2cm, bottom=2cm]{geometry}

\usepackage{../../../../components/components}
\usepackage{hyperref}
\usepackage{fancyhdr}


% Configuration des en-têtes et pieds de page
\pagestyle{fancy}
\fancyhf{} % reset tout

\fancyhead[L]{DL2 Math-Info ASF4}
\fancyhead[C]{Analyse}
\fancyhead[R]{2025-2026}

\fancyfoot[L]{Ewen Rodrigues de Oliveira}
\fancyfoot[R]{\thepage}

\begin{document}

\docTitle{Chapitre 5 : Séries entières}

\section{Introduction}

\definition{
    Une \textbf{série entière} est une série de fonction dont le $n^{\text{ème}}$ terme général est un monôme de degré $n$.\\
    C'est-à-dire que c'est une série de la forme $\sum_{n} a_n z^n$ où $(a_n)_{n \in \mathbb{N}}$ est une suite de nombres complexes ($\mathbb{C}^\mathbb{N}$ ou $\mathbb{R}^\mathbb{N}$) et $z$ est une variable complexe.
}


\remark{
    Comme séries de fonctions, les propositions vues sur le chapitre sur les séries de fonctions s'appliquent aux séries entières.
}

\example{
    \[
        \sum_{n\geq0} z^n \hspace{1.25cm} \sum_{n\geq0} \frac{z^n}{n!} \hspace{1.25cm} \sum_{n\geq0} \frac{z^n}{n} \hspace{1.25cm} \cdots
    \]
}

\subsection{Convergence d'une série entière}

\example{
    La série $\sum_{n\geq0} z^n$ converge si et seulement si $|z| < 1$ sur $\{z \in \mathbb{C} : |z| < 1\} = D(0, 1)$
}

\example{
    Soit $z \in \mathbb{C}$. On pose $u_n = \frac{z^n}{n!}$. Alors on a $\frac{u_{n+1}}{u_n} = \frac{z^{n+1}}{(n+1)!} \cdot \frac{n!}{z^n} = \frac{z}{n+1} \to 0$ lorsque $n \to +\infty$.\\
    Par le critère de d'Alembert, la série $\sum_{n\geq0} \frac{z^n}{n!}$ converge pour tout $z \in \mathbb{C}$ et on dira que le rayon de convergence de la série $\sum_{n\geq0} \frac{z^n}{n!}$ est $+\infty$.
}

\vocabulary{On dit que $1$ est le \textbf{rayon de convergence} de la série $\sum_{n\geq0} z^n$ (et $D(0, 1)$ est le \textbf{disque de convergence} de la série $\sum_{n\geq0} z^n$).}

\theorem{Lemme}{}{false}{
    Soit $\sum_{n\geq0} a_n z^n$ une série entière et $z_0 \in \mathbb{C}$.
    \begin{enumerate}
        \item Si $\sum_{n\geq0} a_n z_0^n$ converge absolument, alors la série $\sum_{n\geq0} a_n z^n$ converge normalement sur $\overline{D}(0, |z_0|) = \{ z \in \mathbb{C} : |z| \leq |z_0| \}$.
        \item Si $(a_n z_0^n)_{n\in\mathbb{N}}$ est bornée, alors la série $\sum_{n\geq0} a_n z^n$ converge normalement sur $\overline{D}(0, r)$ pour tout $r < |z_0|$
    \end{enumerate}
}

\noindent{\textbf{Démonstration :}\\
    Soit $z \in \overline{D}(0, |z_0|)$ et soit $n \in \mathbb{N}$.\\
    On a $|a_n z^n| = |a_n||z|^n \leq |a_n||z_0|^n$.\\
    Et de plus $\sum_{n\geq0} |a_n||z_0|^n$ converge par hypothèse et est indépendant de $z$.\\

    \noindent Donc la série $\sum_{n\geq0} a_n z^n$ converge normalement sur $\overline{D}(0, |z_0|)$.
}

\newpage
\tableofcontents


\end{document}  