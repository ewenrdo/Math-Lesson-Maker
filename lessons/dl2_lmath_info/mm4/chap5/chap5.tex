\documentclass{article}

\usepackage[a4paper, left=1.5cm, right=1.5cm, top=2cm, bottom=2cm]{geometry}

\usepackage{../../../../components/components}
\usepackage{hyperref}
\usepackage{fancyhdr}


% Configuration des en-têtes et pieds de page
\pagestyle{fancy}
\fancyhf{} % reset tout

\fancyhead[L]{DL2 Math-Info ASF4}
\fancyhead[C]{Analyse}
\fancyhead[R]{2025-2026}

\fancyfoot[L]{Ewen Rodrigues de Oliveira}
\fancyfoot[R]{\thepage}

\begin{document}

\docTitle{Chapitre 5 : Séries entières}

\section{Introduction}

\definition{
    Une \textbf{série entière} est une série de fonction dont le $n^{\text{ème}}$ terme général est un monôme de degré $n$.\\
    C'est-à-dire que c'est une série de la forme $\sum_{n} a_n z^n$ où $(a_n)_{n \in \mathbb{N}}$ est une suite de nombres complexes ($\mathbb{C}^\mathbb{N}$ ou $\mathbb{R}^\mathbb{N}$) et $z$ est une variable complexe.
}


\remark{
    Comme séries de fonctions, les propositions vues sur le chapitre sur les séries de fonctions s'appliquent aux séries entières.
}

\example{
    \[
        \sum_{n\geq0} z^n \hspace{1.25cm} \sum_{n\geq0} \frac{z^n}{n!} \hspace{1.25cm} \sum_{n\geq0} \frac{z^n}{n} \hspace{1.25cm} \cdots
    \]
}

\subsection{Convergence d'une série entière}

\theorem{Proposition}{}{false}{
    La série $\sum_{n\geq0} z^n$ converge si et seulement si $|z| < 1$ sur $\{z \in \mathbb{C} : |z| < 1\} = D(0, 1)$
}
\vocabulary{On dit que $1$ est le \textbf{rayon de convergence} de la série $\sum_{n\geq0} z^n$ (et $D(0, 1)$ est le \textbf{disque de convergence} de la série $\sum_{n\geq0} z^n$).}

\newpage
\tableofcontents


\end{document}  