\documentclass{article}

\usepackage[a4paper, left=1.5cm, right=1.5cm, top=2cm, bottom=2cm]{geometry}

\usepackage{../../../../components/components}

\usepackage{fancyhdr}


% Configuration des en-têtes et pieds de page
\pagestyle{fancy}
\fancyhf{} % reset tout

\fancyhead[L]{DL2 Math-Info MM4}
\fancyhead[C]{Algèbre}
\fancyhead[R]{2025-2026}

\fancyfoot[L]{Ewen Rodrigues de Oliveira}
\fancyfoot[R]{\thepage}

\begin{document}

\docTitle{Chapitre 4 : Espaces euclidiens}
\section{Produit scalaire et norme}
\subsection{Produit scalaire}
\textit{On ne rappelera pas les définitions de normes et distances ici et les propositions qui s'ensuivent. Le lecteur est invité à se référer au chapitre 4 du cours AN3.}

\definition{
    Soit $E$ un $\mathbb{R}$-espace vectoriel.\\
    Un \textbf{produit scalaire} sur $E$ est une application $\langle \cdot, \cdot \rangle : E \times E \to \mathbb{R}$ qui vérifie les propriétés suivantes :
    \begin{enumerate}
        \item $\forall x,y \in E, \; \langle x,y \rangle = \langle y,x \rangle$ (symétrie)
        \item $\forall x,x',y \forall \lambda \in \mathbb{R}, \; \langle \lambda x + x', y \rangle = \lambda \langle x,y \rangle + \langle x',y \rangle$ (linéarité en la première variable)
        \item $\forall x \in E, \; \langle x,x \rangle \geq 0$ et $\langle x,x \rangle = 0 \Rightarrow x = 0$ (positivité)
    \end{enumerate}
    En résumé, un produit scalaire est une forme bilinéaire, symétrique et définie positive.
}

\example{
    Dans $\mathbb{R}^n$,$\langle x,y \rangle = \sum_{i=1}^n x_i y_i$ est un produit scalaire.\\
    Dans $\mathcal{C}^0([0,1],\mathbb{R})$, $\langle f,g \rangle = \int_0^1 f(t)g(t) dt$ est un produit scalaire.
}

\remark{Si $E$ est un espace vectoriel normé de dimension finie, il n'y a pas de produit scalaire canonique \textit{a priori}. (On a vu néanmoins que dans $\mathbb{R}^n$, il existe un produit scalaire canonique.}

\definition{
    Un $\mathbb{R}$-espace vectoriel $E$ muni d'un produit scalaire est appelé un \textbf{espace euclidien}.\\
    On le note $(E, \langle \cdot, \cdot \rangle)$.
}

\theorem{Identités remarquables}{}{false}{
    $\langle x+y, x+y \rangle = \langle x,x \rangle + 2\langle x,y \rangle + \langle y,y \rangle$\\
    $\langle x+y, x-y \rangle = \langle x,x \rangle - \langle y,y \rangle$\\
}

\theorem{Théorème}{Inégalité de Cauchy-Schwarz}{false}{
    Soit $(E, \langle \cdot, \cdot \rangle)$ un espace euclidien.\\
    Alors, pour tous $x,y \in E$, on a :
    \[
        |\langle x,y \rangle| \leq \sqrt{\langle x,x \rangle} \sqrt{\langle y,y \rangle}
    \]
    De plus, l'égalité a lieu si et seulement si $x$ et $y$ sont colinéaires.
}
\ndlr{cf. Laurent.}

\subsection{Normes}

\reminder{On rappelle qu'une norme est une application $N\colon E \to \mathbb{R}_+$ qui vérifie l'homogénéité, l'inégalité triangulaire et la séparation.}

\theorem{Proposition}{}{false}{
    Soit $E$ un $\mathbb{R}$-espace vectoriel, muni d'un produit scalaire $\langle \cdot, \cdot \rangle$.\\
    On pose $\| x \| = \sqrt{\langle x,x \rangle}$.\\
    Alors $\| \cdot \|$ est une norme sur $E$, appelée la norme associée au produit scalaire $\langle \cdot, \cdot \rangle$.
}

\noindent{\textbf{Démonstration :}\\
\textbf{Homogénéité :} Soit $\lambda \in \mathbb{R}$ et $x \in E$.\\
$\| \lambda x \| = \sqrt{\langle \lambda x, \lambda x \rangle} = \sqrt{\lambda^2 \langle x,x \rangle} = |\lambda| \sqrt{\langle x,x \rangle} = |\lambda| \| x \|$\\
\textbf{Inégalité triangulaire :} Soit $x,y \in E$.\\
$\| x+y \|^2 = \langle x+y, x+y \rangle = \langle x,x \rangle + 2\langle x,y \rangle + \langle y,y \rangle \leq \langle x,x \rangle + 2|\langle x,y \rangle| + \langle y,y \rangle$\\
$\leq \langle x,x \rangle + 2\sqrt{\langle x,x \rangle} \sqrt{\langle y,y \rangle} + \langle y,y \rangle = (\sqrt{\langle x,x \rangle} + \sqrt{\langle y,y \rangle})^2$\\
$\Rightarrow \| x+y \| \leq \| x \| + \| y \|$\\
\textbf{Séparation :} Soit $x \in E$ tel que $\| x \| = 0$.\\
Alors $\sqrt{\langle x,x \rangle} = 0 \Rightarrow \langle x,x \rangle = 0 \Rightarrow x = 0$\\
}

\remark{$x \mapsto |x|$ est une norme sur $\mathbb{R}$, associée au produit scalaire $\langle x,y \rangle = xy$.}
\reminder{Une distance sur un ensemble $X$ est une application $d : X \times X \to \mathbb{R}_+$ qui vérifie la séparation, la symétrie et l'inégalité triangulaire.}

\theorem{Proposition}{Lien entre norme et distance}{false}{
    Soit $N$ une norme sur un $\mathbb{R}$-espace vectoriel $E$.\\
    Alors $d(x,y) = N(x-y)$ est une distance sur $E$.
}

\remark{En géométrie affine, si $A$ et $B$ sont deux points, le vecteur $\overrightarrow{AB}$ est défini par $\overrightarrow{AB} = B - A$, et $d(A,B) = \| \overrightarrow{AB} \|$.}

\remark{Un produit scalaire donne une norme ($\| x \| = \sqrt{\langle x,x \rangle}$) et une norme donne une distance ($d(x,y) = \| x-y \|$).}
\theorem{Propriété}{Norme associée au produit scalaire (identité du parallélogramme)}{false}{
    $\forall x,y \in E, \; \| x + y \|^2 + \| x - y \|^2 = 2\| x \|^2 + 2\| y \|^2$. \\
    On peut le vérifier avec $\| x \| = \sqrt{\langle x,x \rangle}$ et les identités remarquables.\\\\
    D'où si une norme ne vérifie pas l'identité du parallélogramme, elle n'est pas associée à un produit scalaire.
}

\definition{
    Soit $E$ un $\mathbb{R}$-espace vectoriel, muni d'un produit scalaire $\langle \cdot, \cdot \rangle$.\\
    L'\textbf{angle non-orienté} entre $x,y \in E\setminus \{0\}$ est défini par $\theta = \arccos\left(\frac{\langle x,y \rangle}{\| x \| \| y \|}\right) \in [0, \pi]$.
}

\remark{Cette définition est basée sur la propriété en géométrie euclidienne classique (angle où la direction n'a pas d'importance), et on a : $\langle x,y \rangle = \| x \| \| y \| \cos(\theta)$.}
\remark{Le vecteur normé associé à $x \in E\setminus \{0\}$ est $\frac{x}{\| x \|}$. C'est utile pour calculer des vecteurs unitaires.}

\section{Orthogonalité}

\definition{
    Soit $E$ un $\mathbb{R}$-espace vectoriel, muni d'un produit scalaire $\langle \cdot, \cdot \rangle$.\\
    Deux vecteurs $x,y \in E$ sont dits \textbf{orthogonaux} si $\langle x,y \rangle = 0$.\\\\
    On note $x \perp y$.
}
\remark{L'angle entre deux vecteurs orthogonaux non nuls est $\frac{\pi}{2}$, et tout vecteur est orthogonal au vecteur nul.}

\definition{
    Soit $E$ un $\mathbb{R}$-espace vectoriel, muni d'un produit scalaire $\langle \cdot, \cdot \rangle$.\\
    Une famille de vecteurs $(x_i)_{i \in I}$ est dite \textbf{orthogonale} si $x_i \perp x_j$ pour tous $i \neq j$, \textit{i.e.} si ses vecteurs sont deux à deux orthogonaux.
}

\theorem{Théorème de Pythagore}{}{false}{
    Soit $E$ un $\mathbb{R}$-espace vectoriel, muni d'un produit scalaire $\langle \cdot, \cdot \rangle$.\\
    Soit $(x_1, \ldots, x_k)$ une famille orthogonale (finie) de $E$.\\
    Alors : \[ \| x_1 + \ldots + x_k \|^2 = \| x_1 \|^2 + \ldots + \| x_k \|^2. \]
    De plus, si $k=2$, alors $\| x + y \|^2 = \| x \|^2 + \| y \|^2 \Rightarrow x \perp y$.
}

\noindent{\textbf{Démonstration :}\\
$\| \sum_{i=1}^k x_i \|^2 = \langle \sum_{i=1}^k x_i, \sum_{j=1}^k x_j \rangle = \text{ (bilinéarité) } \sum_{i=1}^k \sum_{j=1}^k \langle x_i, x_j \rangle = \sum_{i=1}^k \langle x_i, x_i \rangle = \sum_{i=1}^k \| x_i \|^2$.\\\\
Pour $k = 2$, 
$\| x + y \| ^ 2 = \|x\|^2 + 2\langle x,y \rangle + \| y \|^2$.\\
Donc si $\| x + y \|^2 = \| x \|^2 + \| y \|^2$, alors $2\langle x,y \rangle = 0 \Rightarrow x \perp y$. $\Box$
}

\theorem{Définition}{}{false}{
    Soit $E$ un $\mathbb{R}$-espace vectoriel, muni d'un produit scalaire $\langle \cdot, \cdot \rangle$.\\
    Une famille de vecteurs $(x_i)_{i \in I}$ est dite \textbf{orthonormée} si elle est orthogonale et que $\| x_i \| = 1$ pour tout $i \in I$.\\\\
    Autrement dit, $\forall i,j \in I, \; \langle x_i, x_j \rangle = \delta_{ij}$.
}

\newpage
\tableofcontents


\end{document}  