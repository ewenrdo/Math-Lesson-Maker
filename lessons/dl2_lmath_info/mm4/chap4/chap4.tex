\documentclass{article}

\usepackage[a4paper, left=1.5cm, right=1.5cm, top=2cm, bottom=2cm]{geometry}

\usepackage{../../../../components/components}

\usepackage{fancyhdr}


% Configuration des en-têtes et pieds de page
\pagestyle{fancy}
\fancyhf{} % reset tout

\fancyhead[L]{DL2 Math-Info MM4}
\fancyhead[C]{Algèbre}
\fancyhead[R]{2025-2026}

\fancyfoot[L]{Ewen Rodrigues de Oliveira}
\fancyfoot[R]{\thepage}

\begin{document}

\docTitle{Chapitre 4 : Espaces euclidiens}

\section{Produit scalaire et norme}
\subsection{Produit scalaire}
\definition{
    Soit $E$ un $\mathbb{R}$-espace vectoriel.\\
    Un \textbf{produit scalaire} sur $E$ est une application $\langle \cdot, \cdot \rangle : E \times E \to \mathbb{R}$ qui vérifie les propriétés suivantes :
    \begin{enumerate}
        \item $\forall x,y \in E, \; \langle x,y \rangle = \langle y,x \rangle$ (symétrie)
        \item $\forall x,x',y \forall \lambda \in \mathbb{R}, \; \langle \lambda x + x', y \rangle = \lambda \langle x,y \rangle + \langle x',y \rangle$ (linéarité en la première variable)
        \item $\forall x \in E, \; \langle x,x \rangle \geq 0$ et $\langle x,x \rangle = 0 \Rightarrow x = 0$ (positivité)
    \end{enumerate}
    En résumé, un produit scalaire est une forme bilinéaire, symétrique et définie positive.
}

\example{
    Dans $\mathbb{R}^n$,$\langle x,y \rangle = \sum_{i=1}^n x_i y_i$ est un produit scalaire.\\
    Dans $\mathcal{C}^0([0,1],\mathbb{R})$, $\langle f,g \rangle = \int_0^1 f(t)g(t) dt$ est un produit scalaire.
}

\remark{Si $E$ est un espace vectoriel normé de dimension finie, il n'y a pas de produit scalaire canonique \textit{a priori}. (On a vu néanmoins que dans $\mathbb{R}^n$, il existe un produit scalaire canonique.}

\definition{
    Un $\mathbb{R}$-espace vectoriel $E$ muni d'un produit scalaire est appelé un \textbf{espace euclidien}.\\
    On le note $(E, \langle \cdot, \cdot \rangle)$.
}

\theorem{Identités remarquables}{}{false}{
    $\langle x+y, x+y \rangle = \langle x,x \rangle + 2\langle x,y \rangle + \langle y,y \rangle$\\
    $\langle x+y, x-y \rangle = \langle x,x \rangle - \langle y,y \rangle$\\
}

\theorem{Théorème}{Inégalité de Cauchy-Schwarz}{false}{
    Soit $(E, \langle \cdot, \cdot \rangle)$ un espace euclidien.\\
    Alors, pour tous $x,y \in E$, on a :
    \[
        |\langle x,y \rangle| \leq \sqrt{\langle x,x \rangle} \sqrt{\langle y,y \rangle}
    \]
    De plus, l'égalité a lieu si et seulement si $x$ et $y$ sont linéairement dépendants.
}
\ndlr{cf. Laurent.}

\subsection{Normes}

\reminder{On rappelle qu'une norme est une application $N\colon E \to \mathbb{R}_+$ qui vérifie l'homogénéité, l'inégalité triangulaire et la séparation.}

\theorem{Proposition}{}{false}{
    Soit $E$ un $\mathbb{R}$-espace vectoriel, muni d'un produit scalaire $\langle \cdot, \cdot \rangle$.\\
    On pose $\| x \| = \sqrt{\langle x,x \rangle}$.\\
    Alors $\| \cdot \|$ est une norme sur $E$, appelée la norme associée au produit scalaire $\langle \cdot, \cdot \rangle$.
}

\noindent{\textbf{Démonstration :}\\
\textbf{Homogénéité :} Soit $\lambda \in \mathbb{R}$ et $x \in E$.\\
$\| \lambda x \| = \sqrt{\langle \lambda x, \lambda x \rangle} = \sqrt{\lambda^2 \langle x,x \rangle} = |\lambda| \sqrt{\langle x,x \rangle} = |\lambda| \| x \|$\\
\textbf{Inégalité triangulaire :} Soit $x,y \in E$.\\
$\| x+y \|^2 = \langle x+y, x+y \rangle = \langle x,x \rangle + 2\langle x,y \rangle + \langle y,y \rangle \leq \langle x,x \rangle + 2|\langle x,y \rangle| + \langle y,y \rangle$\\
$\leq \langle x,x \rangle + 2\sqrt{\langle x,x \rangle} \sqrt{\langle y,y \rangle} + \langle y,y \rangle = (\sqrt{\langle x,x \rangle} + \sqrt{\langle y,y \rangle})^2$\\
$\Rightarrow \| x+y \| \leq \| x \| + \| y \|$\\
\textbf{Séparation :} Soit $x \in E$ tel que $\| x \| = 0$.\\
Alors $\sqrt{\langle x,x \rangle} = 0 \Rightarrow \langle x,x \rangle = 0 \Rightarrow x = 0$\\
}

\remark{$x \mapsto |x|$ est une norme sur $\mathbb{R}$, associée au produit scalaire $\langle x,y \rangle = xy$.}

\newpage
\tableofcontents


\end{document}  