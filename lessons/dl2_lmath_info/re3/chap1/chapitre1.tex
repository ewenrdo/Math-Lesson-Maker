\documentclass{article}

\usepackage[a4paper, left=1.5cm, right=1.5cm, top=2cm, bottom=2cm]{geometry}

\usepackage{../../../../components/components} % <-- ton fichier .sty, avec toutes tes définitions

\usepackage{fancyhdr}


% Configuration des en-têtes et pieds de page
\pagestyle{fancy}
\fancyhf{} % reset tout

\fancyhead[L]{DL2 Math-Info RE3}
\fancyhead[C]{Groupes}
\fancyhead[R]{2025-2026}

\fancyfoot[L]{Ewen Rodrigues de Oliveira}
\fancyfoot[R]{\thepage}

\begin{document}

\docTitle{Chapitre 1 : Groupes}

\section{Loi de composition interne}

\definition{Soit \(E\) un ensemble. Une \textbf{loi de composition interne} sur \(E\) est une application \( * : E \times E \to E \) qui à tout couple \((x, y) \in E \times E\) associe un élément \(x * y \in E\).}
\theorem{Propriété}{Associativité}{false}{\(*\) est associative si \( \forall x, y, z \in E, (x * y) * z = x * (y * z)\).}
\theorem{Propriété}{Elément neutre}{false}{On dit que \(e \in E\) est un élément neutre si \( \forall x \in E, e * x = x * e = x\).}
\remark{L'élément neutre est unique. La démonstration découle du fait que si on prend deux éléments neutres \(e\) et \(e'\), on a \(e * e' = e\) et \(e * e' = e'\), donc \(e = e'\).}
\theorem{Propriété}{Symétrique}{false}{Soient \(a,b \in E\). On dit que \(b\) est symétrique (ou inverse, ou opposé) de \(a\) si \(a * b = b * a = e\), où \(e\) est l'élément neutre.}
\theorem{Propriété}{Commutativité}{false}{\(*\) est commutative si \( \forall x, y \in E, x * y = y * x\).}

\vocabulary{Notations typiques pour les lois de composition interne : \(+\), \(\times\), \(\cdot\), \(\circ\), etc.}
\section{Notions de groupe}
\subsection{Généralités}

\definition{Soit G un ensemble muni d'une loi de composition interne \( * \). On dit que \((G, *)\) est un \textbf{groupe} si les trois propriétés suivantes sont vérifiées : 
\begin{itemize}
    \item \( * \) est associative.
    \item Il existe un élément neutre \(e \in G\).
    \item Tout élément de \(G\) possède un symétrique dans \(G\).
\end{itemize}
Si \( * \) est en plus commutative, on dit que \((G, *)\) est un \textbf{groupe abélien}.}

%% Exemples et contres exemples de groupes

\example{
    \textbf{Exemples de groupes :}
    \begin{itemize}
        \item \((\mathbb{Z}, +)\) : l'ensemble des entiers avec l'addition.
        \item \((\mathbb{R}^*, \times)\), \((\mathbb{Q}^*, \times)\), \((\mathbb{C}^*, \times)\) : l'ensemble des réels, rationnels et complexes non nuls avec la multiplication.
        \item $\left( \{ \text{bijections } X \to X \mid X \text{ est un ensemble} \}, \circ \right)$ : l'ensemble des bijections d'un ensemble \(X\) dans lui-même avec la composition.
    \end{itemize}
    \textbf{Contre-exemples de groupes :}
    \begin{itemize}
        \item \((\mathbb{N}, +)\) : l'ensemble des entiers naturels avec l'addition (pas d'élément neutre dans \(\mathbb{N}\)).
    \end{itemize}
}

\vocabulary{
    \textbf{Systèmes de notations pour les groupes :}
    \begin{itemize}
        \item Système additif : on note le groupe \((G, +)\), l'élément neutre est noté \(0\) et le symétrique de \(x\) est noté \(-x\).
        \item Système multiplicatif : on note le groupe \((G, \times)\) ou \((G, \cdot)\), l'élément neutre est noté \(1\) et le symétrique de \(x\) est noté \(x^{-1}\).
    \end{itemize}
}
\theorem{Propriété}{Produit de lois}{true}{
Soient \((G_1, *_1)\) et \((G_2, *_2)\) deux groupes. On définit une loi de composition interne sur \(G_1 \times G_2\) par * : 
\[
(g_1, g_2) * (h_1, h_2) \mapsto (g_1 *_1 h_1, g_2 *_2 h_2)
\]
pour tout \((x_1, y_1), (x_2, y_2) \in G_1 \times G_2\). Alors \((G_1 \times G_2, *)\) est un groupe.\newline
}
\theorem{Proposition}{Produit cartésien}{false}{Soient \((G_1, *_1)\) et \((G_2, *_2)\) deux groupes. On définit une loi de composition interne sur \(G_1 \times G_2\) par * comme susdit. Alors l'ensemble \((G_1 \times G_2, *)\) est un groupe, appelé le \textbf{groupe produit} de \((G_1, *_1)\) et \((G_2, *_2)\).}
\noindent\textbf{Preuve:}
\begin{itemize}
    \item \textit{Associativité :} Soient \((g_1, g_2), (h_1, h_2), (k_1, k_2) \in G_1 \times G_2\).
    \begin{align*}
        ((g_1, g_2) * (h_1, h_2)) * (k_1, k_2) &= (g_1 *_1 h_1, g_2 *_2 h_2) * (k_1, k_2) \\
        &= ((g_1 *_1 h_1) *_1 k_1, (g_2 *_2 h_2) *_2 k_2) \\
        &= (g_1 *_1 (h_1 *_1 k_1), g_2 *_2 (h_2 *_2 k_2)) \quad (\text{par associativité dans } G_1 \text{ et } G_2) \\
        &= (g_1, g_2) * (h_1 *_1 k_1, h_2 *_2 k_2) \\
        &= (g_1, g_2) * ((h_1, h_2) * (k_1, k_2))
    \end{align*}

    \item \textit{Élément neutre :} Soient \(e_1\) et \(e_2\) les éléments neutres de \(G_1\) et \(G_2\) respectivement. Alors \((e_1, e_2)\) est l'élément neutre de \(G_1 \times G_2\) car pour tout \((g_1, g_2) \in G_1 \times G_2\), \((e_1, e_2) * (g_1, g_2) = (e_1 *_1 g_1, e_2 *_2 g_2) = (g_1, g_2)\) et \((g_1, g_2) * (e_1, e_2) = (g_1 *_1 e_1, g_2 *_2 e_2) = (g_1, g_2)\).
    \item \textit{Symétrique :} Soit \((g_1, g_2) \in G_1 \times G_2\). Comme \(G_1\) et \(G_2\) sont des groupes, il existe \(g_1^{-1} \in G_1\) et \(g_2^{-1} \in G_2\) tels que \(g_1 *_1 g_1^{-1} = e_1\) et \(g_2 *_2 g_2^{-1} = e_2\). Alors le symétrique de \((g_1, g_2)\) dans \(G_1 \times G_2\) est \((g_1^{-1}, g_2^{-1})\) car :
\end{itemize}
\theorem{Propriété}{Produit cartésien et commutativité}{true}{
    Si \((G_1, *_1)\) et \((G_2, *_2)\) sont des groupes abéliens, alors leur produit cartésien \((G_1 \times G_2, *)\) est aussi un groupe abélien.
}
\remark{On pourrait prendre plus de deux groupes et faire le produit cartésien de plusieurs groupes.}
\subsection{Sous-groupes}
\definition{Soit \((G, \cdot)\) un groupe \textit{(on utilise la notation multiplicative, mais cela fonctionne aussi en notation additive)}. Un \textbf{sous-groupe} de \(G\) est un sous-ensemble \(H \subseteq G\) tel que \((H, \cdot)\) est lui-même un groupe.}
\theorem{Propriété}{Lien entre sous-groupe et groupe}{true}{
    Un sous-groupe est lui-même un groupe pour la même loi de composition interne que le groupe dont il est issu.
}

\example{\((Z, +)\) est un sous-groupe de \((\mathbb{R}, +)\).}
\theorem{Proposition}{Sous-groupe}{false}{
    Soit \((H, \cdot)\) un sous-groupe de \((G, \cdot)\ \Leftrightarrow\) %% système
    \begin{itemize}
        \item \(H \neq \emptyset\) : 1
        \item \( \forall h, h' \in H, h \cdot h' \in H\) (stabilité par la loi) : 2
        \item \( \forall h \in H, \exists h^{-1} \in H\) (stabilité par l'inverse) : 3
    \end{itemize}
}

\noindent\textbf{Preuve:}
\begin{itemize}
    \item \(\Rightarrow\)/ : Si \(H\) est un sous-groupe de \(G\), alors par définition de groupe, \(H\) satisfait 1, 2 et 3.
    \item \(\Leftarrow\)/ : Supposons que \(H\) vérifie les trois conditions. Nous devons montrer que \((H, \cdot)\) est un groupe.
    \begin{itemize}
        \item \textit{Associativité :} La loi de composition interne sur \(H\) est la même que celle sur \(G\), donc elle est associative.
        \item \textit{Élément neutre :} Soit \(e\) l'élément neutre de \(G\). Comme \(H\) est non vide, \(\exists h_0\in H\) et par la condition 3, \(h_0^{-1} \in H\). Par la définition de l'élément neutre dans \(G\), on a \(h_0 \cdot h_0^{-1} = e\). Donc \(e \in H\).
        \item \textit{Symétrique :} Par la condition 3, pour tout \(h \in H\), son inverse \(h^{-1}\) appartient à \(H\).
    \end{itemize}
    Ainsi, toutes les propriétés d'un groupe sont satisfaites pour \(H\), donc \(H\) est un sous-groupe de \(G\).   
\end{itemize}

\example{
    \begin{itemize}
        \item \((G, \cdot)\) est un sous-groupe de lui-même.
        \item \(\{1\}\) est un sous-groupe de \(G\).
        \item \((Z, +)\) est un sous-groupe de \((\mathbb{R}, +)\).
    \end{itemize}
}

\theorem{Proposition}{Intersection}{false}{
    Soit \((H_i)_{i \in I}\) une famille de sous-groupes de \((G, \cdot)\). Alors l'intersection \(H = \bigcap_{i \in I} H_i\) est un sous-groupe de \(G\).
}

\noindent\textbf{Preuve:}\\
\carreaux{20}


\theorem{Corollaire}{Sous-groupe engendré}{false}{Soit \(X \subseteq G\). Considérons  \(H = \bigcap_{i \in I} H_i\). C'est un \textbf{sous-groupe de \(G\) engendré par \(X\)}.}
\definition{Soit \(g \in G\).\\
On a posé pour \(n\in \mathbb{Z}, g^n = \underbrace{g \cdot g \cdot ... \cdot g}_{n \text{ fois}}\) si \(n > 0\), \(g^0 = e\) (élément neutre) et \(g^n = \underbrace{g^{-1} \cdot g^{-1} \cdot ... \cdot g^{-1}}_{-n \text{ fois}}\) si \(n < 0\).\newline
On pose \(g^{\mathbb{Z}} = \{ g^n \mid n \in \mathbb{Z} \}\) : c'est \textbf{l'ensemble des itérés} de \(G\).}

\theorem{Proposition}{Sous-groupe engendré}{false}{
    On a que \(g^{\mathbb{Z}}\) est un sous-groupe de \(G\) engendré par \(g\).
}

\noindent\textbf{Preuve:}\\
\carreaux{10}

\remark{En notation additive, l'ensemble des itérés de \(g\) est noté \(\mathbb{Z}g = \{ ng \mid n \in \mathbb{Z} \}\).}

\definition{Si \(G = g^{\mathbb{Z}}\), on dit que \(G\) est monogène et que \(g\) est un générateur de \(G\).}
\example{
    \(\mathbb{Z}\) est monogène et engendré par \(1\).
}
\definition{Si le sous-groupe engendré par \(X\) est \(G\), on dit que \(X\) est un système de générateurs de \(G\).}

\end{document}