\documentclass{article}

\usepackage[a4paper, left=1.5cm, right=1.5cm, top=2cm, bottom=2cm]{geometry}

\usepackage{../../../../components/components}
\usepackage{hyperref}
\usepackage{fancyhdr}


% Configuration des en-têtes et pieds de page
\pagestyle{fancy}
\fancyhf{} % reset tout

\fancyhead[L]{DL2 Math-Info ASF4}
\fancyhead[C]{Analyse}
\fancyhead[R]{2025-2026}

\fancyfoot[L]{Ewen Rodrigues de Oliveira}
\fancyfoot[R]{\thepage}

\begin{document}

\docTitle{Chapitre 5 : Séries entières}

\section{Introduction}

\definition{
    Une \textbf{série entière} est une série de fonction dont le $n^{\text{ème}}$ terme général est un monôme de degré $n$.\\
    C'est-à-dire que c'est une série de la forme $\sum_{n} a_n z^n$ où $(a_n)_{n \in \mathbb{N}}$ est une suite de nombres complexes ($\mathbb{C}^\mathbb{N}$ ou $\mathbb{R}^\mathbb{N}$) et $z$ est une variable complexe.
}


\remark{
    Comme séries de fonctions, les propositions vues sur le chapitre sur les séries de fonctions s'appliquent aux séries entières. Exception faite sur le théorème touchant aux dérivées, car on ne sait pas encore ce qu'est la dérivée d'une fonction de variable complexe.
}

\example{
    \[
        \sum_{n\geq0} z^n \hspace{1.25cm} \sum_{n\geq0} \frac{z^n}{n!} \hspace{1.25cm} \sum_{n\geq0} \frac{z^n}{n} \hspace{1.25cm} \cdots
    \]
}

\reminder{Le $sup$ d'une partie de $\mathbb{R}$ est défini par $sup(A) = min\{ x \in \mathbb{R} : A \subset ]-\infty, x] \}$ et le $inf$ d'une partie de $\mathbb{R}$ est défini par $inf(A) = max\{ x \in \mathbb{R} : A \subset [x, +\infty[ \}$.}

\subsection{Convergence d'une série entière}

\example{
    La série $\sum_{n\geq0} z^n$ converge si et seulement si $|z| < 1$ sur $\{z \in \mathbb{C} : |z| < 1\} = D(0, 1)$
}

\example{
    Soit $z \in \mathbb{C}$. On pose $u_n = \frac{z^n}{n!}$. Alors on a $\frac{u_{n+1}}{u_n} = \frac{z^{n+1}}{(n+1)!} \cdot \frac{n!}{z^n} = \frac{z}{n+1} \to 0$ lorsque $n \to +\infty$.\\
    Par le critère de d'Alembert, la série $\sum_{n\geq0} \frac{z^n}{n!}$ converge pour tout $z \in \mathbb{C}$ et on dira que le rayon de convergence de la série $\sum_{n\geq0} \frac{z^n}{n!}$ est $+\infty$.
}

\vocabulary{On dit que $1$ est le \textbf{rayon de convergence} de la série $\sum_{n\geq0} z^n$ (et $D(0, 1)$ est le \textbf{disque de convergence} de la série $\sum_{n\geq0} z^n$).}

\theorem{Lemme}{}{false}{
    Soit $\sum_{n\geq0} a_n z^n$ une série entière et $z_0 \in \mathbb{C}$.
    \begin{enumerate}
        \item Si $\sum_{n\geq0} a_n z_0^n$ converge absolument, alors la série $\sum_{n\geq0} a_n z^n$ converge normalement sur $\overline{D}(0, |z_0|) = \{ z \in \mathbb{C} : |z| \leq |z_0| \}$.
        \item Si $(a_n z_0^n)_{n\in\mathbb{N}}$ est bornée, alors la série $\sum_{n\geq0} a_n z^n$ converge normalement sur $\overline{D}(0, r)$ pour tout $r < |z_0|$
    \end{enumerate}
}

\noindent{\textbf{Démonstration :}\\
    1. Soit $z \in \overline{D}(0, |z_0|)$ et soit $n \in \mathbb{N}$.\\
    On a $|a_n z^n| = |a_n||z|^n \leq |a_n||z_0|^n$.\\
    Et de plus $\sum_{n\geq0} |a_n||z_0|^n$ converge par hypothèse et est indépendant de $z$.\\

    \noindent Donc la série $\sum_{n\geq0} a_n z^n$ converge normalement sur $\overline{D}(0, |z_0|)$.\\\\

    2. Soit $r < |z_0|$.\\
    Montrons que $\sum_{n\geq0} a_n r^n$ converge absolument.\\

    et $1 \Rightarrow 2$ et $|a_n|r^n = |a_n||z_0|^n \cdot \left(\frac{r}{|z_0|}\right)^n \leq M \cdot \left(\frac{r}{|z_0|}\right)^n$ pour un certain $M = \sup_{n\in\mathbb{N}} |a_n||z_0|^n$.\\
    et comme série géométrique de raison $\frac{r}{|z_0|} < 1$, la série $\sum_{n\geq0} M \cdot \left(\frac{r}{|z_0|}\right)^n$ converge.\\
    D'où la série $\sum_{n\geq0} a_n r^n$ converge absolument.\\
}

\definition{
    On appelle \textbf{rayon de convergence} de la série $\sum_{n\geq0} a_n z^n$ le nombre $R \in [0, +\infty[]$ défini par $R = \sup \{ r \in [0, +\infty[] : (a_n r^n)_{n\in\mathbb{N}} \text{ est bornée} \} = \sup \{ r \in [0, +\infty[] : \sum_{n\geq0} a_n r^n \text{ converge} \}$.
}

\definition{
    On appelle \textbf{disque ouvert de convergence} de la série $\sum_{n\geq0} a_n z^n$ l'ensemble $D(0, R) = \{ z \in \mathbb{C} : |z| < R \}$ où $R$ est le rayon de convergence de la série $\sum_{n\geq0} a_n z^n$.
}

\remark{
    Si $R = 0$, $D(0, R) = \emptyset$\\
    Si $R = +\infty$, $D(0, R) = \mathbb{C}$
}

\theorem{Théorème}{}{false}{
    Une série entière converge absolument en tout point du disque ouvert de convergence et converge normalement sur tout disque fermé inclus dans le disque ouvert de convergence. En particulier, la somme est continue en tout point du disque ouvert de convergence.
}

\ndlr{cf. Laurent pour la preuve}

\remark{
    On déduit que $\{ r \in [0, +\infty[ : \sum_{n\geq0} a_n r^n \text{ converge} \} = [0, R[$ ou $[0, R]$.\\
    Autrement dit, il y a convergence sur $D(0, R) = \{ z \in \mathbb{C} : |z| < R \}$ et il y a divergence sur $\{ z \in \mathbb{C} : |z| > R \}$.
    Par contre, on ne peut rien dire sur le cercle $\{ z \in \mathbb{C} : |z| = R \}$.
}

\example{
    \begin{itemize}
        \item $\sum_{n=0}^{+\infty} z^n$ a pour rayon de convergence $R = 1$. En effet, $\sum_{n=0}^{+\infty} z^n$ converge si et seulement si $|z| < 1$. En particulier, il y a divergence si $|z| = 1$.
        \item $\sum_{n=0}^{+\infty} \frac{z^n}{n^2}$ a pour rayon de convergence $R = 1$. En effet, $\sum_{n=0}^{+\infty} \frac{z^n}{n^2}$ converge si et seulement si $|z| \leq 1$ (sinon, divergence grossière). En particulier, il y a divergence si $|z| > 1$.
    \end{itemize}
}

\textbf{Calculs de rayons de convergence :}
$\sum_{n\geq0} a_n z^n$ une série entière.\\
Si $(a_n r^n)$ est bornée sur $R \geq r$
Si $(a_n r^n)$ est non bornée alors $R \leq r$

\theorem{Théorème}{Critères de d'Alembert et Cauchy}{false}{
    Soit $R$ le rayon de convergence de $\sum_{n=0}^{+\infty} a_n z^n$.
    \begin{enumerate}
        \item Si $|a_n|^{\frac{1}{n}} \to l$ lorsque $n \to +\infty$, alors $R = \frac{1}{l}$
        \item Si $a_n \neq 0$ à partir d'un certain rang et si $\frac{|a_{n+1}|}{|a_n|} \to l$ lorsque $n \to +\infty$, alors $R = \frac{1}{l}$.
    \end{enumerate}
}

\remark{
    Lorsque $R = +\infty, l=0$ et lorsque $R = 0, l = +\infty$.
}

\example{
    $\sum_{n=0}^{+\infty} \frac{z^n}{n^2}$ a pour rayon de convergence $R = 1$ car $\frac{1}{n^2}^{\frac{1}{n}} \to 1$ lorsque $n \to +\infty$.
}

\remark{
    Si $f$ est définie sur $D(0,1)$. Alors $z \mapsto f(az)$ est définie sur $D(0, \frac{1}{a})$.
}

\ndlr{cf. Laurent pour l'exemple}
\ndlr{Une proposition hors programme sur la limsup a été vue en cours, mais elle n'est pas au programme, donc pas ici.}

\theorem{Proposition}{}{false}{
    Soit $\frac{P}{Q}$ une fraction rationnelle. Alors, pour $n_0$ assez grand $\sum_{n \geq n_0} \frac{P(n)}{Q(n)} a_n z^n$ est bien défini et son rayon et convergence est le même que $\sum_{n = 0}^{+\infty} a_n z^n$.
}

\ndlr{cf. Laurent pour la preuve}

\example{
    $\sum a_nz^n$, $\sum \frac{a_n}{n} z^n$, $\sum \frac{a_n}{n^2} z^n$ ont le même rayon de convergence.
}

\theorem{Proposition}{}{false}{
    $\sum a_n z^n$ de rayon de convergence $R_a$.\\
    $\sum b_n z^n$ de rayon de convergence $R_b$.\\
    Alors $\sum (a_n + b_n) z^n$ a pour rayon de convergence $R \geq min(R_a, R_b)$.
}

\newpage
\tableofcontents

\end{document}  