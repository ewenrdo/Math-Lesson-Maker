\documentclass{article}

\usepackage[a4paper, left=1.5cm, right=1.5cm, top=2cm, bottom=2cm]{geometry}

\usepackage{../../../../components/components} % <-- ton fichier .sty, avec toutes tes définitions

\usepackage{hyperref}
\usepackage{fancyhdr}


% Configuration des en-têtes et pieds de page
\pagestyle{fancy}
\fancyhf{} % reset tout

\fancyhead[L]{DL2 Math-Info}
\fancyhead[C]{Réduction des endomorphismes}
\fancyhead[R]{2025-2026}

\fancyfoot[L]{Ewen Rodrigues de Oliveira}
\fancyfoot[R]{\thepage}

\begin{document}

\docTitle{Chapitre 6 : Théorie spectrale}


\section{Valeurs propres, vecteurs propres, sous-espaces propres}

On se donne $E$ un $K$-espace vectoriel et $u \in End(E)$ un endomorphisme de $E$.
\definition{
    Soit $\lambda \in K$. On dit que $\lambda$ est une \textbf{valeur propre (vp)} de $u$ s'il existe un vecteur $x \in E \setminus \{0_E\}$ tel que $u(x) = \lambda x$. Un tel vecteur $x$ est appelé un \textbf{vecteur propre ($\overrightarrow{vp}$)} associé à la valeur propre $\lambda$ de $u$.
}

\definition{
    On note $E_\lambda$ le \textbf{sous-espace propre} de $u$ associé à la valeur propre $\lambda$ :
    \[
        E_\lambda = \{ x \in E \mid u(x) = \lambda x \} = Ker(u - \lambda Id_E)
    \]
}

\theorem{Propriété}{}{true}{
    On a $\lambda$ est une valeur propre de $u$ si et seulement si $E_\lambda \neq \{0_E\}$.\\
    Ce qui est vrai si et seulement si $u - \lambda Id_E$ n'est pas inversible (autrement dit, $det(u - \lambda Id_E) = 0$).
}

\vocabulary{On dit que $u$ est \textbf{diagonalisable} s'il existe une base de $E$ formée de $\overrightarrow{vp}$ de $u$. (vecteurs propres)}
\theorem{Proposition}{}{false}{
    $u$ est diagonalisable si et seulement si $Mat_B(u)$ est diagonale pour une certaine base $B$ de $E$. (E de dimension finie)
}

\noindent{\textbf{Preuve :}\\
Soient $(x_1, \ldots, x_n)$ une base de $E$ formée de $\overrightarrow{vp}$ de $u$ avec $u(x_i) = \lambda_i x_i$. On a donc :
\[
    Mat_B(u) = 
    \begin{pmatrix}
        \lambda_1 & 0 & \ldots & 0 \\
        0 & \lambda_2 & \ldots & 0 \\
        \vdots & \vdots & \ddots & \vdots \\
        0 & 0 & \ldots & \lambda_n
    \end{pmatrix}
\] $\Box$
}

\vocabulary{On dit que $u$ est \textbf{trigonalisable} s'il existe une base de $E$ dans laquelle la matrice de $u$ est triangulaire supérieure. (vecteurs propres)}
\remark{Alors $x$ est un $\overrightarrow{vp}$ de $u$. Un endomorphisme sans $\overrightarrow{vp}$ n'est pas diagonalisable.}

\example{Si $E = K^2$ et $Mat_B(u) = \begin{pmatrix}
    5 & -1 \\
    4 & 1
\end{pmatrix}$ ($B$ = canonique), alors $u$ est trigonalisable pour la base $B' = \left\{ \begin{pmatrix}
    1 \\
    2   
\end{pmatrix}, \begin{pmatrix}
    0 \\
    -1
\end{pmatrix} \right\}$. En effet, on a :
$u(\begin{pmatrix}
    1 \\
    2\end{pmatrix}) = \begin{pmatrix}
    3 \\
    6
\end{pmatrix} = 3 \begin{pmatrix}
    1 \\
    2
\end{pmatrix}$
}

\example{
    Si $Mat_B(u) = \begin{pmatrix}
        1 & 1 \\
        0 & 2
    \end{pmatrix}$, alors $u$ est diagonalisable.
}
\example{
    Si $Mat_B(u) = \begin{pmatrix}
        0 & -1 \\
        1 & 0
    \end{pmatrix}$, n'est pas diagonalisable si $K = \mathbb{R}$ (pas de $\overrightarrow{vp}$), mais est diagonalisable si $K = \mathbb{C}$ (car $\lambda = i$ et $\lambda = -i$ sont des vp).
}

\section{Projections, symétries et rotations}
Posons $E = F \oplus G$ avec $F, G$ sous-espaces vectoriels de $E$.

\theorem{Proposition}{}{true}{
    Soit $p \in \mathcal{L}(E)$ la projection sur $F$ parallèlement à $G$ (i.e. $p(x + y) = x$ pour tout $(x, y) \in F \times G$).\\
    Alors les valeurs propres de $p$ sont contenues dans $\{0, 1\}$.
    De plus, $F = E_1$ et $G = E_0$ et $p$ est diagonalisable.
}

\ndlr{cf. Laurent pour la démonstration}

\theorem{Proposition}{}{true}{
    Soit $s \in \mathcal{L}(E)$ la symétrie par rapport à $F$ parallèlement à $G$ (i.e. $s(x + y) = x - y$ pour tout $(x, y) \in F \times G$).\\
    Alors les valeurs propres de $s$ sont contenues dans $\{-1, 1\}$.
    De plus, $F = E_1$ et $G = E_{-1}$ et $s$ est diagonalisable.
}

Supposons $E = \mathbb{K}^2, K=\mathbb{R}$.\\
Considérons $u$ tel que $Mat_B(u) = \begin{pmatrix}
    \cos \theta & -\sin \theta \\
    \sin \theta & \cos \theta \end{pmatrix}$ dans la base canonique $B$ de $\mathbb{R}^2$ avec $\theta \in \mathbb{R} \setminus \pi\mathbb{Z}$.

\theorem{Proposition}{}{true}{
    $u$ n'est pas diagonalisable si $K = \mathbb{R}$ (pas de $\overrightarrow{vp}$), mais est diagonalisable si $K = \mathbb{C}$ (car $\lambda = \cos \theta + i \sin \theta$ et $\lambda = \cos \theta - i \sin \theta$ sont des vp).
}

\section{Rappels sur les polynômes}
\ndlr{cf. AL2}

\definition{
    Un polynôme $P$ est irréductible sur $K$ si $P=QR$ entraine que $Q$ ou $R$ est de degré $0$ (c'est-à-dire une constante).\\
    Cela dépend du corps $K$.
}

\definition{
    $P$ est sciendé sur $K$ si $P$ est produit de polynômes de degré $1$ sur $K$.
}
\example{
    Sur $\mathbb{R}$, $X^2 + 1$ est irréductible. Sur $\mathbb{C}$, $X^2 + 1 = (X - i)(X + i)$ est scindé.
}

\vocabulary{$P$ est scindé à racines simples si $P$ est scindé et si toutes ses racines sont de multiplicité $1$, i.e. $P = c(X - \alpha_1)(X - \alpha_2) \ldots (X - \alpha_n)$ avec $\alpha_i$ distincts.}

\theorem{Théorème d'Alembert-Gauss}{}{true}{
    \begin{enumerate}
        \item Si $K = \mathbb{C}$, tout polynôme de degré $\geq 1$ est scindé. (i.e. $\mathbb{C}$ est un corps algébriquement clos)
        \item Si $K = \mathbb{R}$, tout polynôme de degré $\geq 1$ est produit de polynômes irréductibles de degré $1$ ou $2$.
    \end{enumerate}
}

\section{Polynôme caractéristique d'un endomorphisme}

Soit $E$ un $K$-espace vectoriel de dimension finie $n$ et $u \in End(E)$.\\
\reminder{Soit $\lambda \in K$. C'est une valeur propre de $u$ si et seulement si $det(u - \lambda Id_E) = 0$.}
\example{
    $Mat_B(u) = \begin{pmatrix}
        5 & 1 \\
        2 & 4 \end{pmatrix}$. $E=\mathbb{R}^2$, $K=\mathbb{R}$ et $B$ la base canonique.\\
        On a :
    \[
    det(u - \lambda Id_E) = det\begin{pmatrix}
        5 - \lambda & 1 \\
        2 & 4 - \lambda
    \end{pmatrix} = \lambda^2 - 9\lambda + 18 = (\lambda - 6)(\lambda - 3)
    \]
    Donc les valeurs propres de $u$ sont $3$ et $6$.
}

\theorem{Théorème}{}{true}{
    Posons $P_u(\lambda) = det(\lambda Id_E - u)$. C'est un polynôme de $K[\lambda]$ de degré $n$ appelé \textbf{polynôme caractéristique} de $u$.\\
    Plus précisément, on a $P_u(\lambda) = (-1)^n \lambda^n + (-1)^{n-1} tr(u) \lambda^{n-1} + \ldots + det(u)$.
}
\example{
    Pour $n=2$, on a $P_u(\lambda) = \lambda^2 - tr(u) \lambda + det(u)$.\\
}

\theorem{Corollaire}{}{true}{
    L'endomorphisme $u$ admet au plus $n$ valeurs propres (distinctes).
}

\section{Polynôme caractéristique d'une matrice}

Soit $A \in M_n(K)$ une matrice carrée, soit $\lambda \in K$.
On dit que $\lambda$ est une valeur propre de $A$ s'il existe un vecteur $X \in K^n \setminus \{0\}$ tel que $AX = \lambda X$.
On pose $E_{\lambda}(A) = Ker(A - \lambda I_n)$ le sous-espace propre de $A$ associé à la valeur propre $\lambda$.
Et on pose $P_A(\lambda) = det(\lambda I_n - A)$ le polynôme caractéristique de $A$.

\theorem{Proposition}{}{true}{
    Soit $AB \in M_n(K)$ tel que $B$ sont semblables $A$ (i.e. il existe $P \in GL_n(K)$ tel que $B = P^{-1}AP$).\\
    Alors $A$ et $B$ ont même polynôme caractéristique ($P_A = P_B$) et donc les mêmes valeurs propres.
}

\theorem{Proposition}{}{true}{
    Supposons $A$ triagonalisable (i.e. semblable à une matrice triangulaire).\\
    Alors $P_A$ est scindé sur $K$.
}

\remark{La réciproque est vraie (vu plus tard), $P_A$ scindé $\implies A$ trigonalisable.}
\attention{Retenir que $P_A$ non scindé $\implies A$ non trigonalisable.}

\example{
    Si $K=\mathbb{R}$ et $A = \begin{pmatrix}
        0 & 1 \\
        0 & 0
    \end{pmatrix}$. On a $P_A(\lambda) = \lambda^2$. Mais $A$ n'est pas diagonalisable.
}

\section{Étude des sous-espaces propres (sep)}

Soit $E$ un $K$-espace vectoriel de dimension finie $n$ et $u \in End(E)$.
Soit $F$ un sev de $E$.

\definition{
    On dit que $F$ est \textbf{stable} par $u$ si $u(F) \subset F$.
    Alors $u_{|F} \in End(F)$ est la restriction de $u$ à $F$.
}

\theorem{Proposition}{}{true}{
    Le polynôme caractéristique de $u_{|F}$ divise celui de $u$.
    Autrement dit : $P_{u_{|F}}(\lambda) | P_u(\lambda)$.\\
    Autrement dit : $\exists Q \in K[\lambda], P_u(\lambda) = P_{u_{|F}}(\lambda) Q(\lambda)$.
}

\textit{La preuve utilise le déterminant par blocs.}

\theorem{Proposition}{}{true}{
    Soit $k \leq n$ et soient $A \in M_k(K)$ et $B \in M_{k,n-k}(K)$ et $D \in M_{n-k}(K)$.\\
    On a : $det\begin{pmatrix}
        A & B \\
        0 & D\end{pmatrix} = det(A) det(D)$.
}

\theorem{Proposition}{}{true}{
    Soit $\lambda \in K$. Alors $E_\lambda$ est stable par $u$.
}
En effet, si $u(x) = \lambda x$, alors $u(u(x)) = u(\lambda x) = \lambda u(x)$, donc $u(x) \in E_\lambda$.

\theorem{Corollaire}{}{true}{
    Soit $\lambda \in K$, et soit $X \in E_\lambda \setminus \{0_E\}$.\\
On a que $u_{|E_\lambda}$ est une homothétie de rapport $\lambda$.\\
Donc $P_{u_{|E_\lambda}}(\lambda) = (\lambda - X)^{\dim(E_\lambda)}$\\.
En particulier, $P_{u_{|E_\lambda}}$ divise $P_u$.\\
}

\vocabulary{On appelle la \textbf{multiplicité} de la valeur propre $\lambda$ pour $u$ le plus grand entier $m$ tel que $(\lambda - X)^m$ divise $P_u(X)$. On la note $m_{a}(\lambda)$.}
\vocabulary{On appelle la \textbf{multiplicité géométrique} de la valeur propre $\lambda$ pour $u$ l'entier $\dim(E_\lambda)$. On la note $m_{g}(\lambda)$.}

\theorem{Proposition}{}{true}{
    On a $1 \leq m_{g}(\lambda) \leq m_{a}(\lambda)$ si $\lambda$ est une valeur propre de $u$.
}

\example{
    Soit $A = \begin{pmatrix}
        3 & 0 & 0 \\
        0 & 3 & 1 \\
        0 & 0 & 3
    \end{pmatrix}$. On a $P_A(\lambda) = (3 - \lambda)^3$. Donc $m_a(3) = 3$.\\
    On a $E_3 = Ker(A - 3I_3) = \left\{ \begin{pmatrix}
        x \\
        y \\
        0
    \end{pmatrix} \mid (x, y) \in K^2 \right\}$. Donc $\dim(E_3) = 2$. Donc $m_g(3) = 2$.
}

\theorem{Proposition}{Somme directe des sep}{true}{
    Soit $\lambda_1, \lambda_2, \ldots, \lambda_k$ des valeurs propres distinctes de $u$.\\
    Alors $E_{\lambda_1} + E_{\lambda_2} + \ldots + E_{\lambda_k}$ est une somme directe.\\
    i.e. $\sum_{i=1}^k E_{\lambda_i} = \bigoplus_{i=1}^k E_{\lambda_i}$.\\\\
    Autrement dit, les sous-espaces propres associés à des valeurs propres distinctes sont en somme directe.
}

\theorem{Corollaire}{}{true}{
    Toute famille de vecteurs propres associés à des valeurs propres distinctes est libre.
}
\theorem{Corollaire}{}{true}{
    Si $u$ admet $n$ valeurs propres distinctes (avec $n = \dim(E)$), alors $u$ est diagonalisable.
}
\example{
    Soit $A = \begin{pmatrix}
        1 & 2 & 3 \\
        2 & 4 & 6 \\
        3 & 6 & 9
    \end{pmatrix}$. On a $P_A(\lambda) = -\lambda(\lambda^2 - 15\lambda -18)$ qui est scindé sur $\mathbb{R}$ à racines simples. Les valeurs propres sont $0, \frac{15 + \sqrt{297}}{2}, \frac{15 - \sqrt{297}}{2}$, donc distinctes. Donc $A$ est diagonalisable.
}

\section{Diagonalisabilité}

Soit $E$ un $K$-espace vectoriel de dimension finie $n$ et $u \in End(E)$.
Soit $F$ un sev de $E$.

\theorem{Théorème}{}{true}{
    $u$ est diagonalisable si et seuelemnt si $E = \bigoplus_{\lambda \text{ vp de } u} E_\lambda$.\\\\

    Autrement dit, $u$ est diagonalisable si et seulement si la somme des dimensions des sous-espaces propres de $u$ est égale à $\dim(E)$.\\
}

\theorem{Théorème}{}{true}{
    $u$ est diagonalisable si et seulement si le polynôme caractéristique de $u$ est scindé et on a $m_a(\lambda) = m_g(\lambda)$ pour toute valeur propre $\lambda$ de $u$.
}

\theorem{Proposition}{}{true}{
    Supposons $u$ diagonalisable. Soit $F$ un sev de $E$ stable par $u$.\\
    Alors la restriction $u_{|F} \in \mathcal{L}(F)$ est diagonalisable.
}

\theorem{Lemme}{}{true}{
    On a $F = \bigoplus_{\lambda \text{ vp de } u} (F \cap E_\lambda)$.
}

\theorem{Théorème}{}{true}{
    Soit $v \in End(E)$ tel que $u \circ v = v \circ u$ (i.e. $u$ et $v$ commutent).\\
    Supposons que $u$ et $v$ sont diagonalisables.\\
    Alors il existe une base $B$ de $E$ telle que les matrices de $u$ et $v$ dans cette base sont diagonales. 
}

\vocabulary{On dit que $u$ et $v$ sont \textbf{simultanément diagonalisables}.}

\theorem{Corollaire}{}{true}{
    Soit $v \in End(E)$ tel que $u \circ v = v \circ u$ (i.e. $u$ et $v$ commutent).\\
    Alors toute combinaison linéaire de $u$ et $v$ est diagonalisable.\\
    De plus, $u \circ v$ et $v \circ u$ sont diagonalisables.
}

\section{Trigonalisation}

\theorem{Théorème}{}{true}{
    $u$ est trigonalisable si et seulement si le polynôme caractéristique de $u$ est scindé.
}

\theorem{Corollaire}{}{true}{
    Si $K = \mathbb{C}$, tout endomorphisme de $E$ est trigonalisable.
}

\example{Soit $\theta \in \mathbb{R} \setminus \pi \mathbb{Z}$ et $u$ tel que $Mat_B(u) = \begin{pmatrix}
    \cos \theta & -\sin \theta \\
    \sin \theta & \cos \theta
\end{pmatrix}$ dans la base canonique $B$ de $\mathbb{R}^2$.\\
On a $P_u(\lambda) = \lambda^2 - 2\cos \theta \lambda + 1$. Ce polynôme n'est pas scindé sur $\mathbb{R}$ car ses racines sont $\cos \theta \pm i \sin \theta$. Donc $u$ n'est pas trigonalisable sur $\mathbb{R}$. Cependant, $P_u$ est scindé sur $\mathbb{C}$, donc $u$ est trigonalisable sur $\mathbb{C}$.
}
\end{document}
