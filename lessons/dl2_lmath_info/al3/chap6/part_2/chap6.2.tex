\documentclass{article}

\usepackage[a4paper, left=1.5cm, right=1.5cm, top=2cm, bottom=2cm]{geometry}

\usepackage{../../../../../components/components}

\usepackage{hyperref}
\usepackage{fancyhdr}


% Configuration des en-têtes et pieds de page
\pagestyle{fancy}
\fancyhf{} % reset tout

\fancyhead[L]{DL2 Math-Info}
\fancyhead[C]{Réduction des endomorphismes}
\fancyhead[R]{2025-2026}

\fancyfoot[L]{Ewen Rodrigues de Oliveira}
\fancyfoot[R]{\thepage}

\begin{document}

\docTitle{Chapitre 6.2 : Sous-espaces caractéristiques}

Soit $E$ un $K$-espace vectoriel de dimension finie $n$.\\
Soit $u \in \mathcal{L}(E)$ un endomorphisme de $E$.\\
Soit $\lambda \in K$ une valeur propre de $u$.\\
\reminder{On appelle $m_a(\lambda)$ la multiplicité algébrique de $\lambda$, c'est-à-dire la multiplicité de $\lambda$ comme racine du polynôme caractéristique de $u$.}
\reminder{On appelle $m_g(\lambda)$ la multiplicité géométrique de $\lambda$, c'est-à-dire la dimension du sous-espace propre $E_\lambda = Ker(u - \lambda Id_E)$.}

\definition{
    On pose $N_\lambda = Ker\big((u - \lambda Id_E)^{m_a(\lambda)}\big)$.\\
    Il s'agit du \textbf{sous-espace caractéristique} de $u$ associé à la valeur propre $\lambda$.
}

\theorem{Théorème}{}{true}{
    On suppose $P_u$ scindé (i.e. trigonalisable).\\
    Alors :
    \begin{enumerate}
        \item $N_\lambda$ est un sev stable par $u$ de dimension $m_a(\lambda)$.
        \item $E_\lambda \subset N_\lambda$.
        \item $E = \bigoplus_{\lambda \in Sp(u)} N_\lambda$.
        \item Soit $\pi_\lambda$ la projection de $E$ sur $N_\lambda$ parallèlement à $\bigoplus_{\mu \in Sp(u), \mu \neq \lambda} N_\mu$.\\
        Alors $\pi_\lambda \in K[u]$.
        \item Si $\lambda \neq \mu$, $\pi_\lambda \circ \pi_\mu = 0$
    \end{enumerate}
}

\example{
    $Mat(u) = \begin{pmatrix}
        1 & 1 & 0\\
        0 & 1 & 0\\
        0 & 0 & 2
    \end{pmatrix}$. Les valeurs propres sont $1$ et $2$.
    $P_u(X) = -(X-1)^2(X-2)$.\\
    Donc :
    \begin{itemize}
        \item $\lambda = 2$, $m_a(2)=m_g(1)=1$ donc $E_2 = Ke_3 = N_2$.
        \item $\lambda = 1$, $m_a(1)=2$ et $m_g(1)=1$ donc $E_1 = Ke_1$ et $N_1 = Ke_1 \oplus Ke_2$.
    \end{itemize}
}

\theorem{Théorème}{}{true}{
    $N_\lambda$ est stable par $u$ et posons $u_\lambda = u_{|N_\lambda}$.\\
    Alors : 
    \begin{enumerate}
        \item $u_\lambda$ a une seule valeur propre qui est $\lambda$.
        \item On a $P_u = \pm (X - \lambda)^{m_a(\lambda)}$.
        \item On a $dim N_\lambda = m_a(\lambda)$.
        \item $\exists B_\lambda$ une base de $N_\lambda$ telle que $Mat_{B_\lambda}(u_\lambda) = \lambda I_{m_a(\lambda)}$.
    \end{enumerate}
}

\theorem{Corollaire}{}{true}{
    On suppose $P_u$ scindé.\\
    Il existe une base $B$ de trigonalisation de $u$ telle que la $Mat_B(u)$ est de la forme suivante :\\
% Forme bloc-triangulaire : blocs diagonaux A_{lambda_k}
\[
\mathrm{Mat}_B(u) =
\begin{pmatrix}
A_{\lambda_1} & 0            & \cdots & 0 \\
0             & A_{\lambda_2} & \ddots & \vdots \\
\vdots        & \ddots       & \ddots & 0 \\
0             & \cdots       & 0      & A_{\lambda_k}
\end{pmatrix},
\]
\[
\text{avec}\qquad
A_{\lambda_i} =
\begin{pmatrix}
\lambda_i & \ast      & \ast & \cdots & \ast\\
0         & \lambda_i & \ast & \cdots & \ast\\
\vdots    & \ddots    & \ddots & \ddots & \vdots\\
0         & \cdots    & 0      & \lambda_i & \ast\\
0         & \cdots    & \cdots & 0      & \lambda_i
\end{pmatrix}
\]
}

\example{
    $E = \mathbb{R}^3$ et $u$ tel que $Mat(u) = \begin{pmatrix}
        2 & 1 & 3\\
        5 & 3 & 6\\
        -2 & -1 & -2\end{pmatrix}$.\\
    $P_u(X) = - (X-1)^3$.\\
    Donc $1$ est la seule valeur propre de $u$ avec $m_a(1)=3$.\\
    Or $u$ n'est pas diagonalisable car si $u$ diagonalisable dans $B$, on a $Mat_B(u) = I_3$ .
    Donc $u = Id_E$. Absurde.\\\\
    $u$ est scindé, donc trigonalisable.
}


\section{Nilpotence}

\definition{
    On dit que $u \in End(E)$ est \textbf{nilpotent} s'il existe $k \in \mathbb{N}^*$ tel que $u^k = 0_{End(E)}$.
}
\vocabulary{On appelle l'indice de nilpotence le plus petit entier $k$ tel que $u^k = 0$.}

\theorem{Proposition}{}{true}{
    L'endomorphisme $u$ est nilpotent si et seulement si $P_u$ est scindé avec pour seule racine $0$ (c'est à dire que sa seule valeur propre est $0$).\\
    En particulier, $u$ est trigonalisable strictement avec des 0 sur la diagonale. On a que l'indice de nilpotence de $u$ est inférieur ou égal à $dim(E)$.
}
\theorem{Corollaire}{}{true}{
    $u$ nilpotent et diagonalisable $\Rightarrow u = 0_{End(E)}$.
}

\theorem{Proposition}{}{true}{
    Si $u$ est nilpotent d'indice $k$ et $x \in E$ avec $u^{k-1}(x) \neq 0$, alors la famille $\{x, u(x), u^2(x), \ldots, u^{k-1}(x)\}$ est libre.\\
    Ainsi, on a $k \leq dim(E)$.
}

\theorem{Proposition}{}{true}{
    Si $u$ est nilpotent d'indice $n = dim(E)$, alors il existe une base $B$ de $E$ telle que :
    \[
    Mat_B(u) =
    \begin{pmatrix}
        0 & 1 & 0 & \cdots & 0\\
        0 & 0 & 1 & \cdots & 0\\
        \vdots & \vdots & \ddots & \ddots & \vdots\\
        0 & 0 & \cdots & 0 & 1\\
        0 & 0 & \cdots & 0 & 0
    \end{pmatrix}
    \]
}

\remark{
    La matrice ci-dessus est un cas particulier de forme de Jordan.
}
\remark{\textbf{Comment trouver une base de trigonalisation de $u$ lorsque $u$ est nilpotent d'indice $k \leq n=dim(E)$ ?}\\
    On a $Ker(u) \subset Ker(u^2) \subset \cdots \subset Ker(u^{k-1}) \subsetneq Ker(u^k) = E$.\\
    On considère $B_1$ une base de $Ker(u)$, que l'on complète en une base $B_2$ de $Ker(u^2)$, que l'on complète en une base $B_3$ de $Ker(u^3)$, et ainsi de suite jusqu'à obtenir une base $B_k$ de $E$.\\
    On a ainsi construit une base $B_k$ de $E$ telle que $Mat_{B_k}(u)$ est triangulaire supérieure avec des 0 sur la diagonale.\\
}

\theorem{Proposition}{Combinaison linéaire}{true}{
    Soient $u,v \in End(E)$ deux endomorphismes nilpotents qui commutent.\\
    Alors toute combinaison linéaire $au + bv$ avec $a,b \in K$ est nilpotente.
}

\section{Décomposition de Jordan Chévalley (ou Dunford)}

\theorem{Théorème}{}{true}{
    Soit $u \in End(E)$ tel que $P_u$ est scindé.\\
    Il existe un unique $(d,e) \in End(E)^2$ tel que :
    \begin{itemize}
        \item $u = d + e$,
        \item $d$ est diagonalisable,
        \item $e$ est nilpotent,
        \item $d$ et $e$ commutent.
        \item $d,e \in K[u]$ (polynômes en $u$).
    \end{itemize}.\\
    C'est la \textbf{décomposition de Jordan Chévalley} (ou Dunford) de $u$.
}

\definition{
    Soit $A \in M_n(K)$.\\
    On dit que $A$ est nilpotente si il existe $k \geq 1$ tel que $A^k = 0$ (matrice nulle).
}

\theorem{Théorème}{Décomposition de Jordan Chévalley matricielle}{true}{
    Soit $A \in M_n(K)$ tel que son polynôme caractéristique est scindé.\\
    Il existe un unique couple $(D,N) \in M_n(K)^2$ tel que :
    \begin{itemize}
        \item $A = D + N$,
        \item $D$ est diagonalisable,
        \item $N$ est nilpotente,
        \item $D$ et $N$ commutent,
        \item $D,N$ sont des polynômes en $A$.
    \end{itemize}
}

\example{
    Soit $A = \begin{pmatrix}
        3 & -1 & 1\\
        2 & 0 & 1\\
        1 & -1 & 2
    \end{pmatrix}$.\\
    On calcule facilement que :
    \begin{itemize}
        \item $P_A(X) = -(X-2)^2(X-1)$,
        \item $m_a(2) = 2$, $m_g(2) = 1$,
        \item $m_a(1) = m_g(1) = 1$.
        \item $E_2 = Ker(A - 2I_3) = Vect\{(1,1,0)^T\}$,
        \item $N_1 = E_1 = = Ker(A - I_3) = Vect\{(0,1,1)^T\}$.
        \item $N_2 = Ker\big((A - 2I_3)^2\big) = Vect\{(1,1,0)^T, (0,0,1)^T\}$.
    \end{itemize}
    On a donc $N_1 + N_2 = \mathbb{R}^3$. (car $dim(N_1) + dim(N_2) = 3$ et $N_1 \cap N_2 = \{0\}$).\\
    Donc $N_1 \oplus N_2 = \mathbb{R}^3$.\\
    Par le théorème de Jordan Chévalley, il existe un unique $(D,N)$ tel que $A = D + N$ avec $D$ diagonalisable, $N$ nilpotente et $DN = ND$.\\
    Or :
    \begin{itemize}
        \item $d(e_1) = 2e_1 + e_2 + e_3$
        \item $d(e_2) = e_2 - e_2$
        \item $d(e_3) = 2e_3$
    \end{itemize}
    Donc $D = Mat_{can}d = \begin{pmatrix}
        2 & 1 & 0\\
        1 & 1 & 1\\
        1 & 0 & 1
    \end{pmatrix}$ et $N = A - D = \begin{pmatrix}
        1 & -1 & 1\\
        1 & -1 & 1\\
        0 & 0 & 0
    \end{pmatrix}$.\\
    De plus, $N^2 = 0_3$. Donc $N$ est nilpotente d'indice $2$.
    Donc : $A = \begin{pmatrix}
        2 & 1 & 0\\
        1 & 1 & 1\\
        1 & 0 & 1
    \end{pmatrix} + \begin{pmatrix}
        1 & -1 & 1\\
        1 & -1 & 1\\
        0 & 0 & 0
    \end{pmatrix}$ : décomposition de Jordan Chévalley de $A$.\\\\
    \textbf{Pour calculer $A^n$ :}\\
    On utilise le fait que $D$ et $N$ commutent, le binôme de Newton donc et la nilpotence de $N$ :
    \[A^n = (D + N)^n = \sum_{k=0}^{n} \binom{n}{k} D^{n-k}N^k = D^n + nD^{n-1}N\] (tous les termes avec $k \geq 2$ sont nuls car $N^2 = 0$).
}


\end{document}
