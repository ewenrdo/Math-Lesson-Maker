\documentclass{article}

\usepackage[a4paper, left=1.5cm, right=1.5cm, top=2cm, bottom=2cm]{geometry}

\usepackage{../../../../components/components} % <-- ton fichier .sty, avec toutes tes définitions

\usepackage{hyperref}
\usepackage{fancyhdr}


% Configuration des en-têtes et pieds de page
\pagestyle{fancy}
\fancyhf{} % reset tout

\fancyhead[L]{DL2 Math-Info}
\fancyhead[C]{Réduction des endomorphismes}
\fancyhead[R]{2025-2026}

\fancyfoot[L]{Ewen Rodrigues de Oliveira}
\fancyfoot[R]{\thepage}

\begin{document}

\docTitle{Chapitre 6.2 : Sous-espaces caractéristiques}

Soit $E$ un $K$-espace vectoriel de dimension finie $n$.\\
Soit $u \in \mathcal{L}(E)$ un endomorphisme de $E$.\\
Soit $\lambda \in K$ une valeur propre de $u$.\\
\reminder{On appelle $m_a(\lambda)$ la multiplicité algébrique de $\lambda$, c'est-à-dire la multiplicité de $\lambda$ comme racine du polynôme caractéristique de $u$.}
\reminder{On appelle $m_g(\lambda)$ la multiplicité géométrique de $\lambda$, c'est-à-dire la dimension du sous-espace propre $E_\lambda = Ker(u - \lambda Id_E)$.}

\definition{
    On pose $N_\lambda = Ker\big((u - \lambda Id_E)^{m_a(\lambda)}\big)$.\\
    Il s'agit du \textbf{sous-espace caractéristique} de $u$ associé à la valeur propre $\lambda$.
}

\theorem{Théorème}{}{true}{
    On suppose $P_u$ scindé (i.e. trigonalisable).\\
    Alors :
    \begin{enumerate}
        \item $N_\lambda$ est un sev stable par $u$ de dimension $m_a(\lambda)$.
        \item $E_\lambda \subset N_\lambda$.
        \item $E = \bigoplus_{\lambda \in Sp(u)} N_\lambda$.
        \item Soit $\pi_\lambda$ la projection de $E$ sur $N_\lambda$ parallèlement à $\bigoplus_{\mu \in Sp(u), \mu \neq \lambda} N_\mu$.\\
        Alors $\pi_\lambda \in K[u]$.
        \item Si $\lambda \neq \mu$, $\pi_\lambda \circ \pi_\mu = 0$
    \end{enumerate}
}

\example{
    $Mat(u) = \begin{pmatrix}
        1 & 1 & 0\\
        0 & 1 & 0\\
        0 & 0 & 2
    \end{pmatrix}$. Les valeurs propres sont $1$ et $2$.
    $P_u(X) = -(X-1)^2(X-2)$.\\
    Donc :
    \begin{itemize}
        \item $\lambda = 2$, $m_a(2)=m_g(1)=1$ donc $E_2 = Ke_3 = N_2$.
        \item $\lambda = 1$, $m_a(1)=2$ et $m_g(1)=1$ donc $E_1 = Ke_1$ et $N_1 = Ke_1 \oplus Ke_2$.
    \end{itemize}
}

\theorem{Théorème}{}{true}{
    $N_\lambda$ est stable par $u$ et posons $u_\lambda = u_{|N_\lambda}$.\\
    Alors : 
    \begin{enumerate}
        \item $u_\lambda$ a une seule valeur propre qui est $\lambda$.
        \item On a $P_u = \pm (X - \lambda)^{m_a(\lambda)}$.
        \item On a $dim N_\lambda = m_a(\lambda)$.
        \item $\exists B_\lambda$ une base de $N_\lambda$ telle que $Mat_{B_\lambda}(u_\lambda) = \lambda I_{m_a(\lambda)}$.
    \end{enumerate}
}

\theorem{Corollaire}{}{true}{
    On suppose $P_u$ scindé.\\
    Il existe une base $B$ de trigonalisation de $u$ telle que la $Mat_B(u)$ est de la forme suivante :\\
% Forme bloc-triangulaire : blocs diagonaux A_{lambda_k}
\[
\mathrm{Mat}_B(u) =
\begin{pmatrix}
A_{\lambda_1} & 0            & \cdots & 0 \\
0             & A_{\lambda_2} & \ddots & \vdots \\
\vdots        & \ddots       & \ddots & 0 \\
0             & \cdots       & 0      & A_{\lambda_k}
\end{pmatrix},
\]
\[
\text{avec}\qquad
A_{\lambda_i} =
\begin{pmatrix}
\lambda_i & \ast      & \ast & \cdots & \ast\\
0         & \lambda_i & \ast & \cdots & \ast\\
\vdots    & \ddots    & \ddots & \ddots & \vdots\\
0         & \cdots    & 0      & \lambda_i & \ast\\
0         & \cdots    & \cdots & 0      & \lambda_i
\end{pmatrix}
\]
}

\example{
    $E = \mathbb{R}^3$ et $u$ tel que $Mat(u) = \begin{pmatrix}
        2 & 1 & 3\\
        5 & 3 & 6\\
        -2 & -1 & -2\end{pmatrix}$.\\
    $P_u(X) = - (X-1)^3$.\\
    Donc $1$ est la seule valeur propre de $u$ avec $m_a(1)=3$.\\
    Or $u$ n'est pas diagonalisable car si $u$ diagonalisable dans $B$, on a $Mat_B(u) = I_3$ .
    Donc $u = Id_E$. Absurde.\\\\
    $u$ est scindé, donc trigonalisable.
}

\end{document}
