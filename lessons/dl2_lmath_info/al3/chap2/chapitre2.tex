\documentclass{article}

\usepackage[a4paper, left=1.5cm, right=1.5cm, top=2cm, bottom=2cm]{geometry}

\usepackage{../../../../components/components} % <-- ton fichier .sty, avec toutes tes définitions

\usepackage{fancyhdr}


% Configuration des en-têtes et pieds de page
\pagestyle{fancy}
\fancyhf{} % reset tout

\fancyhead[L]{DL2 Math-Info AL3}
\fancyhead[C]{Théorie des groupes}
\fancyhead[R]{2025-2026}

\fancyfoot[L]{Ewen Rodrigues de Oliveira}
\fancyfoot[R]{\thepage}

\begin{document}

\docTitle{Chapitre 2 : Groupes symétriques}

\section{Permutations}

\definition{Soient $X$ un ensemble et $S(X)$ l'ensemble des bijections de $X$ dans $X$.\\
On appelle permutation de $X$ tout élément de $S(X)$.\\\\

Intuitivement, c'est l'ensemble des réarrangements des éléments de $X$.}


\theorem{Propriété}{Ensemble des permutations}{true}{
    $(S(X), \circ)$ est un groupe \textit{(en général non commutatif)}.
}
\vocabulary{C'est le groupe symétrique sur $X$.}
\noindent{
    \textbf{Démonstration :}\\
    \begin{itemize}
        \item  La composée de deux bijections est une bijection, donc $\circ$ est une loi interne sur $S(X)$.
        \item La loi $\circ$ est associative.
        \item L'élément neutre est l'identité $id_X$.
        \item L'inverse d'une bijection est une bijection (la bijection réciproque). $\Box$ 
    \end{itemize}
}

\theorem{Proposition}{}{false}{
    Soit Y un ensemble avec une bijection $b : X \rightarrow Y$.\\
    L'application $\varphi_b : S(X) \rightarrow S(Y)$ définie par $\sigma \mapsto b \circ \sigma \circ b^{-1}$ est un isomorphisme de groupe.
}

\ndlr{À quoi ça sert ? Permet de montrer que le groupe symétrique ne dépend pas de l'ensemble, mais seulement de son cardinal.}

\remark{Donc $S(Y)$ est isomorphe à $S(X)$.}
\noindent{
    \textbf{Démonstration :}\\
    $\varphi_b$ est bien définie : comme $b$ et $\sigma$ sont bijectives, $b\circ\sigma\circ b^{-1}$ est bijective.\\
    $\varphi_b$ est un morphisme $\forall \sigma, \sigma' \in S(X)$. On a :\\
    \[
        \varphi_b(\sigma\circ\sigma') = b \circ (\sigma \circ \sigma') \circ b^{-1}\\
                                    = b \circ \sigma \circ b^{-1}\circ b \circ \sigma' \circ b^{-1}\\
                                    = (b\circ \sigma \circ b^{-1}) \circ (b\circ \sigma' \circ b^{-1})
                                    = \varphi_b(\sigma) \circ  \varphi_b(\sigma')
    \]
    $\varphi_b$ est bijective car sa réciproque est donnée par $\tau = b^{-1} \circ \tau \circ b$. $\Box$
}

\definition{
Supposons X fini de cardinal n.\\
Il existe une bijection $\{1,2,..,n\} \rightarrow X$ (numérotation de X).\\
On prend $S_n = S({1,2,...n})$ : c'est le \textbf{groupe symétrique sur n lettres}. Il est isomorphe à $S(X)$}

Notation par tableau : $\sigma$
\[
\begin{array}{c|ccccc}
i & 1 & 2 & 3 & \cdots & n \\
\hline
\sigma(i) & \sigma(1) & \sigma(2) & \sigma(3) & \cdots & \sigma(n)
\end{array}
\]

\definition{Soit $\sigma \in S(X)$.\\
Le support de $\sigma$ est $\{x \in X \mid \sigma(x) \neq x\}$.\\\\
Intuitivement, c'est l'ensemble des éléments de $X$ que $\sigma$ "déplace".}


\example{
    Prenons $S(X)=S_6$.\\
\[
\begin{array}{c|cccccc}
i & 1 & 2 & 3 & 4 & 5 & 6 \\
\hline
\sigma(i) & 3 & 2 & 1 & 6 & 5 & 4
\end{array}
\]
    $\sigma$ a pour support $\{1,3,4,6\}$.
}

\theorem{Proposition}{}{false}{
    Soient $\sigma, \sigma' \in S(X)$ de supports disjoints.\\
    Alors $\sigma$ et $\sigma'$ commutent, \textit{i.e.} $\sigma \circ \sigma' = \sigma' \circ \sigma$
}

\noindent{
    \textbf{Démonstration :}\\
    Soient S et S' les supports de $\sigma$ et $\sigma'$.
    On a $\sigma \circ \sigma'(x) = \sigma'(\sigma(x)) = \sigma'(x)$.\\
    On a $\sigma'(x) \notin S$, sinon $\sigma'(x) \notin S'$ et $\sigma'(\sigma'(x)) = \sigma'(x)$\\
    donc $\sigma'(x)=x$, donc $\sigma'(x) \notin S$.\\
    Donc $\sigma\circ\sigma'(x) = \sigma'(x) = \sigma'\circ\sigma(x)$.\\\\
    De même, si $x \in X-S'$, on a : $\sigma\circ\sigma'(x) = \sigma'\circ\sigma(x)$.\\
    Comme $S \cap S' = \emptyset$, on a : $\sigma\circ\sigma'(x)=\sigma'\circ\sigma(x)$ $\forall x \in X$. $\Box$
   }   

\theorem{Propriété}{Ordre de $S_n$}{false}{
    Le groupe $S_n$ est d'ordre $n!$.
}
\noindent{
    \textbf{Démonstration :}\\
    Soient X,Y deux ensembles à $n$ éléments.\\
    Montrons que $\#\{bijections X \rightarrow Y\} = n!$.\\
    En effet, si $X={x_1,...,x_n}$ et $f : X\rightarrow Y$ est une bijection, il y a :
    \begin{itemize}
        \item $n$ possibilités pour $f(x_1)$
        \item $n-1$ possibilités pour $f(x_2)$
        \\$\vdots$
        \item 1 possibilité pour $f(x_n)$
    \end{itemize}
}

\section{Cycles}

\definition{
    Soit $X$ un ensemble et soit $k \geq 2$ un entier.\\
    Un $k$-cycle de $S(X)$ est donné par $a_1,a_2,\ldots,a_k \in X \mid a_i \neq a_j si i \neq j$.\\\\
    et $\sigma(a_i) = a_{i+1}$ pour $1 \leq i < k$ et $\sigma(a_k) = a_1$ et $\sigma$ de support ${a_1, a_2, ...n a_k}$.\\
    On le note $(a_1 \cdots a_k)$.
}

\example{
    Soit $X = \{1,2,3,4,5\}$.\\
    Alors $(1\ 3\ 4)$ est un 3-cycle de $S(X)$ défini par : 
    \[
    \begin{array}{c|ccccc}
    i & 1 & 2 & 3 & 4 & 5 \\
    \hline
    \sigma(i) & 3 & 2 & 4 & 1 & 5
    \end{array}
    \]
}

\remark{
    Un k-cycle est juste une notation compacte pour une permutation, par exemple :\\
    $(a_1 a_2 \cdots a_k)$ est la permutation $\sigma$ définie par :\\
    $\sigma(a_1) = a_2, \sigma(a_2) = a_3, \ldots, \sigma(a_{k-1}) = a_k, \sigma(a_k) = a_1$ et $\sigma(x) = x$ si $x \notin \{a_1, a_2, \ldots, a_k\}$.
}

\attention{La notation n'est pas unique : $(a_{i}a_{i+1}\cdots a_{k}a_{1}a_{2} \cdots a_{i-1}) = (a_1 \cdots a_k)$}

\vocabulary{On dit qu'une permutation $c$ est un cycle s'il existe $k\geq2 \mid c$ est un k-cycle. Alors $k$ s'appelle la \textbf{longueur} de $c$.}

\theorem{Proposition}{}{false}{
    Comme élément du groupe $S(X)$ un k-cycle $c$ est d'ordre k.
}
\noindent{
    \textbf{Démonstration :}\\
    Posons $c = (a_1 \cdots a_k)$.\\
    On a $\varepsilon(a_1)=a_{1+j}\neq a_{1}$.\\
    Donc $ordre(c)\geq k$. On a $c^k(a_i)=a_i \forall i$, donc $c$ est d'ordre $k$. $\Box$
}

\remark{\textbf{Rappel}\\
Des cycles à supports disjoints communtent.\\
Soient $c = (a_1 \cdots a_k)$ et $c' = (a'_1 \cdots a'_{k'})$ deux cycles de $S(X)$ tels que $S(c) \cap S(c') = \emptyset$.\\
avec ${a_1, \ldots, a_k} \cap  {a'_1, \ldots, a'_{k'}} = \emptyset$.\\
On a $c \circ c' = c' \circ c$
}

\definition{
    Soit $x \in X$, l'\textbf{orbite de $x$ sous $\sigma$} est $\{\sigma^m(x) \mid m\in Z\}$.
}

\remark{
On a $x \notin Support(\sigma)$ si $\sigma(x)=x \Leftrightarrow$ orbite de $x$ est un singleton.\\
Si $\sigma$ est un $k$-cycle de support $S$ et $x\in S$, l'orbite de $x$ a $k$ éléments, c'est $S$.

}

\theorem{Théorème}{}{false}{
    Si $X$ est fini, tout élément de $S(X)$ s'écrit comme produit de cycles à supports disjoints.\\
    Cette écriture est unique à l'ordre des facteurs près.
}
\noindent{
    \textbf{Démonstration :}\\
    \begin{itemize}
        \item \textbf{Existence :} (par récurrence)\\
        {
            Si $Support(\sigma) = \emptyset$, on a $\sigma = id_X$ : c'est bien un produit (vide) de cycles.\\
            Supposons maintenant que $Support(\sigma) \neq \emptyset$. Soit $x\in Support(\sigma)$.\\
            Soit $\sigma' \in S(X)$ donnée par $\sigma'(y)=\sigma(y)$ si $y \notin$ orbite de $x$, $\sigma'(y)=y$ sinon.\\
            Considérons le cycle $c$ donné par : $(x \sigma(x)\sigma^2(x)-\sigma^k(x))$ avec $k=min\{m \mid \sigma^m(x)=x\}$.\\
            C'est un $k$-cycle de support l'orbite de $x$.\\
            Si $y \in$ orbite de $x$ on a $\sigma(y)=c(y)$.\\
            Alors $\sigma$ et $c$ sont de supports disjoints et on a : $\sigma = \sigma' c = c \sigma'$.\\
            En effet, soit $y\in X$,\\
            $y \notin$ orbite de $x$ on a $\sigma'(y)=c(y)$
        }
    \end{itemize}
}
\attention{Démonstration non terminée \textit{(le prof n'écrivait pas clair au tableau)}}

\example{Soit $X = \{1,2,3,4,5\}$ et $\sigma \in S(X)$ défini par :
\[
\sigma(1)=3, \quad \sigma(2)=5, \quad \sigma(3)=1, \quad \sigma(4)=4, \quad \sigma(5)=2
\]
Alors $\sigma$ s'écrit comme produit de cycles à supports disjoints :
\[
\sigma = (1\ 3)(2\ 5)
\]
Cette écriture est unique à l'ordre des facteurs près.}

\section{Signature}

\definition{
    Soit $X$ un ensemble fini et notons $S(X)$ le groupe symétrique sur $X$.\\
    Posons $Z = \{(i,j) \mid i,j \in X, i \neq j\}$.\\
    Soit $R$ la relation sur $Z$ donnée par : $(i,j) R (i',j') \Leftrightarrow (i,j) = (i',j')$ ou $(i,j) = (j',i')$. \textit(i.e. $\{i,j\} = \{i',j'\}$\textit).\\
    C'est une relation d'équivalence. Soit $S$ un système de représentants de $R$.\\\\
    Soit $\sigma \in S(X)$.\\
    Alors si $(i,j) \in Z$, on a $(\sigma(i),\sigma(j)) \in Z$.\\
    De plus, $(i,j) \mapsto (\sigma(i),\sigma(j))$ est une bijection de $Z$ notée $\sigma^2$.\\\\
    
    Soit $(i,j)\in S$.\\
    On dit qu'on a une \textbf{inversion} en $(i,j)$ pour $\sigma$ si $(\sigma(i),\sigma(j)) \notin S$.\\
}

\example{
    Si $X={1,2,\ldots,n}$, on peut prendre $S = \{(i,j) \in X^2 \mid i < j\}$.\\
    Alors $(i,j) \in S$ est une inversion pour $\sigma \Leftrightarrow \sigma(j) < \sigma(i)$.
}

\theorem{Propriété}{Signature}{false}{
On pose $\varepsilon_S(\sigma) = (-1)^{\#\{\text{inversions de $\sigma$}\}} \in \{-1,1\}$.
On a $\varepsilon_S(\sigma)$ ne dépend pas du choix de $S$.

On le note $\varepsilon(\sigma)$ et on l'appelle la \textbf{signature} de $\sigma$.
}

\noindent{
    \textbf{Démonstration :}\\
    Soit $(i_0, j_0) \in S$. Posons $S' = S - \{(i_0, j_0)\} \cup \{(j_0, i_0)\}$.\\
    Si $(i,j) \neq (i_0, j_0)$ et $(i,j)\neq(j_0,i_0)$, on a $(i,j) \in S$ est une inversion pour $S \Leftrightarrow (i,j) \in S'$ est une inversion pour $S'$.\\
    Si $(i,j) = (i_0, j_0)$, on a $(i,j) \in S\backslash S'$ et $(j,i) \in S'\backslash S$.\\
    On a une inversion $(i_0,j_0)$ pour $S \Leftrightarrow$ on a une inversion $(j_0,i_0)$ pour $S'$.\\
    Donc $\#\{\text{inversions de $\sigma$ pour $S$}\} \equiv \#\{\text{inversions de $\sigma$ pour $S'$}\}$.\\
    Donc $\varepsilon_S(\sigma) = \varepsilon_{S'}(\sigma)$ de proche en proche on a $\varepsilon_S$ indépendant de $S$. $\Box$
}

\theorem{Proposition}{}{false}{
    Soit \functionSets{f}{X \rightarrow Y} injective.\\
    On a : \[
    \varepsilon(\sigma) = \prod_{(i,j) \in S} \frac{f(\sigma(j))-f(\sigma(i))}{f(j)-f(i)}
    \]
}

\example{
    Si $X = \{1,2,\ldots,n\}$ et $f = id_X$, on a : \[
    \varepsilon(\sigma) = \prod_{1 \leq i < j \leq n} \frac{\sigma(j)-\sigma(i)}{j-i}
    \]
}

\noindent{
    \textbf{Démonstration :}\\
    On a $(\sigma(i),\sigma(j)) \in S \Leftrightarrow (i,j)$ est une inversion.\\
    Sinon, on a $(\sigma(j),\sigma(i)) \in S$.\\
    Donc : \[
    \prod_{(i,j) \in S} \frac{f(\sigma(j))-f(\sigma(i))}{f(j)-f(i)} = \prod_{\substack{(i,j) \in S \\ \text{pas une inversion}}} \frac{f(\sigma(j))-f(\sigma(i))}{f(j)-f(i)} \times \prod_{\substack{(i,j) \in S \\ \text{inversion}}} \frac{f(\sigma(j))-f(\sigma(i))}{f(j)-f(i)} \]\[
    = \prod_{\substack{(i,j) \in S}} \frac{1}{f(j)-f(i)} \times \prod_{\substack{(i,j) \in S \\ \text{inversion}}} (f(\sigma(j))-f(\sigma(i))) \times \prod_{\substack{(i,j) \in S \\ \text{pas une inversion}}} (f(\sigma(j))-f(\sigma(i)))\\
    \]
    Si $S=\{(\sigma(i), \sigma(j)) \mid \sigma \text{ pas une inversion sur } S\cup \{(\sigma(j), \sigma(i)) \mid \sigma \text{ inversion sur } S\}\}$.\\
    Donc : 
    %% \prod_{(i,j) \in S} \frac{1}{f(j)-f(i)} = \prod_{\substack{(i,j) \in S\\ \text{\sigma pas inversion}}} (f(\sigma(j))-f(\sigma(i))) \times \prod_{\substack{(i,j) \in S \\ \text{\sigma inversion}}} (f(\sigma(j))-f(\sigma(i)))\\
    \attention{Démonstration non terminée \textit{(le prof n'écrivait pas clair au tableau et c'était verbeux)}}
}

\theorem{Théorème}{Signature et morphisme}{false}{
    La signature est un morphisme de groupe de $S(X)$ dans $\{-1,1\}$.\\
    \textit{i.e.} $\forall \sigma, \varrho \in S(X)$, on a : $\varepsilon(\sigma \circ \varrho) = \varepsilon(\sigma) \times \varepsilon(\varrho)$.
}

\noindent{
    \textbf{Démonstration :}\\
    \theorem{Lemme}{\textit{pour démontrer le théorème}}{false}{
        On a $\sigma^2(S) = \{(\sigma(i),\sigma(j)) \mid (i,j) \in S\}$ est un système de représentants de $R$.
    }

    \ndlr{Demander la démonstration à Laurent}
    
}

\definition{
    Soit $\sigma\in S(X)$.\\
    Si $\varepsilon(\sigma) = 1$, on dit que $\sigma$ est une \textbf{paire}.\\
    Si $\varepsilon(\sigma) = -1$, on dit que $\sigma$ est une \textbf{impaire}.
}

\theorem{Proposition}{}{false}{
    On pose $\mathcal{A}(X) = \{\sigma \in S(X) \mid \varepsilon(\sigma) = 1\}$.\\
    C'est un sous-groupe de $S(X)$ appelé le \textbf{groupe alterné} sur $X$.\\\\

    En particulier, si $X = \{1,2,\ldots,n\}$, on le note $\mathcal{A}_n$. (groupe alterné sur n lettres)\\
    Et on a $\#\mathcal{A}_n = ord(\mathcal{A}_n) = \frac{n!}{2}$ pour $n\geq2$.
}
\noindent{
    \textbf{Démonstration :}\\
    On a $\mathcal{A}(X) = \{\sigma \in S(X) \mid \varepsilon(\sigma) = 1\} = Ker(\varepsilon)$, donc c'est un sous-groupe de $S(X)$.\\
    Supposons que $\mathcal{A}_n$ a un élément $\tau$ de signature -1, c'est vrai si $n\geq2$.\\
    Alors $S(X) = \mathcal{A}(X) \cup \tau \mathcal{A}(X)$ et $\mathcal{A}(X) \cap \tau \mathcal{A}(X) = \emptyset$.\\
    En effet, si $\sigma \in \mathcal{A}(X)$ OK.
    Sinon si $\sigma \notin \mathcal{A}(X)$, on a $\varepsilon(\sigma) = -1$ et donc $\varepsilon(\sigma\tau^{-1}) = -1 \times -1 = 1$, donc $\sigma\tau^{-1} \in \mathcal{A}(X)$ et $\sigma \in \tau \mathcal{A}(X)$.\\
    On a $\mathcal{A}(X) \cap \tau \mathcal{A}(X) = \emptyset$.\\
    On a une bijection \function{}{\mathcal{A}(X) \rightarrow \tau \mathcal{A}(X)}{\sigma \mapsto \tau\sigma}.\\
    Donc $\#\mathcal{A}(X) = \#\tau \mathcal{A}(X)$ et $\#S(X) = 2\#\mathcal{A}(X)$.\\
    Donc $\#\mathcal{A}(X) = \frac{\#S(X)}{2}$. Donc $\#\mathcal{A}_n = \frac{n!}{2}$ pour $n\geq2$.
    $\Box$
}

\section{Transpositions}

\definition{
    Une transposition de $X$ est un 2-cycle. On la note $(a\ b)$.
}

\theorem{Propriété}{Transpositions et signature}{false}{
    Soit $\sigma \in S(X)$.\\
    On a $\varepsilon(\sigma) = (-1)$. (une transposition existe ssi $\#X \geq 2$)
}

\noindent{
    \textbf{Démonstration :}\\
    Soit $\sigma = (a\ b)$ avec $(a,b) \in S$.\\
    Soit \functionSets{f}{X \rightarrow \mathbb{R}} et $f(a), f(b) : f(b) > f(c) > f(a)$ si $c \neq a,b$.\\
    On a : $\varepsilon(\sigma) = \ldots$\\
    \ndlr{Il y a une erreur dans la démonstration du prof, il écrit n'importe quoi au tableau}
}

\theorem{Formulaire}{}{false}{
    Soit $c = (a_1 \cdots a_k)$ un $l$-cycle.\\
    \begin{enumerate}
    \item On a $c = (a_1 a_2)(a_2 a_3) \cdots (a_{k-1} a_k)$ \textit{i.e. c'est un produit de transpositions}.
    \item Soit $\sigma \in S(X)$.\\
    On a $\sigma c \sigma^{-1} = (\sigma(a_1) \cdots \sigma(a_k))$ (formule de conjugaison).
    \item Soient $c_1,\ldots,c_k$ des cycles.\\
    On a $\sigma c_1 \cdots c_k \sigma^{-1} = (\sigma c_1 \sigma^{-1}) \cdots (\sigma c_k \sigma^{-1})$.
    \end{enumerate}
}

\ndlr{Démonstration laissée à l'appréciation du lecteur, elle n'était pas à mon appréciation}

\theorem{Corollaire}{}{false}{
    Le groupe $S(X)$ est engendré par les transpositions.\\
    \textit{i.e.} toute permutation $\sigma$ de $S(X)$ s'écrit comme produit de transpositions $\sigma = \tau_1 \cdots \tau_r$ et $\varepsilon(\sigma) = (-1)^k$. (non unique)  
}

\ndlr{Démonstration laissée à l'appréciation du lecteur, elle n'était pas à mon appréciation}

\theorem{Proposition}{}{false}{
    Soit $c$ un cycle de longueur $l$.\\
    Alors $\varepsilon(c) = - (-1)^{l} = (-1)^{l-1}$.
}

\noindent{
    \textbf{Démonstration :}\\
    On a $c = (a_1 a_2\cdots a_l) = (a_1 a_2)(a_2 a_3) \cdots (a_{l-1} a_l)$, donc $c$ est produit de $l-1$ transpositions.\\
    Donc $\varepsilon(c) = \varepsilon((a_1 a_2)) \times \cdots \times \varepsilon((a_{l-1} a_l)) = (-1)^{l-1}$. $\Box$
}

\example{
    Soit $\sigma \in S_{12}$ donné par :\\
    \[
    \begin{array}{c|cccccccccccc}
        i & 1 & 2 & 3 & 4 & 5 & 6 & 7 & 8 & 9 & 10 & 11 & 12 \\
        \hline
        \sigma(i) & 10 & 8 & 7 & 4 & 2 & 9 & 1 & 12 & 3 & 11 & 5 & 6
    \end{array}
    \]
    Elle vaut $(1\ 10\ 11\ 5\ 2\ 8\ 12\ 6\ 9\ 3\ 7)$.\\
    Donc $\varepsilon(\sigma) = (-1)^{11-1} = 1$.
}

\section{Compléments sur les groupes cycliques}

\theorem{Proposition}{Ordre d'une permutation}{false}{
    Soit $G$ un groupe et $g \in G$ d'ordre fini $n$.\\
    L'application \function{exp_g}{\mathbb{Z} \rightarrow G}{k \mapsto g^k}  se factorise \textit{(i.e. passe au quotient)} par $\mathbb{Z}/n\mathbb{Z}$.\\
    L'application quotient est un isomorphisme de $\mathbb{Z}/n\mathbb{Z}$ sur $g^\mathbb{Z} = \{g^k \mid k \in \mathbb{Z}\}$.
}

\theorem{Corollaire}{}{false}{
    Tout groupe cyclique d'ordre $n$ est isomorphe à $\mathbb{Z}/n\mathbb{Z}$.
}

\noindent{
    \textbf{Démonstration de la proposition :}\\
    Il faut montrer que si $k,k' \in \mathbb{Z}$ et $k \equiv k' [n]$, alors $exp_g(k) = exp_g(k')$.\\
    Comme $k \equiv k' [n]$, $\exists t \in \mathbb{Z}$ tel que $k' - k = tn$ et donc $k' = k + tn$.\\
    On a $exp_g(k') = exp_g(k + tn) = g^{k+tn} = g^k (g^{n})^{t} = g^k e_G = g^k = exp_g(k)$. $\Box$\\\\

    Pour $k\in \mathbb{Z}$, notons $f(\overline{k}) = exp_g(k)$.\\
    On a $f(\overline{k}+\overline{k'}) = f(\overline{k+k'}) = exp_g(k+k') = g^{k+k'} = g^k g^{k'} = f(\overline{k}) + f(\overline{k'})$.\\
    Donc $f$ est un morphisme de groupe.\\ 
    Pour $k\in\mathbb{Z}$, on a $f(\overline{k}) =g^{\mathbb{Z}}$. Donc $Im(f) = g^{\mathbb{Z}}$.\\
    Montrons $f$ injective.\\
    Soit $k\in \mathbb{Z}$ tel que $f(\overline{k}) = e_G$.\\
    On a $g^k = e_G$. Donc $n \mid k$ et donc $\overline{k} = \overline{0}$. $\Box$
}

\ndlr{Fin du chapitre, amen.}

\end{document}

