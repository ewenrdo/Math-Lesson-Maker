\documentclass{article}

\usepackage[a4paper, left=1.5cm, right=1.5cm, top=2cm, bottom=2cm]{geometry}

\usepackage{../../../../components/components} % <-- ton fichier .sty, avec toutes tes définitions

\usepackage{hyperref}
\usepackage{fancyhdr}


% Configuration des en-têtes et pieds de page
\pagestyle{fancy}
\fancyhf{} % reset tout

\fancyhead[L]{DL2 Math-Info}
\fancyhead[C]{Espaces vectoriels}
\fancyhead[R]{2025-2026}

\fancyfoot[L]{Ewen Rodrigues de Oliveira}
\fancyfoot[R]{\thepage}

\begin{document}

\docTitle{Chapitre 5 : Déterminants}

\section{Aire algébrique dans le plan}

Soit $E$ un plan sur un corps $K$.
Soient $u,v \in E$.
Comment définir $\mathcal{A}(u,v)$, l'aire algébrique du parallélogramme construit sur les vecteurs $u$ et $v$ ?
Propriéts attendues :
\begin{itemize}
    \item $\mathcal{A}(\lambda u,v) = \lambda \mathcal{A}(u,v)$ pour tout $\lambda \in K$ (linéarité par rapport au premier argument) ;
    \item $\mathcal{A}(u,\lambda v) = \lambda \mathcal{A}(u,v)$ pour tout $\lambda \in K$ (linéarité par rapport au premier argument) ;
    \item $\mathcal{A}(u_1+u_2, v) = \mathcal{A}(u_1,v) + \mathcal{A}(u_2,v)$ (additivité par rapport au premier argument) ;
    \item $\mathcal{A}(u,v_1+v_2) = \mathcal{A}(u,v_1) + \mathcal{A}(u,v_2)$ (additivité par rapport au second argument) ;
\end{itemize}
Donc les applications $u \mapsto \mathcal{A}(u,v)$ et $v \mapsto \mathcal{A}(u,v)$ sont linéaires.
De plus, $\mathcal{A}(u,u) = 0$ pour tout $u \in E$.
Cela entraine : $\mathcal(u+v, u+v) = 0$\\
et donc : $\mathcal{A}(u,u+v) + \mathcal{A}(v,u+v) = 0$\\
et donc : $\mathcal{A}(u,u) + \mathcal{A}(u,v) + \mathcal{A}(v,u) + \mathcal{A}(v,v) = 0$\\
et donc : $\mathcal{A}(v,u) = -\mathcal{A}(u,v)$ (antisymétrie/alternée).

Soit $(e_1,e_2)$ une base de $E$.\\
Soient $u=ae_1 + be_2$ et $v=ce_1 + de_2$, avec $a,b,c,d \in K$.\\
On a : $\mathcal{A}(u,v) = \mathcal{A}(ae_1 + be_2, ce_1 + de_2)$\\
$= ac\mathcal{A}(e_1,e_1) + ad\mathcal{A}(e_1,e_2) + bc\mathcal{A}(e_2,e_1) + bd\mathcal{A}(e_2,e_2)$\\
$= (ad - bc)\mathcal{A}(e_1,e_2) = $déterminant de $u$ et $v$ dans la base $(e_1,e_2)$ multiplié par $\mathcal{A}(e_1,e_2)$.

\attention{
    Pour avoir une meilleure idée que le ramassis de maths crachées au tableau, voir \href{https://www.youtube.com/watch?v=Ip3X9LOh2dk}{cette vidéo} (en anglais).
}

\section{Formes multilinéaires}

\definition{
Soit $E$ un espace vectoriel sur un corps $K$, soit $p$ un entier naturel non nul.\\
Soit \function{f}{E^p}{K} une application.\\
On dit que $f$ est une \textbf{forme multilinéaire} si pour tout $i$ et $(u_1,\ldots,u_{i-1},u_{i+1},\ldots,u_p) \in E^{p}$, $u \mapsto f(u_1,\ldots,u_{i-1},u,u_{i+1},\ldots,u_p)$ est une application linéaire de $E$ dans $K$.\\
Autrement dit, $f$ est linéaire par rapport à chacun de ses variables.
}
\vocabulary{On dit que $f$ est $p$-linéaire. C'est une forme linéaire si $p=1$, une forme bilinéaire si $p=2$.}

\definition{
    On dit que $f$ est \textbf{alternée} si pour tout $(u_1,\ldots,u_p) \in E^p$ et pour tout $i \neq j$, $f(u_1,\ldots,u_i,\ldots,u_j,\ldots,u_p) = 0_K$ dès que $u_i = u_j$.\\\\
    Autrement dit, $f$ s'annule si deux de ses arguments sont égaux.
}

\definition{
    On dit que $f$ est \textbf{antisymétrique} si pour tous $i,j, i \neq j$ on a $f(u_1,\ldots,u_i,\ldots,u_j,\ldots,u_p) = - f(u_1,\ldots,u_j,\ldots,u_i,\ldots,u_p)$.\\\\
    Autrement dit, permuter deux arguments de $f$ change le signe de son image.
}

\theorem{Proposition}{}{false}{
    Soit $\sigma \in S_p$ une permutation et soit $f\colon E^p \to K$ une forme $p$-linéaire.\\
    L'application $(u_1,\ldots,u_p) \mapsto f(u_{\sigma(1)},\ldots,u_{\sigma(p)})$ est une forme $p$-linéaire notée $f^\sigma$.\\
    De plus, si $f$ est antisymétrique, on a $f^\sigma = \varepsilon(\sigma) f$ où $\varepsilon(\sigma)$ est la signature de la permutation $\sigma$.
}

\ndlr{cf. Laurent pour la preuve}

\theorem{Théorème}{Relation entre alternance et antisymétrie}{false}{
    Toute forme $p$-linéaire alternée est antisymétrique.\\
    De plus, si $2 \neq 0$ (i.e. 2 est inversible dans $K$), toute forme $p$-linéaire antisymétrique est alternée.
}

\remark{Si $K = \mathbb{Z}/2\mathbb{Z} := \mathbb{F}_2$ et $E=\mathbb{K}$, alors :
$(u=(x,y), v=(z,t)) \mapsto xz + yt$ n'est pas alternée car $(1,0)$ et $(0,1)$ sont distincts mais l'image est $1$. Cependant, cette application est antisymétrique car $1 = -1$ dans $\mathbb{F}_2$.}

\theorem{Proposition}{}{false}{
    Soit $f$ une forme $p$-linéaire alternée.\\
    Soit $(u_1,\ldots,u_p) \in E^p$ et soit $i \in \{1,\ldots,p\}$.\\
    Soit $v_i = u_i$ + combinaison linéaire de $u_1, \ldots, \hat{u_i}, \ldots, u_p$ (c'est-à-dire des $u_j$ avec $j \neq i$).\\
    Alors $f(u_1,\ldots,u_{i-1},v_i,u_{i+1},\ldots,u_p) = f(u_1,\ldots,u_p)$.
}

\vocabulary{On note $(u_1,\ldots,\hat{u_i},\ldots,u_p)$ la famille obtenue en supprimant $u_i$ de la famille $(u_1,\ldots,u_p)$.}

\end{document}