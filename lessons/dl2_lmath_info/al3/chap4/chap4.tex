\documentclass{article}

\usepackage[a4paper, left=1.5cm, right=1.5cm, top=2cm, bottom=2cm]{geometry}

\usepackage{../../../../components/components} % <-- ton fichier .sty, avec toutes tes définitions

\usepackage{fancyhdr}


% Configuration des en-têtes et pieds de page
\pagestyle{fancy}
\fancyhf{} % reset tout

\fancyhead[L]{DL2 Math-Info}
\fancyhead[C]{Espaces vectoriels}
\fancyhead[R]{2025-2026}

\fancyfoot[L]{Ewen Rodrigues de Oliveira}
\fancyfoot[R]{\thepage}

\begin{document}

\docTitle{Chapitre 4 : Matrices}

\section{Généralités}
Voir le cours de MM1.

\theorem{Proposition}{Symbole de Kronecker}{false}{
    Soit \(n \in \mathbb{N}^*\). Le symbole de Kronecker est défini par :
    \[
        \delta_{ij} = \begin{cases}
            1 & \text{si } i = j \\
            0 & \text{sinon}
        \end{cases}
    \]
    pour \(i, j \in \{1, 2, \ldots, n\}\).
}

\remark{De là découle immédiatement les matrices élémentaires suivantes : $E_{ij} = (\delta_{ik}\delta_{jl})_{1 \leq k,l \leq n}$.}

\section{Structure d'un espace vectoriel}

\subsection{Ensemble des matrices}

\theorem{Proposition}{}{false}{
    Soient $n,m \in \mathbb{N}$.\\
    Comme $M_{n,m}(\mathbb{K}) = \{ \varphi : \{1, \ldots, n\} \times \{1, \ldots, m\} \to \mathbb{K} \}$, on en déduit que $M_{n,m}(\mathbb{K})$ est un espace vectoriel de dimension $nm$.
}

\remark{Une base de $M_{n,m}(\mathbb{K})$ est donnée par les matrices élémentaires $\{ E_{ij} \mid 1 \leq i \leq n, 1 \leq j \leq m \}$.}

\subsection{Structure multiplicative}

\theorem{Proposition}{}{false}{
    Soient $n, m, p, q \in \mathbb{N}^*$.\\
    Soient $A = (a_{ij}) \in M_{n,p}(\mathbb{K})$ et $B = (b_{jk}) \in M_{p,q}(\mathbb{K})$.\\
    Le produit matriciel $C = AB$ est défini par $C = (c_{ik})$ où $c_{ik} = \sum_{j=1}^{p} a_{ij} b_{jk}$.
}

\remark{Le produit matriciel est associatif.}

\theorem{Proposition}{Linéarité}{false}{
    Si on fixe $A \in M_{n,p}(\mathbb{K})$, l'application $M_{p,q}(\mathbb{K}) \to M_{n,q}(\mathbb{K}), B \mapsto AB$ est $K$-linéaire.\\
    \textit{Réciproquement $M_{n,p}(\mathbb{K}) \to M_{n,q}(\mathbb{K}), A \mapsto AB$ est $K$-linéaire si on fixe $B \in M_{p,q}(\mathbb{K})$.}
}

\remark{Si $n=p$, on a une \textbf{loi associative interne} sur $M_n(\mathbb{K})$. La matrice identité $I_n$ est l'élément neutre.}

\definition{
    Soit $A \in M_n(\mathbb{K})$. On dit que $A$ est inversible s'il existe $B \in M_n(\mathbb{K})$ tel que $AB = BA = I_n$. Dans ce cas, $B$ est unique et on le note $A^{-1}$.
}

\vocabulary{On note $GL_n(\mathbb{K})$ l'ensemble des matrices inversibles de $M_n(\mathbb{K})$.}

\theorem{Proposition}{Groupe}{false}{
    L'ensemble $GL_n(\mathbb{K})$ muni du produit matriciel est un groupe. (non abélien si $n \geq 2$)
}

\remark{
    \begin{enumerate}
        \item $GL_n(\mathbb{K})$ n'est pas un espace vectoriel.
        \item $GL_1(\mathbb{K}) = \mathbb{K}^*$ qui est abélien.
    \end{enumerate}
}

\section{Transposition}

\definition{
    Soient $n,p \in \mathbb{N}^*$ et $A = (a_{ij}) \in M_{n,p}(\mathbb{K})$.\\
    La transposée de $A$ est la matrice ${ }^t A = (a_{ji}) \in M_{p,n}(\mathbb{K})$.
}

\theorem{Propriété}{}{false}{
    L'application transposée $M_{n,p}(\mathbb{K}) \to M_{p,n}(\mathbb{K}), A \mapsto { }^t A$ est une isomorphisme linéaire.\\
    Elle est symétrique si ${ }^t({ }^t A) = A$ et antisymétrique si ${ }^t({ }^t A) = -A$.
}

\section{Représentation matricielle}

\definition{
    Soit $E$ un $K$-espace vectoriel de dimension $n$. Soit $B=(e_1, \ldots, e_n)$ une base de $E$.\\
    Soit $(u_1, \ldots, u_p)$ une famille de $p$ vecteurs de $E$. Posons pour $1 \leq j \leq p$, $u_j = \sum_{i=1}^{n} a_{ij} e_i$.\\\\

    On définit $Mat_B(u_1, \ldots, u_p) = A = (a_{ij}) \in M_{n,p}(\mathbb{K})$ la matrice des coordonnées des vecteurs $(u_1, \ldots, u_p)$ dans la base $B$.
}

\theorem{Propriété}{}{false}{
    Si $p=1$, on a $u=u_1$ et $Mat_B(u) = \begin{pmatrix}
        a_{1} \\
        a_{2} \\
        \vdots \\
        a_{n}
    \end{pmatrix}$\\
    On a que l'application $E \to K^n, u \mapsto Mat_B(u)$ est un isomorphisme linéaire.
}

\definition{
    Soient $E,F$ des $K$-espaces vectoriels avec $dimE=p, dimF=n$.\\
    Soit $f \in \mathcal{L}(E,F)$. Soient $B=(e_1, \ldots, e_p)$ une base de $E$ et $B'=(v_1, \ldots, v_n)$ une base de $F$.\\

    On définit la matrice de $f$ dans les bases $B$ et $B'$ par $Mat_{B,B'}(f) = Mat(f(e_1), \ldots, f(e_p)) \in M_{n,p}(\mathbb{K})$.
}

\theorem{Proposition}{}{false}{
    Soit $u \in E$.\\
    Posons $X = Mat(u, B)$, $Y= Mat(f(u), B)$. Posons A $ = Mat_{B,B'}(f)$.\\
    On a $Y = AX$.
}

\theorem{Proposition}{}{false}{
    L'application $\mathcal{L}(E,F) \to M_{p,n}(\mathbb{K}), f \mapsto Mat_{B,B'}(f)$ est un isomorphisme linéaire.
}

\theorem{Proposition}{Composition}{false}{
    Soit $G$ un $K$-espace vectoriel de base finie $B''$. Soit $g \in \mathcal{L}(F,G)$.\\
    Alors $Mat_{B,B''}(g \circ f) = Mat_{B',B''}(g) \cdot Mat_{B,B'}(f)$.
}

\theorem{Proposition}{Bijection et inversibilité}{false}{
    On a $f \in \mathcal{L}(E,F)$ est un bijective si et seulement si $Mat_{B,B'}(f) \in GL_n(\mathbb{K})$.\\
    Alors $Mat_{B',B}(f^{-1}) = (Mat_{B,B'}(f))^{-1}$.
}

\remark{Soit $A\in M_{p,n}(\mathbb{K})$. Il lui correspond $L(A) : K^p \to K^n$ définie par $L(A)(X) = AX$.\\
On a $L(AB) = L(A) \circ L(B)$.}

\definition{
    \begin{enumerate}
        \item Noyau de A : $\text{Ker}(A) = \{ X \in K^m \mid AX = 0 \} = Ker(L(A))$.
        \item Image de A : $\text{Im}(A) = \{ Y \in K^n \mid \exists X \in K^m, Y = AX \} = Im(L(A))$.
        \item Rang de A : $\text{rg}(A) = dim(Im(A)) = rg(L(A))$.
    \end{enumerate}
}

\section{Matrice de changement de base}

\definition{Soit $E$ un $K$-espace vectoriel de dimension $n$.
Soient $B = (e_1, \ldots, e_n)$ et $B' = (v_1, \ldots, v_n)$ des bases de $E$.\\
Si on pose $v_j = \sum_{i=1}^{n} a_{ij} e_j$, alors : $P_{B,B'} = (a_{ij}) \in M_n(\mathbb{K})$ est la matrice de passage de la base $B$ à la base $B'$.}

...

\theorem{Proposition}{}{false}{
    Les conditions suivantes sont équivalentes :
    \begin{enumerate}
        \item $A$ est inversible.
        \item Pour tout $B$, $AX=B$ a une unique solution $X$.
        \item Pour tout $B$, $AX=B$ admet au moins une solution $X$.
        \item Pour tout $B$, $AX=B$ admet au plus une solution $X$.
        \item $AX=0$ a pour solution $X=0$.
    \end{enumerate}
}

\vocabulary{On appelle système de Cramer un système de la forme $AX=B$ où $A$ est inversible.}

\end{document}