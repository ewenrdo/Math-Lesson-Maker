\documentclass{article}

\usepackage[a4paper, left=1.5cm, right=1.5cm, top=2cm, bottom=2cm]{geometry}

\usepackage{../../../../components/components} % <-- ton fichier .sty, avec toutes tes définitions

\usepackage{fancyhdr}


% Configuration des en-têtes et pieds de page
\pagestyle{fancy}
\fancyhf{} % reset tout

\fancyhead[L]{DL2 Math-Info}
\fancyhead[C]{Espaces vectoriels}
\fancyhead[R]{2025-2026}

\fancyfoot[L]{Ewen Rodrigues de Oliveira}
\fancyfoot[R]{\thepage}

\begin{document}

\docTitle{Chapitre 3 : Espaces vectoriels}


\section{Corps}

\definition{Un \textbf{corps} est un ensemble $K$ muni de deux lois de composition interne notées $+$ et $\times$ telles que :\\
    \begin{itemize}
        \item $(K,+)$ est un groupe abélien
        \item $(K\setminus\{0\},\times)$ est un groupe abélien
        \item La loi $\times$ est distributive par rapport à la loi $+$
    \end{itemize}
    Si de plus la loi $\times$ est commutative, on dit que $K$ est un \textbf{corps commutatif}.
}

\reminder{Distributivité : $\forall a,b,c \in K, a \times (b + c) = a \times b + a \times c$}
\example{$\mathbb{Q}, \mathbb{R}, \mathbb{C}, \mathbb{Z}/p\mathbb{Z}, p \text{ premier}$ sont des corps.}

\section{Espaces vectoriels}

\definition{
    Soient $K$ un corps et $E$ un groupe abélien.\\
    Soit une loi \function{}{K \times E \rightarrow E}{(\lambda, v) \mapsto \lambda \cdot v} \textit{(multiplication externe)}.\\
    On dit que $(E, +, \cdot)$ est un \textbf{$K$-espace vectoriel} si on a $\forall \lambda, \mu \in K, \forall v \in E$ :\\
    \begin{itemize}
        \item $\lambda \cdot (\mu \cdot v) = (\lambda \times \mu) \cdot v$
        \item $1 \cdot v = v$
        \item $(\lambda + \mu) \cdot v = \lambda \cdot v + \mu \cdot v$ \textit{(on a deux + différents)}
        \item $\lambda \cdot (v + w) = \lambda \cdot v + \lambda \cdot w$
    \end{itemize}
}

\vocabulary{Les éléments de $E$ sont appelés \textbf{vecteurs}. Les éléments de $K$ sont appelés \textbf{scalaires}.}
\example{$\mathbb{R}^n$ est un $\mathbb{R}$-espace vectoriel. De même pour $\{0\}$, $\mathbb{R}[X]$, $M_n(\mathbb{R})$.\\
On peut voir $\mathbb{C}$ comme un $\mathbb{R}$-espace vectoriel.}

\definition{
    Soit $E$ un $K$-ev, et soit $(v_i)_{i \in I}$ une famille de vecteurs de $E$.\\
    Soit $(\lambda_i)_{i \in I}$ une famille de scalaires de $K$.\\
    On dit que $(\lambda_i)_{i \in I}$ est presque nulle si : \{$i \in I, \lambda_i \neq 0\}$ est fini.\\
    
    Alors on considère $\sum_{i \in I, \lambda \neq 0} \lambda_i v_i$ noté $\sum_{i \in I} \lambda_i v_i$. C'est une \textbf{combinaison linéaire} des $v_i$.
}

\definition{
    Soit $X \subset E$. 
    Une combinaison linéaire de vecteurs de $X$ est de la forme $\sum_{v \in X} \lambda_v v$ avec $(\lambda_v)_{v \in X}$ presque nulle.\\
}

\vocabulary{Les $(\lambda_v)_{v \in X}$ sont appelés les \textbf{coefficients} de la combinaison linéaire.}

\section{Sous-espaces vectoriels}

\definition{
    Soit $E$ un $K$-ev. Soit $F \subset E$.\\
    On dit que $F$ est un \textbf{sous-espace vectoriel} (sous-ev) de $E$ si :
    \begin{itemize}
        \item $F \neq \emptyset$
        \item $\forall u,v \in F, \lambda,\mu \in K, \lambda u + \mu v \in F$
    \end{itemize}
}

\theorem{Proposition}{Caractérisation des sous-ev}{false}{
    Tout sous-espace vectoriel est un espace vectoriel pour les lois induites par $E$.
}

\noindent{\textbf{Preuve:}\\
Montrons $(F, +)$ est un sous-groupe de $(E, +)$ :\\
\begin{itemize}
    \item $F \neq \emptyset$ donc $\exists u \in F$.
    \item $\lambda = \mu = 1 \implies u + v \in F, \forall u,v \in F$ donc $F$ est stable par $+$
    \item $u + (-1)u = u(1 + (-1)) = 0_E \in F$. On a donc $-u \in F, \forall u \in F$.
\end{itemize}
Donc on a bien un sous-groupe.\\
Les autres propriétés sont vérifiables et immédiates, on a bien un espace vectoriel
}

\example{
    Soit $E$ un $\mathbb{K}$-ev.\\
    \begin{itemize}
        \item $\{0_E\}$ et $E$ sont des sous-ev de $E$.
        \item $\{(x,y) \mid ax+by=0\} \subset \mathbb{R}^2$ est un sous-ev de $\mathbb{R}^2$.
    \end{itemize}
}

\theorem{Proposition}{Intersection de sev}{false}{
    Soit $E$ un $K$-ev. Soit $(F_i)_{i \in I}$ une famille de sev de $E$.\\
    Alors $\bigcap_{i \in I} F_i$ est un sev de $E$.
}

\noindent{\textbf{Preuve:}\\
Montrons que $\bigcap_{i \in I} F_i$ est stable par combinaison linéaire.\\
Soient $x,y \in \bigcap_{i \in I} F_i$ et $\lambda, \mu \in K$.\\
On a $x,y \in F_i$ pour tout $i \in I$, donc $\lambda x + \mu y \in F_i$ pour tout $i \in I$.\\
Donc $\lambda x + \mu y \in \bigcap_{i \in I} F_i$. $\Box$}

\remark{L'union de sev n'est pas forcément un sev.}

\theorem{Proposition}{Sous-ev engendré}{false}{
    Soit $E$ un $K$-ev. Soit $X \subset E$.\\
    Alors il existe un plus petit sev de $E$ contenant $X$, noté $\text{Vect}(X)$ et appelé le \textbf{sev engendré} par $X$.\\
    On a $\text{Vect}(X) = \{\sum_{x\in X} \lambda_x x \mid (\lambda_x)_{x\in X} \text{est presque nulle}, \lambda_x \in K\} = \bigcap_{\substack{F \text{ sev de } E \\ X \subset F}} F$.\\\\
    Intuitivement, c'est l'ensemble des combinaisons linéaires d'éléments de $X$.
}

\reminder{"Presque nulle" signifie que tous les coefficients sont nuls sauf un nombre fini d'entre eux.}

\noindent{\textbf{Preuve:}\\
Montrons que $Vect(X)$ est un sev de E.\\
Soient $u,v \in Vect(X)$ et $\lambda, \mu \in K$.\\
On a $u = \sum_{x \in X} \lambda_x x$ et $v = \sum_{x \in X} \mu_x x$ avec $(\lambda_x)_{x \in X}$ et $(\mu_x)_{x \in X}$ presque nulles.\\
Donc $\lambda u + \mu v = \sum_{x \in X} (\lambda \lambda_x + \mu \mu_x) x$ est une combinaison linéaire d'éléments de $X$ avec des coefficients presque nuls.\\
Donc $\lambda u + \mu v \in Vect(X)$. C'est bien un sev.

On a $X \subset \{CL de X\}$ car $x = 1 \cdot x + 0 \cdot y, \forall x \in X, y \in E$.\\
Donc $Vect(X) \subset \{CL de X\}$.\\
Réciproquement, $Vect(X)$ est stable par combinaison linéaire et contient $X$, donc $\{CL de X\} \subset Vect(X)$. Donc $Vect(X) = \{CL de X\}$. $\Box$.}

\remark{La démonstration est générée par IA, elle diffère de celle du cours.}

\theorem{Proposition}{Addition de sev}{false}{
    Soit $E$ un $K$-ev. Soit $(F_i)_{i \in I}$ une famille de sev de $E$.\\
    On peut considérer $Vect(\bigcup_{i \in I} F_i)$, noté $\sum_{i \in I} F_i$ et appelé la \textbf{somme} de la famille $(F_i)_{i \in I}$.
}

\remark{On note $F_1 + F_2 + \cdots + F_n$ au lieu de $\sum_{i=1}^n F_i$.}

\theorem{Proposition}{Caractérisation de la somme}{false}{
    Soit $E$ un $K$-ev. Soit $(F_i)_{i \in I}$ une famille de sev de $E$.\\
    Alors $\sum_{i \in I} F_i = \{\sum_{i \in I} x_i \mid x_i \in F_i, \text{ presque tous nuls}\}$.
}

\noindent{\textbf{Preuve:}\\
\ndlr{cf. Laurent}}

\example{Soient $F$ et $G$ deux sev de $E$.\\
Alors $F + G = \{x + y \mid x \in F, y \in G\}$.}

\theorem{Proposition}{Application}{false}{
    On a une application \function{\varphi}{F_1 \times F_2 \times \cdots \times F_n \rightarrow F_1 + F_2 + \cdots + F_n}{(x_1, x_2, \ldots, x_n) \mapsto x_1 + x_2 + \cdots + x_n}.\\
    Elle est surjective.
}

\vocabulary{Si de plus l'application $\varphi$ est injective, alors on dit que la somme $F_1 + F_2 + \cdots + F_n$ est \textbf{directe} et on note $F_1 \oplus F_2 \oplus \cdots \oplus F_n$.}

\theorem{Proposition}{Caractérisation de la somme directe}{false}{
    Soient $F_1, F_2, \ldots, F_n$ des sev de $E$. Les assertions suivantes sont équivalentes :
    \begin{itemize}
        \item $F_1 + F_2 + \cdots + F_n$ est une somme directe.
        \item $\forall (u_1, u_2, \ldots, u_n) \in F_1 \times F_2 \times \cdots \times F_n, u_1 + u_2 + \cdots + u_n = 0_E \implies u_1 = u_2 = \cdots = u_n = 0_E$
        \item $\forall i \in \{1, 2, \ldots, n\}, F_i \cap (F_1 + \cdots + F_{i-1}) = \{0_E\}$
    \end{itemize}
}
\noindent{\textbf{Preuve:}\\
\ndlr{cf. Laurent}}

\attention{On a $F_1$ somme directe avec $F_2$ ssi $F_1 \cap F_2 = \{0_E\}$. Mais pour avoir $F_1 \oplus F_2$ et $F_1 \oplus F_3$, et $F_2 \oplus F_3 = \{0_E\}$, mais pas forcément $F_1 \oplus F_2 \oplus F_3$.}

\example{On prend 3 droites dans le plan passant par l'origine, deux par deux distinctes.\\
Alors elles sont en somme, mais pas en somme directe.}

\vocabulary{Si $F \oplus G = E$, on dit que $F$ et $G$ sont des \textbf{supplémentaires}.}
\attention{Le supplémentaire n'est pas unique.}

\section{Familles}

\definition{
    Soit $E$ un $K$-ev. Soit $I$ un ensemble.\\
    Soit $(x_i)_{i \in I}$ une famille d'éléments de $E$.\\\\
    Soit $J \subset I$. On dit que $(x_i)_{i \in J}$ est une \textbf{sous-famille} de $(x_i)_{i \in I}$.
}

\vocabulary{À l'inverse, $(x_i)_{i \in I}$ est une \textbf{sur-famille} de $(x_i)_{i \in J}$.}

\remark{Dans la pratique, on prend souvent $I = \{1, 2, \ldots, n\}$.}

\remark{
    Soit $X \subset E$. On a une famille $(x)_{x \in X}$ indexée par $X$.\\
    Donc une combinaison linéaire de X est une combinaison linéaire de la famille $(x)_{x \in X}$. \textit{(la réciproque n'est pas vraie, car il peut y avoir des répétitions dans la famille: x+x n'est pas une combinaison linéaire de X)}
}
\definition{
Soit $(x_i)_{i \in I}$ une famille de vecteurs de $E$.\\
On dit que $(x_i)_{i \in I}$ est \textbf{libre} si :\\
$\forall (\lambda_i)_{i \in I}$ presque nulle, $\sum_{i \in I} \lambda_i x_i = 0_E \implies \forall i \in I, \lambda_i = 0$.\\
On dit que $(x_i)_{i \in I}$ est \textbf{liée} sinon.\\\\

Pour $X \subset E$, on dit que $X$ est une partie libre (resp. liée) si la famille $(x)_{x \in X}$ est libre (resp. liée).
}

\example{
    Dans $\mathbb{R}^3$, les vecteurs $(1,0,0)$ et $(0,1,0)$ sont libres.\\
    Les vecteurs $(1,0,0)$ et $(2,0,0)$ sont liés.
}

\definition{
    Soit $(x_i)_{i \in I}$ une famille de vecteurs de $E$.\\
    On dit que c'est une \textbf{famille génératrice} de $E$ si $Vect(\{x_i \mid i \in I\}) = E$.\\\\
    Alors tout élément de $E$ s'écrit comme une combinaison linéaire des $x_i$.\\
    Pour $X \subset E$, on dit que $X$ est une partie génératrice de $E$ si la famille $(x)_{x \in X}$ est génératrice de $E$.
}

\theorem{Proposition}{Sous-familles}{false}{
    Soit $(x_i)_{i \in I}$ une famille de vecteurs de $E$.\\
    Alors toute sous-famille $(x_j)_{j \in J}$ de $(x_i)_{i \in I}$ est libre (resp. liée) si $(x_i)_{i \in I}$ est libre (resp. liée).\\
    Et toute sur-famille $(x_k)_{k \in K}$ de $(x_i)_{i \in I}$ est génératrice (resp. non génératrice) si $(x_i)_{i \in I}$ est génératrice (resp. non génératrice).\\\\
    \textit{(de même pour un sous-ensemble et un sur-ensemble d'une partie de $E$)}
}

\section{Bases}
\definition{
    Soit $(x_i)_{i \in I}$ une famille de vecteurs de $E$.\\
    On dit que $(x_i)_{i \in I}$ est une \textbf{base} de $E$ si c'est une famille libre et génératrice de $E$.\\\\
    Pour $X \subset E$, on dit que $X$ est une partie basique de $E$ si c'est une partie libre et génératrice de $E$. Alors la famille $(x)_{x \in X}$ est une base de $E$.
}

\vocabulary{Une partie basique appelée une \textbf{base} par abus de langage.}

\theorem{Théorème}{Coordonnées}{false}{
    Soit $(b_i)_{i \in I}$ une base de $E$, et soit $u \in E$.\\
    Alors il existe une unique famille $(\lambda_i)_{i \in I}$ presque nulle telle que $u = \sum_{i \in I} \lambda_i b_i$.
}
\vocabulary{On dit que $\lambda_i$ est la \textbf{i-ème coordonnée} de $u$ dans la base $B$.}
\noindent{\textbf{Preuve:}\\
Montrons l'existence.\\
Comme B est une base, c'est une famille génératrice, donc $u$ s'écrit comme une combinaison linéaire de vecteurs de B.\\
Montrons l'unicité.\\
Supposons qu'il existe $(\lambda_i)_{i \in I}$ et $(\mu_i)_{i \in I}$ presque nulles telles que $u = \sum_{i \in I} \lambda_i b_i = \sum_{i \in I} \mu_i b_i$.\\
On a donc $\sum_{i \in I} (\lambda_i - \mu_i) b_i = 0_E$. Comme B est libre, on a $\forall i \in I, \lambda_i - \mu_i = 0$, donc $\lambda_i = \mu_i$. $\Box$}

\example{E=$\{0\}$, la seule base est la base vide.\\Dans $\mathbb{R}^2$, la famille $((1,0),(0,1))$ est une base : c'est la base canonique.}
\vocabulary{Si $E$ est engendré par une partie finie, on dit que $E$ est de \textbf{dimension finie}.}
\remark{$K^n$ est de dimension finie, $K^X$ avec $X$ infini est de dimension infinie.}

\theorem{Proposition}{}{false}{
    Soit $L \subset E$ avec une partie libre. Soit $u \in E$.\\
    Alors $L \cup \{u\}$ est libre ssi $u \notin Vect(L)$.
}

\noindent{\textbf{Preuve:}\\
Si $u \in Vect(L)$, alors $u = \sum_{v \in L} \lambda_v v$ avec $(\lambda_v)_{v \in L}$ presque nulle.\\
Donc $u - \sum_{v \in L} \lambda_v v = 0
$ avec $1 \neq 0$, donc $L \cup \{u\}$ est liée.\\
Réciproquement, si $L \cup \{u\}$ est liée, alors il existe $(\lambda_v)_{v \in L}$ presque nulle et $\mu \neq 0$ tels que $\sum_{v \in L} \lambda_v v + \mu u = 0_E$.\\
Donc $u = -\frac{1}{\mu} \sum_{v \in L} \lambda_v v \in Vect(L)$. $\Box$}

\attention{Démonstration générée par IA, différente de celle du cours.}
\ndlr{cf. Laurent pour une autre preuve}

\theorem{Théorème}{Base incomplète}{false}{
    Soit $G$ une partie génératrice finie de $E$.\\
    Soit $L \subset G$ une partie libre.\\
    Alors il existe une partie basique $B$ de $E$ telle que $L \subset B \subset G$.\\\\
}

\ndlr{cf. Laurent pour les preuve}

\theorem{Corollaire}{Existence de base}{false}{
    Si $E$ est un $K$-ev de dimension finie, alors $E$ admet une base.
}

\noindent{\textbf{Preuve:}\\
On applique le théorème de la base incomplète avec $L = \emptyset$. $\Box$}

\theorem{Corollaire}{Caractérisation des bases}{false}{
    Si $L$ est une partie libre à $p$ éléments de $E$ et $G$ une partie génératrice finie à $q$ éléments de $E$.\\
    On a $p \leq q$.
}
\ndlr{cf. Laurent pour les preuve}
  


\theorem{Théorème}{Bases finies}{false}{
    Si $E$ est un $K$-ev de dimension finie, alors toutes les bases de $E$ ont le même cardinal, appelé la \textbf{dimension} de $E$ et noté $\dim(E)$.
}

Si $E$ est de dimension finie $n$ et $(u_1,\ldots,u_n)$ est une famille libre de E, il existe $(u_{p+1}, \ldots, u_n)$ tels que $(u_1, \ldots, u_n)$ est une base de $E$.\\

\theorem{Théorème}{Assertions équivalentes}{false}{
    Les assertions suivantes sont équivalentes :
    \begin{enumerate}
        \item E est un ev de dimension n.
        \item Toute base de E a $n$ vecteurs.
        \item Toute base génératrice de E a au moins $n$ vecteurs.
        \item Toute partie génératrice de E a $n$ vecteurs est libre.
        \item Toute partie libre de $E$ a au plus $n$ vecteurs.
        \item Toute partie libre $E$ à $n$ éléments est génératrice.
    \end{enumerate}
}

\ndlr{cf. Laurent pour les preuves}

\theorem{Proposition}{}{false}{
Soit $E$ un $K$-ev. Supposons $E$ de dimension finie.\\
Soit $F$ un sev de $E$. Alors $F$ est de dimension finie et $\dim(F) \leq \dim(E)$.\\\\
Si de plus $\dim(F) = \dim(E)$, alors $F = E$.
}
\ndlr{cf. Laurent pour la preuve}

\section{Applications linéaires}

\definition{
    Soient $E$ et $F$ des $K$-ev.\\
    Soit une application \functionSets{f}{E \rightarrow F}.\\
    On dit que $f$ est une \textbf{application linéaire} si $\forall x,y \in E, \forall \lambda, \mu \in K, f(\lambda x + \mu y) = \lambda f(x) + \mu f(y)$.\\\\
    On note $\mathcal{L}(E,F)$ l'ensemble des applications linéaires de $E$ dans $F$.
}

\vocabulary{On appelle aussi les applications linéaires des \textbf{morphismes} d'espaces vectoriels.}
\remark{$f$ est un morphisme de groupes de $(E,+)$ dans $(F,+)$.}

\vocabulary{On dit que $f$ est un \textbf{endomorphisme} si $E=F$.\\
On dit que $f$ est un \textbf{isomorphisme} si $f$ est bijective.\\
On dit que $f$ est un \textbf{automorphisme} si $f$ est un endomorphisme et un isomorphisme.}

\vocabulary{Si $F=K$, on dit que $f$ est une \textbf{forme linéaire}.}

\example{
    \begin{enumerate}
        \item Soit $\alpha \in K$. L'application \function{\varphi_\alpha}{E \rightarrow E}{x \mapsto \alpha x} est linéaire. C'est l'\textbf{homotétie} de rapport $\alpha$.
        \item Soit $B$ une base de $E$.\\
        L'application \function{\varphi_B}{E \rightarrow}{x \mapsto \text{ième coordonnée de } x \text{ dans la base } B} est linéaire. C'est \textbf{l'application coordonnées} dans la base $B$.
    \end{enumerate}
}

\theorem{Proposition}{}{false}{
    Soit $f \in \mathcal{L}(E,F)$. Soit $B$ une base de $E$.\\
    Alors $f$ est déterminée par ses valeurs sur $B$. \textit{(i.e. si $g \in \mathcal{L}(E,F)$ vérifie $f(e_i) = g(e_i), \forall i$, alors $f=g$)}
}

\remark{
    \begin{itemize}
        \item $B$ génératrice suffit.
        \item Si $E$ est de dimension finie, $f$ est déterminée par un nombre fini de valeurs.
    \end{itemize}
}

\ndlr{cf. Laurent pour la preuve}


\end{document}