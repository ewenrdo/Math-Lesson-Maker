\documentclass{article}

\usepackage[a4paper, left=1.5cm, right=1.5cm, top=2cm, bottom=2cm]{geometry}

\usepackage{../../../../components/components} % <-- ton fichier .sty, avec toutes tes définitions

\usepackage{fancyhdr}


% Configuration des en-têtes et pieds de page
\pagestyle{fancy}
\fancyhf{} % reset tout

\fancyhead[L]{DL2 Math-Info}
\fancyhead[C]{Espaces vectoriels}
\fancyhead[R]{2025-2026}

\fancyfoot[L]{Ewen Rodrigues de Oliveira}
\fancyfoot[R]{\thepage}

\begin{document}

\docTitle{Chapitre 3 : Espaces vectoriels}


\section{Corps}

\definition{Un \textbf{corps} est un ensemble $K$ muni de deux lois de composition interne notées $+$ et $\times$ telles que :\\
    \begin{itemize}
        \item $(K,+)$ est un groupe abélien
        \item $(K\setminus\{0\},\times)$ est un groupe abélien
        \item La loi $\times$ est distributive par rapport à la loi $+$
    \end{itemize}
    Si de plus la loi $\times$ est commutative, on dit que $K$ est un \textbf{corps commutatif}.
}

\reminder{Distributivité : $\forall a,b,c \in K, a \times (b + c) = a \times b + a \times c$}
\example{$\mathbb{Q}, \mathbb{R}, \mathbb{C}, \mathbb{Z}/p\mathbb{Z}, p \text{ premier}$ sont des corps.}

\section{Espaces vectoriels}

\definition{
    Soient $K$ un corps et $E$ un groupe abélien.\\
    Soit une loi \function{}{K \times E \rightarrow E}{(\lambda, v) \mapsto \lambda \cdot v} \textit{(multiplication externe)}.\\
    On dit que $(E, +, \cdot)$ est un \textbf{$K$-espace vectoriel} si on a $\forall \lambda, \mu \in K, \forall v \in E$ :\\
    \begin{itemize}
        \item $\lambda \cdot (\mu \cdot v) = (\lambda \times \mu) \cdot v$
        \item $1 \cdot v = v$
        \item $(\lambda + \mu) \cdot v = \lambda \cdot v + \mu \cdot v$ \textit{(on a deux + différents)}
        \item $\lambda \cdot (v + w) = \lambda \cdot v + \lambda \cdot w$
    \end{itemize}
}

\vocabulary{Les éléments de $E$ sont appelés \textbf{vecteurs}. Les éléments de $K$ sont appelés \textbf{scalaires}.}
\example{$\mathbb{R}^n$ est un $\mathbb{R}$-espace vectoriel. De même pour $\{0\}$, $\mathbb{R}[X]$, $M_n(\mathbb{R})$.}
\end{document}