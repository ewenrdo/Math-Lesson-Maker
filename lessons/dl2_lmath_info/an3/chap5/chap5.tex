\documentclass{article}

\usepackage[a4paper, left=1.5cm, right=1.5cm, top=2cm, bottom=2cm]{geometry}

\usepackage{amsmath,amssymb}

\usepackage{../../../../components/components} % <-- ton fichier .sty, avec toutes tes définitions

\usepackage{fancyhdr}


% Configuration des en-têtes et pieds de page
\pagestyle{fancy}
\fancyhf{} % reset tout

\fancyhead[L]{DL2 Math-Info}
\fancyhead[C]{Analyse}
\fancyhead[R]{2025-2026}

\fancyfoot[L]{Ewen Rodrigues de Oliveira}
\fancyfoot[R]{\thepage}

\begin{document}

\docTitle{Chapitre 5 : Fonctions de plusieurs variables}

\textbf{Cadre de travail :}
\begin{itemize}
    \item $p,q \in \mathbb{N}^*$
    \item $U$ ouvert de $\mathbb{R}^n$
    \item $f : U \to \mathbb{R}^m$
    \item $\mathbb{R}^p$ et $\mathbb{R}^q$ sont des espaces vectoriels qui peuvent être normés, donc des espaces métriques. Ainsi, toutes les propriétés et des notions des espaces métriques s'appliquent.
\end{itemize}

\section{Continuité}

\reminder{Une suite $(X_n)$ de composante $\begin{pmatrix}x^{(n)}_1 \\ x^{(n)}_2 \\ \vdots \\ x^{(n)}_p\end{pmatrix} \in \mathbb{R}^p$ tend vers $X = \begin{pmatrix}x_1 \\ x_2 \\ \vdots \\ x_p\end{pmatrix} \in \mathbb{R}^p$ si et seulement si chacune de ses composantes converge, c'est-à-dire :
\[
    \forall i \in \{1, \ldots, p\}, \quad x^{(n)}_i \xrightarrow[n \to +\infty]{} x_i
\]}

\theorem{Proposition}{}{false}{
    Soit \functionSets{f}{U \to \mathbb{R}^q} et $a \in U$. On écrit $f(x) = (f_1(x), f_2(x), \ldots, f_q(x)), x\in U$.\\
    On a : $f$ est continue en $a \Leftrightarrow \forall i=1,\ldots,q, f_i(x)$ est continue en $a$.
}

\remark{$\forall i=1,\ldots,q$, les fonctions $x \mapsto f_i(x)$ sont à valeur dans $\mathbb{R}$, donc on peut utiliser les résultats de la continuité pour les fonctions à valeur réelle.}
\vocabulary{On appelle les $f_i$ les \textbf{fonctions composantes} de $f$.}

\noindent{\textbf{Preuve :}\\
    On rappelle que $f$ est continue en $a$ ssi par toute suite $(x_n)$ qui converge vers $a$, la suite $(f(x_n))$ converge vers $f(a)$.\\
    On a donc $f(x_n) = (f_1(x_n), f_2(x_n), \ldots, f_q(x_n)) \to f(a) = (f_1(a), f_2(a), \ldots, f_q(a))$ ssi $\forall i=1,\ldots,q, f_i(x_n) \to f_i(a)$.\\
    Donc $f$ est continue en $a$ ssi $\forall i=1,\ldots,q, f_i$ est continue en $a$.
}

Pour comprendre la continuité, il suffit de se restreindre aux fonctions $f : U \to \mathbb{R}$.\\
\example{
    \begin{enumerate}
        \item $f(x,y) = \frac{x^2y}{x^2+y^2}$ et $f(0,0)=0$.\\
        On a $U = \mathbb{R}^2$, $p=2$ et $q=1$.\\
        Montrons que $f$ est continue.\\\\
        $f$ est continue sur $\mathbb{R}^2 \setminus \{(0,0)\}$ car c'est un quotient de fonctions continues (polynômes) dont le dénominateur ne s'annule pas.\\
        Montrons que $f$ est continue en $(0,0)$.\\
        On doit montrer que si $(x,y) \to (0,0)$, alors $f(x,y) \to 0$.\\
        $|f(x,y)| = \left|\frac{x^2y}{x^2+y^2}\right| = \frac{x^2 |y|}{x^2+y^2} \leq \frac{x^2 |y|}{y^2} = |y|$.\\
        Donc si $(x,y) \to (0,0)$, alors $|y| \to 0$ et donc $f(x,y) \to 0$.\\
        Ainsi, $f$ est continue en $(0,0)$.\\
        Donc $f$ est continue sur $\mathbb{R}^2$.
        \item $f(x,y) = \frac{xy}{x^2+y^2}$.\\
        $f$ peut-elle être continue sur $\mathbb{R}^2$ ?\\
        Sur $\mathbb{R}^2 \setminus \{(0,0)\}$, $f$ est continue (même raison).\\
        Problème en $(0,0)$.\\
        Montrons que $f(x,y)$ n'a pas de limite en $(0,0)$.\\
        Trouvons $(x_n,y_n) \to (0,0)$ et $(a_n,b_n) \to (0,0)$ tels que $f(x_n,y_n)$ et $f(a_n,b_n)$ aient des limites différentes.\\
        Prenons $(x_n,y_n) = \left(\frac{1}{n},\frac{1}{n}\right)$.
        On a : $f(x_n,y_n) = \frac{\frac{1}{n} \cdot \frac{1}{n}}{\left(\frac{1}{n}\right)^2 + \left(\frac{1}{n}\right)^2} = \frac{\frac{1}{n^2}}{\frac{2}{n^2}} = \frac{1}{2}$.\\
        De plus, prenons $(a_n,b_n) = \left(\frac{2}{n},\frac{1}{n}\right)$.
        On a : $f(a_n,b_n) = \frac{\frac{2}{n} \cdot \frac{1}{n}}{\left(\frac{2}{n}\right)^2 + \left(\frac{1}{n}\right)^2} = \frac{\frac{2}{n^2}}{\frac{4}{n^2} + \frac{1}{n^2}} = \frac{2}{5}$.\\
        Ainsi, $f(x_n,y_n) \to \frac{1}{2}$ et $f(a_n,b_n) \to \frac{2}{5}$.\\
        Donc $f$ n'a pas de limite en $(0,0)$ et donc $f$ n'est pas continue en $(0,0)$.
        Ainsi, $f$ n'est pas continue sur $\mathbb{R}^2$.
    \end{enumerate}
}

\section{Vers une bonne notion de dérivée pour les fonctions à plusieurs variables}

\textbf{Introduction :} Soit $f \colon ]a,b[ \to \mathbb{R}$ où $]a,b[ \subset \mathbb{R}$.
On dit que $f$ est dérivable en $x \in ]a,b[$ si il existe un réel $f'(x)$ tel que : 
\[
    f(x+h) = f(x) + hf'(x) + o(h)
\]
Ce qui équivaut à : $\lim_{h \to 0} \frac{f(x+h) - f(x)}{h}$ existe.\\
Si $f$ est dérivable en $x$ alors on sait que $f$ est continue en $x$.\\
$|f(x+h) - f(x)| \to 0$ quand $h \to 0$.

\subsection{Rappels : fonctions à valeurs dans $\mathbb{R}^q, q \geq 1$}
Soit \function{f}{]a,b[ \to \mathbb{R}^q}{x \mapsto (f_1(x), f_2(x), \ldots, f_q(x))}. (fonction coordonnées)

\definition{
    On dit que $f$ est dérivable en $x \in ]a,b[$ si $\lim_{h \to 0} \frac{f(x) - f(x+h)}{h}$ existe. (avec $\frac{f(x) - f(x+h)}{h} \in \mathbb{R}^q$).\\
    \textit{i.e.} $\exists L \in \mathbb{R}^q$ tel que $\| \frac{f(x) - f(x+h)}{h} - L \| \to 0$ quand $h \to 0$ où $\|.\|$ est une norme $\mathbb{R}^q$ (elles sont équivalentes).
}

\theorem{Proposition}{Dérivabilité}{false}{
    $f$ est dérivable en $x$ $\Leftrightarrow \forall i=1,\ldots,q, f_i$ est dérivable en $x$ et on a : $f'(x) = (f_1'(x), f_2'(x), \ldots, f_q'(x))$.\\
    Autrement dit, $f$ est dérivable en $x$ si et seulement si chacune de ses composantes est dérivable en $x$.
}

\noindent{\textbf{Preuve :}\\
    $\Rightarrow/$ Supposons $f$ dérivable en $x$.\\
    $\exists L = (L_1, L_2, \ldots, L_q) \in \mathbb{R}^q$ tel que $\| \frac{f(x+h) - f(x)}{h} - L \| \to 0$ quand $h \to 0$.\\
    On a le choix de $\|.\|$ car les normes sont toutes équivalentes (en dimension finie). Prenons la norme infinie.\\
    Alors $\forall i \in \{1,\ldots,q\}, |\frac{f_i(x+h) - f_i(x)}{h} - L_i| \leq \| \frac{f(x+h) - f(x)}{h} - L \|_{+\infty} \to 0$ quand $h \to 0$.\\
    Ainsi, $\forall i \in \{1,\ldots,q\}, f_i$ est dérivable en $x$ et $f_i'(x) = L_i$.\\
    Et donc $f'(x) = (f_1'(x), f_2'(x), \ldots, f_q'(x))$.\\
    $\Leftarrow/$ Supposons que $\forall i=1,\ldots,q, f_i$ est dérivable en $x$.\\
    Donc $\forall i \in \{1,\ldots,q\}, \frac{f_i(x+h) - f_i(x)}{h} \to f_i'(x)$ quand $h \to 0$.\\
    Montrons que $\lim_{h \to 0} \frac{f(x+h) - f(x)}{h}$ existe.\\
    Posons $L = (f_1'(x), f_2'(x), \ldots, f_q'(x)) \in \mathbb{R}^q$.\\
    On veut $\| \frac{f(x+h) - f(x)}{h} - L \| \to 0$ quand $h \to 0$.\\
    Prenons $\|.\|_1$ la norme 1.\\
    On a :
    \[
        \left\| \frac{f(x+h) - f(x)}{h} - L \right\|_1 = \sum_{i=1}^q \left| \frac{f_i(x+h) - f_i(x)}{h} - L_i \right|
    \]
    On a une somme finie donc chaque terme (par hypothèse) tend vers 0 quand $h \to 0$.\\
    En passant à la limite on obtient : $\| \frac{f(x+h) - f(x)}{h} - L \|_1 \to 0$ quand $h \to 0$.\\
    Donc $f$ est dérivable en $x$ et $f'(x) = L = (f_1'(x), f_2'(x), \ldots, f_q'(x))$.
}

\subsection{Dérivées partielles}

Soit $U \subset \mathbb{R}^p$ ouvert et $p \in \mathbb{N}^*$.
On considère $f : U \to \mathbb{R}^q$.

\remark{Penser à une fonction $f$ définie de $U=\mathbb{R}^2$ dans $\mathbb{R}$, donc $p=2$ et $q=1$. (tous les problèmes se posent déjà dans ce cas)}

On veut définir la notion de dérivées partielles en un point $a = (a_1, a_2, \ldots, a_p) \in U$.\\
Il y a $p$ dérivées partielles (car $p$ variables). Prenons $k \in \{1,\ldots,p\}$ et définissons la $k$-ième dérivée partielle de $f$ en $a=(a_1, a_2, \ldots, a_p)$.\\
Considérons la $k$-ième fonction partielle d'une variable réelle $g_k \colon x_k \mapsto f(a_1, \ldots, a_{k-1}, x_k, a_{k+1}, \ldots, a_p)$.\\
Les points $a_1, \ldots, a_{k-1}, a_{k+1}, \ldots, a_p$ sont fixés dans $\mathbb{R}$ et $x_k$ varie dans un voisinage de $a_k$.\\

On dit que $f$ admet une $k$-ième dérivée partielle en $a$ si $g_k$ est dérivable en $a_k$ et on a :
\[
    \frac{\partial f}{\partial x_k}(a) = g_k'(a_k)
\]

\example{
    Soit $f \colon (x,y) \mapsto x^2 + y^2$.\\
    Prenons $a = (a_1, a_2) \in \mathbb{R}^2$.
    Alors :
    \[
        \frac{\partial f}{\partial x}(a) = 2a_1 \quad \text{et} \quad \frac{\partial f}{\partial y}(a) = 2a_2
    \]\\

On utilise plutôt les notations suivantes :
    \[
        \frac{\partial f}{\partial x}(x, y) = 2x \quad \text{et} \quad \frac{\partial f}{\partial y}(x, y) = 2y
    \]
}

\attention{
    Avant de calculer une dérivée partielle en un point, il faut justifier qu'elles existent (i.e. que les fonctions partielles sont dérivables en les points correspondants).
}

\example{(suite) Ici, il faut dire :\\
$\forall y$ fixé dans $\mathbb{R}$, la fonction $x \mapsto f(x,y) = x^2 + y^2$ avec $y$ constant est dérivable en tout $x \in \mathbb{R}$ (c'est un polynôme). Donc $\frac{\partial f}{\partial x}(x,y)$ existe pour tout $(x,y) \in \mathbb{R}^2$.\\
De même, $\forall x$ fixé dans $\mathbb{R}$, la fonction $y \mapsto f(x,y) = x^2 + y^2$ avec $x$ constant est dérivable en tout $y \in \mathbb{R}$. Donc $\frac{\partial f}{\partial y}(x,y)$ existe pour tout $(x,y) \in \mathbb{R}^2$.
}

\remark{$f$ admet une $k$-ième dérivée partielle en $a$ $\Leftrightarrow$ la fonction $t \mapsto f(a+te_k)$ (où $e_k$ est le $k$-ième vecteur de la base canonique de $\mathbb{R}^p$) est dérivable en $0$.}

\definition{
    On dit que $f$ admet des dérivées partielles en $a$ si $\forall k \in \{1,\ldots,p\}$, $\frac{\partial f}{\partial x_k}(a)$ existe. (\textit{i.e.} toutes les dérivées partielles existent en $a$)
}

\attention{Problème : Malheureusement, ce n'est pas parce que $f$ admet des dérivées partielles en $a$ que $f$ est continue en $a$. (d'où le fait de chercher une meilleure notion de dérivabilité)}

\cexample{
    Soit $f(x,y) = \frac{xy}{x^2 + y^2}$ si $(x,y) \neq (0,0)$ et $f(0,0) = 0$.\\
    On a $U \subset \mathbb{R}^2$ ouvert, $p=2$ et $q=1$.\\
    Montrons que $f$ admet des dérivées partielles sur $\mathbb{R}^2$ mais que $f$ n'est pas continue en $(0,0)$.\\
    Montrons que $x \mapsto \frac{\partial f}{\partial x}(x,y)$ existe pour tout $y$.\\
    Fixons $y \in \mathbb{R}$.\\
    Considérons $x \mapsto f(x,y) = \begin{cases}
        \frac{xy}{x^2 + y^2} & \text{si } (x,y) \neq (0,0)\\
        0 & \text{si } (x,y) = (0,0)
    \end{cases}$ est dérivée en tout $x \in \mathbb{R}$.\\
    Ainsi $\forall y \in \mathbb{R} x \mapsto f(x,y)$ est dérivable en tout $x \in \mathbb{R}$. Donc $\frac{\partial f}{\partial x}(x,y)$ existe pour tout $(x,y) \in \mathbb{R}^2$.\\

    De même, par symétrie on montre que $y \mapsto f(x,y)$ est dérivable en tout $y \in \mathbb{R}$ pour tout $x \in \mathbb{R}$.\\
    Ainsi $f$ admet des dérivées partielles en tout point de $\mathbb{R}^2$.\\
    En particulier, $\frac{\partial f}{\partial x}(0,0)$ et $\frac{\partial f}{\partial y}(0,0)$ existent.\\
    Pourtant, on a vu que $f$ n'est pas continue en $(0,0)$.\\
}

\noindent{\textit{Transition :} On peut voir le problème de continuité en $(0,0)$ comme le fait que $f(x,x)=\frac{1}{2}$ qui ne tend pas vers $0$ quand $x \to 0$.\\
En fait, on regarde la trajectoire $t \mapsto (t,t)$ qui n'est pas alignée avec les axes.\\
Une notion plus forte que la dérivée partielle est la dérivée directionnelle.}

\subsection{Dérivées directionnelles}

Soit $f : U \to \mathbb{R}^q$ où $U \subset \mathbb{R}^p$ ouvert.\\
Soit $a \in U$ et soit $h$ un vecteur de $\mathbb{R}^p$.\\
Considérons la fonction d'une variable réelle $\varphi \colon t \mapsto f(a + th)$ où $t$ varie dans un voisinage de $0$ tel que $a + th \in U$.\\

\definition{On dit que $f$ admet une dérivée directionnelle en $a$ selon la direction $h$ si $\varphi$ est dérivable en $0$.\\
    On a :
    \[
        \varphi'(0) = \lim_{t \to 0} \frac{f(a + th) - f(a)}{t}
    \]
}

\theorem{Proposition}{Lien avec la dérivée partielle}{false}{
    Si on choisit comme direction $h = e_k$ (le $k$-ième vecteur de la base canonique de $\mathbb{R}^p$), alors la dérivée directionnelle de $f$ en $a$ selon la direction $e_k$ est la $k$-ième dérivée partielle de $f$ en $a$.\\
}
\noindent{\textbf{Preuve :}\\
    On a $\varphi(t) = f(a + te_k) = f(a_1, \ldots, a_{k-1}, a_k + t, a_{k+1}, \ldots, a_p)$.\\
    Donc $\varphi$ est dérivable en $0$ ssi la fonction $x_k \mapsto f(a_1, \ldots, a_{k-1}, x_k, a_{k+1}, \ldots, a_p)$ est dérivable en $a_k$.\\
    Ainsi, la dérivée directionnelle de $f$ en $a$ selon la direction $e_k$ est égale à la $k$-ième dérivée partielle de $f$ en $a$.
}
\training{
    Étudier la dérivée directionnelle en toutes les directions au point $(0,0)$ de la fonction $f(x,y) = \frac{xy}{x^2 + y^2}$ si $(x,y) \neq (0,0)$ et $f(0,0) = 0$.\\
}
\noindent\carreaux{15}

\attention{Cette notion de dérivée directionnelle est encore trop faible pour garantir la continuité de $f$ en $a$.}

\training{Considérer $f(x,y)=\frac{y^2}{x}$ si $x \neq 0$ et $f(x,y)=y$ sinon.\\
\begin{enumerate}
    \item Montrer que $f$ admet des dérivées directionnelles en tout point de $\mathbb{R}^2$ et en toute direction.
    \item Montrer que $f$ n'est pas continue sur $\mathbb{R}^2$.
    \textit{Indication : Considérer le point $(0,1)$ et $(\frac{1}{n},1) \to (0,1)$.}
\end{enumerate}
}
\noindent\carreaux{15}

\subsection{Dérivées différentielles (cette fois-ci c'est la bonne !)}
\textit{La dérivée différentielle est la notion qui assure que si une fonction est différentiable en un point, alors elle est continue en ce point.}
Soit $f \colon \mathbb{R}^p \to \mathbb{R}^q$ et $a \in U$, où $U \subset \mathbb{R}^p$ est ouvert.

\definition{
    On dit que $f$ est différentiable en $a$ $\Leftrightarrow \exists Df(a) \in \mathcal{L}(\mathbb{R}^p, \mathbb{R}^q)$ telle que :
    \[
        f(a+h) = f(a) + Df(a) \cdot h + o(\|h\|) \quad \text{quand } h \to 0
    \]
    avec $\forall h \in \mathbb{R}^p, f(a) \in \mathbb{R}^q, Df(a) \cdot h \in \mathbb{R}^q$, et $o(\|h\|) = \|h\| \varepsilon(h)$ où $\varepsilon(h) \to 0$ quand $\|h\| \to 0$.
} 

\example{
    $f \colon \mathbb{R} \to \mathbb{R}, f$ dérivable en $a \in \mathbb{R}$.\\
    $Df(a) \in \mathcal{L}(\mathbb{R}, \mathbb{R})$.\\
    Dans le cas d'une fonction d'une variable réelle, $Df(a) \colon h \mapsto f'(a)h$. (c'est une multiplication par un scalaire)
}

\example{
    Soit $f \in \mathcal{L}(\mathbb{R}^p, \mathbb{R}^q)$.\\
    $\forall a \in \mathbb{R}^p, \forall h \in \mathbb{R}^q, f(a+h) = f(a) + f(h)$.\\
    Ainsi, la différentielle de $f$ en $a$ est : $\forall h \in \mathbb{R}^p, Df(a) \cdot h = f(h)$.\\
    Autrement dit, $Df(a) = f$, et la différentielle est indépendante de $a$.
}

\theorem{Proposition}{Différentiabilité implique continuité}{false}{
    Si $f$ est différentiable en $a$, alors $f$ est continue en $a$.
}

\noindent{\textbf{Preuve :}\\
    $\forall h \in \mathbb{R}^p, f(a+h) = f(a) + Df(a) \cdot h + o(\|h\|)$.\\
    On a : $\|f(a+h) - f(a)\| \leq \|Df(a) \cdot h\| + \|o(\|h\|)\|$. (car $\| \cdot \|$ est une norme)\\
    $= \|Df(a) \cdot h\| + \|h\| \|\varepsilon_a(h)\|$.\\
    $\leq C_a \|h\| + \|h\| \|\varepsilon_a(h)\|$ où $C_a$ est une constante telle que $\|Df(a) \cdot h\| \leq C_a \|h\|$ (par continuité automatique des applications linéaires en dimension finie).\\
    Or $\|h\| \varepsilon_a(h) \to 0$ quand $\|h\| \to 0$.\\
    Donc $\|f(a+h) - f(a)\| \to 0$ quand $\|h\| \to 0$.\\
    Ainsi, $f$ est continue en $a$.
}

\theorem{Proposition}{Lien avec les dérivées directionnelles}{false}{
    Si $f$ est différentiable en $a$, alors $f$ admet des dérivées directionnelles en $a$ selon toutes les directions et on a :
    \[
        \forall h \in \mathbb{R}^p, \quad \text{la dérivée directionnelle de } f \text{ en } a \text{ selon la direction } h \text{ est } Df(a) \cdot h
    \]
}

\noindent{\textbf{Preuve :}\\
    On a $\forall t \in \mathbb{R}, \forall h \in \mathbb{R}^p, f(a+th) = f(a) + Df(a) \cdot (th) + \|th\| \varepsilon_a(th)$. (car $f$ est différentiable en $a$)\\
    $= f(a) + t(Df(a) \cdot h) + |t| \|h\| \varepsilon_a(th)$.\\
    On veut montrer que $t \mapsto f(a + th)$ est dérivable en $0$.\\
    Par ce qui précède, on a :\\
    $f(a+h)-f(a) = t(Df(a) \cdot h) + |t| \|h\| \varepsilon_a(th)$.\\
    $\Rightarrow \frac{f(a+th) - f(a)}{t} = Df(a) \cdot h + \frac{|t|}{t} \|h\| \varepsilon_a(th)$.\\
    En passant à la limite quand $t \to 0$, on obtient : $\
    \lim_{t \to 0} \frac{f(a+th) - f(a)}{t} = Df(a) \cdot h$.\\
    Donc $t \mapsto f(a + th)$ est dérivable en $0$ et sa dérivée en $0$ est $Df(a) \cdot h$.\\
    Ainsi, $f$ admet une dérivée directionnelle en $a$ selon la direction $h$.
}

\remark{Il existe des fonctions qui admettent des dérivées directionnelles en un point selon toutes les directions mais qui ne sont pas différentiables en ce point.}

\cexample{
    Considérons $\|\cdot\| \colon \mathbb{R}^p \to \mathbb{R}$ une norme quelconque. C'est une fonction à plusieurs variables.\\
    Montrons que $\|\cdot\|$ n'est pas différentiable en $a$ (si elle l'était, par la proposition précédente, elle serait dérivable dans toutes les directions en $a$).\\
    On a $\forall t \in \mathbb{R}, \forall h \in \mathbb{R}^p, \|th\| = |t| \|h\|$.\\
    On a $\frac{\|0 + th\| - \|0\|}{t} = \frac{\|th\|}{t} = \frac{|t|}{t} \|h\|$.\\
    En passant à la limite quand $t \to 0$, on constate que la limite n'existe pas (car $\frac{|t|}{t}$ n'a pas de limite quand $t \to 0$).\\
    Ainsi, $\|\cdot\|$ n'admet pas de dérivée directionnelle en $0$ selon la direction $h$.\\
    Donc $\|\cdot\|$ n'est pas différentiable en $0$.
}

\end{document}

