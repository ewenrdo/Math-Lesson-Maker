\documentclass{article}

\usepackage[a4paper, left=1.5cm, right=1.5cm, top=2cm, bottom=2cm]{geometry}

\usepackage{../../../../components/components} % <-- ton fichier .sty, avec toutes tes définitions

\usepackage{fancyhdr}


% Configuration des en-têtes et pieds de page
\pagestyle{fancy}
\fancyhf{} % reset tout

\fancyhead[L]{DL2 Math-Info AN3}
\fancyhead[C]{Séries}
\fancyhead[R]{2025-2026}

\fancyfoot[L]{Ewen Rodrigues de Oliveira}
\fancyfoot[R]{\thepage}

\begin{document}

\docTitle{Chapitre 2 : Séries numériques}

%% 17/09/2025

\section{Séries et sommes d'une série}

\definition{
    Soit \((u_n)\) une suite dans $\mathbb{K} = \mathbb{R}$ ou $\mathbb{C}$.\\
    On considère \(\forall N \in \mathbb{N}, S_N = \sum_{n=0}^{N} u_n \in \mathbb{K}\).\\\
    On a donc une suite $(S_N)_{N \in \mathbb{N}}$ associée à la suite $(u_n)_{n \in \mathbb{N}}$.
}
\definition{On appelle \textbf{série de terme général \(u_n\)} que l'on note \(\sum_{n \geq 0} u_n\) la suite $(S_N)_{N \in \mathbb{N}}$.}

\ndlr{Les deux définitions précédentes gagneraient à être fusionnées.}

\vocabulary{On dit que \((S_N)\) est la suite des sommes partielles de la série.}
\remark{\((S_N)\) correspond aux $N+1$ premiers termes de la suite.}

\subsection{Correspondance suite - série}

\noindent{
\textbf{Raisonnement :}
Par définition une série est une suite. Expliquons comment une suite peut-être vue comme une série.\\
Si $(u_n)$ est une suite, considérons la série de terme général \(v_n = u_n - u_{n-1} \forall n \in \mathbb{N}\) (avec la convention \(v_{0} = u_{0}\)).\\
Ainsi, \(u_n = \sum_{k=0}^{n} v_k\).}

\remark{Cependant la série associée à une suite $(u_n)$ va s'étudier en tant que telle (que série) grâce à $u_n$.}


\subsection{Opérations sur les séries}

\theorem{Propriété}{Opérations sur les séries}{true}{
    Soient \(\sum_{n\geq0} u_n\) et \(\sum_{n\geq0} v_n\) deux séries.\\
    Alors, pour tout \(\lambda \in \mathbb{K}\) :
    \begin{itemize}
        \item \textbf{Somme} : \(\sum_{n\geq0} (u_n + v_n) = \sum_{n\geq0} u_n + \sum_{n\geq0} v_n\) définie comme $(S_N  + S'_N)$
        \item \textbf{Produit par un scalaire} : \(\sum_{n\geq0} \lambda u_n = \lambda \sum_{n\geq0} u_n\) définie comme \((\lambda S_N)\)
    \end{itemize}
}

\example{Si $u_n=0, \forall n \in \mathbb{N}$, alors \(\sum_{n\geq0} u_n = 0\) est la série nulle.}


\subsection{Troncature d'une série}


\definition{
    Si $(u_n)$ est une suite définie pour \(n \geq n_0 \mid n_0 \in \mathbb{N}\).
    On peut considérer la série \(\sum_{n \geq n_0} u_n\) où \(u_{0} = u_{1} = ... = u_{n_0 - 1} = 0\), ou bien on peut écrire $\sum_{n \geq n_0} u_n$.

    Si $\sum_{n \geq 0} u_n$ est une série de terme général $u_n$, une \textbf{troncature} de la série est $\sum_{n \geq n_0} u_n$. C'est la suite $(S_N)$ où \(S_N = \sum_{n=n_0}^{N} u_n\).
}

\ndlr{Cette définition pourraît être synthétisée.}

\example{
    \begin{itemize}
        \item la série nulle
        \item la série géométrique de raison $q\in\mathbb{C^*}$ : \(\sum_{n\geq0} q^n\) de terme général \(q^n\) ;
        \item la série harmonique : \(\sum_{n\geq1} \frac{1}{n}\) de terme général \(\frac{1}{n}\) ;
        \item la série $\sum_{n\geq1} \frac{1}{n^\alpha}, \alpha\in\mathbb{R}$.
    \end{itemize}
}

\section{Convergence d'une série}
\subsection{Définitions et nature d'une série}
\definition{
    Soit \(\sum_{n\geq0} u_n\) une série.\\
    On dit que la série converge, si la suite $(S_N)$ converge, et on note $S$ la limite de $S_N$.\\
    S s'appelle la somme de la série.\\\\

    Dans ce cas, on écrit : \(\sum_{n=0}^{\infty} u_n = S \in \mathbb{R}\) \textit{(c'est une "somme infinie", un objet-limite).}
}

\vocabulary{Si $(S_N)$ diverge, alors on dit que la série $\sum_{n\geq0} u_n$ diverge.}
\attention{Si $S$ n'existe pas, alors on écrit \textbf{jamais} la notation avec \(\infty\)}

\vocabulary{La convergence ou la divergence d'une série s'appelle la \textbf{nature} de la série.}
\theorem{Proposition}{Stabilité de la limite par troncature}{true}{
    La nature d'une série n'est pas modifée par troncature.
}
\noindent\textbf{Preuve:}\\
\carreaux{12}
\ndlr{Indication : les premiers termes n'influencent pas la convergence.}

\subsection{Quelques applications...}

\example{
    \begin{itemize}
        \item Si $(u_n)$ est nulle à partir d'un rang $N_0$ alors la série $\sum_{n\geq0} u_n$ est converge, et $\sum_{n=0}^{\infty} u_n =  \sum_{n=0}^{N_0} u_n$.
        \item Série géométrique $\sum_{n\geq0} q^n$ :\\
        On considère la suite des sommes partielles $(S_N)$ où \(S_N = \sum_{n=0}^{N} q^n\) = \(\frac{1-q^{N+1}}{1-q}\) avec \(q \neq 1\). On a plusieurs cas :
        \begin{itemize}
            \item Si \(|q| < 1\), $q^{N+1} \xrightarrow[N\to\infty]{} 0$ donc \(S_N \xrightarrow[N\to\infty]{} \frac{1}{1-q} \Leftrightarrow \sum_{n=0}^{\infty} q^n = \frac{1}{1-q}\).\\
            La série $\sum_{n\geq0} q^n$ converge et on arrive à trouver S !
            \item Si \(|q| > 1\), alors $\sum_{n\geq0} q^n$ diverge.
            \item Si \(q = 1\), alors $\sum_{n\geq0} q^n = N+1 \Rightarrow \sum_{n\geq0} q^n$ diverge.
        \end{itemize}
        \item $\sum_{n\geq1}{log(1+1/n)}$ :\\
        On a \(\forall N \geq 1, S_N = \sum_{n=1}^{N} log(\frac{n+1}{n}) = log(N+1)\) (télescopage).\\
        Or \(log(N+1) \xrightarrow[N\to\infty]{} +\infty\), donc la série $\sum_{n\geq1}{log(1+1/n)}$ diverge.
        \item $\sum_{n\geq1} \frac{1}{n(n+1)}$ :\\
        On a \(\forall N \geq 1, S_N = \sum_{n=1}^{N} \frac{1}{n(n+1)} = \sum_{n=1}^{N} (\frac{1}{n} - \frac{1}{n+1}) = 1 - \frac{1}{N+1}\) (télescopage).\\
        Or \(1 - \frac{1}{N+1} \xrightarrow[N\to\infty]{} 1\), donc la série $\sum_{n\geq1} \frac{1}{n(n+1)}$ converge et \(\sum_{n=1}^{\infty} \frac{1}{n(n+1)} = 1\).
        \item \textbf{Important, démontré plus tard :} $\sum_{n\geq1} \frac{1}{n}$ (série harmonique) diverge.
        \item $\sum_{n\geq1} \frac{(-1)^n}{n}$ converge. \textit{(idée : montrer que $S_N$ converge en montant $A_N= S_{2N}$ et $B_N=S_{2N+1}$ sont adjacentes)}
    \end{itemize}
}

\attention{Ces six exemples sont à connaître et comprendre parfaitement.}

\training{Étudier la convergence de la série géométrique pour \(|q| = 1\) et \(q = -1\) ($q\in\mathbb{C}$).}
\noindent\carreaux{5}

\subsection{Propriétés des séries convergentes}

\theorem{Propriété}{Convergence de la combinaison linéaire}{true}{
    Soient \(\sum_{n\geq0} u_n\) et \(\sum_{n\geq0} v_n\) deux séries convergentes.\\
    Alors $\forall \lambda,\mu \in \mathbb{K}$ : \(\sum_{n\geq0} (\lambda u_n + \mu v_n) = \lambda \sum_{n\geq0} u_n + \mu \sum_{n\geq0} v_n\), cette série converge (vers la combinaison linéaire des limites).
}
\remark{En d'autres termes, la somme de deux séries convergentes est une série $\sum_{n\geq0} (u_n + v_n)$ qui converge.}
\noindent{
\textbf{Preuve :}\\
La suite de sommes partielles associée à $\sum_{n\geq}{(u_n + v_n)}$ est $\sum_{n=0}^{N} (u_n + v_n) = \sum_{n=0}^{N} (u_n) + \sum_{n=0}^{N} (v_n)$\\
Comme $\sum_{n=0}^{N} (u_n)$ et $\sum_{n=0}^{N} (v_n)$ sont convergentes, on a $\sum_{n=0}^{N} (u_n + v_n)$ est convergente et sa limite est \(\sum_{n=0}^{\infty} (u_n+v_n)  = \sum_{n=0}^{\infty} (u_n) + \sum_{n=0}^{\infty} (v_n)\).
}

\example{
    \textbf{Retour : Divergence de la série harmonique $\sum_{n\geq1} \frac{1}{n}$}\\
    But : minorer $\sum_{n\geq1}^{N}\frac{1}{n} \forall N \in \mathbb{N}$.\\
    $n \leq t \in \mathbb{R} \leq n+1 \Rightarrow \frac{1}{n+1} \leq \frac{1}{t} \leq \frac{1}{n}$\\
    Intégrons entre $n$ et $n+1$ : \(\int_{n}^{n+1} \frac{1}{t} dt \leq \frac{1}{n}\)\\
    Donc en sommant : \(\sum_{n=1}^{N} \int_{n}^{n+1} \frac{1}{t} dt \leq \sum_{n=1}^{N} \frac{1}{n}\)\\
    donc par Chasles : \(\int_{1}^{N+1} \frac{1}{t} dt \leq \sum_{n=1}^{N} \frac{1}{n} \forall n \in \mathbb{N}\)\\
    Or \(\int_{1}^{N+1} \frac{1}{t} dt = \ln(N+1) \xrightarrow[N\to\infty]{} +\infty\) donc \(\sum_{n=1}^{N} \frac{1}{n} \xrightarrow[N\to\infty]{} +\infty\).\\
    Donc la série harmonique diverge.
}

\theorem{Propriété}{Divergence de la combinaison linéaire}{true}{
    Soient \(\sum_{n\geq0} u_n\) une série convergente et \(\sum_{n\geq0} v_n\) une série divergente.\\
    Alors \(\sum_{n\geq0} (u_n + v_n)\)  diverge.
}
\noindent{
\textbf{Preuve :}\\
$\sum_{n=0}^{N} (u_n + v_n) = \sum_{n=0}^{N} (u_n) + \sum_{n=0}^{N} (v_n)$\\
Comme $\sum_{n=0}^{N} (u_n)$ est convergente et $\sum_{n=0}^{N} (v_n)$ est divergente, on a $\sum_{n=0}^{N} (u_n + v_n)$ est divergente.
}

\attention{Quand on considère deux séries divergentes, la situation est à étudier au cas par cas.}
\example{
    Considérons $\sum_{n\geq1} u_n$ avec $u_n = 1 \forall n\in\mathbb{N}$ et $\sum_{n\geq1} v_n$ avec $v_n = -1 \forall n\in\mathbb{N}$.\\
    D'une part $\sum_{n\geq1} u_n$ diverge, et $\sum_{n\geq1} v_n$ diverge aussi.\\
    Mais \(\sum_{n\geq1} (u_n + v_n) = \sum_{n\geq1} 0 = 0\) converge.\\\\
    Mais si on considère $v_n = u_n$, alors \(\sum_{n\geq1} (u_n + v_n) = \sum_{n\geq1} 2u_n\) diverge.
}

\attention{\textbf{Source d'erreur classique :} Si $\sum_{n\geq0} u_n + v_n$ est convergente, \textit{\textcolor{danger}{\textbf{a priori}}} on ne peut pas écrire que $\sum_{n=0}^{\infty} u_n+v_n = \sum_{n=0}^{\infty} u_n + \sum_{n=0}^{\infty} v_n$ car les séries de termes généraux $u_n$ et $v_n$ peuvent être divergentes (il faut donc vérifier leur convergence).}

\theorem{Proposition}{}{true}{
    Soit $\sum_{n\geq0} u_n$ une série numérique où $u_n \in \mathbb{C}$ $\forall n \in \mathbb{N}$.\\
    On a $\sum_{n\geq0} u_n$ converge \(\Leftrightarrow\) les suites $(Re(u_n))$ et $(Im(u_n))$ sont convergentes.\\
}

\training{Montrer la proposition précédente.}
\noindent{\textbf{Indication pour la preuve:}\\écrire $u_n = Re(u_n) + i Im(u_n)$ et utiliser la propriété sur les combinaisons linéaires.}\\
\noindent{\carreaux{10}}

\theorem{Théorème}{Lien entre convergence et limite des termes}{false}{
    Si $\sum_{n\geq0} u_n$ converge, alors $u_n \xrightarrow[n\to\infty]{} 0$.
}
\noindent{\textbf{Preuve:}\\
    Considérons $(S_N)$ la suite des sommes partielles associée à $\sum_{n\geq0} u_n$.\\
    On a $S_{N+1}-S_N = u_{N+1}$ $\forall N \in\mathbb{N}$.\\
    Or $\sum_{n\geq0} u_n$ converge $\Rightarrow (S_N)$ converge.
    Donc $\lim_{N\to\infty} S_{N} - \lim_{N\to\infty} S_{N+1} = 0 \Rightarrow \lim_{N\to\infty} u_{N} = 0$.
}

\attention{La réciproque est fausse. Par exemple la série harmonique $\sum_{n\geq1} \frac{1}{n}$ diverge mais $\frac{1}{n} \xrightarrow[n\to\infty]{} 0$.}
\vocabulary{Si $u_n \nrightarrow 0$, on dit que la série $\sum_{n\geq0} u_n$ \textcolor{purple}{\textbf{diverge grossièrement}}.}

\subsection{Reste d'une série}

\definition{
On suppose que $\sum_{n\geq0} u_n$ converge. On note $S = \sum_{n=0}^{\infty} u_n$ sa somme et $(S_N)$ la suite des sommes partielles.\\
Le \textbf{reste} de la série au rang $N$ est $R_N = S - S_N = \sum_{n=N+1}^{\infty} u_n$.
}

\theorem{Proposition}{Comportement du reste}{false}{
    Si $\sum_{n\geq0} u_n$ converge, alors $R_N \xrightarrow[N\to\infty]{} 0$.
}
\noindent{\textbf{Preuve:}\\
    Par définition, $R_N = S - S_N$. Or $S_N \xrightarrow[N\to\infty]{} S$. Donc $R_N \xrightarrow[N\to\infty]{} 0$.
}

\section{Série absolument convergente (ACV)}
\subsection{Critère de Cauchy pour les séries numériques}

Ce qui a été fait dans le \textbf{Chapitre 1 - Suites de Cauchy} sur les suites réelles reste valable si on considère des suites complexes.

\definition{
    On dit que la série $\sum_{n\geq0} u_n$ vérifie le \textbf{critère de Cauchy} si : \[\forall \varepsilon > 0, \exists N_\varepsilon \in \mathbb{N}, \forall N \geq N_\varepsilon, \forall p \in \mathbb{N}, |\sum_{k=N}^{N+p} u_k| < \varepsilon\]
}

\theorem{Proposition}{Convergence et critère de Cauchy}{false}{
    $\sum_{n\geq0} u_n$ vérifie le critère de Cauchy \(\Leftrightarrow\) $\sum_{n\geq0} u_n$ converge.
}

\noindent{\textbf{Preuve: (par équivalence)}\\
    $\sum_{n\geq0} u_n$ converge \(\Leftrightarrow\) $(S_N)$ converge
    \(\Leftrightarrow\) $(S_N)$ est une suite de Cauchy \textit{(car l'espace est complet)}
    \(\Leftrightarrow\) \(\forall \varepsilon > 0, \exists N_\varepsilon \in \mathbb{N}, \forall N \geq N_\varepsilon, \forall p \in \mathbb{N}, |S_{N+p} - S_N| < \varepsilon\)
    \(\Leftrightarrow\) \(\forall \varepsilon > 0, \exists N_\varepsilon \in \mathbb{N}, \forall N \geq N_\varepsilon, \forall p \in \mathbb{N}, |\sum_{n=N}^{N+p} u_n| < \varepsilon\)
}\newline


\remark{Autre preuve de la divergence de la série harmonique :\\
    Soit $\varepsilon = 1/2$. Pour tout $N \in \mathbb{N}$, on peut choisir $p = N$ et on a : \(|\sum_{k=N}^{2N} \frac{1}{k}| \geq \sum_{k=N}^{2N} \frac{1}{2N} = \frac{1}{2}\).\\
    Donc la série harmonique ne vérifie pas le critère de Cauchy, donc elle diverge.
}

\subsection{Définitions et propriétés}
\definition{
    On dit que la série $\sum_{n\geq0} u_n$ est absolument convergente (ACV) si la série $\sum_{n\geq0} |u_n|$ converge.
}

\theorem{Théorème}{Série ACV et convergence}{false}{
    Série ACV \(\Rightarrow\) série convergente et \(|\sum_{n=0}^{\infty} u_n| \leq \sum_{n=0}^{\infty} |u_n|\).
}
\noindent{\textbf{Preuve:}\\
    Soit $\sum_{n\geq0} u_n$ une série ACV.\\
    Donc $\sum_{n\geq0} |u_n|$ converge.\\
    Donc $\sum_{n\geq0} |u_n|$ vérifie le critère de Cauchy : \(\forall \varepsilon > 0, \exists N_\varepsilon \in \mathbb{N}, \forall N \geq N_\varepsilon, \forall p \in \mathbb{N}, |\sum_{k=N+1}^{N+p} |u_k|| < \varepsilon\)\\
    Donc \(\forall \varepsilon > 0, \exists N_\varepsilon \in \mathbb{N}, \forall N \geq N_\varepsilon, \forall p \in \mathbb{N}, |\sum_{k=N+1}^{N+p} u_k| \leq \sum_{k=N+1}^{N+p} |u_k| < \varepsilon\)\\
    Ainsi \(\forall \varepsilon > 0, \exists N_\varepsilon \in \mathbb{N}, \forall N \geq N_\varepsilon, \forall p \in \mathbb{N}, |\sum_{k=N}^{N+p} u_k| < \varepsilon\)\\
    Donc $\sum_{n\geq0} u_n$ vérifie le critère de Cauchy.\\
    Donc $\sum_{n\geq0} u_n$ converge et on a \(|\sum_{n=0}^{N} u_n| \leq \sum_{n=0}^{N} |u_n| \implies |\sum_{n=0}^{\infty} u_n| \leq \sum_{n=0}^{\infty} |u_n|\).
}\newline


\attention{La réciproque est fausse.}
\example{
    La série $\sum_{n=1}^{\infty} \frac{(-1)^{n}}{n}$ est convergente, mais elle n'est pas absolument convergente car $\sum_{n=1}^{\infty} \frac{1}{n}$ diverge.
}

\section{Convergence absolue d'une série}
\ndlr{Correspond à II. dans le plan de cours du prof.}
\subsection{Séries à termes positifs}
\theorem{Théorème}{}{false}{
    Soit $(u_n)_{n\in\mathbb{N}}$ une suite à valeurs dans $\mathbb{R}^+$.\\
    Alors la série $\sum_{n\geq0} u_n$ ($u_n \geq 0$) converge \(\Leftrightarrow\) la suite $(S_N)$ des sommes partielles est bornée.
}

\noindent{\textbf{Preuve:}\\
    En effet, $S_{N+1} - S_N = u_{N+1} \geq 0$ donc $(S_N)$ est croissante (à termes positifs).\\
    Ainsi $(S_N)$ converge \(\Leftrightarrow\) $(S_N)$ est bornée \textit{(théorème de convergence monotone)}.\\
    Or \(\sum_{n\geq0} u_n\) converge \(\Leftrightarrow\) $(S_N)$ converge.\\
    Donc \(\sum_{n\geq0} u_n\) converge \(\Leftrightarrow\) $(S_N)$ est bornée.
}

\remark{
    Si $(S_N)$ n'est pas bornée, alors $S_N \xrightarrow[N\to\infty]{} +\infty$. On tolère la notation \(\sum_{n=0}^{\infty} u_n = +\infty\).
}

\training{\textbf{Application du théorème.}\\
Soit $\sum_{n\geq0} u_n$ et $\sum_{n\geq0} v_n$ deux séries à termes positifs.\\
Montrons que la série $\sum_{n\geq0} \sqrt{u_n v_n}$ converge.\\
En effet, utilisons l'inégalité de Cauchy-Schwarz.\\
$\forall N \in \mathbb{N}, \sum_{n=0}^{N} \sqrt{u_n v_n} \leq \sqrt{\sum_{n=0}^{N} u_n} \sqrt{\sum_{n=0}^{N} v_n}$.\\
Or les deux termes de droite sont bornés, donc $\forall N \in \mathbb{N}, \sum_{n=0}^{N} \sqrt{u_n v_n}$ est bornée.\\
Donc $\sum_{n\geq0} \sqrt{u_n v_n}$ converge.
\\\\
\textbf{Autre preuve (sans Cauchy-Schwarz) :}\\
$(a-b)^2 \geq 0 \Leftrightarrow ab \leq \frac{a^2 + b^2}{2} \forall a,b \in \mathbb{R}$.\\
Donc $\sum_{n=0}^{N} \sqrt{u_n v_n} \leq \frac{1}{2} (\sum_{n=0}^{N} u_n + \sum_{n=0}^{N} v_n)$.\\
Or les deux termes de droite sont bornés, donc $\forall N \in \mathbb{N}, \sum_{n=0}^{N} \sqrt{u_n v_n}$ est bornée.\\
Donc $\sum_{n\geq0} \sqrt{u_n v_n}$ converge.
}

\ndlr{On a pas encore abordé Cauchy-Schwarz.}

\theorem{Proposition}{}{false}{
    Soient $\sum_{n\geq0} u_n$ et $\sum_{n\geq0} v_n$ deux séries convergentes \textit{(pas forcément à termes positifs mais réels)}.\\
    Si $u_n \leq v_n \forall n \in \mathbb{N}$, alors $\sum_{n=0}^{\infty} u_n \leq \sum_{n=0}^{\infty} v_n$.
}

\noindent{\textbf{Preuve:}\\
    On considère la série à termes positifs $\sum_{n\geq0} (v_n - u_n)$. C'est une série convergente.\\
    On a $\sum_{n=0}^{\infty} (v_n - u_n) \geq 0$.\\
    Or $\sum_{n\geq0} v_n$ et $\sum_{n\geq0} u_n$ sont convergentes.\\
    Donc on peut écrire : \(\sum_{n=0}^{\infty} v_n - \sum_{n=0}^{\infty} u_n = \sum_{n=0}^{\infty} (v_n - u_n) \geq 0\).\\
    Donc \(\sum_{n=0}^{\infty} u_n \leq \sum_{n=0}^{\infty} v_n\).
}

\subsection{Critère de comparaison}
Tout cela est fait pour des séries à termes positifs.
\theorem{Théorème}{Critère de comparaison ("Hyper important")}{false}{
    Soit $\sum_{n\geq0} u_n$ et $\sum_{n\geq0} v_n$ deux séries à termes positifs.\\
    Supposons que $\forall n \in \mathbb{N}, 0 \leq u_n \leq v_n$.\\
    Alors :
    \begin{itemize}
        \item Si $\sum_{n\geq0} v_n$ converge, alors $\sum_{n\geq0} u_n$ converge.
        \item Si $\sum_{n\geq0} u_n$ diverge, alors $\sum_{n\geq0} v_n$ diverge.
    \end{itemize}
}

\noindent{\textbf{Preuve:}\\
    \begin{itemize}
        \item On a $\forall N \in \mathbb{N}, 0 \leq \sum_{n=0}^{N} u_n \leq \sum_{n=0}^{N} v_n$.\\
        Or $\sum_{n\geq0} v_n$ converge, donc la suite des sommes partielles$(\sum_{n=0}^{N} v_n)$ est bornée.\\
        Donc la suite des sommes partielles $(\sum_{n=0}^{N} u_n)$ est bornée et donc $\sum_{n\geq0} u_n$ converge.
        \item Comme $\sum_{n\geq0} u_n$ diverge, la suite des sommes partielles $(\sum_{n=0}^{N} u_n)$ n'est pas bornée.\\
        Et comme $\forall N \in \mathbb{N}, 0 \leq \sum_{n=0}^{N} u_n \leq \sum_{n=0}^{N} v_n$, la suite des sommes partielles $(\sum_{n=0}^{N} v_n)$ n'est pas bornée.\\
        Et donc par le théorème de convergence des séries à termes positifs on a que $\sum_{n\geq0} v_n$ diverge.
    \end{itemize}
}

\theorem{Corollaire}{}{false}{
    Soient $\sum_{n\geq0} u_n$ et $\sum_{n\geq0} v_n$ deux séries à termes positifs.\\
    $\exists n_0 \in \mathbb{N}, \forall n \geq n_0, \frac{u_{n+1}}{u_n} \leq \frac{v_{n+1}}{v_n}$.\\
    Alors :
    \begin{itemize}
        \item Si $\sum_{n\geq0} v_n$ converge, alors $\sum_{n\geq0} u_n$ converge.
        \item Si $\sum_{n\geq0} u_n$ diverge, alors $\sum_{n\geq0} v_n$ diverge.
    \end{itemize}
}

\noindent{\textbf{Preuve:}\\
    Pour $n \geq n_0$,\\
    \(
        \frac{u_{n+1}}{u_n} \times \frac{u_n}{u_{n-1}} \times \ldots \times \frac{u_{n_0+1}}{u_{n_0}} \leq \frac{v_{n+1}}{v_n} \times \frac{v_n}{v_{n-1}} \times \ldots \times \frac{v_{n_0+1}}{v_{n_0}}\\
        \Rightarrow \frac{u_{n+1}}{u_{n_0}} \leq \frac{v_{n+1}}{v_{n_0}} \Rightarrow u_{n+1} \leq k v_{n+1}\) avec \(k = \frac{u_{n_0}}{v_{n_0}} \in \mathbb{R}_{+}^*\)

    \begin{itemize}
        \item On suppose que $\sum_{n\geq0} v_n$ converge.\\
        Donc $\sum_{n\geq0} k v_n$ converge.\\
        Donc par le théorème précédent, comme $\forall n \geq n_0, 0 \leq u_n \leq k v_n$, on a que $\sum_{n\geq0} u_n$ converge.
        \item \textit{(non démontré en cours)}
    \end{itemize}
}

\training{\textit{applications aux séries absolument convergentes}\\

    \theorem{Proposition}{}{false}{
        Soit $\sum_{n\geq0} u_n$ une série à termes réels.\\
        Définissons $u_{n}^{+} = \max(u_n,0) \geq 0$ et $u_{n}^{-} = \max(-u_n,0) \geq 0$.\\

        On a $\sum_{n\geq0} u_n$ est ACV.\\
       $\sum_{n\geq0} |u_n|$ converge \(\Leftrightarrow\) $\sum_{n\geq0} u_{n}^{+}$ et $\sum_{n\geq0} u_{n}^{-}$ convergent.\\
    }

    \noindent{\textbf{Preuve:}\\
        $\Rightarrow$/ On a $\forall n \in \mathbb{N} 0 \leq u_{n}^{+} \leq |u_n|$ et $0 \leq u_{n}^{-} \leq |u_n|$.\\
        Donc par le théorème de comparaison, $\sum_{n\geq0} u_{n}^{+}$ et $\sum_{n\geq0} u_{n}^{-}$ convergent.\\
        $\Leftarrow$/ On remarque que $|u_n| = u_{n}^{+} + u_{n}^{-}$.\\
        Si $\sum_{n\geq0} u_{n}^{+}$ et $\sum_{n\geq0} u_{n}^{-}$ convergent, alors $\sum_{n\geq0} |u_n|$ converge $\Rightarrow \sum_{n\geq0} u_n$ est ACV.
    }

    \theorem{Proposition}{}{false}{
        Soit $\sum_{n\geq0} u_n$ une série à termes complexes.\\
        On a $\sum_{n\geq0} u_n$ est ACV \(\Leftrightarrow\) $\sum_{n\geq0} Re(u_n)$ et $\sum_{n\geq0} Im(u_n)$ sont ACV.
    }
}

\training{Montrer la proposition précédente.}
\noindent\carreaux{10}


\subsection{Domination, convergence et équivalence}

\reminder{
    Soient ($u_n$) et ($v_n$) deux suites.\\
    \begin{itemize}
        \item $u_n = O(v_n)$ ssi \(\exists M > 0, |u_n| \leq M |v_n|\) au voisinage de l'infini (n assez grand) $\Leftrightarrow |\frac{u_n}{v_n}|$ est bornée.
        \item $u_n = o(v_n)$ ssi $\frac{u_n}{v_n} \xrightarrow[n\to\infty]{} 0$. ($u_n$ est négligeable devant $v_n$)
        \item $u_n = o(v_n) \Rightarrow u_n = O(v_n)$
        \item $u_n \sim v_n$ ssi $\frac{u_n}{v_n} \xrightarrow[n\to\infty]{} 1$. ($u_n$ est équivalent à $v_n$)
    \end{itemize}
}

\theorem{Proposition}{}{true}{
    Soient $\sum_{n\geq0} u_n$ et $\sum_{n\geq0} v_n$ deux séries à termes positifs.\\
    On suppose $u_n = O_{+\infty}(v_n)$.\\
    \begin{itemize}
        \item Si $\sum_{n\geq0} v_n$ converge, alors $\sum_{n\geq0} u_n$ converge.
        \item Si $\sum_{n\geq0} u_n$ diverge, alors $\sum_{n\geq0} v_n$ diverge.
    \end{itemize}
}

\noindent{\textbf{Indication pour la preuve:}\\
    Il suffit de remarquer que $\sum_{n\geq0} u_n$ et $\sum_{n\geq0} M v_n$ sont de même nature ; et $M$ est tel que $u_n \leq M v_n$
}

\attention{
    Si on sait que $\sum_{n\geq0} v_n$ alors pour montrer que $\sum_{n\geq0} u_n$ converge, il suffit de montrer que $u_n = o_{+\infty}(v_n)$.\\
    \textit{(en réalité il faudrait montrer grand O, mais o $\Rightarrow$ O donc c'est plus fort et plus simple à montrer)}
}

\theorem{Corollaire}{}{true}{
    Soit $\sum_{n\geq0} u_n$ une série à terme général dans $\mathbb{C}$ et soit $\sum_{n\geq0} v_n$ une série à terme général positif tel que $\sum_{n\geq0} v_n$ converge.\\
    Si $u_n = O_{+\infty}(v_n)$, alors $\sum_{n\geq0} u_n$ converge absolument (ACV).
}

\training{Montrer le corollaire précédent.}
\noindent\carreaux{10}

\theorem{Théorème}{"Hyper² important"}{false}{
    Soit $\sum_{n\geq0} u_n$ une série à terme général dans $\mathbb{C}$ et soit $\sum_{n\geq0} v_n$ une série à termes positifs.\\
    On suppose $u_n \sim_{+\infty} v_n$.\\ \textit{(on pourrait mettre une constante)}\\
    On a :
    \begin{itemize}
        \item Si $\sum_{n\geq0} v_n$ converge alors $\sum_{n\geq0} u_n$ converge absolument (ACV).
        \item Si $\sum_{n\geq0} v_n$ diverge alors $\sum_{n\geq0} u_n$ diverge.
    \end{itemize}
}

\remark{
    Si $u_n \geq 0$ alors $\sum_{n\geq0} u_n$ et $\sum_{n\geq0} v_n$ sont de même nature.
}

\section{Séries de références}

\subsection{Série de Riemann}

\theorem{Théorème}{}{false}{
    Soit $\alpha \in \mathbb{R}$. Soit la série $\sum_{n\geq1} \frac{1}{n^{\alpha}}$, dite \textbf{série de Riemann}.\\
    La série converge \(\Leftrightarrow \alpha > 1\).
}

\noindent{\textbf{Preuve:}\\
    On a vu que pour $\alpha = 1$, la série diverge (série harmonique).\\
    \begin{itemize}
        \item Si $\alpha \leq 1, \frac{1}{n^{\alpha}} \geq \frac{1}{n}$. Donc par le théorème de comparaison, $\sum_{n\geq1} \frac{1}{n^{\alpha}}$ diverge.
        \item $\Leftarrow/$ Supposons $\alpha > 1$.\\
        Considérons la série $\sum_{n\geq1} u_n$ de terme général $u_n = \frac{1}{n^{\alpha-1}} - \frac{1}{(n+1)^{\alpha-1}}$.\\
        \textbf{Observation 1 :} $\forall N \in \mathbb{N}^*, \sum_{n=1}^{N} u_n = 1 - \frac{1}{(N+1)^{\alpha-1}}$ donc $\sum_{n\geq1} u_n$ converge (car $\alpha - 1 > 0$). (téléscopage)\\
        \textbf{Observation 2 :} Déterminons un équivalent de $u_n$.\\
        $u_n = \frac{1}{n^{\alpha-1}} - \frac{1}{(n+1)^{\alpha-1}} = \frac{1}{n^{\alpha-1}} (1 - (\frac{n}{n+1})^{\alpha-1})$.\\
        On a $(\frac{n}{n+1})^{\alpha-1} = (\frac{n+1-1}{n+1})^{\alpha-1} = (1 - \frac{1}{n+1})^{\alpha-1} = 1 - \frac{\alpha-1}{n} + o_{+\infty}(\frac{1}{n})$ (DL ordre 1).\\
        $\Rightarrow 1 - (\frac{n}{n+1})^{\alpha-1} = \frac{\alpha-1}{n} + o_{+\infty}(\frac{1}{n}) \sim_{+\infty} \frac{\alpha-1}{n}$.\\
        Donc $u_n \sim_{+\infty} \frac{1}{n^{\alpha-1}} \times \frac{\alpha-1}{n} = \frac{\alpha-1}{n^{\alpha}} > 0$.\\
        On a deux séries à termes positifs $\sum_{n\geq1} u_n$ et $\sum_{n\geq1} \frac{\alpha-1}{n^{\alpha}}$ qui sont de même nature car équivalentes ($u_n \sim_{+\infty} \frac{\alpha-1}{n^{\alpha}}$).\\
        On en déduit que $\sum_{n\geq1} \frac{\alpha - 1}{n^{\alpha}}$ converge pour $\alpha > 1$ par le théorème sur les équivalents.\\
        De plus la nature d'une série n'est pas modifiée quand le terme général est multiplié par un scalaire non nul.\\
        Donc $\sum_{n\geq1} \frac{1}{n^{\alpha}}$ est de même nature que $\sum_{n\geq1} \frac{\alpha - 1}{n^{\alpha}}$.\\
        Donc $\sum_{n\geq1} \frac{1}{n^{\alpha}}$ converge.
    \end{itemize}
}

\attention{Démonstration probablement en question de cours au partiel/CC :)}

\theorem{Règles de comparaisons avec les séries de Riemann}{}{false}{
    Soient $\sum u_n$ une série de terme général dans $\mathbb{C}$.
    \begin{enumerate}
        \item Si $u_n \sim_{+\infty} k \frac{1}{n^{\alpha}}$ avec $k \in \mathbb{C}^*$.
        \begin{itemize}
            \item Si $\alpha > 1$ alors $\sum_{n\geq1} u_n$ converge absolument (ACV).
            \item Si $\alpha \leq 1$ alors $\sum_{n\geq1} u_n$ diverge.
        \end{itemize}
        \item Si $\exists \alpha > 1, n^{\alpha} |u_n|$ bornée \textit(i.e. $u_n = O(\frac{1}{n^{\alpha}})$), alors $\sum u_n$ converge absolument (ACV).\\
            \textit{il suffit de montrer que $u_n = o(\frac{1}{n^{\alpha}})$}
        \item On se restreint à $u_n \in \mathbb{R}$. Si $\exists \alpha \leq 1, n^{\alpha} u_n \xrightarrow[n\to\infty]{} +\infty$, alors $\sum u_n$ diverge.
 
    \end{enumerate}
}

\remark{Penser $u_n$ à terme réel positif et $k \in \mathbb{R}_{+}^{*}$ pour la compréhension. \textit{(suffisant pour la compréhension et la plupart des exercices)}}

\training{Montrer les règles de comparaison avec les séries de Riemann.}
\noindent\carreaux{25}
\training{Etudier la nature de la série de terme général $u_n = \sqrt{n^2+n+1} - \sqrt[3]{n^3+an^2+bn+c}$ avec $a,b,c \in \mathbb{R}$.}
\noindent\carreaux{10}

\subsection{Série géométrique}

\reminder{
    La série $\sum_{n\geq0} q^n$ converge \(\Leftrightarrow |q| < 1\) et dans ce cas $\sum_{n=0}^{\infty} q^n = \frac{1}{1-q}$.
}
\noindent{\textbf{Preuve:}\\
    $\Leftarrow$ Si $|q| < 1$, alors $S_N = \sum_{n=0}^{N} q^n = \frac{1-q^{N+1}}{1-q} \xrightarrow[N\to\infty]{} \frac{1}{1-q}$.\\
    $\Rightarrow$ Si $|q| \geq 1$, alors $q^n \nrightarrow 0$ donc la série diverge (grossièrement).
}

\theorem{Règle de Cauchy}{}{false}{
    Soit $\sum_{n\geq0} u_n$ une série à terme général dans $\mathbb{C}$.\\
    On suppose que $\lim_{n\to\infty} |u_n|^{\frac{1}{n}} = l$ (existe et égale à $l\in [0,+\infty]$, $+\infty$ autorisé).
    \begin{enumerate}
        \item Si $l < 1$, alors $\sum_{n\geq0} u_n$ converge absolument (ACV).
        \item Si $l > 1$, alors $\sum_{n\geq0} u_n$ diverge.
        \item Si $l = 1$, on ne peut rien conclure.
    \end{enumerate}
}
\remark{Comprendre la règle précédente dans le cas réel, terme positif.}

\noindent{\textbf{Preuve:}\\
    \begin{enumerate}
        \item Si $l < 1$, prenons $\varepsilon > 0$ tel que $l + \varepsilon < 1$.\\
        Or $|u_n|^{\frac{1}{n}} \xrightarrow[n\to\infty]{} l$, donc $\exists N \in \mathbb{N}, \forall n \geq N, |u_n|^{\frac{1}{n}} \leq l + \varepsilon$.\\
        Donc $|u_n| \leq (l + \varepsilon)^n$ pour $n \geq N$.\\
        Or la série de terme général $(l + \varepsilon)^n$ est une série géométrique de raison $l + \varepsilon < 1$, donc elle converge.\\
        Donc $\sum_{n\geq0} u_n$ converge.
        \item Laissée à la douce appréciation du lecteur.
        \item Trouvons une série $\sum_{n\geq0} u_n$  où $|u_n|^{\frac{1}{n}} \xrightarrow[n\to\infty]{} 1$ et où on ne peut rien conclure sur la nature de la série.\\
        Si on prend $u_n = \frac{1}{n^{\alpha}} = e^{-\alpha \ln(n)}$, on a bien $u_n^{\frac{1}{n}} = e^{-\alpha \frac{\ln(n)}{n}} \xrightarrow[n\to\infty]{} 1 \forall \alpha$.\\
        Or on a convergence pour $\alpha > 1$ et divergence pour $\alpha \leq 1$, on ne peut rien conclure.
    \end{enumerate}
}
\newpage

\training{Etudier la nature de la série de terme général $u_n = \cosh(\frac{1}{n})^{-n^3}$.}
\noindent\carreaux{10}

\theorem{Règle de d'Alembert}{}{false}{
    Soit $\sum u_n$ une série à terme général dans $\mathbb{C}$ .\\
    On suppose que $\lim_{n\to\infty} |\frac{u_{n+1}}{u_n}| = l$ (existe et égale à $l\in [0,+\infty]$, $+\infty$ autorisé).
    \begin{enumerate}
        \item Si $l < 1$, alors $\sum_{n\geq0} u_n$ converge absolument (ACV).
        \item Si $l > 1$, alors $\sum_{n\geq0} u_n$ diverge.
        \item Si $l = 1$, on ne peut rien conclure.
    \end{enumerate}
}

\noindent{\textbf{Preuve:}\\
    \begin{enumerate}
        \item Si $l < 1$, prenons $\varepsilon > 0$ tel que $l + \varepsilon < 1$.\\
        Or $|\frac{u_{n+1}}{u_n}| \xrightarrow[n\to\infty]{} l$, donc $\exists N \in \mathbb{N}, \forall n \geq N, |\frac{u_{n+1}}{u_n}| \leq l + \varepsilon$.\\
        Posons $q = l + \varepsilon < 1$.\\
        Ainsi, $|\frac{u_{n+1}}{u_n}| \leq \frac{q^{n+1}}{q^n}$ pour $n \geq N$.\\
        On a une comparaison du type $\frac{a_{n+1}}{a_n} \leq \frac{b_{n+1}}{b_n}$.\\
        On a vu que dans ce cas, $sum b_n$ converge $\Rightarrow \sum a_n$ converge.\\
        Or $\sum q^n$ converge (série géométrique de raison $q < 1$) donc $\sum u_n$ converge (ACV).
        \item Comme $\lim_{n\to\infty} |\frac{u_{n+1}}{u_n}| = l > 1, \exists N \in \mathbb{N}, \forall n \geq N, |\frac{u_{n+1}}{u_n}| \geq 1 \Rightarrow |u_n|$ est minorée par $n$ assez grand.\\
        Donc $\sum u_n$ diverge.
        \item Prendre $\sum_{n\geq1} \frac{1}{n^{\alpha}}$. On a bien $\frac{(n+1)^{\alpha}}{n} \xrightarrow[n\to\infty]{} 1$ et la nature dépend de $\alpha$.
    \end{enumerate}
}

\training{Etudier la nature de la série de terme général $u_n = \frac{n!}{n^n}$.}
\noindent\carreaux{13}

\newpage
\ndlr{On a évoqué en cours la formule de Stirling pour la culture, mais elle est hors programme : $n! \sim \sqrt{2 \pi n} (\frac{n}{e})^n$.}

\theorem{Proposition}{Comparaison des règles de d'Alembert et de Cauchy}{false}{
    Soit $\sum u_n$ une série à terme général positif ou nul.\\
    On suppose que $\frac{u_{n+1}}{u_n} \xrightarrow[n\to\infty]{} l \in [0,+\infty]$.\\
    Alors $u_n^{\frac{1}{n}} \xrightarrow[n\to\infty]{} l$.
}

\noindent{\textbf{Preuve:}\\
    On suppose $\lim_{n\to\infty} \frac{u_{n+1}}{u_n} = l, l>0, l\neq +\infty$.\\
    On a $\forall l_1, 0 < l_1 < l$, $\sum_{n\geq0} \frac{l_1^n}{u_n}$ converge par la règle de d'Alembert.\\
    En effet, $\frac{l_1^{n+1}}{u_{n+1}} \times \frac{u_n}{l_1^n} = l_1 \times \frac{u_n}{u_{n+1}} \xrightarrow[n\to\infty]{} \frac{l_1}{l} < 1$.\\
    Par convergence de la série on a que $\frac{l_1^n}{u_n} \xrightarrow[n\to\infty]{} 0$.\\*
    À partir d'un certain rang, $\frac{l_1^n}{u_n} \leq 1 \Rightarrow l_1^n \leq u_n \Rightarrow l_1 \leq u_n^{\frac{1}{n}}$.\\\\
    On a $\forall l_2, 0 < l < l_2$, $\sum_{n\geq0} \frac{u_n}{l_2^n}$ converge par la règle de d'Alembert.\\
    À partir d'un certain rang (même argument que pour $l_1$), $u_n \leq l_2^n \Rightarrow u_n^{\frac{1}{n}} \leq l_2$.\\\\
    Donc $l_1 \leq u_n^{\frac{1}{n}} \leq l_2$, $\forall l_1 < l < l_2$ pour un $n$ assez grand.\\
    On fait tendre $n$ vers $\infty$ puis $l_1$ et $l_2$ vers $l$ et on en déduit que $u_n^{\frac{1}{n}} \xrightarrow[n\to\infty]{} l$.\\
}

\attention{La réciproque est fausse.}
\example{Contre-exemple.\\
Soit $0 < a < b$. Posons :
\[
    u_n = \begin{cases}
        a^p b^p & \text{si n = 2p}\\
        a^{p+1} b^p & \text{si n = 2p + 1}
    \end{cases}
\]

On a $u_n^{\frac{1}{n}} \xrightarrow[n\to\infty]{} ab$ (peu importe la parité de $n$).\\
Mais $\frac{u_{n+1}}{u_n}$ dépend de la parité de $n$.
}

\remark{Donc on préfère la règle de d'Alembert à celle de Cauchy. Mais si la règle d'Alembert ne donne rien, la règle de Cauchy ne donnera rien non plus.}

\section{Séries semi-convergentes}
\example{
    $\sum_{n\geq1} \frac{e^{in\theta}}{n}$, on voit qu'elle n'est pas ACV, mais $|\frac{e^{in\theta}}{n}| = \frac{1}{n}$ $\forall \theta \in \mathbb{R}$.\\
    Mais on a pas les outils pour voir si elle est "seulement" convergente. \textit{Idem} pour la série $\sum_{n\geq1} \frac{(-1)^n}{n}$.
}

Donc le but ici, c'est de trouver des critères de convergence pour des séries qui ne sont pas ACV.

\subsection{Définitions et premières propriétés}

\definition{Une série est dite \textbf{semi-convergente} (SCV) si elle est convergente mais pas absolument convergente.}
\remark{On considère ici les les séries à terme général $u_n \in \mathbb{C}$ ou $\mathbb{R}$ (on a pas $u_n \geq 0$).}

\theorem{Proposition}{"étrange"}{false}{
    Soit $\sum_{n\geq0} u_n$ une série à terme général dans $\mathbb{R}$.\\
    On considère la série $\sum_{n\geq0} u_n^{+}$ et $\sum_{n\geq0} u_n^{-}$.\\

    On a $\sum_{n\geq0} u_n$ est SCV \(\Rightarrow\) $\sum_{n\geq0} u_n^{+}$ et $\sum_{n\geq0} u_n^{-}$ divergent.
}

\noindent{\textbf{Preuve:}\\
    On rappelle que $u_n^{+} = \max(u_n,0)$ et $u_n^{-} = \max(-u_n,0)$ et donc $u_n = u_n^{+} - u_n^{-}$ (2) et $|u_n| = u_n^{+} + u_n^{-}$ (1).\\
    \begin{itemize}
        \item Si $\sum_{n\geq0} u_n^{+}$ et $\sum_{n\geq0} u_n^{-}$ convergent, alors $\sum_{n\geq0} u_n$ ACV par (1) : \textbf{absurde}.
        \item Si l'une des séries converge et l'autre diverge, alors $\sum_{n\geq0} u_n$ diverge par (2).
        \item Seule possibilité donc : $\sum_{n\geq0} u_n^{+}$ et $\sum_{n\geq0} u_n^{-}$ divergent.
    \end{itemize}
}

\theorem{Proposition}{}{false}{
    Considérons $\sum_{n\geq0} u_n$ une série à terme général dans $\mathbb{C}$.\\
    Alors on a :\\
    $\sum_{n\geq0} u_n$ est SCV \(\Leftrightarrow\) $\sum_{n\geq0} Re(u_n)$ et $\sum_{n\geq0} Im(u_n)$ sont CV et l'une d'entre elles est SCV.
}

\noindent{\textbf{Preuve:}\\
    $\Rightarrow/$ Si $\sum_{n\geq0} u_n$ est CV, alors $\sum{n=0}^{N} u_n = \sum_{n=0}^{N} Re(u_n) + i \sum_{n=0}^{N} Im(u_n)$.\\
    Donc on a la CV des séries $\sum_{n\geq0} Re(u_n)$ et $\sum_{n\geq0} Im(u_n)$.\\
    Montrons que l'une des deux séries n'est pas ACV.\\
    En effet on a $\forall n \in \mathbb{N}, |u_n| \leq |Re(u_n)| + |Im(u_n)|$ si $\sum_{n\geq0} |Re(u_n)|$ et $\sum_{n\geq0} |Im(u_n)|$ ACV.\\
    $\Rightarrow$ une des deux séries n'est pas ACV.\\
    $\Rightarrow$ une des deux séries est ACV.\\\\

    $\Leftarrow/$ On a que $\sum_{n\geq0} Re(u_n)$ et $\sum_{n\geq0} Im(u_n)$ sont CV.\\
    Donc $\sum_{n\geq0} u_n$ est CV.\\
    Montrons que $\sum_{n\geq0} u_n$ est SCV.\\
    On a : $|Re(u_n)| \leq |u_n|$ et $|Im(u_n)| \leq |u_n|$, $\forall n \in \mathbb{N}$.\\
    Si $\sum_{n\geq0} u_n$ était ACV, alors $\sum_{n\geq0} Re(u_n)$ et $\sum_{n\geq0} Im(u_n)$ seraient ACV ce qui est contraite à l'hypothèse "l'une d'entre elles est SCV".
}

\subsection{Critère d'Abel}
\training{On veut donner un critère pour la convergence d'une série du type $\sum_{n\geq1} \frac{e^{in\theta}}{n} = a_n b_n$ avec $a_n = e^{in\theta}$ et $b_n = \frac{1}{n}$.}


\theorem{Théorème}{Critère d'Abel}{false}{
    On considère la série $\sum_{n\geq0} u_n$ où $\sum_{n\geq0} u_n \in \mathbb{C}$, avec $u_n = a_n b_n$ tels quels :
    \begin{enumerate}
        \item $(a_n)$ est \textbf{réelle}, \textbf{décroissante}, et \textbf{\(\lim_{n\to\infty} a_n = 0\)}.
        \item $(b_n)$ est \textbf{complexe} telle que $B_N = \sum_{n=0}^{N} b_n$, \textit{i.e.} $(B_N)$ est \textbf{bornée}.
    \end{enumerate}
    Alors la série $\sum_{n\geq0} a_n b_n$ converge.
}

\reminder{Une suite complexe est bornée : $\exists M > 0, \forall n \in \mathbb{N}, |z_n| \leq M$, où $|z_n| = \sqrt{Re(z_n)^2 + Im(z_n)^2}$.}

\noindent{\textbf{Preuve:}\\
On va utiliser la "transformation d'Abel".\\
On a $B_N = \sum_{n=0}^{N} b_n$.\\
Alors $B_k - B_{k-1} = b_k$, $\forall k \geq 1$ et $B_0 = b_0$.\\

On part de la somme partielle de la série :\\
\(
\forall n \in \mathbb{N}, \sum_{k=1}^{N} a_k b_k = a_0 b_0 + \sum_{k=1}^{N} a_k (B_k - B_{k-1})\\
    = a_0 b_0 + \sum_{k=1}^{N} a_k B_k - \sum_{k=1}^{N} a_k B_{k-1}\\
    = a_0 b_0 + \sum_{k=1}^{N} a_k B_k - \sum_{k=0}^{N-1} a_{k+1} B_k\\
    = a_0 b_0 + \sum_{k=1}^{N-1} (a_k - a_{k+1}) B_k + a_N B_N - a_1 b_0
    = \sum_{k=0}^{N} (a_k - a_{k+1}) B_k + a_n B_N \text{ avec } a_n \text{ tend vers 0 et } B_n \text{ bornée}\\
    \)

Etude de $\sum_{k=0}^{N} (a_k - a_{k+1}) B_k$, séries à termes dans \(\mathbb{C}\).\\
Etudions donc l'ACV :\\
\(
|a_k - a_{k+1} B_k| |a_k - a_{k+1}| |B_k| \leq (a_k - a_{k+1}) M \text{ car } |B_k| \leq M\\
\)
Or la série $\sum_{k=0}^{N} (a_k - a_{k+1}) M$ est de même nature que $\sum_{k=0}^{N} a_k - a_{k+1}$ (car $M$ est un scalaire non nul).\\
Et la CV de cette série téléscopique est évidente.\\
}

\ndlr{Il y avait beaucoup d'indices et d'infos, j'attends la vérification de Laurent pour être sûr que c'est correct (j'ai un doute sur la fin).}

\training{Pour $\theta \neq 2\pi k, \sum_{n\geq1} \frac{e^{in\theta}}{n}$ converge par le critère d'Abel, $a_n = \frac{1}{n}$ et $b_n = e^{in\theta}$ et $\sum_{n\geq1} b_n = \frac{1-e^{i(N+1)\theta}}{1-e^{i\theta}}$ et $|B_N| = |\frac{1-e^{i(N+1)\theta}}{1-e^{i\theta}}| \leq \frac{|1| + |e^{i(N+1)\theta}|}{|1-e^{i\theta}|} = \frac{2}{|1-e^{i\theta}|}$.\\
    Donc la série $\sum_{n\geq1} \frac{e^{in\theta}}{n}$ converge, mais pas ACV, donc elle est SCV.
}

\training{Etudier la convergence, l'absolue convergence et la semi-convergence de la série $\sum_{n\geq1} \frac{e^{in\theta}}{n^{\alpha}}$ avec $\theta \in \mathbb{R}, \alpha \in \mathbb{R}_{+}^*$.}
\noindent\carreaux{15}

\remark{Dans le critère d'Abel, comme $(a_n)$ est décroissante et $a_n \xrightarrow[n\to\infty]{} 0$, $a_n \geq 0$ (car $a_n \in \mathbb{R}$).}

\subsection{Séries alternées}

\definition{Une série $\sum_{n\geq0} u_n$ est dite \textbf{alternée} si $u_n = (-1)^n a_n$ ou $u_n = (-1)^{n+1} a_n$ avec $a_n \geq 0$.}

\example{$\sum_{n\geq0} \frac{(-1)^n}{n}$, $\sum_{n\geq0} (-1)^n$ sont des séries alternées.}

\remark{$(-1)^n \cdot u_n = (-1)^{2n}a_n = a_n$ ou $u_n = -a_n \Rightarrow (-1)^n u_n$ est de signe constant}
\remark{Une définition équivalente est : une série est alternée si le signe de $(-1)^n \cdot u_n$ est constant.}


\theorem{Théorème}{Critère spécial des séries alternées (CSSA)}{false}{
    Soit $\sum_{n\geq0} u_n$ une série de terme général $u_n = (-1)^n a_n$, avec $a_n \geq 0$.\\
    Si :
    \begin{enumerate}
        \item $(a_n)$ est décroissante.
        \item $\lim_{n\to\infty} a_n = 0$.
    \end{enumerate}
    Alors la série $\sum_{n\geq0} u_n$ converge.\\
}

\noindent{\textbf{Preuve:}\\
    On applique le critère d'Abel avec $a_n = a_n$ et $b_n = (-1)^n$.\\
    On a bien $a_n \xrightarrow[n\to\infty]{} 0$ et $(a_n)$ décroissante.\\
    De plus, $B_N = \sum_{n=0}^{N} (-1)^n = \frac{1-(-1)^{N+1}}{1-(-1)}$ est bornée (égale à 0 ou 1).\\
    Donc la série $\sum_{n\geq0} u_n$ converge.
}

\theorem{Proposition}{}{false}{
    Soit $\sum_{n\geq0} (-1)^n a_n$ une série alternée vérifiant les hypothèses du CSSA (donc $(a_n)$ est décroissante et $\lim_{n\to\infty} a_n = 0$).\\
    On considère la suite des sommes partielles $(S_N)$ avec $S_N = \sum_{k=0}^{N} (-1)^k a_k$.\\
    Soit $S$ la somme de la série.\\

    Alors :\\
    $S_{2N+1} \leq S \leq S_{2N}$ et $|R_N| = |S - S_N| \leq a_{N+1}$.
}

\noindent{\textbf{Preuve:}\\
    On pouse $A_N = S_{2N}$ et $B_N = S_{2N+1}$.\\
    On observe que $S_{2N+1}-S_{2N} = -a_{2N+1} \leq 0$\\
    $\Leftrightarrow S_{2N+1} \leq S_{2N}$.\\\\

    \noindent\textit{Variations de $(A_N)$ et $(B_N)$}\\
    $A_{N+1} - A_N = S_{2N+2} - S_{2N} = a_{2N+2} - a_{2N+1} \leq 0$ car $(a_n)$ décroissante.\\
    $\Leftrightarrow A_{N+1} \leq A_N$. Donc $(A_N)$ est décroissante et $B_{N+1} - B_N = S_{2N+3} - S_{2N+1} = a_{2N+2} - a_{2N+3} \geq 0$ car $(a_n)$ décroissante.\\
    $\Leftrightarrow B_{N+1} \geq B_N$. Donc $(B_N)$ est croissante.\\\\

    \noindent De plus, on a $B_N - A_N \xrightarrow[N\to\infty]{} 0$ donc $(A_N)$ et $(B_N)$ sont adjacentes, et convergent vers la même limite $S$.\\
    et donc $B_N \leq S \leq A_N$ où $S = \lim_{N\to\infty} A_N = \lim_{N\to\infty} B_N$.\\\\
    On a bien $S_{2N+1} \leq S \leq S_{2N}, \forall N \in \mathbb{N}$.\\\\

    \noindent Etudions maintenant le reste.\\
    $R_N = S - S_N$, on veut montrer que $|R_N| \leq a_{N+1}$.\\
    Séparons le cas $N$ pair et impair :\\
    \begin{itemize}
        \item Si $N = 2p+1$, alors $S_{2p+1} \leq S \implies S - S_{2p+1} \geq 0$.\\
        $\Rightarrow |R_{2p+1}| = S - S_{2p+1} \leq S_{2p+2} - S_{2p+1} = a_{2p+2} = a_{N+1}$.
        \item \textit{Laissé en exercice au lecteur :)} $\Box$
    \end{itemize}
   
}
\newpage

\attention{
    \begin{enumerate}
        \item Si deux suites sont équivalentes ($\sim$) et l'une monotone, l'autre ne l'est pas forcément.
        \example{$a_n = \frac{1}{\sqrt{n}+(-1)^n}$ et $b_n = \frac{1}{\sqrt{n}}$. On a $a_n \sim b_n$ mais $(a_n)$ n'est pas monotone (on le montre en encadrant/calculant 3 termes consécutifs (2p, 2p+1, 2p+2), alors que $(b_n)$ l'est).}
        \item Considérons $\sum_{n\geq0} (-1)^n a_n$. On remarque que $\sum_{n\geq0} (-1)^n a_n$ n'est pas ACV. Est-elle semi-convergente ?\\
        Le CSSA ne s'applique pas. Mais $(-1)^n a_n \sim \frac{(-1)^n}{\sqrt{n}}$ QUI N'IMPLIQUE PAS "$\sum_{n\geq0} (-1)^n a_n$ CV car $\sum_{n\geq0} \frac{(-1)^n}{\sqrt{n}}$ CV" (car $\sum_{n\geq0} \frac{(-1)^n}{\sqrt{n}}$ n'est pas positive).\\
        \textbf{À faire :} Montrer que $(-1)^n a_n = \frac{(-1)^n}{\sqrt(n)} + b_n$, où $b_n = \frac{-1}{\sqrt{n}(\sqrt{n}+(-1)^n)}$ et en déduire que $\sum u_n$ DV.

    \end{enumerate}
    Donc $u_n \sim v_n$, $\sum v_n$ CV $\implies \sum u_n CV$ que si $v_n est \geq 0 ou \leq 0$
}

\section{Produit de Cauchy de deux séries}

\definition{
    Soient $\sum_{n\geq0} a_n$ et $\sum_{n\geq0} b_n$ deux séries.\\
    La \textbf{série produit (de Cauchy)} est définie par la série $\sum_{n\geq0} c_n$ où $c_n = \sum_{p+q=n} a_p b_q = \sum_{k=0}^{n} a_k b_{n-k}, \forall n \in \mathbb{N}$.
}

\remark{Supposons que $a_n = 0 = b_n$ pour $n > N\in\mathbb{N}$. Considérons $P(X) = a_0 + a_1 X + ... + a_n X^n$ et $Q(X) = b_0 + b_1 X + ... + b_n X^n$.\\
Alors $(PQ)(X) = c_0 + c_1 X + ... + c_{2N} X^{2N}$.  On peut penser au produit de Cauchy comme une "généralisation".}

\theorem{Proposition}{}{false}{
    On considère $\sum_{n\geq0} a_n$ et $\sum_{n\geq0} b_n$ deux séries à termes positifs et convergentes.\\
    Alors la série produit $\sum_{n\geq0} c_n$ est convergente et on a : $\sum_{n=0}^{\infty} c_n = (\sum_{n=0}^{\infty} a_n)(\sum_{n=0}^{\infty} b_n)$.
}

\noindent{\textbf{Preuve:}\\
    Soient $A_n = \sum_{n=0}^{N} a_n$ et $B_N = \sum_{n=0}^{N} b_n$.\\
    Notons $C_N = \sum_{n=0}^{N} c_n$ = $\sum_{n=0}^{N} \sum_{k=0}^{n} a_k b_{n-k}$.\\\\
    On veut monrer que $(C_N)$ converge et déterminer sa limite.\\\\
    $(C_n)$ est une somme partielle à termes positifs, donc $(C_N)$ est croissante.\\
    Posons $I_N = \{ 0, \ldots, N\} \subset \mathbb{N}$\\
    \ndlr{Dessin $I_N x I_N$}
    Considérons $A_N B_N = \sum_{(p,q) \in I_N x I_N} a_p b_q$.\\
    Mais $C_N = \sum_{n=0}^{N} c_n = \sum_{n=0}^{N} \sum_{p+q=n} a_p b_q = \sum_{(p,q) \in I_N^2, p+q \leq N} a_p b_q$.\\
    On a $\{ (p,q) \mid p+q \leq N\} \subset \{(p,q) \mid p,q\in I_N\}$, donc $C_N \leq A_N B_N$  (1) qui est bornée car $A_N$ CV et $B_N$ CV $\implies C_N$ bornée.\\
    On a aussi l'inégalité : $A_N B_N \leq C_{2N} (2)$\\
    \ndlr{Deuxième schema}
    car $\{(p,q) \mid p+q \leq N\} \supset \{(p,q) \mid 0 \leq p,q \leq N\}$.
    On obtient $\lim_{+\infty} c_n =  \lim_{+\infty} (A_N B_N) = (\lim_{+\infty} A_N)\cdot(\lim_{+\infty} B_N)$. $\Box$
}

\theorem{Théorème}{}{false}{
    Soient $\sum_{n\geq0} u_n$ et $\sum_{n\geq0} v_n$ deux séries à termes dans $\mathbb{C}$.\\
    Si les séries sont ACV, alors la série produit $\sum_{n\geq0} c_n$ est ACV.
}

\noindent{\textbf{Preuve:}\\
    On considère $A_N = \sum_{n=0}^{N} |a_n|$, $B_N = \sum_{n=0}^{N} |b_n|$ et $C_N = \sum_{n=0}^{N} |c_n|$.\\
    D'après la proposition précédente et sa démonstration, on a $A_N B_N - C_N \xrightarrow[N\to\infty]{} 0$. (on va utiliser cette propriété)\\
    On a $\forall N \in \mathbb{N}, |(\sum_{n=0}^{N} a_n)(\sum_{n=0}^{N} |b_n|) - (\sum_{n=0}^{N} |c_n|)|$.\\
    On peut donc écrire : \[
|(\sum_{n=0}^{N} a_n)(\sum_{n=0}^{N} |b_n|) - (\sum_{n=0}^{N} |c_n|)|
= |\sum_{p\in I_N} \sum_{q \in I_N} a_p b_q - \sum_{(p,q) \in I_N^2, p+q \leq N} a_p b_q|\\
= |\sum_{(p,q) \in I_N^2} a_p b_q  - \sum_{(p,q) \in J_N^2} a_p b_q| \\ \text{ où } J_N = \{(p,q) \mid p+q \leq N\}\]
Or $J_N \subset I_N^2$, donc \[
|\sum_{(p,q) \in I_N^2} a_p b_q  - \sum_{(p,q) \in J_N^2} a_p b_q| 
= |\sum_{(p,q) \in I_N^2 \setminus J_N^2} a_p b_q| = \sum_{(p,q) \in K_N} |a_p b_q|\] où $K_N = I_N^2 \setminus J_N^2 = \{(p,q) \mid p+q > N\}$.\\
\[\leq \sum_{(p,q) \in K_N} |a_p| |b_q| = (\sum_{n=0}^{N} |a_n|)(\sum_{n=0}^{N} |b_n|) - \sum_{n=0}^{N} |c_n| = A_N B_N - C_N \xrightarrow[N\to\infty]{} 0\] par la proposition précédente et l'inégalité triangulaire.
}

\ndlr{À changer dans la démo $A_N$ en $A_N'$ et $B_N$ en $B_N'$.\\
De plus, il faut mettre $C_N' = \sum_{n=0}^{N} |c_n'| = \sum_{n=0}^{N} |a_k| |b_{n-k}|$ et pas $|\sum_{n=0}^{N} a_k b_{n-k}|$.
}

\remark{
    L'hypothèse d'absolue convergence pour $\sum a_n$ et $\sum b_n$ est très importante dans le théorème.\\
    L'hypothèse de positivité dans la proposition qui précède le théorème est fondamentale.
}

\example{
    On considère la série de terme général $u_n = \frac{(-1)^n}{\sqrt{n}}$.\\
    \begin{itemize}
        \item $u_n$ n'est pas positive.
        \item On a pas l'absolue convergence.
        \item Le CSSA s'applique car $a_n = \frac{1}{\sqrt{n}}$ est positive, décroissante et tend vers 0.
    \end{itemize}

    Considérons le produit de Cauchy.
    $(\sum_{n \geq 1} u_n)(\sum_{n \geq 1} u_n) = \sum_{n \geq 1} c_n$ où $c_n = \sum_{k=1}^{n-1} \frac{(-1)^k}{\sqrt{k}} \cdot \frac{(-1)^{n-k}}{\sqrt{n-k}}$\\
    Montrons que $\sum_{n \geq 1} c_n$ diverge (en montrant que ça ne tend pas vers 0).\\
    On a $|c_n| = |\sum_{k=1}^{n-1} \frac{1}{\sqrt{k \cdot (n-k)}}|$
    
    On a $k(n-k) \leq kn-k^2 \leq kn \leq (n-1)n$.\\
    Donc $|c_n| = |\sum_{k=1}^{n-1} \frac{1}{\sqrt{k \cdot (n-k)}}| \geq \sum_{k=1}^{n-1} \frac{1}{\sqrt{(n-1)n}} = \frac{n-1}{\sqrt{(n-1)n}} = \sqrt{\frac{n-1}{n}}$.\\
    Donc $|c_n| \nrightarrow 0$.\\
    \textbf{Conclusion :} Pour faire le produit de Cauchy de deux séries, il faut :
    \begin{enumerate}
        \item Que les deux séries soient ACV.
        \item ou Que les deux séries soient à termes positifs et CV.
    \end{enumerate}
}

\training{
    Fixons $z \in \mathbb{C}$. Etudions la convergence de la série $\sum_{n\geq0} \frac{z^n}{n!}$.\\
    1. Montrons que $\forall n \in \mathbb{N}$ la série ACV.\\ 
    On va utiliser la règle de d'Alembert : $\frac{|u_{n+1}|}{|u_n|} = \frac{|z|}{(n+1)}.$\\
    Donc $\forall z \in \mathbb{C}, \frac{z^n}{n!} \xrightarrow[n\to\infty]{} 0$.\\
    Donc $\forall z \in \mathbb{C}, \sum_{n\geq0} u_n(z)$ est ACV.\\
    \remark{On a le bon goût de pouvoir appeller $\sum_{n\geq0} \frac{z^n}{n!} := exp(z)$. \textit{(je dis bon goût mais ça risque de faire mal bientôt)}}
    2. Calculons $exp(z) \cdot exp(z')$ avec $z,z' \in \mathbb{C}$.\\
    Comme les deux séries sont ACV, on peut faire le produit de Cauchy.\\
    $exp(z) \cdot exp(z') = \sum_{n\geq0} c_n$ où $c_n = \sum_{k=0}^{n} \frac{z^k}{k!} \cdot \frac{(z')^{n-k}}{(n-k)!}$\\
    On a $c_n = \frac{1}{n!} \sum_{k=0}^{n} \frac{n!}{k!(n-k)!} z^k (z')^{n-k} = \frac{(z+z')^n}{n!}$ par le binôme de Newton.\\
    Donc $exp(z) \cdot exp(z') = \sum_{n\geq0}^{+\infty} \frac{(z+z')^n}{n!} = exp(z+z')$.
}

\section{Hors-programme : Séries commutativement convergentes}
\ndlr{On a traité de ça en parlant rapidement de permutations. À voir chez Laurent si c'est nécessaire à mettre, mais je l'omets ici pour l'instant.}


\end{document}