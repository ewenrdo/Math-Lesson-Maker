\documentclass{article}

\usepackage[a4paper, left=1.5cm, right=1.5cm, top=2cm, bottom=2cm]{geometry}

\usepackage{../../../../../components/components} % <-- ton fichier .sty, avec toutes tes définitions

\usepackage{fancyhdr}


% Configuration des en-têtes et pieds de page
\pagestyle{fancy}
\fancyhf{} % reset tout

\fancyhead[L]{DL2 Math-Info AN3}
\fancyhead[C]{Séries numériques}
\fancyhead[R]{2025-2026}

\fancyfoot[L]{Ewen Rodrigues de Oliveira}
\fancyfoot[R]{\thepage}

\begin{document}

\docTitle{Chapitre 2.3 : Séries semi-convergentes et produit de Cauchy}

\section{Séries semi-convergentes}
\example{
    $\sum_{n\geq1} \frac{e^{in\theta}}{n}$, on voit qu'elle n'est pas ACV, mais $|\frac{e^{in\theta}}{n}| = \frac{1}{n}$ $\forall \theta \in \mathbb{R}$.\\
    Mais on a pas les outils pour voir si elle est "seulement" convergente. \textit{Idem} pour la série $\sum_{n\geq1} \frac{(-1)^n}{n}$.
}

Donc le but ici, c'est de trouver des critères de convergence pour des séries qui ne sont pas ACV.

\subsection{Définitions et premières propriétés}

\definition{Une série est dite \textbf{semi-convergente} (SCV) si elle est convergente mais pas absolument convergente.}
\remark{On considère ici les les séries à terme général $u_n \in \mathbb{C}$ ou $\mathbb{R}$ (on a pas $u_n \geq 0$).}

\theorem{Proposition}{"étrange"}{false}{
    Soit $\sum_{n\geq0} u_n$ une série à terme général dans $\mathbb{R}$.\\
    On considère la série $\sum_{n\geq0} u_n^{+}$ et $\sum_{n\geq0} u_n^{-}$.\\

    On a $\sum_{n\geq0} u_n$ est SCV \(\Rightarrow\) $\sum_{n\geq0} u_n^{+}$ et $\sum_{n\geq0} u_n^{-}$ divergent.
}

\noindent{\textbf{Preuve:}\\
    On rappelle que $u_n^{+} = \max(u_n,0)$ et $u_n^{-} = \max(-u_n,0)$ et donc $u_n = u_n^{+} - u_n^{-}$ (2) et $|u_n| = u_n^{+} + u_n^{-}$ (1).\\
    \begin{itemize}
        \item Si $\sum_{n\geq0} u_n^{+}$ et $\sum_{n\geq0} u_n^{-}$ convergent, alors $\sum_{n\geq0} u_n$ ACV par (1) : \textbf{absurde}.
        \item Si l'une des séries converge et l'autre diverge, alors $\sum_{n\geq0} u_n$ diverge par (2).
        \item Seule possibilité donc : $\sum_{n\geq0} u_n^{+}$ et $\sum_{n\geq0} u_n^{-}$ divergent.
    \end{itemize}
}

\theorem{Proposition}{}{false}{
    Considérons $\sum_{n\geq0} u_n$ une série à terme général dans $\mathbb{C}$.\\
    Alors on a :\\
    $\sum_{n\geq0} u_n$ est SCV \(\Leftrightarrow\) $\sum_{n\geq0} Re(u_n)$ et $\sum_{n\geq0} Im(u_n)$ sont CV et l'une d'entre elles est SCV.
}

\noindent{\textbf{Preuve:}\\
    $\Rightarrow/$ Si $\sum_{n\geq0} u_n$ est CV, alors $\sum{n=0}^{N} u_n = \sum_{n=0}^{N} Re(u_n) + i \sum_{n=0}^{N} Im(u_n)$.\\
    Donc on a la CV des séries $\sum_{n\geq0} Re(u_n)$ et $\sum_{n\geq0} Im(u_n)$.\\
    Montrons que l'une des deux séries n'est pas ACV.\\
    En effet on a $\forall n \in \mathbb{N}, |u_n| \leq |Re(u_n)| + |Im(u_n)|$ si $\sum_{n\geq0} |Re(u_n)|$ et $\sum_{n\geq0} |Im(u_n)|$ ACV.\\
    $\Rightarrow$ une des deux séries n'est pas ACV.\\
    $\Rightarrow$ une des deux séries est ACV.\\\\

    $\Leftarrow/$ On a que $\sum_{n\geq0} Re(u_n)$ et $\sum_{n\geq0} Im(u_n)$ sont CV.\\
    Donc $\sum_{n\geq0} u_n$ est CV.\\
    Montrons que $\sum_{n\geq0} u_n$ est SCV.\\
    On a : $|Re(u_n)| \leq |u_n|$ et $|Im(u_n)| \leq |u_n|$, $\forall n \in \mathbb{N}$.\\
    Si $\sum_{n\geq0} u_n$ était ACV, alors $\sum_{n\geq0} Re(u_n)$ et $\sum_{n\geq0} Im(u_n)$ seraient ACV ce qui est contraite à l'hypothèse "l'une d'entre elles est SCV".
}

\subsection{Critère d'Abel}
\training{On veut donner un critère pour la convergence d'une série du type $\sum_{n\geq1} \frac{e^{in\theta}}{n} = a_n b_n$ avec $a_n = e^{in\theta}$ et $b_n = \frac{1}{n}$.}


\theorem{Théorème}{Critère d'Abel}{false}{
    On considère la série $\sum_{n\geq0} u_n$ où $\sum_{n\geq0} u_n \in \mathbb{C}$, avec $u_n = a_n b_n$ tels quels :
    \begin{enumerate}
        \item $(a_n)$ est \textbf{réelle}, \textbf{décroissante}, et \textbf{\(\lim_{n\to\infty} a_n = 0\)}.
        \item $(b_n)$ est \textbf{complexe} telle que $B_N = \sum_{n=0}^{N} b_n$, \textit{i.e.} $(B_N)$ est \textbf{bornée}.
    \end{enumerate}
    Alors la série $\sum_{n\geq0} a_n b_n$ converge.
}

\reminder{Une suite complexe est bornée : $\exists M > 0, \forall n \in \mathbb{N}, |z_n| \leq M$, où $|z_n| = \sqrt{Re(z_n)^2 + Im(z_n)^2}$.}

\noindent{\textbf{Preuve:}\\
On va utiliser la "transformation d'Abel".\\
On a $B_N = \sum_{n=0}^{N} b_n$.\\
Alors $B_k - B_{k-1} = b_k$, $\forall k \geq 1$ et $B_0 = b_0$.\\

On part de la somme partielle de la série :\\
\(
\forall n \in \mathbb{N}, \sum_{k=1}^{N} a_k b_k = a_0 b_0 + \sum_{k=1}^{N} a_k (B_k - B_{k-1})\\
    = a_0 b_0 + \sum_{k=1}^{N} a_k B_k - \sum_{k=1}^{N} a_k B_{k-1}\\
    = a_0 b_0 + \sum_{k=1}^{N} a_k B_k - \sum_{k=0}^{N-1} a_{k+1} B_k\\
    = a_0 b_0 + \sum_{k=1}^{N-1} (a_k - a_{k+1}) B_k + a_N B_N - a_1 b_0
    = \sum_{k=0}^{N} (a_k - a_{k+1}) B_k + a_n B_N \text{ avec } a_n \text{ tend vers 0 et } B_n \text{ bornée}\\
    \)

Etude de $\sum_{k=0}^{N} (a_k - a_{k+1}) B_k$, séries à termes dans \(\mathbb{C}\).\\
Etudions donc l'ACV :\\
\(
|a_k - a_{k+1} B_k| |a_k - a_{k+1}| |B_k| \leq (a_k - a_{k+1}) M \text{ car } |B_k| \leq M\\
\)
Or la série $\sum_{k=0}^{N} (a_k - a_{k+1}) M$ est de même nature que $\sum_{k=0}^{N} a_k - a_{k+1}$ (car $M$ est un scalaire non nul).\\
Et la CV de cette série téléscopique est évidente.\\
}

\ndlr{Il y avait beaucoup d'indices et d'infos, j'attends la vérification de Laurent pour être sûr que c'est correct (j'ai un doute sur la fin).}

\training{Pour $\theta \neq 2\pi k, \sum_{n\geq1} \frac{e^{in\theta}}{n}$ converge par le critère d'Abel, $a_n = \frac{1}{n}$ et $b_n = e^{in\theta}$ et $\sum_{n\geq1} b_n = \frac{1-e^{i(N+1)\theta}}{1-e^{i\theta}}$ et $|B_N| = |\frac{1-e^{i(N+1)\theta}}{1-e^{i\theta}}| \leq \frac{|1| + |e^{i(N+1)\theta}|}{|1-e^{i\theta}|} = \frac{2}{|1-e^{i\theta}|}$.\\
    Donc la série $\sum_{n\geq1} \frac{e^{in\theta}}{n}$ converge, mais pas ACV, donc elle est SCV.
}

\training{Etudier la convergence, l'absolue convergence et la semi-convergence de la série $\sum_{n\geq1} \frac{e^{in\theta}}{n^{\alpha}}$ avec $\theta \in \mathbb{R}, \alpha \in \mathbb{R}_{+}^*$.}
\noindent\carreaux{15}

\remark{Dans le critère d'Abel, comme $(a_n)$ est décroissante et $a_n \xrightarrow[n\to\infty]{} 0$, $a_n \geq 0$ (car $a_n \in \mathbb{R}$).}

\subsection{Séries alternées}

\definition{Une série $\sum_{n\geq0} u_n$ est dite \textbf{alternée} si $u_n = (-1)^n a_n$ ou $u_n = (-1)^{n+1} a_n$ avec $a_n \geq 0$.}

\example{$\sum_{n\geq0} \frac{(-1)^n}{n}$, $\sum_{n\geq0} (-1)^n$ sont des séries alternées.}

\remark{$(-1)^n \cdot u_n = (-1)^{2n}a_n = a_n$ ou $u_n = -a_n \Rightarrow (-1)^n u_n$ est de signe constant}
\remark{Une définition équivalente est : une série est alternée si le signe de $(-1)^n \cdot u_n$ est constant.}


\theorem{Théorème}{Critère spécial des séries alternées (CSSA)}{false}{
    Soit $\sum_{n\geq0} u_n$ une série de terme général $u_n = (-1)^n a_n$, avec $a_n \geq 0$.\\
    Si :
    \begin{enumerate}
        \item $(a_n)$ est décroissante.
        \item $\lim_{n\to\infty} a_n = 0$.
    \end{enumerate}
    Alors la série $\sum_{n\geq0} u_n$ converge.\\
}

\noindent{\textbf{Preuve:}\\
    On applique le critère d'Abel avec $a_n = a_n$ et $b_n = (-1)^n$.\\
    On a bien $a_n \xrightarrow[n\to\infty]{} 0$ et $(a_n)$ décroissante.\\
    De plus, $B_N = \sum_{n=0}^{N} (-1)^n = \frac{1-(-1)^{N+1}}{1-(-1)}$ est bornée (égale à 0 ou 1).\\
    Donc la série $\sum_{n\geq0} u_n$ converge.
}

\theorem{Proposition}{}{false}{
    Soit $\sum_{n\geq0} (-1)^n a_n$ une série alternée vérifiant les hypothèses du CSSA (donc $(a_n)$ est décroissante et $\lim_{n\to\infty} a_n = 0$).\\
    On considère la suite des sommes partielles $(S_N)$ avec $S_N = \sum_{k=0}^{N} (-1)^k a_k$.\\
    Soit $S$ la somme de la série.\\

    Alors :\\
    $S_{2N+1} \leq S \leq S_{2N}$ et $|R_N| = |S - S_N| \leq a_{N+1}$.
}

\noindent{\textbf{Preuve:}\\
    On pouse $A_N = S_{2N}$ et $B_N = S_{2N+1}$.\\
    On observe que $S_{2N+1}-S_{2N} = -a_{2N+1} \leq 0$\\
    $\Leftrightarrow S_{2N+1} \leq S_{2N}$.\\\\

    \noindent\textit{Variations de $(A_N)$ et $(B_N)$}\\
    $A_{N+1} - A_N = S_{2N+2} - S_{2N} = a_{2N+2} - a_{2N+1} \leq 0$ car $(a_n)$ décroissante.\\
    $\Leftrightarrow A_{N+1} \leq A_N$. Donc $(A_N)$ est décroissante et $B_{N+1} - B_N = S_{2N+3} - S_{2N+1} = a_{2N+2} - a_{2N+3} \geq 0$ car $(a_n)$ décroissante.\\
    $\Leftrightarrow B_{N+1} \geq B_N$. Donc $(B_N)$ est croissante.\\\\

    \noindent De plus, on a $B_N - A_N \xrightarrow[N\to\infty]{} 0$ donc $(A_N)$ et $(B_N)$ sont adjacentes, et convergent vers la même limite $S$.\\
    et donc $B_N \leq S \leq A_N$ où $S = \lim_{N\to\infty} A_N = \lim_{N\to\infty} B_N$.\\\\
    On a bien $S_{2N+1} \leq S \leq S_{2N}, \forall N \in \mathbb{N}$.\\\\

    \noindent Etudions maintenant le reste.\\
    $R_N = S - S_N$, on veut montrer que $|R_N| \leq a_{N+1}$.\\
    Séparons le cas $N$ pair et impair :\\
    \begin{itemize}
        \item Si $N = 2p+1$, alors $S_{2p+1} \leq S \implies S - S_{2p+1} \geq 0$.\\
        $\Rightarrow |R_{2p+1}| = S - S_{2p+1} \leq S_{2p+2} - S_{2p+1} = a_{2p+2} = a_{N+1}$.
        \item \textit{Laissé en exercice au lecteur :)} $\Box$
    \end{itemize}
   
}
\newpage

\attention{
    \begin{enumerate}
        \item Si deux suites sont équivalentes ($\sim$) et l'une monotone, l'autre ne l'est pas forcément.
        \example{$a_n = \frac{1}{\sqrt{n}+(-1)^n}$ et $b_n = \frac{1}{\sqrt{n}}$. On a $a_n \sim b_n$ mais $(a_n)$ n'est pas monotone (on le montre en encadrant/calculant 3 termes consécutifs (2p, 2p+1, 2p+2), alors que $(b_n)$ l'est).}
        \item Considérons $\sum_{n\geq0} (-1)^n a_n$. On remarque que $\sum_{n\geq0} (-1)^n a_n$ n'est pas ACV. Est-elle semi-convergente ?\\
        Le CSSA ne s'applique pas. Mais $(-1)^n a_n \sim \frac{(-1)^n}{\sqrt{n}}$ QUI N'IMPLIQUE PAS "$\sum_{n\geq0} (-1)^n a_n$ CV car $\sum_{n\geq0} \frac{(-1)^n}{\sqrt{n}}$ CV" (car $\sum_{n\geq0} \frac{(-1)^n}{\sqrt{n}}$ n'est pas positive).\\
        \textbf{À faire :} Montrer que $(-1)^n a_n = \frac{(-1)^n}{\sqrt(n)} + b_n$, où $b_n = \frac{-1}{\sqrt{n}(\sqrt{n}+(-1)^n)}$ et en déduire que $\sum u_n$ DV.

    \end{enumerate}
    Donc $u_n \sim v_n$, $\sum v_n$ CV $\implies \sum u_n CV$ que si $v_n est \geq 0 ou \leq 0$
}

\section{Produit de Cauchy de deux séries}

\definition{
    Soient $\sum_{n\geq0} a_n$ et $\sum_{n\geq0} b_n$ deux séries.\\
    La \textbf{série produit (de Cauchy)} est définie par la série $\sum_{n\geq0} c_n$ où $c_n = \sum_{p+q=n} a_p b_q = \sum_{k=0}^{n} a_k b_{n-k}, \forall n \in \mathbb{N}$.
}

\remark{Supposons que $a_n = 0 = b_n$ pour $n > N\in\mathbb{N}$. Considérons $P(X) = a_0 + a_1 X + ... + a_n X^n$ et $Q(X) = b_0 + b_1 X + ... + b_n X^n$.\\
Alors $(PQ)(X) = c_0 + c_1 X + ... + c_{2N} X^{2N}$.  On peut penser au produit de Cauchy comme une "généralisation".}

\theorem{Proposition}{}{false}{
    On considère $\sum_{n\geq0} a_n$ et $\sum_{n\geq0} b_n$ deux séries à termes positifs et convergentes.\\
    Alors la série produit $\sum_{n\geq0} c_n$ est convergente et on a : $\sum_{n=0}^{\infty} c_n = (\sum_{n=0}^{\infty} a_n)(\sum_{n=0}^{\infty} b_n)$.
}

\noindent{\textbf{Preuve:}\\
    Soient $A_n = \sum_{n=0}^{N} a_n$ et $B_N = \sum_{n=0}^{N} b_n$.\\
    Notons $C_N = \sum_{n=0}^{N} c_n$ = $\sum_{n=0}^{N} \sum_{k=0}^{n} a_k b_{n-k}$.\\\\
    On veut monrer que $(C_N)$ converge et déterminer sa limite.\\\\
    $(C_n)$ est une somme partielle à termes positifs, donc $(C_N)$ est croissante.\\
    Posons $I_N = \{ 0, \ldots, N\} \subset \mathbb{N}$\\
    \ndlr{Dessin $I_N x I_N$}
    Considérons $A_N B_N = \sum_{(p,q) \in I_N x I_N} a_p b_q$.\\
    Mais $C_N = \sum_{n=0}^{N} c_n = \sum_{n=0}^{N} \sum_{p+q=n} a_p b_q = \sum_{(p,q) \in I_N^2, p+q \leq N} a_p b_q$.\\
    On a $\{ (p,q) \mid p+q \leq N\} \subset \{(p,q) \mid p,q\in I_N\}$, donc $C_N \leq A_N B_N$  (1) qui est bornée car $A_N$ CV et $B_N$ CV $\implies C_N$ bornée.\\
    On a aussi l'inégalité : $A_N B_N \leq C_{2N} (2)$\\
    \ndlr{Deuxième schema}
    car $\{(p,q) \mid p+q \leq N\} \supset \{(p,q) \mid 0 \leq p,q \leq N\}$.
    On obtient $\lim_{+\infty} c_n =  \lim_{+\infty} (A_N B_N) = (\lim_{+\infty} A_N)\cdot(\lim_{+\infty} B_N)$. $\Box$
}

\theorem{Théorème}{}{false}{
    Soient $\sum_{n\geq0} a_n$ et $\sum_{n\geq0} b_n$ deux séries à termes dans $\mathbb{C}$.\\
    Si les séries sont ACV, alors la série produit $\sum_{n\geq0} c_n$ est ACV. (où $c_n = \sum_{k=0}^{n} a_k b_{n-k}$)\\
}

% COURS DU 8 OCTOBRE

\noindent{\textbf{Preuve:}\\
    On considère $A_N = \sum_{n=0}^{N} |a_n|$, $B_N = \sum_{n=0}^{N} |b_n|$ et $C_N = \sum_{n=0}^{N} |c_n|$.\\
    D'après la proposition précédente et sa démonstration, on a $A_N B_N - C_N \xrightarrow[N\to\infty]{} 0$. (on va utiliser cette propriété)\\
    On a $\forall N \in \mathbb{N}, |(\sum_{n=0}^{N} a_n)(\sum_{n=0}^{N} |b_n|) - (\sum_{n=0}^{N} |c_n|)|$.\\
    On peut donc écrire : \[
|(\sum_{n=0}^{N} a_n)(\sum_{n=0}^{N} |b_n|) - (\sum_{n=0}^{N} |c_n|)|
= |\sum_{p\in I_N} \sum_{q \in I_N} a_p b_q - \sum_{(p,q) \in I_N^2, p+q \leq N} a_p b_q|\\
= |\sum_{(p,q) \in I_N^2} a_p b_q  - \sum_{(p,q) \in J_N^2} a_p b_q| \\ \text{ où } J_N = \{(p,q) \mid p+q \leq N\}\]
Or $J_N \subset I_N^2$, donc \[
|\sum_{(p,q) \in I_N^2} a_p b_q  - \sum_{(p,q) \in J_N^2} a_p b_q| 
= |\sum_{(p,q) \in I_N^2 \setminus J_N^2} a_p b_q| = \sum_{(p,q) \in K_N} |a_p b_q|\] où $K_N = I_N^2 \setminus J_N^2 = \{(p,q) \mid p+q > N\}$.\\
\[\leq \sum_{(p,q) \in K_N} |a_p| |b_q| = (\sum_{n=0}^{N} |a_n|)(\sum_{n=0}^{N} |b_n|) - \sum_{n=0}^{N} |c_n| = A_N B_N - C_N \xrightarrow[N\to\infty]{} 0\] par la proposition précédente et l'inégalité triangulaire.
}

\ndlr{À changer dans la démo $A_N$ en $A_N'$ et $B_N$ en $B_N'$.\\
De plus, il faut mettre $C_N' = \sum_{n=0}^{N} |c_n'| = \sum_{n=0}^{N} |a_k| |b_{n-k}|$ et pas $|\sum_{n=0}^{N} a_k b_{n-k}|$.
}

\remark{
    L'hypothèse d'absolue convergence pour $\sum a_n$ et $\sum b_n$ est très importante dans le théorème.\\
    L'hypothèse de positivité dans la proposition qui précède le théorème est fondamentale.
}

\example{
    On considère la série de terme général $u_n = \frac{(-1)^n}{\sqrt{n}}$.\\
    \begin{itemize}
        \item $u_n$ n'est pas positive.
        \item On a pas l'absolue convergence.
        \item Le CSSA s'applique car $a_n = \frac{1}{\sqrt{n}}$ est positive, décroissante et tend vers 0.
    \end{itemize}

    Considérons le produit de Cauchy.
    $(\sum_{n \geq 1} u_n)(\sum_{n \geq 1} u_n) = \sum_{n \geq 1} c_n$ où $c_n = \sum_{k=1}^{n-1} \frac{(-1)^k}{\sqrt{k}} \cdot \frac{(-1)^{n-k}}{\sqrt{n-k}}$\\
    Montrons que $\sum_{n \geq 1} c_n$ diverge (en montrant que ça ne tend pas vers 0).\\
    On a $|c_n| = |\sum_{k=1}^{n-1} \frac{1}{\sqrt{k \cdot (n-k)}}|$
    
    On a $k(n-k) \leq kn-k^2 \leq kn \leq (n-1)n$.\\
    Donc $|c_n| = |\sum_{k=1}^{n-1} \frac{1}{\sqrt{k \cdot (n-k)}}| \geq \sum_{k=1}^{n-1} \frac{1}{\sqrt{(n-1)n}} = \frac{n-1}{\sqrt{(n-1)n}} = \sqrt{\frac{n-1}{n}}$.\\
    Donc $|c_n| \nrightarrow 0$.\\
    \textbf{Conclusion :} Pour faire le produit de Cauchy de deux séries, il faut :
    \begin{enumerate}
        \item Que les deux séries soient ACV.
        \item ou Que les deux séries soient à termes positifs et CV.
    \end{enumerate}
}

\training{
    Fixons $z \in \mathbb{C}$. Etudions la convergence de la série $\sum_{n\geq0} \frac{z^n}{n!}$.\\
    1. Montrons que $\forall n \in \mathbb{N}$ la série ACV.\\ 
    On va utiliser la règle de d'Alembert : $\frac{|u_{n+1}|}{|u_n|} = \frac{|z|}{(n+1)}.$\\
    Donc $\forall z \in \mathbb{C}, \frac{z^n}{n!} \xrightarrow[n\to\infty]{} 0$.\\
    Donc $\forall z \in \mathbb{C}, \sum_{n\geq0} u_n(z)$ est ACV.\\
    \remark{On a le bon goût de pouvoir appeller $\sum_{n\geq0} \frac{z^n}{n!} := exp(z)$. \textit{(je dis bon goût mais ça risque de faire mal bientôt)}}
    2. Calculons $exp(z) \cdot exp(z')$ avec $z,z' \in \mathbb{C}$.\\
    Comme les deux séries sont ACV, on peut faire le produit de Cauchy.\\
    $exp(z) \cdot exp(z') = \sum_{n\geq0} c_n$ où $c_n = \sum_{k=0}^{n} \frac{z^k}{k!} \cdot \frac{(z')^{n-k}}{(n-k)!}$\\
    On a $c_n = \frac{1}{n!} \sum_{k=0}^{n} \frac{n!}{k!(n-k)!} z^k (z')^{n-k} = \frac{(z+z')^n}{n!}$ par le binôme de Newton.\\
    Donc $exp(z) \cdot exp(z') = \sum_{n\geq0}^{+\infty} \frac{(z+z')^n}{n!} = exp(z+z')$.
}

\section{Compléments}

\subsection{Hors-programme : Séries commutativement convergentes}
\ndlr{On a traité de ça en parlant rapidement de permutations. À voir chez Laurent si c'est nécessaire à mettre, mais je l'omets ici pour l'instant.}

\subsection{Introduction aux séries de Taylor d'une fonction}

\definition{
Considérons $f : I \to \mathbb{R}$ où $I$ est un intervalle ouvert contenant $o$.\\
Supposons que $f \in \mathcal{C}^{\infty}(I)$. \textit{(i.e. $f$ est dérivable autant de fois qu'on veut sur $I$ et les dérivées sont continues)}.\\
La \textbf{série de Taylor} associée à $f$ au voisinage de 0 est la série $\sum_{n\geq0} \frac{f^{(n)}(0)}{n!} x^n$.
}

Ici, $x \in \mathcal{V}(0)$ \textit{(cela peut être I tout entier)}. Et il s'agit en fait d'une série de fonctions: $x \in \mathcal{V}(0) \subset I \mapsto \frac{f^{(n)}(0)}{n!} x^n$.\\
À ce stade du cours, on y pense comme une série numérique à $x$ fixé.

\textbf{Deux questions se posent :}
\begin{enumerate}
    \item Pour quels $x \in I$ la série de Taylor $\sum_{n\geq0} \frac{f^{(n)}(0)}{n!} x^n$ converge-t-elle ?
    \item Si elle converge "pour des $x$", a-t-elle pour somme $f(x)$ ?
\end{enumerate}

\remark{Plus généralement, si $I$ est quelconque et que on prend $a < b \in I$, les mêmes questions se posent de la façon suivante :
$f(b) = \sum_{n\geq0} \frac{(b-a)^n}{n!} f^{(n)}(a)$ ?}

\remark{Les sommes partielles de la série de Taylor associée à $f$, i.e. $\sum_{n=0}^{N} \frac{f^{(n)}(0)}{n!} x^n, x \in I$ sont appelées \textbf{polynômes de Taylor}}

\textbf{Réponses partielles aux questions.}
\begin{enumerate}
    \item Utiliser les règles de d'Alembert ou de Cauchy pour déterminer les $x \in I$ tels que $\sum_{n\geq0} \frac{f^{(n)}(0)}{n!} x^n$ converge.
    \item Utilisons la formule de Taylor avec reste intégral si on veut montrer que pour les $x$ où $\sum_{n\geq0} \frac{f^{(n)}(0)}{n!} x^n$ converge, on a $f(x) = \sum_{n\geq0}^{+\infty} \frac{f^{(n)}(0)}{n!} x^n$.\\
    Ou de manière équivalente $\sum_{n=0}^{N} \frac{f^{(n)}(0)}{n!} x^n \xrightarrow[N\to\infty]{} f(x)$. (convergence à $x$ fixé !)
\end{enumerate}

\training{Retrouvons la formule de Taylor avec reste intégral.}
\theorem{Théorème fondamental de l'analyse}{Rappel}{true}{
    Soit $x \in I, x > 0$.\\
    Alors $f(x) - f(0) = \int_{0}^{x} f'(t) dt$.
}

$\int_{0}^{x} f'(t) dt = \int_{0}^{x} (t - x)' f'(t) dt = [(t-x)f'(t)]_{0}^{x} - \int_{0}^{x} (t-x) f''(t) dt$.\\
$= -x f'(0) + \int_{0}^{x} (x-t) f''(t) dt$.\\
Ce qui se réécrit : $f(x) - f(0) - xf'(0) = \int_{0}^{x} (x-t) f''(t) dt$.\\
On refait la même chose pour $f''$ :\\
$\int_{0}^{x} (x-t) f''(t) dt = \int_{0}^{x} ((x-t)^2/2)' f''(t) dt = - [(x-t)^2/2 f''(t)]_{0}^{x} + \int_{0}^{x} (x-t)^2/2 f^{(3)}(t) dt$.\\
$= -x^2/2 f''(0) + \int_{0}^{x} (x-t)^2/2 f^{(3)}(t) dt$.\\
Ce qui se réécrit : $f(x) - f(0) - xf'(0) - x^2/2 f''(0) = \int_{0}^{x} (x-t)^2/2 f^{(3)}(t) dt$.\\
Puis on continue par récurrence et on a :
\theorem{Théorème}{Taylor avec R.I.}{false}{
    $f \in \mathcal{C}^{\infty}(I), I \ni 0$\\
    On a $\forall x \in I, x > 0, f(x) - \sum_{k=0}^{n} \frac{f^{(k)}(0)}{k!} x^k = \int_{0}^{x} \frac{(x-t)^n}{n!} f^{(n+1)}(t) dt$.
}

\remark{Pour montrer que $\sum_{n\geq0} \frac{f^{(n)}(0)}{n!} x^n$ converge vers $f(x)$, il suffit de montrer que le $R_n(x) \xrightarrow[n\to\infty]{} 0$, où $R_n$ est le reste.\\
On a $|R_n(x)| = |\int_{0}^{x} \frac{(x-t)^n}{n!} f^{(n+1)}(t) dt|$. En principe, majorer $R_n(x)$ !
}

\training{On prend $f(x) = exp(x)$, $I = \mathbb{R}$.\\
On a que $\forall n \in \mathbb{N}, f^{(n)}(x) = exp(x)$. Sa série de Taylor au voisinage de 0 est $\sum_{n\geq0} \frac{f^{(n)}(0)}{n!} x^n = \sum_{n\geq0} \frac{x^n}{n!}$.\\
\textbf{1. Convergence}\\
$u_n(x) = \frac{x^n}{n!}$.\\
On utilise la règle de d'Alembert : $\frac{|u_{n+1}(x)|}{|u_n(x)|} = \frac{|x|}{n+1} \xrightarrow[n\to\infty]{} 0, \forall x \in \mathbb{R}$.\\
Or $0 < 1$, donc $\sum_{n\geq0} \frac{x^n}{n!}$ est ACV, $\forall x \in \mathbb{R}$.\\

\textbf{2. Somme}\\
On veut montrer que $\sum_{n\geq0}^{+\infty} \frac{x^n}{n!} = exp(x), \forall x \in \mathbb{R}$.\\
Par la formule de Taylor avec reste intégral, on a $|f(x) - \sum_{k=0}^{n} \frac{f^{(k)}(0)}{k!} x^k| = |R_n(x)| = |\int_{0}^{x} \frac{(x-t)^n}{n!} e(t) dt| \\ 
\leq \frac{|x|^n}{n!}e^x \xrightarrow[n\to\infty]{} 0, \forall x \in \mathbb{R}$.\\
Donc $exp(x) = \sum_{n\geq0}^{+\infty} \frac{x^n}{n!}, \forall x \in \mathbb{R}$.
}

\example{
Calculons la valeur de $\sum_{n\geq0}^{+\infty} \frac{1}{n!}$.\\
On sait que $\sum_{n\geq0}^{+\infty} \frac{x^n}{n!} = exp(x)$, donc en prenant $x = 1$, on obtient $\sum_{n\geq0}^{+\infty} \frac{1}{n!} = exp(1) = e$.
}

\cexample{Voici une fonction $C^{\infty}$ sur $\mathbb{R}$ dont la série de Taylor ne converge pas vers la fonction (sauf en 0).\\
Considérons \function{f}{\mathbb{R} \mapsto \mathbb{R}}{x \mapsto \begin{cases}
    0 & \text{ si } x = 0\\
    e^{-\frac{1}{x^2}} & \text{ si } x \neq 0
\end{cases}}.\\

$f$ est continue sur $\mathbb{R}$.\\
$x \mapsto e^{-\frac{1}{x^2}}$ est dérivable sur $\mathbb{R}^*$, et $\lim_{x\to0} f(x) = 0 = f(0)$.\\\\

On montre que $f$ est $C^{\infty}$ : elle se prolonge en une fonction $C^{\infty}$ sur $\mathbb{R}$ telle que $f^{(n)}(0) = 0, \forall n \in \mathbb{N}$.\\
Il faut vérifier $\lim_{x\to0} f^{(n)}(x) = 0, \forall n \in \mathbb{N}$.
}
\end{document}
