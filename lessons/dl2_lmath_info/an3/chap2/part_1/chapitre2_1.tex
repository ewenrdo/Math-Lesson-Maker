\documentclass{article}

\usepackage[a4paper, left=1.5cm, right=1.5cm, top=2cm, bottom=2cm]{geometry}

\usepackage{../../../../../components/components} % <-- ton fichier .sty, avec toutes tes définitions

\usepackage{fancyhdr}


% Configuration des en-têtes et pieds de page
\pagestyle{fancy}
\fancyhf{} % reset tout

\fancyhead[L]{DL2 Math-Info AN3}
\fancyhead[C]{Séries numériques}
\fancyhead[R]{2025-2026}

\fancyfoot[L]{Ewen Rodrigues de Oliveira}
\fancyfoot[R]{\thepage}

\begin{document}

\docTitle{Chapitre 2.1 : Introduction aux séries numériques}

\section{Séries et sommes d'une série}

\definition{
    Soit \((u_n)\) une suite dans $\mathbb{K} = \mathbb{R}$ ou $\mathbb{C}$.\\
    On considère \(\forall N \in \mathbb{N}, S_N = \sum_{n=0}^{N} u_n \in \mathbb{K}\).\\\
    On a donc une suite $(S_N)_{N \in \mathbb{N}}$ associée à la suite $(u_n)_{n \in \mathbb{N}}$.
}
\definition{On appelle \textbf{série de terme général \(u_n\)} que l'on note \(\sum_{n \geq 0} u_n\) la suite $(S_N)_{N \in \mathbb{N}}$.}

\ndlr{Les deux définitions précédentes gagneraient à être fusionnées.}

\vocabulary{On dit que \((S_N)\) est la suite des sommes partielles de la série.}
\remark{\((S_N)\) correspond aux $N+1$ premiers termes de la suite.}

\subsection{Correspondance suite - série}

\noindent{
\textbf{Raisonnement :}
Par définition une série est une suite. Expliquons comment une suite peut-être vue comme une série.\\
Si $(u_n)$ est une suite, considérons la série de terme général \(v_n = u_n - u_{n-1} \forall n \in \mathbb{N}\) (avec la convention \(v_{0} = u_{0}\)).\\
Ainsi, \(u_n = \sum_{k=0}^{n} v_k\).}

\remark{Cependant la série associée à une suite $(u_n)$ va s'étudier en tant que telle (que série) grâce à $u_n$.}


\subsection{Opérations sur les séries}

\theorem{Propriété}{Opérations sur les séries}{true}{
    Soient \(\sum_{n\geq0} u_n\) et \(\sum_{n\geq0} v_n\) deux séries.\\
    Alors, pour tout \(\lambda \in \mathbb{K}\) :
    \begin{itemize}
        \item \textbf{Somme} : \(\sum_{n\geq0} (u_n + v_n) = \sum_{n\geq0} u_n + \sum_{n\geq0} v_n\) définie comme $(S_N  + S'_N)$
        \item \textbf{Produit par un scalaire} : \(\sum_{n\geq0} \lambda u_n = \lambda \sum_{n\geq0} u_n\) définie comme \((\lambda S_N)\)
    \end{itemize}
}

\example{Si $u_n=0, \forall n \in \mathbb{N}$, alors \(\sum_{n\geq0} u_n = 0\) est la série nulle.}


\subsection{Troncature d'une série}


\definition{
    Si $(u_n)$ est une suite définie pour \(n \geq n_0 \mid n_0 \in \mathbb{N}\).
    On peut considérer la série \(\sum_{n \geq n_0} u_n\) où \(u_{0} = u_{1} = ... = u_{n_0 - 1} = 0\), ou bien on peut écrire $\sum_{n \geq n_0} u_n$.

    Si $\sum_{n \geq 0} u_n$ est une série de terme général $u_n$, une \textbf{troncature} de la série est $\sum_{n \geq n_0} u_n$. C'est la suite $(S_N)$ où \(S_N = \sum_{n=n_0}^{N} u_n\).
}

\ndlr{Cette définition pourraît être synthétisée.}

\example{
    \begin{itemize}
        \item la série nulle
        \item la série géométrique de raison $q\in\mathbb{C^*}$ : \(\sum_{n\geq0} q^n\) de terme général \(q^n\) ;
        \item la série harmonique : \(\sum_{n\geq1} \frac{1}{n}\) de terme général \(\frac{1}{n}\) ;
        \item la série $\sum_{n\geq1} \frac{1}{n^\alpha}, \alpha\in\mathbb{R}$.
    \end{itemize}
}

\section{Convergence d'une série}
\subsection{Définitions et nature d'une série}
\definition{
    Soit \(\sum_{n\geq0} u_n\) une série.\\
    On dit que la série converge, si la suite $(S_N)$ converge, et on note $S$ la limite de $S_N$.\\
    S s'appelle la somme de la série.\\\\

    Dans ce cas, on écrit : \(\sum_{n=0}^{\infty} u_n = S \in \mathbb{R}\) \textit{(c'est une "somme infinie", un objet-limite).}
}

\vocabulary{Si $(S_N)$ diverge, alors on dit que la série $\sum_{n\geq0} u_n$ diverge.}
\attention{Si $S$ n'existe pas, alors on écrit \textbf{jamais} la notation avec \(\infty\)}

\vocabulary{La convergence ou la divergence d'une série s'appelle la \textbf{nature} de la série.}
\theorem{Proposition}{Stabilité de la limite par troncature}{true}{
    La nature d'une série n'est pas modifée par troncature.
}
\noindent\textbf{Preuve:}\\
\carreaux{12}
\ndlr{Indication : les premiers termes n'influencent pas la convergence.}

\subsection{Quelques applications...}

\example{
    \begin{itemize}
        \item Si $(u_n)$ est nulle à partir d'un rang $N_0$ alors la série $\sum_{n\geq0} u_n$ est converge, et $\sum_{n=0}^{\infty} u_n =  \sum_{n=0}^{N_0} u_n$.
        \item Série géométrique $\sum_{n\geq0} q^n$ :\\
        On considère la suite des sommes partielles $(S_N)$ où \(S_N = \sum_{n=0}^{N} q^n\) = \(\frac{1-q^{N+1}}{1-q}\) avec \(q \neq 1\). On a plusieurs cas :
        \begin{itemize}
            \item Si \(|q| < 1\), $q^{N+1} \xrightarrow[N\to\infty]{} 0$ donc \(S_N \xrightarrow[N\to\infty]{} \frac{1}{1-q} \Leftrightarrow \sum_{n=0}^{\infty} q^n = \frac{1}{1-q}\).\\
            La série $\sum_{n\geq0} q^n$ converge et on arrive à trouver S !
            \item Si \(|q| > 1\), alors $\sum_{n\geq0} q^n$ diverge.
            \item Si \(q = 1\), alors $\sum_{n\geq0} q^n = N+1 \Rightarrow \sum_{n\geq0} q^n$ diverge.
        \end{itemize}
        \item $\sum_{n\geq1}{log(1+1/n)}$ :\\
        On a \(\forall N \geq 1, S_N = \sum_{n=1}^{N} log(\frac{n+1}{n}) = log(N+1)\) (télescopage).\\
        Or \(log(N+1) \xrightarrow[N\to\infty]{} +\infty\), donc la série $\sum_{n\geq1}{log(1+1/n)}$ diverge.
        \item $\sum_{n\geq1} \frac{1}{n(n+1)}$ :\\
        On a \(\forall N \geq 1, S_N = \sum_{n=1}^{N} \frac{1}{n(n+1)} = \sum_{n=1}^{N} (\frac{1}{n} - \frac{1}{n+1}) = 1 - \frac{1}{N+1}\) (télescopage).\\
        Or \(1 - \frac{1}{N+1} \xrightarrow[N\to\infty]{} 1\), donc la série $\sum_{n\geq1} \frac{1}{n(n+1)}$ converge et \(\sum_{n=1}^{\infty} \frac{1}{n(n+1)} = 1\).
        \item \textbf{Important, démontré plus tard :} $\sum_{n\geq1} \frac{1}{n}$ (série harmonique) diverge.
        \item $\sum_{n\geq1} \frac{(-1)^n}{n}$ converge. \textit{(idée : montrer que $S_N$ converge en montant $A_N= S_{2N}$ et $B_N=S_{2N+1}$ sont adjacentes)}
    \end{itemize}
}

\attention{Ces six exemples sont à connaître et comprendre parfaitement.}

\training{Étudier la convergence de la série géométrique pour \(|q| = 1\) et \(q = -1\) ($q\in\mathbb{C}$).}
\noindent\carreaux{5}

\subsection{Propriétés des séries convergentes}

\theorem{Propriété}{Convergence de la combinaison linéaire}{true}{
    Soient \(\sum_{n\geq0} u_n\) et \(\sum_{n\geq0} v_n\) deux séries convergentes.\\
    Alors $\forall \lambda,\mu \in \mathbb{K}$ : \(\sum_{n\geq0} (\lambda u_n + \mu v_n) = \lambda \sum_{n\geq0} u_n + \mu \sum_{n\geq0} v_n\), cette série converge (vers la combinaison linéaire des limites).
}
\remark{En d'autres termes, la somme de deux séries convergentes est une série $\sum_{n\geq0} (u_n + v_n)$ qui converge.}
\noindent{
\textbf{Preuve :}\\
La suite de sommes partielles associée à $\sum_{n\geq}{(u_n + v_n)}$ est $\sum_{n=0}^{N} (u_n + v_n) = \sum_{n=0}^{N} (u_n) + \sum_{n=0}^{N} (v_n)$\\
Comme $\sum_{n=0}^{N} (u_n)$ et $\sum_{n=0}^{N} (v_n)$ sont convergentes, on a $\sum_{n=0}^{N} (u_n + v_n)$ est convergente et sa limite est \(\sum_{n=0}^{\infty} (u_n+v_n)  = \sum_{n=0}^{\infty} (u_n) + \sum_{n=0}^{\infty} (v_n)\).
}

\example{
    \textbf{Retour : Divergence de la série harmonique $\sum_{n\geq1} \frac{1}{n}$}\\
    But : minorer $\sum_{n\geq1}^{N}\frac{1}{n} \forall N \in \mathbb{N}$.\\
    $n \leq t \in \mathbb{R} \leq n+1 \Rightarrow \frac{1}{n+1} \leq \frac{1}{t} \leq \frac{1}{n}$\\
    Intégrons entre $n$ et $n+1$ : \(\int_{n}^{n+1} \frac{1}{t} dt \leq \frac{1}{n}\)\\
    Donc en sommant : \(\sum_{n=1}^{N} \int_{n}^{n+1} \frac{1}{t} dt \leq \sum_{n=1}^{N} \frac{1}{n}\)\\
    donc par Chasles : \(\int_{1}^{N+1} \frac{1}{t} dt \leq \sum_{n=1}^{N} \frac{1}{n} \forall n \in \mathbb{N}\)\\
    Or \(\int_{1}^{N+1} \frac{1}{t} dt = \ln(N+1) \xrightarrow[N\to\infty]{} +\infty\) donc \(\sum_{n=1}^{N} \frac{1}{n} \xrightarrow[N\to\infty]{} +\infty\).\\
    Donc la série harmonique diverge.
}

\theorem{Propriété}{Divergence de la combinaison linéaire}{true}{
    Soient \(\sum_{n\geq0} u_n\) une série convergente et \(\sum_{n\geq0} v_n\) une série divergente.\\
    Alors \(\sum_{n\geq0} (u_n + v_n)\)  diverge.
}
\noindent{
\textbf{Preuve :}\\
$\sum_{n=0}^{N} (u_n + v_n) = \sum_{n=0}^{N} (u_n) + \sum_{n=0}^{N} (v_n)$\\
Comme $\sum_{n=0}^{N} (u_n)$ est convergente et $\sum_{n=0}^{N} (v_n)$ est divergente, on a $\sum_{n=0}^{N} (u_n + v_n)$ est divergente.
}

\attention{Quand on considère deux séries divergentes, la situation est à étudier au cas par cas.}
\example{
    Considérons $\sum_{n\geq1} u_n$ avec $u_n = 1 \forall n\in\mathbb{N}$ et $\sum_{n\geq1} v_n$ avec $v_n = -1 \forall n\in\mathbb{N}$.\\
    D'une part $\sum_{n\geq1} u_n$ diverge, et $\sum_{n\geq1} v_n$ diverge aussi.\\
    Mais \(\sum_{n\geq1} (u_n + v_n) = \sum_{n\geq1} 0 = 0\) converge.\\\\
    Mais si on considère $v_n = u_n$, alors \(\sum_{n\geq1} (u_n + v_n) = \sum_{n\geq1} 2u_n\) diverge.
}

\attention{\textbf{Source d'erreur classique :} Si $\sum_{n\geq0} u_n + v_n$ est convergente, \textit{\textcolor{danger}{\textbf{a priori}}} on ne peut pas écrire que $\sum_{n=0}^{\infty} u_n+v_n = \sum_{n=0}^{\infty} u_n + \sum_{n=0}^{\infty} v_n$ car les séries de termes généraux $u_n$ et $v_n$ peuvent être divergentes (il faut donc vérifier leur convergence).}

\theorem{Proposition}{}{true}{
    Soit $\sum_{n\geq0} u_n$ une série numérique où $u_n \in \mathbb{C}$ $\forall n \in \mathbb{N}$.\\
    On a $\sum_{n\geq0} u_n$ converge \(\Leftrightarrow\) les suites $(Re(u_n))$ et $(Im(u_n))$ sont convergentes.\\
}

\training{Montrer la proposition précédente.}
\noindent{\textbf{Indication pour la preuve:}\\écrire $u_n = Re(u_n) + i Im(u_n)$ et utiliser la propriété sur les combinaisons linéaires.}\\
\noindent{\carreaux{10}}

\theorem{Théorème}{Lien entre convergence et limite des termes}{false}{
    Si $\sum_{n\geq0} u_n$ converge, alors $u_n \xrightarrow[n\to\infty]{} 0$.
}
\noindent{\textbf{Preuve:}\\
    Considérons $(S_N)$ la suite des sommes partielles associée à $\sum_{n\geq0} u_n$.\\
    On a $S_{N+1}-S_N = u_{N+1}$ $\forall N \in\mathbb{N}$.\\
    Or $\sum_{n\geq0} u_n$ converge $\Rightarrow (S_N)$ converge.
    Donc $\lim_{N\to\infty} S_{N} - \lim_{N\to\infty} S_{N+1} = 0 \Rightarrow \lim_{N\to\infty} u_{N} = 0$.
}

\attention{La réciproque est fausse. Par exemple la série harmonique $\sum_{n\geq1} \frac{1}{n}$ diverge mais $\frac{1}{n} \xrightarrow[n\to\infty]{} 0$.}
\vocabulary{Si $u_n \nrightarrow 0$, on dit que la série $\sum_{n\geq0} u_n$ \textcolor{purple}{\textbf{diverge grossièrement}}.}

\subsection{Reste d'une série}

\definition{
On suppose que $\sum_{n\geq0} u_n$ converge. On note $S = \sum_{n=0}^{\infty} u_n$ sa somme et $(S_N)$ la suite des sommes partielles.\\
Le \textbf{reste} de la série au rang $N$ est $R_N = S - S_N = \sum_{n=N+1}^{\infty} u_n$.
}

\theorem{Proposition}{Comportement du reste}{false}{
    Si $\sum_{n\geq0} u_n$ converge, alors $R_N \xrightarrow[N\to\infty]{} 0$.
}
\noindent{\textbf{Preuve:}\\
    Par définition, $R_N = S - S_N$. Or $S_N \xrightarrow[N\to\infty]{} S$. Donc $R_N \xrightarrow[N\to\infty]{} 0$.
}

\end{document}