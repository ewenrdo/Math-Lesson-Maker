\documentclass{article}

\usepackage[a4paper, left=1.5cm, right=1.5cm, top=2cm, bottom=2cm]{geometry}

\usepackage{../../../../components/components} % <-- ton fichier .sty, avec toutes tes définitions

\usepackage{fancyhdr}


% Configuration des en-têtes et pieds de page
\pagestyle{fancy}
\fancyhf{} % reset tout

\fancyhead[L]{DL2 Math-Info AN3}
\fancyhead[C]{Intégration}
\fancyhead[R]{2025-2026}

\fancyfoot[L]{Ewen Rodrigues de Oliveira}
\fancyfoot[R]{\thepage}

\begin{document}

\docTitle{Chapitre 3 : Intégrales impropres}

% Jusqu'à présent, les intégrales qu'on a étudiées étaient des intégrales de Riemann, approchant l'aire sous une courbe.
\attention{Tous les exemples de calculs d'intégrales dans ce chapitre sont à connaître par cœur (il s'agit d'intégrales de référence), et peuvent être utilisés dans des exercices.}

\section{Généralités sur les intégrales impropres}

\subsection{Sur un intervalle du type $[a, +\infty[, a\in\mathbb{R}$}
\definition{
    Soit \functionSets{f}{[a, +\infty[ \rightarrow \mathbb{R} \text{ ou } \mathbb{C}} une fonction continue.\\
    On dit que l'intégrale impropre \textit{(ou généralisée)} au voisiage de $+\infty$ converge (ou existe) si $\lim_{x \to +\infty} \int_a^x f(t) dt$ existe et est finie.\\\\
    Dans ce cas, on note: $\int_a^{+\infty} f(t)\,dt$ l'intégrale impropre. Si une telle limite n'existe pas, on dit que l'intégrale diverge.
}

\remark{On utilisera la notation $\int_a^{+\infty} f(t)\,dt$ pour l'intégrale impropre qui converge ou diverge (on précise toujours si elle converge ou diverge).}

\theorem{Propriété}{}{false}{
    Supposons que $\int_a^{+\infty} f(t)\,dt$ converge.\\
    Alors, $\lim_{A \to +\infty} \int_A^{+\infty} f(t)\,dt = 0$ converge. C'est le reste de l'intégrale impropre.
}

\noindent{\textbf{Preuve:}\\
    Soit $x > A > a$.\\
    On a : $\int_a^x f(t)\,dt = \int_a^A f(t)\,dt + \int_A^x f(t)\,dt$.\\
    En faisant tendre $x$ vers $+\infty$, on obtient : $\int_a^{+\infty} f(t)\,dt = \int_a^A f(t)\,dt + \int_A^{+\infty} f(t)\,dt$.\\
    On passe à la limite sur $A, A \to +\infty$ et on obtient :\\
    $\int_a^{+\infty} f(t)\,dt = \int_a^{+\infty} f(t)\,dt + \lim_{A \to +\infty} \int_A^{+\infty} f(t)\,dt$.\\
    Donc $\lim_{A \to +\infty} \int_A^{+\infty} f(t)\,dt = 0$ $\Box$.
}

\remark{Si on continue $f$ continue de $]-\infty, a]$ dans $\mathbb{R}$ ou $\mathbb{C}$, on peut définir l'intégrale impropre $\int_{-\infty}^{a} f(t)\,dt$ par $\lim_{x \to -\infty} \int_x^a f(t)\,dt$.}

\example{
    $\int_1^{+\infty} \frac{1}{t}\,dt$ converge-t-elle?
    \begin{enumerate}
        \item D'abord, $t \mapsto \frac{1}{t}$ est continue sur $[1, +\infty[$.
        \item Soit $x > 1$, $\int_1^x \frac{1}{t}\,dt = [\ln(t)]_1^x = \ln(x)$.\\
    Donc, $\lim_{x \to +\infty} \int_1^x \frac{1}{t}\,dt = \lim_{x \to +\infty} \ln(x) = +\infty$.\\
    \textbf{Donc $\int_1^{+\infty} \frac{1}{t}\,dt$ diverge.}
    \end{enumerate}
}
\example{
    $\int_0^{+\infty} cos(t)\,dt$ converge-t-elle?
    \begin{enumerate}
        \item D'abord, $t \mapsto cos(t)$ est continue sur $[0, +\infty[$.
        \item Soit $x > 0$, $\int_0^x cos(t)\,dt = [sin(t)]_0^x = sin(x)$.\\
    Donc, $\lim_{x \to +\infty} \int_0^x cos(t)\,dt = \lim_{x \to +\infty} sin(x)$ n'existe pas.\\
    \textbf{Donc $\int_0^{+\infty} cos(t)\,dt$ diverge.}
    \end{enumerate}
}

\example{
    $\int_0^{+\infty} \frac{1}{t^\alpha}\,dt$ converge-t-elle? ($\alpha \in \mathbb{R}$)
    \begin{enumerate}
        \item D'abord, $t \mapsto \frac{1}{t^\alpha}$ est continue sur $[0, +\infty[$ pour $\alpha > 0$.
        \item Soit $x > 0$, $\int_0^x \frac{1}{t^\alpha}\,dt = \left[ \frac{t^{1-\alpha}}{1-\alpha} \right]_0^x = \frac{x^{1-\alpha}}{1-\alpha}$.\\
    Donc, $\lim_{x \to +\infty} \int_0^x \frac{1}{t^\alpha}\,dt = \lim_{x \to +\infty} \frac{x^{1-\alpha}}{1-\alpha}$.
    \begin{itemize}
        \item Si $\alpha < 1$, alors $1-\alpha > 0$ et $\lim_{x \to +\infty} \frac{x^{1-\alpha}}{1-\alpha} = +\infty$.
        \item Si $\alpha = 1$, alors $\lim_{x \to +\infty} \frac{x^{1-\alpha}}{1-\alpha}$ est indéfini.
        \item Si $\alpha > 1$, alors $1-\alpha < 0$ et $\lim_{x \to +\infty} \frac{x^{1-\alpha}}{1-\alpha} = 0$.
    \end{itemize}
    \textbf{Donc $\int_0^{+\infty} \frac{1}{t^\alpha}\,dt$ converge si et seulement si $\alpha > 1$.}
    \end{enumerate}
}

\theorem{Proposition}{Convergence des intégrales de Riemann}{false}{
    Les intégrales de la forme $\int_1^{+\infty} \frac{1}{t^\alpha}\,dt$ sont appelées des intégrales de Riemann et convergent si et seulement si $\alpha > 1$, et on a :
    \[
        \int_1^{+\infty} \frac{1}{t^\alpha}\,dt = 
        \begin{cases}
            \frac{1}{\alpha - 1} & \text{si } \alpha > 1\\
            \text{indéterminée} & \text{si } \alpha = 1\\
            +\infty & \text{si } \alpha < 1
        \end{cases}
    \]
}

\example{
    $\int_0^{+\infty} e^{-\alpha t}\,dt$ converge-t-elle?
    \begin{enumerate}
        \item D'abord, $t \mapsto e^{-\alpha t}$ est continue sur $[0, +\infty[$ pour $\alpha > 0$.
        \item Soit $x > 0$, $\int_0^x e^{-\alpha t}\,dt = \left[ -\frac{e^{-\alpha t}}{\alpha} \right]_0^x = \frac{1}{\alpha}(1 - e^{-\alpha x})$.\\
        La convergence de $\int_0^{+\infty} e^{-\alpha t}\,dt$ est la même que $\frac{1}{\alpha}(1 - e^{-\alpha x})$, quand $x \to +\infty$.\\
        Donc, $\lim_{x \to +\infty} \int_0^x e^{-\alpha t}\,dt = \lim_{x \to +\infty} \frac{1}{\alpha}(1 - e^{-\alpha x}) = \frac{1}{\alpha}$.\\
        \textbf{Donc $\int_0^{+\infty} e^{-\alpha t}\,dt$ converge si, et seulement si, $\alpha > 0$, et vaut $\frac{1}{\alpha}$.}
    \end{enumerate}
}
\attention{Si $\alpha \leq 0$, l'intégrale diverge.}

\vocabulary{
    La nature de l'intégrale impropre $\int_a^{+\infty} f(t)\,dt$ est la convergence ou la divergence de cette intégrale.
}

\training{Déterminer la nature de l'intégrale impropre suivante : $\int_0^{+\infty} \frac{1}{t^2 + 1}\,dt$.}

\subsection{Sur un intervalle du type $]-\infty, +\infty[$}

\definition{
    Soit \functionSets{f}{]-\infty, +\infty[ \rightarrow \mathbb{R} \text{ ou } \mathbb{C}} une fonction continue.\\
    On dit que l'intégrale impropre (ou généralisée) sur $]-\infty, +\infty[$ converge si pour $a \in \mathbb{R}$, les deux intégrales $\int_{-\infty}^a f(t)\,dt$ et $\int_a^{+\infty} f(t)\,dt$ convergent.
}

\example{
    $\int_{-\infty}^{+\infty} \frac{1}{1+t^2}\,dt$ converge-t-elle?
    \begin{enumerate}
        \item D'abord, $t \mapsto \frac{1}{1+t^2}$ est continue sur $]-\infty, +\infty[$.
        \item On a $\int_0^{+\infty} \frac{1}{1+t^2}\,dt = \lim_{x \to +\infty} [\arctan(t)]_0^{x} = \frac{\pi}{2}$ (converge).
        \item On a $\int_{-\infty}^{0} \frac{1}{1+t^2}\,dt = \lim_{y \to -\infty} [\arctan(t)]_{y}^{0} = \frac{\pi}{2}$ (converge).
    \end{enumerate}
    \textbf{Donc $\int_{-\infty}^{+\infty} \frac{1}{1+t^2}\,dt$ converge et vaut $\pi$.}
}

\subsection{Intégrales impropres sur des intervalles $[a, b[, b<+\infty$}

\definition{
    Soient $a < b \in \mathbb{R}$.
    Considérons une fonction continue sur $[a, b[$ \textit{(a priori, f n'est pas continue en b car sinon $\int_a^b f(t)\,dt$ n'est pas impropre)}.\\
     
    On définit l'intégrale impropre $\int_a^b f(t)\,dt$ comme la limite $\lim_{x \to b^-} \int_a^x f(t)\,dt$, si cette limite existe et est finie.\\
    Si cette limite n'existe pas, on dit que l'intégrale diverge.
}

\remark{Si \functionSets{f}{]a, b] \rightarrow \mathbb{R}} est continue, on peut définir l'intégrale impropre $\int_a^b f(t)\,dt$ par $\lim_{x \to a^+} \int_x^b f(t)\,dt$.}

\example{
    $\int_0^1 \frac{1}{t^{\alpha}}\,dt$ converge-t-elle?
    $\int_0^1 \ln(t)\,dt$ converge-t-elle?

}
\remark{Il se peut qu'on rencontre des intégrales impropres aux deux bornes $a$ et $b$ avec \functionSets{f}{]a, b[ \rightarrow \mathbb{R}}. Pour étudier $\int_a^b f(t)\,dt$ on peut utiliser la relation de Chasles ($c \in ]a, b[$), ce qui nous ramène à des intégrales impropres à une seule borne : $\int_a^b f(t)\,dt = \int_a^c f(t)\,dt + \int_c^b f(t)\,dt$.}

\training{\textbf{Résolution d'un faux problème aux bornes par prolongement par continuité.}
Prenons $\int_0^1 \frac{\sin(t)}{t}\,dt$.
\begin{itemize}
    \item $t \mapsto \frac{\sin(t)}{t}$ est $\mathcal{C}^0(]0, 1], \mathbb{R}$ (i.e. continue sur $]0, 1]$).
    \textit{A priori}, on a une intégrale impropre en $0$.
    \item On remarque que $t \mapsto \frac{\sin(t)}{t}$ peut être prolongée par continuité en $0$ par la valeur $1$ (car $\lim_{t \to 0} \frac{\sin(t)}{t} = 1$).
    Donc on définit $\tilde{f}(t) : ]0, 1] \rightarrow \mathbb{R}$, $t \mapsto \begin{cases}
        \frac{\sin(t)}{t} & \text{si } t \in ]0, 1]\\
        1 & \text{si } t = 0
    \end{cases}$.
    \item Maintenant, $\tilde{f}$ est continue sur $[0, 1]$, donc l'intégrale $\int_0^1 \tilde{f}(t)\,dt$ est une intégrale de Riemann classique.
    Et en fait, $\int_0^1 \frac{\sin(t)}{t}\,dt = \int_0^1 \tilde{f}(t)\,dt$. (ne dépend pas des points clés)    
\end{itemize}
}

\theorem{Proposition}{Prolongement par continuité}{false}{
    Soient $a < b \in \mathbb{R}$ et \functionSets{f}{]a, b[ \rightarrow \mathbb{R}} une fonction continue.\\
    Si $f$ peut être prolongée par continuité en $a$ (resp. en $b$), alors l'intégrale impropre $\int_a^b f(t)\,dt$ converge si et seulement si l'intégrale de Riemann $\int_a^b \tilde{f}(t)\,dt$ converge, où $\tilde{f}$ est le prolongement par continuité de $f$ en $a$ (resp. en $b$).\\\\
    Dans ce cas, on a : $\int_a^b f(t)\,dt = \int_a^b \tilde{f}(t)\,dt$.
}

\noindent{\textbf{Preuve:}\\
    C'est par définition de l'intégrale de Riemann : retirer un nombre fini de points d'un intervalle n'affecte pas la valeur de l'intégrale (de l'aire sous la courbe).\\
    \textit{En réalité, ça se démontre via les intégrales de Lebesgue et la mesure, mais ce n'est pas au programme ici.}
    $\Box$
}

\remark{Toutes les propositions évoquées jusqu'ici restent valables pour une fonction continue par morceaux.}
\newpage
\training{Etudier la convergence de $\int_1^2 \frac{\ln(t)}{t-1}\,dt$.}
\noindent{\textbf{Indication:} On peut élargir la classe de fonctions que l'on intègre (fonctions continues) en considérant la classe des fonctions continues par morceaux.}\\
\carreaux{10}

\theorem{Rappel}{Fonctions continues par morceaux}{false}{
    Soit \functionSets{f}{[a, b] \rightarrow \mathbb{R}} est $\mathcal{C}^0 p.m$ (par morceaux) sur $[a, b]$ s'il existe une subdivision de $[a,b]$ donnée par $a=a_0 < a_1 < ... < a_n = b$ telle que :
    \begin{itemize}
        \item $f$ restreinte à chaque intervalle $]a_{i-1}, a_i[$ est continue.
        \item Pour tout $i \in [|1, n-1|]$, les limites $\lim_{x \to a_i^-} f(x)$ et $\lim_{x \to a_i^+} f(x)$ existent et sont finies.
    \end{itemize}
}

\example{La fonction $x \mapsto \lfloor x \rfloor$ est continue par morceaux sur tout intervalle $[a, b]$.}
\cexample{Soit \functionSets{f}{[0, 1] \rightarrow \mathbb{R}}, $f(x) = \begin{cases}
    \frac{1}{t} & \text{si } t > 0\\
    0 & \text{si } t = 0
\end{cases}$ n'est pas continue par morceaux sur $[0, 1]$ car $\lim_{x \to 0^+} f(x) = +\infty$.}

\subsection{Propriétés des intégrales convergences}

Ici, on va considérer $[a,b[$ avec $a\in\mathbb{R}$ et $b \in \mathbb{R} \cup \{+\infty\}$.
\theorem{Propriété}{Linéarité des intégrales impropres}{false}{
    Soient $f,g \in \mathcal{C}^0([a,b[,\mathbb{R})$ et $\lambda, \mu \in \mathbb{R}$ telles que les intégrales impropres $\int_a^b f(t)\,dt$ et $\int_a^b g(t)\,dt$ convergent.\\
    Alors l'intégrale impropre $\int_a^b (\lambda f(t) + \mu g(t))\,dt = \lambda \int_a^b f(t)\,dt + \mu \int_a^b g(t)\,dt$ converge.
}

\theorem{Propriété}{Positivité}{false}{
    Soit $f \in \mathcal{C}^0([a,b[,\mathbb{R})$ telle que $f(t) \geq 0$ pour tout $t \in [a,b[$ et que l'intégrale impropre $\int_a^b f(t)\,dt$ converge.\\
    Alors, $\int_a^b f(t)\,dt \geq 0$.\\\\

    De plus, si $\int_a^b f(t)\,dt = 0$, alors $f(t) = 0$ pour tout $t \in [a,b[$.
}

\theorem{Propriété}{Relation de Chasles}{false}{
    Soit $f \in \mathcal{C}^0([a,b[,\mathbb{R})$ telle que l'intégrale impropre $\int_a^b f(t)\,dt$ converge.\\
    Alors $\forall c \in ]a,b[$, les intégrales impropres $\int_a^c f(t)\,dt$ et $\int_c^b f(t)\,dt$ convergent et on a :
    \[
        \int_a^b f(t)\,dt = \int_a^c f(t)\,dt + \int_c^b f(t)\,dt
    \]
}

\theorem{Propriété}{Intégration par parties}{false}{
    Soient $u,v \in \mathcal{C}^1([a,b[,\mathbb{R})$.\\
    On a $\forall x \in [a,b[$, $\int_a^x u(t)v'(t)\,dt = [u(t)v(t)]_a^x - \int_a^x u'(t)v(t)\,dt$.\\\\
    $= u(x)v(x) - u(a)v(a) - \int_a^x u'(t)v(t)\,dt$.\\\\
    On part de $\int_a^b u(t)v'(t)\,dt$ converge.\\
    Dans ce cas, si $\lim_{x \to b} u(x)v(x)$ et $\lim_{x \to b} \int_a^x u'(t)v(t)\,dt$ existent, alors on peut écrire :
    \[
        \int_a^b u(t)v'(t)\,dt = \lim_{x \to b} [u(x)v(x)]_a^x - \lim_{x \to b} \int_a^x u'(t)v(t)\,dt
    \]
}

\attention{Savoir refaire ce calcul plutôt que de le réciter.}

\training{Montrer que $\int_0^{+\infty} t^n e^{-\lambda t}\,dt$ = $\frac{n!}{\lambda^{n+1}}$ pour tout $n\in \mathbb{N}^*$}
\carreaux{10}

\theorem{Théorème}{Changement de variable}{false}{
    Soient \functionSets{f}{[a,b[ \rightarrow \mathbb{R}} (b peut être infini) une fonction continue et \functionSets{\varphi}{[\alpha,\beta[ \rightarrow [a,b[} une fonction $\mathcal{C}^1$ bijective et strictement croissante. (en fait $\varphi^{-1}(a) = \alpha$ et $\varphi^{-1}(b) = \beta$)\\\\
    Alors $\int_a^b f(t)\,dt$ converge si et seulement si $\int_{\alpha}^{\beta} f(\varphi(t)) \varphi'(t)\,dt$ converge.\\
    Et dans ce cas, on a :
    \[
        \int_a^b f(t)\,dt = \int_{\alpha}^{\beta} f(\varphi(t)) \varphi'(t)\,dt
    \]
}

\noindent{\textbf{Preuve:}\\
    On considère $\int_a^A f(t)\,dt$ pour $A > a$.\\
    On a : $\int_a^A f(t)\,dt = \int_{\varphi^{-1}(a)}^{\varphi^{-1}(A)} f(\varphi(t)) \varphi'(t)\,dt$.\\\\
    En fait, $\varphi^{-1}$ est continue, donc $\mathcal{C}^1$.\\
    Donc si $A \to b$, alors $\varphi^{-1}(A) \to \varphi^{-1}(b)$ et donc $\lim_{A \to +\infty} \int_a^A f(t)\,dt = \\lim_{\varphi^{-1}(A) \to \varphi^{-1}(b)} \int_{\varphi^{-1}(a)}^{\varphi^{-1}(A)} f(\varphi(t)) \varphi'(t)\,dt$.\\\\
    Et donc $\int_a^b f(t)\,dt = \int_{\alpha}^{\beta} f(\varphi(t)) \varphi'(t)\,dt$. $\Box$
}

\remark{On ne change pas la nature de l'intégrale impropre avec un changement de variable.}

\example{Convergence et calcul de $\int_0^{+\infty} \frac{e^t}{1+e^{2t}}\,dt$.\\
En pratique, on note $\varphi = u$. Ici, posons $u(t) = e^t$, donc $du = e^t dt$.\\
Alors $\int_0^{+\infty} \frac{e^t}{1+e^{2t}}\,dt = \int_{u(0)}^{u(+\infty)} \frac{1}{1+u^2}\,du = \int_1^{+\infty} \frac{1}{1+u^2}\,du$.\\
On sait que cette intégrale converge et vaut $\frac{\pi}{4}$ (arctan).}

\section{Fonctions positives}

On se donne \functionSets{f}{[a,b[ \rightarrow \mathbb{R}} une fonction continue.

\remark{Toutes les propositions qui suivent sont analogues aux séries numériques à termes positifs.}

\theorem{Proposition}{}{false}{
    $\int_a^b f(t)\,dt$ converge si et seulement si la fonction $x \mapsto \int_a^x f(t)\,dt$ est bornée.
}
\remark{Ici, c'est une série converge ssi la suite des sommes partielles est bornée.}
\noindent{\textbf{Preuve:}\\
    La fonction $F: x \mapsto \int_a^x f(t)\,dt$ est croissante.\\
    Si $x' > x$ alors $F(x') - F(x) = \int_a^{x'} f(t)\,dt - \int_a^x f(t)\,dt = \int_x^{x'} f(t)\,dt \geq 0$ (car $f$ positive).\\\\
}

\theorem{Théorème}{Comparaison}{false}{
    Soient $f,g \in \mathcal{C}^0([a,b[,\mathbb{R})$. Supposons $0 \leq f(t) \leq g(t)$ pour tout $t \in [a,b[$.\\
    Alors :
    \begin{itemize}
        \item Si $\int_a^b g(t)\,dt$ converge, alors $\int_a^b f(t)\,dt$ converge.
        \item Si $\int_a^b f(t)\,dt$ diverge, alors $\int_a^b g(t)\,dt$ diverge.
    \end{itemize}
}

\theorem{Théorème}{Comparaison par équivalences}{false}{
    Soient $f,g \in \mathcal{C}^0([a,b[,\mathbb{R})$.
    On suppose $f$ positive (ou signe constant) au moins au voisinage de $b$ et que $f(t) \sim_{t \to b} g(t)$.\\
    Alors les intégrales impropres $\int_a^b f(t)\,dt$ et $\int_a^b g(t)\,dt$ sont de même nature.
}

\end{document}