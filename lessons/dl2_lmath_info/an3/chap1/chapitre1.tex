\documentclass{article}

\usepackage[a4paper, left=1.5cm, right=1.5cm, top=2cm, bottom=2cm]{geometry}

\usepackage{../../../../components/components} % <-- ton fichier .sty, avec toutes tes définitions

\usepackage{fancyhdr}


% Configuration des en-têtes et pieds de page
\pagestyle{fancy}
\fancyhf{} % reset tout

\fancyhead[L]{DL2 Math-Info AN3}
\fancyhead[C]{Suites de Cauchy}
\fancyhead[R]{2025-2026}

\fancyfoot[L]{Ewen Rodrigues de Oliveira}
\fancyfoot[R]{\thepage}

\begin{document}

\docTitle{Chapitre 1 : Suites de Cauchy}

\section{Rappels sur les suites}

\subsection{Définitions générales}

On ne rappelera que ce qui n'est pas "évident" dans le cours de L1.

\definition{
    Une sous-suite (ou suite extraite) d'une suite $(u_n)$ est une suite $(v_n)$ : \(\exists \varphi : \mathbb{N} \to \mathbb{N}, \text{ strictement croissante tq } v_n = u_{\varphi(n)} \)
}

\vocabulary{
    Une sous-suite de $(u_n)$ est aussi notée $(u_{n_k})$.
}

\definition{
    Une suite $(u_n)$ converge vers $l\in\mathbb{R}$ si : \(\forall \varepsilon > 0, \exists N_\varepsilon \in \mathbb{N}, \forall n \geq N_\varepsilon, |u_n - l| < \varepsilon\)
}
\vocabulary{
    Si elle ne converge pas on dit qu'elle diverge. \\
    Attention : une suite peut diverger mais avoir une limite (une suite qui tend vers l'infini).
}


\theorem{Propriété}{Bornes}{true}{Si une suite $(u_n)$ converge, alors elle est bornée : \(\exists M > 0, \forall n \in \mathbb{N}, |u_n| \leq M\).}
\theorem{Propriété}{Convergence des sous-suites}{true}{Si une suite $(u_n)$ converge vers $l$, alors toute sous-suite $(u_{n_k})$ converge aussi vers $l$.}

\subsection{Propriétés et théorèmes fondamentaux}
\theorem{Propriété}{Espace-vectoriel}{true}{L'ensemble des suites réelles convergeantes est un $\mathbb{R}$-espace vectoriel.}
\theorem{Théorème}{Suites adjacentes}{true}{
    Deux suites $(u_n)$ et $(v_n)$ sont dites adjacentes si :
    \begin{itemize}
        \item $(u_n)$ est croissante et $(v_n)$ est décroissante
        \item \(u_n \leq v_n, \forall n \in \mathbb{N}\)
        \item \(v_n - u_n \xrightarrow[n\to\infty]{} 0\)
    \end{itemize}
    Si deux suites sont adjacentes, alors elles convergent vers la même limite.
}
\theorem{Théorème}{Bolzano-Weierstrass}{true}{Toute suite réelle bornée admet une sous-suite convergente.}

\section{Suites de Cauchy}
\definition{
    Une suite $(u_n)$ est une suite de Cauchy si : \(\forall \varepsilon > 0, \exists N_\varepsilon \in \mathbb{N}, \forall p,q \geq N_\varepsilon, |u_p - u_q| < \varepsilon\)
}
\theorem{Propriété}{Convergence}{false}{Toute suite convergente est une suite de Cauchy.}
\noindent\textbf{Preuve:}\\
\carreaux{7}
\theorem{Proposition}{Bornes}{false}{Toute suite de Cauchy est bornée.}
\noindent\textbf{Preuve:}\\
\carreaux{7}

\theorem{Théorème}{}{false}{Toute suite de Cauchy dans $\mathbb{R}$ converge dans $\mathbb{R}$.}
\noindent\textbf{Preuve:}\\
\carreaux{10}
\remark{On dit que $\mathbb{R}$ est complet.}

\definition{
    On dit que $(\mathbb{R}, |\cdot|)$ est complet.
}
\example{
    Notion de complétude\\
    $(\mathbb{Q}, |\cdot|)$ n'est pas complet : la suite définie par $u_n = \text{la partie décimale de } \sqrt{2} \text{ à la } n\text{-ième décimale}$ est une suite de Cauchy dans $\mathbb{Q}$ qui ne converge pas dans $\mathbb{Q}$ (car $\sqrt{2} \notin \mathbb{Q}$).\\
    Par contre, elle converge dans $\mathbb{R}$.
}

\end{document}