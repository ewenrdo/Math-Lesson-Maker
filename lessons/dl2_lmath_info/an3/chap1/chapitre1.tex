\documentclass{article}

\usepackage[a4paper, left=1.5cm, right=1.5cm, top=2cm, bottom=2cm]{geometry}

\usepackage{../../../../components/components} % <-- ton fichier .sty, avec toutes tes définitions

\usepackage{fancyhdr}


% Configuration des en-têtes et pieds de page
\pagestyle{fancy}
\fancyhf{} % reset tout

\fancyhead[L]{DL2 Math-Info AN3}
\fancyhead[C]{Suites de Cauchy}
\fancyhead[R]{2025-2026}

\fancyfoot[L]{Ewen Rodrigues de Oliveira}
\fancyfoot[R]{\thepage}

\begin{document}

\docTitle{Chapitre 1 : Suites de Cauchy}

\section{Rappels sur les suites}

\subsection{Définitions générales}

On ne rappelera que ce qui n'est pas "évident" dans le cours de L1.

\definition{
    Une sous-suite (ou suite extraite) d'une suite $(u_n)$ est une suite $(v_n)$ : \(\exists \varphi : \mathbb{N} \to \mathbb{N}, \text{ strictement croissante tq } v_n = u_{\varphi(n)} \)
}

\vocabulary{
    Une sous-suite de $(u_n)$ est aussi notée $(u_{n_k})$.
}

\definition{
    Une suite $(u_n)$ converge vers $l\in\mathbb{R}$ si : \(\forall \varepsilon > 0, \exists N_\varepsilon \in \mathbb{N}, \forall n \geq N_\varepsilon, |u_n - l| < \varepsilon\)
}
\vocabulary{
    Si elle ne converge pas on dit qu'elle diverge. \\
    Attention : une suite peut diverger mais avoir une limite (une suite qui tend vers l'infini).
}


\theorem{Propriété}{Bornes}{true}{Si une suite $(u_n)$ converge, alors elle est bornée : \(\exists M > 0, \forall n \in \mathbb{N}, |u_n| \leq M\).}
\theorem{Propriété}{Convergence des sous-suites}{true}{Si une suite $(u_n)$ converge vers $l$, alors toute sous-suite $(u_{n_k})$ converge aussi vers $l$.}

\subsection{Propriétés et théorèmes fondamentaux}
\theorem{Propriété}{Espace-vectoriel}{true}{L'ensemble des suites réelles convergeantes est un $\mathbb{R}$-espace vectoriel.}
\theorem{Théorème}{Suites adjacentes}{true}{
    Deux suites $(u_n)$ et $(v_n)$ sont dites adjacentes si :
    \begin{itemize}
        \item $(u_n)$ est croissante et $(v_n)$ est décroissante
        \item \(u_n \leq v_n, \forall n \in \mathbb{N}\)
        \item \(v_n - u_n \xrightarrow[n\to\infty]{} 0\)
    \end{itemize}
    Si deux suites sont adjacentes, alors elles convergent vers la même limite.
}
\theorem{Théorème}{Bolzano-Weierstrass}{true}{Toute suite réelle bornée admet une sous-suite convergente.}

\section{Suites de Cauchy}
\definition{
    Une suite $(u_n)$ est une suite de Cauchy si : \(\forall \varepsilon > 0, \exists N_\varepsilon \in \mathbb{N}, \forall p,q \geq N_\varepsilon, |u_p - u_q| < \varepsilon\)
}
\theorem{Propriété}{Convergence}{false}{Toute suite convergente est une suite de Cauchy.}
\noindent\textbf{Preuve:}\\
\carreaux{7}
\theorem{Proposition}{Bornes}{false}{Toute suite de Cauchy est bornée.}
\noindent\textbf{Preuve:}\\
\carreaux{7}

\theorem{Théorème}{}{false}{Toute suite de Cauchy dans $\mathbb{R}$ converge dans $\mathbb{R}$.}
\noindent\textbf{Preuve:}\\
\carreaux{10}
\remark{On dit que $\mathbb{R}$ est complet.}

\definition{
    On dit que $(\mathbb{R}, |\cdot|)$ est complet.
}
\example{
    Notion de complétude\\
    $(\mathbb{Q}, |\cdot|)$ n'est pas complet : la suite définie par $u_n = \text{la partie décimale de } \sqrt{2} \text{ à la } n\text{-ième décimale}$ est une suite de Cauchy dans $\mathbb{Q}$ qui ne converge pas dans $\mathbb{Q}$ (car $\sqrt{2} \notin \mathbb{Q}$).\\
    Par contre, elle converge dans $\mathbb{R}$.
}
\section{Topologie de $\mathbb{R}$}
\subsection{Rappels}
\textcolor{primary}{a) Ouvert}
\definition{Soit \(x\in\mathbb{R}\) et \(V\subset \mathbb{R}\). On dit que $V$ est un voisinage de $x$ si : \(\exists \varepsilon > 0, ]x-\varepsilon, x+\varepsilon[ \subset V\)}
\definition{\(U\subset \mathbb{R}\) est un ouvert de $\mathbb{R}$ si : \(\forall x \in U, U \text{ est un voisinage de } x\)}

\example{
    \begin{itemize}
        \item \(\mathbb{R}\) est un ouvert de $\mathbb{R}$.
        \item \(]a,b[ \text{ est un ouvert de } \mathbb{R}\)
        \item \(]a,+\infty[ \text{ est un ouvert de } \mathbb{R}\)
        \item \(]-\infty,a[ \text{ est un ouvert de } \mathbb{R}\)
        \item L'ensemble vide est un ouvert de $\mathbb{R}$.
    \end{itemize}  
}

\theorem{Propriété}{Opérations sur les ouverts}{true}{
    \begin{itemize}
        \item L'intersection finie d'ouverts est un ouvert.
        \item L'union quelconque d'ouverts est un ouvert.
    \end{itemize}
}

\remark{
    %% Grande intersection de ]-1/n, 1/n[ = {0} en symbole maths
    L'intersection infinie d'ouverts n'est pas forcément un ouvert : \(\bigcap_{n=1}^{\infty} ]-\frac{1}{n}, \frac{1}{n}[ = \{0\}\) qui n'est pas un ouvert de $\mathbb{R}$.\\
    Toutefois, pour un \(n_{max}\) donné l'intersection de \(n_{max}\) ouverts est un ouvert.

}

\textcolor{primary}{b) Fermé}

\definition{
    \(F\subset \mathbb{R}\) est un fermé de $\mathbb{R}$ si : \(\mathbb{R} \setminus F \text{ est un ouvert de } \mathbb{R}\)
}

\example{
    \begin{itemize}
        \item \(\mathbb{R}\) est un fermé de $\mathbb{R}$.
        \item \([a,b] \text{ est un fermé de } \mathbb{R}\)
        \item Toute famille finie d'éléments de $\mathbb{R}$ est un fermé de $\mathbb{R}$.
        \item L'ensemble vide est un fermé de $\mathbb{R}$.
    \end{itemize}  
}

\remark{$\mathbb{Q}$ n'est ni ouvert ni fermé dans $\mathbb{R}$.}

\theorem{Propriété}{Opérations sur les fermés}{true}{
    \begin{itemize}
        \item L'union finie de fermés est un fermé.
        \item L'intersection quelconque de fermés est un fermé.
    \end{itemize}
}

\theorem{Théorème}{Caractérisation zéquentielle des fermés}{false}{
    \(F\subset \mathbb{R}\) fermé $\Leftrightarrow$ toute suite $(u_n)$ d'éléments de $F$ qui converge dans $\mathbb{R}$ a sa limite dans $F$.
}
\noindent{\textbf{Preuve:}\\
$\Rightarrow$/ Supposons F fermé, on a \(\mathbb{R} \setminus F\) ouvert.\\
Par l'absurde, on suppose qu'il existe une suite $(u_n)$ d'éléments de $F$ qui converge vers \(l \in \mathbb{R} \setminus F\).\\
Comme \(\mathbb{R} \setminus F\) est un ouvert, $\exists \varepsilon > 0, ]l-\varepsilon, l+\varepsilon[ \subset \mathbb{R} \setminus F$.\\
Par convergence de $(u_n)$, $\exists N_\varepsilon \in \mathbb{N}, \forall n \geq N_\varepsilon, |u_n - l| < \varepsilon \Rightarrow u_n \in ]l-\varepsilon, l+\varepsilon[ \subset \mathbb{R} \setminus F$ ce qui est absurde car $(u_n)$ est une suite d'éléments de $F$.\\\\

$\Leftarrow$/ On raisonne par contraposée.\\
Si $\mathbb{R} \setminus F$ n'est pas un ouvert, $\exists l \in \mathbb{R} \setminus F$ tel que $\forall r > 0, ]l-r, l+r[ \cap F \neq \mathbb{R}\setminus F$ car $\mathbb{R}\setminus F$ est au voisinage de $l$.\\
Supposons qu'il existe une suite $(u_n)$ d'éléments de $F$.\\
En particulier, $\forall n \in \mathbb{N}^*, u_n \in ]l-\frac{1}{n}, l+\frac{1}{n}[ \cap F \implies u_n \xrightarrow[n\to\infty]{} l \text{ et } (u_n) \in F$.\\
}
\remark{Ce théorème est utile pour montrer qu'on a un ensemble fermé.   }


\definition{Soit $A\subset\mathbb{R}$.\\
On définit l'adhérence de $A$, notée $\overline{A}$, comme suit : \(\overline{A} = \bigcap_{F \text{ fermé},\, A \subset F} F\).\\
C'est le plus petit fermé contenant $A$.
}

\theorem{Lemme}{}{false}{
    Soit \(A\subset \mathbb{R}\).\\
    $x\in\overline{A} \Leftrightarrow \forall \varepsilon > 0, ]x-\varepsilon, x+\varepsilon[ \cap A \neq \emptyset$.
}
\noindent\textbf{Preuve:}\\
\carreaux{10}

\theorem{Théorème}{}{false}{
    Soit \(A\subset \mathbb{R}\).\\
    Alors \(\overline{A} = \{x\in\mathbb{R}, \exists (u_n) \text{ suite d'éléments de } A, u_n \xrightarrow[n\to\infty]{} x\}\).
}
\remark{
    Autrement dit, l'adhérence de $A$ est l'ensemble des limites de suites d'éléments de $A$.
}
\noindent\textbf{Preuve:}\\
\carreaux{10}

\subsection{Complétude}

\definition{
    $F\subset \mathbb{R}$ est complet si toute suite de Cauchy d'éléments de $F$ converge dans $F$.
}

\example{\(\mathbb{R}\) est complet.}

\theorem{Théorème}{Caractérisation des parties complètes de $\mathbb{R}$}{false}{
    \(F\subset \mathbb{R}\) est complet $\Leftrightarrow$ $F$ est fermé.
}
\noindent\textbf{Preuve:}\\
\carreaux{12}

\end{document}