\documentclass{article}

\usepackage[a4paper, left=1.5cm, right=1.5cm, top=2cm, bottom=2cm]{geometry}

\usepackage{amsmath,amssymb}

\usepackage{../../../../../components/components} % <-- ton fichier .sty, avec toutes tes définitions

\usepackage{fancyhdr}


% Configuration des en-têtes et pieds de page
\pagestyle{fancy}
\fancyhf{} % reset tout

\fancyhead[L]{DL2 Math-Info}
\fancyhead[C]{Topologie}
\fancyhead[R]{2025-2026}

\fancyfoot[L]{Ewen Rodrigues de Oliveira}
\fancyfoot[R]{\thepage}

\begin{document}

\docTitle{Chapitre 4.2 : Topologie des espaces métriques et des espaces vectoriels normés}


\section{Topologie sur les espaces métriques}
\subsection{Ouverts et fermés}
\remark{Un evn étant un espace métrique avec la distance $d(x,y) = N(y - x)$, toutes les notions de topologie vues ici s'appliquent aux evn.}

\definition{Soit $(X,d)$ un espace métrique.\\
    Une boule ouverte de centre $a\in X$ de rayon $r>0$ est $B(a,r) = \{ x \in X \mid d(x,a) = d(a,x) < r \}$.
}

\definition{Soit $(X,d)$ un espace métrique.\\
    Une boule fermée de centre $a\in X$ de rayon $r>0$ est $B_F(a,r) = \{ x \in X \mid d(a,x) \leq r \}$.
}

\definition{Soit $(X,d)$ un espace métrique.\\
    La sphère de centre $a\in X$ de rayon $r>0$ est $S(a,r) = \{ x \in X \mid d(a,x) = r \}$.
}

\example{
    Dans $(\mathbb{R}, |\cdot|)$, $B(1,1) = ]0,2[$
}

\example{
    Dans $\mathbb{R}^2$ avec trois métriques :\\
    $d_1 = \|y-x\|_1$, $d_2 = \|y-x\|_2$ et $d_{\infty} = \|y-x\|_{\infty}$.\\
    Traçons les boules de centre 0 de rayon 1 associées à chaque distance.\\
    $B(0,1) = \{ x\in \mathbb{R}^2 \mid d(x,0) < 1\}$.\\
    On a : $\|x\|_{1,2,\infty} < 1$.\\
    \begin{itemize}
        \item Pour $d_1$ : $\|x\|_1 = |x_1| + |x_2| < 1$. C'est un losange.
        \item Pour $d_2$ : $\|x\|_2 = \sqrt{x_1^2 + x_2^2} < 1$. C'est un disque.
        \item Pour $d_{\infty}$ : $\|x\|_{\infty} = \max(|x_1|, |x_2|) < 1$. C'est un carré.
    \end{itemize}
}

\remark{Les bordures correspondent aux sphères $S(0,1)$ associées à chaque distance. Les boules fermées $B_F(0,1)$ correspondent aux mêmes figures mais en incluant les bordures.}

\definition{
    Une partie $U$ de $X$ est un ouvert de $(X,d)$ si $\forall x \in U, \exists r > 0$ tel que $B(x,r) \subset U$.
}

\example{
    $]0,1[ \subset \mathcal{C}(]0,1[,\mathbb{R})$ est un ouvert : $\forall x \in ]0,1[, B(x, \min(x, 1-x)) \subset ]0,1[$.
}

\definition{
    Une topologie sur $(X,d)$ est l'ensemble des ouverts de $(X,d)$. Autrement dit, $\tau = \{ U \subset X \mid U \text{ est un ouvert de } (X,d) \}$.
}

\theorem{Proposition}{}{false}{
    \begin{enumerate}
        \item Toute boule ouverte de $(X,d)$ est un ouvert de $(X,d)$.
        \item Si $(U_i)_{i \in I}$ est une famille d'ouverts de $(X,d)$, alors $\bigcup_{i \in I} U_i$ est un ouvert de $(X,d)$. (I un ensemble quelconque)
        \item Si $(U_1, U_2, \ldots, U_n)$ est une famille finie d'ouverts de $(X,d)$, alors $\bigcap_{i=1}^{n} U_i$ est un ouvert de $(X,d)$.
    \end{enumerate}
}

\noindent{\textbf{Preuve :} 
    \begin{enumerate}
        \item Soit $B(a,r)$ une boule ouverte de $(X,d)$. Soit $x \in B(a,r)$. On a $d(a,x) < r$. Posons $s = r - d(a,x) > 0$. Montrons que $B(x,s) \subset B(a,r)$.\\
        Soit $y \in B(x,s)$. On a $d(x,y) < s$. D'après l'inégalité triangulaire, on a $d(a,y) \leq d(a,x) + d(x,y) < d(a,x) + s = d(a,x) + (r - d(a,x)) = r$. Donc $y \in B(a,r)$. Ainsi, $B(x,s) \subset B(a,r)$ et donc $B(a,r)$ est un ouvert de $(X,d)$.
        \item Soit $x \in (U_i)_{i \in I}$. Alors $\exists i_0 \in I$ tel que $x \in U_{i_0}$. Comme $U_{i_0}$ est un ouvert de $(X,d)$, il existe $r > 0$ tel que $B(x,r) \subset U_{i_0} \subset \bigcup_{i \in I} U_i$. Donc $\bigcup_{i \in I} U_i$ est un ouvert de $(X,d)$.
        \item Soit $x \in \bigcap_{i=1}^{n} U_i$, où on a écrit $I = \{1, 2, \ldots, n\}, n \in \mathbb{N}^*$.\\
        Alors $\forall i \in \{1, 2, \ldots, n\}, \exists r_i > 0$ tel que $B(x,r_i) \subset U_i$. Posons $r = \min(r_1, r_2, \ldots, r_n) > 0$. Alors $B(x,r) \subset \bigcap U_i$ pour tout $i \in \{1, 2, \ldots, n\}$.\\
        En effet, soit $z \in B(x,r), d(z,x) < r \leq r_i$ donc $z \in B(x,r_i) \subset U_i$ pour tout $i \in \{1, 2, \ldots, n\}$. Donc $z \in \bigcap_{i=1}^{n} U_i$. Ainsi, $\bigcap_{i=1}^{n} U_i$ est un ouvert de $(X,d)$.
    \end{enumerate}
}

\cexample{Si $I$ est infini, la propriété 3 n'est pas vraie en général. Par exemple, dans $(\mathbb{R}, |\cdot|)$, considérons la famille d'ouverts $U_n = ]-1/n, 1/n[$ pour $n \in \mathbb{N}^*$. Alors $\bigcap_{n=1}^{\infty} U_n = \{0\}$ qui n'est pas un ouvert de $(\mathbb{R}, |\cdot|)$.}

\definition{
    Soit $(X,d)$ un espace métrique.\\
    Une partie $F$ de $X$ est un fermé de $(X,d)$ si son complémentaire $X \setminus F$ est un ouvert de $(X,d)$.
}

\theorem{Proposition}{}{false}{
    \begin{enumerate}
        \item Toute boule fermée de $(X,d)$ est un fermé de $(X,d)$.
        \item Si $(F_i)_{i \in I}$ est une famille de fermés de $(X,d)$, alors $\bigcap_{i \in I} F_i$ est un fermé de $(X,d)$. (I un ensemble quelconque)
        \item Si $(F_1, F_2, \ldots, F_n)$ est une famille finie de fermés de $(X,d)$, alors $\bigcup_{i=1}^{n} F_i$ est un fermé de $(X,d)$.
    \end{enumerate}
}

\noindent{\textbf{Preuve :} 
    \begin{enumerate}
        \item Soit $B_F(x,r)$ une boule fermée de $(X,d)$. C'est un fermé $\Leftrightarrow$ $X \setminus B_F(x,r)$ est un ouvert de $(X,d)$.\\
        Soit $y \in X \setminus B_F(x,r)$. On a $d(x,y) > r$. Posons $\varrho = d(y,x) - r > 0$. Montrons que $B(y,\varrho) \subset X \setminus B_F(x,r)$.\\
        Soit $z \in B(y,\varrho)$. On a $d(y,z) < \varrho$.\\
        $d(z,y) < d(y,x) - r \Rightarrow r < d(y,x) - d(z,y) \leq d(z,x)$ (inégalité triangulaire) $\Rightarrow z \in X \setminus B_F(x,r)$.\\
        Ainsi, $B(y,\varrho) \subset X \setminus B_F(x,r)$ et donc $X \setminus B_F(x,r)$ est un ouvert de $(X,d)$.\\
        Donc $B_F(x,r)$ est un fermé de $(X,d)$.
        \item Laissé en exercice au lecteur.
        \item Laissé en exercice au lecteur.
    \end{enumerate}
}

\remark{
    Dans $(\mathbb{R}, |\cdot|)$, tout intervalle fermé est un fermé, et tout intervalle ouvert est un ouvert. ($\mathbb{R}$ est ouvert).\\
    On a : $]a, +\infty[ = \bigcup_{n=1}^{\infty} ]a, a+n[$ est un ouvert.
}

\remark{$X$ et $\emptyset$ sont des ouverts et des fermés de $(X,d)$ (prendre $r$ arbitrairement grand pour $X$ et $r$ quelconque pour $\emptyset$).}

\definition{Soit $\mathcal{V} \subset X$ une partie de l'espace métrique $(X,d)$.\\
    On dit que $\mathcal{V}$ est un voisinage de $a \in X$ si $\exists r > 0$ tel que $B(a,r) \subset \mathcal{V}$.
}

\theorem{Proposition}{}{false}{
    On dit aussi que $U$ est un ouvert de $(X,d)$ si et seulement si $U$ est un voisinage de chacun de ses points.
}

\subsection{Intérieur et adhérence}
Soit $(X,d)$ un espace métrique.
\definition{Soit $A \subset X$ une partie de l'espace métrique $(X,d)$.\\
    On appelle intérieur de $A$ l'ensemble des points $a \in A$ tels que $A$ est un voisinage de $a$. On le note : $\overset{\circ}{A}$.\\
    On a : \[\overset{\circ}{A} = \bigcup \{ U \subset A \mid U \text{ est un ouvert de } (X,d) \}\]
}

\definition{Soit $A \subset X$ une partie de l'espace métrique $(X,d)$.\\
    On appelle adhérence de $A$ l'ensemble des points $x \in X$ tels que $\forall r > 0, B(x,r) \cap A \neq \emptyset$. On le note : $\overline{A}$.\\
    On a : \[\overline{A} = \bigcap _{F \supset A, F \text{ fermé de } (X,d)} F\]
}

\theorem{Proposition}{Lien avec les ouverts}{false}{
    Soit $A \subset X$ une partie de l'espace métrique $(X,d)$.
    \begin{enumerate}
        \item $\overset{\circ}{A}$ est un ouvert contenu dans $A$.
        \item Si $U \subset A$ est un ouvert de $(X,d)$, alors $U \subset \overset{\circ}{A}$.\\
        Autrement dit, $\overset{\circ}{A}$ est le plus grand ouvert contenu dans $A$.
    \end{enumerate}
}

\noindent{\textbf{Preuve :} 
    \begin{enumerate}
        \item C'est une union quelconque d'ouverts indus dans $A$, donc $\overset{\circ}{A}$ est un ouvert. De plus, par définition, $\overset{\circ}{A} \subset A$.
        \item Par définition $\overset{\circ}{A} = \bigcup \{ U \subset A \mid U \text{ est un ouvert de } (X,d) \}$. Donc si $U \subset A$ est un ouvert de $(X,d)$, alors $U$ est dans la famille indexée par l'union, donc $U \subset \overset{\circ}{A}$.
    \end{enumerate}
}

\theorem{Proposition}{Lien avec les fermés}{false}{
    Soit $A \subset X$ une partie de l'espace métrique $(X,d)$.
    \begin{enumerate}
        \item $\overline{A}$ est un fermé contenant $A$.
        \item Si $F \supset A$ est un fermé de $(X,d)$, alors $\overline{A} \subset F$.\\
        Autrement dit, $\overline{A}$ est le plus petit fermé contenant $A$.
    \end{enumerate}
}

\noindent{\textbf{Preuve :} 
    \begin{enumerate}
        \item C'est une intersection quelconque de fermés contenant $A$, donc $\overline{A}$ est un fermé. De plus, par définition, $A \subset \overline{A}$.
        \item Par définition $\overline{A} = \bigcap _{F \supset A, F \text{ fermé de } (X,d)} F$. Donc si $F \supset A$ est un fermé de $(X,d)$, alors $F$ est dans la famille indexée par l'intersection, donc $\overline{A} \subset F$.
    \end{enumerate}
}

\theorem{Proposition}{}{false}{
    \begin{enumerate}
        \item $\overset{\circ}{A} = X \setminus \overline{(X \setminus A)}$
        \item $\overline{A} = X \setminus \overset{\circ}{(X \setminus A)}$
        \item $x \in \overline{A} \Leftrightarrow \forall r > 0, B(x,r) \cap A \neq \emptyset$
    \end{enumerate}
}

\noindent{\textbf{Preuve :} 
    \begin{enumerate}
        \item ($\Rightarrow$) $x \in \overset{\circ}{A}$. Par définition d'un ouvert, $\exists r > 0$ tel que $B(x,r) \subset \overset{\circ}{A} \subset A$.\\
        ($\Leftarrow$) On a $B(x,r) \subset A$ qui est un ouvert.\\
        Donc $B(x,r) \subset \overset{\circ}{A}$ car $\overset{\circ}{A}$ est le plus grand ouvert contenu dans $A$. Donc $x \in \overset{\circ}{A}$, donc $x \in \overset{\circ}{A}$.\\
        \item $\overset{\circ}{A} \Leftrightarrow X\setminus\overset{\circ}{A} = \overline{(X \setminus A)}$.\\
        Or $\overset{\circ}{A} = \bigcup \{ U \subset A \mid U \text{ est un ouvert de } (X,d) \}$.\\
        Donc $X \setminus \overset{\circ}{A} = X \setminus \bigcup \{ U \subset A \mid U \text{ est un ouvert de } (X,d) \} = \bigcap_{F \supset X \setminus A, F \text{ fermé de } (X,d)} F = \overline{(X \setminus A)}$.\\
        Faire de même avec $\overline{A} = X \setminus \overset{\circ}{(X \setminus A)}$.
        \item $x \in \overline{A} \Leftrightarrow x \in X \setminus \overset{\circ}{(X \setminus A)}$ (d'après 2) $\Leftrightarrow x \notin \overset{\circ}{(X \setminus A)}$ $\Leftrightarrow$ pour tout $r > 0$, $B(x,r) \not\subset X \setminus A$ $\Leftrightarrow$ pour tout $r > 0$.
        $\Leftarrow B(x,r) \cap A \neq \emptyset$.
    \end{enumerate}
}

\theorem{Proposition}{}{false}{
    \begin{itemize}
    \item $U$ est ouvert $\Leftrightarrow \overset{\circ}{U} = U$.
    \item $F$ est fermé $\Leftrightarrow \overline{F} = F$.
    \end{itemize}
}

\subsection{Suites dans un espace métrique}

\subsubsection{Définitions}

\definition{
    On dit qu'une suite $(x_n)$, $x_n \in X$ pour tout $n \in \mathbb{N}$, converge si $\exists x \in X$ tel que $d(x_n,x) \to 0$ quand $n \to +\infty$.\\
    Autrement dit, \[\forall \varepsilon > 0, \exists N \in \mathbb{N}, \forall n \geq N, d(x_n,x) < \varepsilon\]
}

\remark{Si $(X,d) = (\mathbb{R}, |\cdot|)$, on retrouve la définition usuelle de la convergence des suites réelles.}

\definition{
    On dit que $x \in X$ est une valeur d'adhérence d'une suite $(x_n)$ si il existe une sous suite qui converge vers $x$.
    i.e. $\exists (n_k)_{k \in \mathbb{N}}$ strictement croissante telle que $d(x_{n_k}, x) \to 0$ quand $k \to +\infty$.
}

\theorem{Proposition}{}{false}{
    Soit $(x_n)$ une suite de $(X,d)$ qui converge vers $x$.\\
    Alors $x$ est la seule valeur d'adhérence de $(x_n)$. En particulier, la limite de $(x_n)$ est unique.
}


\noindent{\textbf{Preuve :}\\
Soit $x$ la limite de $(x_n)$. Suppons qu'il existe une sous-suite $(x_{n_k})$ qui converge vers une autre valeur $y \neq x$.\\
On a $d(x,y) > 0$. Posons $\varepsilon = \frac{d(x,y)}{2} > 0$.\\
$\exists k_1, (\varepsilon, y) \in \mathbb{N}, \forall k \geq k_1, d(x_{n_k}, y) < \varepsilon$. (convergence de la sous-suite)\\
Et comme $x_n \to x$, alors $x_{n_k} \to x$ aussi.\\
Donc $\exists k_2, (\varepsilon, x) \in \mathbb{N}, \forall k \geq k_2, d(x_{n_k}, x) < \varepsilon$.\\
Alors on a pour $k \geq \max(k_1, k_2)$ :\\
$d(x,y) \leq d(x, x_{n_k}) + d(x_{n_k}, y) < \varepsilon + \varepsilon = 2\varepsilon = d(x,y)$, ce qui est absurde. Donc $x$ est la seule valeur d'adhérence de $(x_n)$.
}\\

\noindent{\textbf{Fermés}\\Soit $A$ une partie quelconque de l'espace métrique $(X,d)$.
}

\theorem{Proposition}{Caractérisation de l'adhérence par des suites}{false}{
    $\overline{A} = \{ x \in X \mid \exists (x_n) \text{ suite de } A, x_n \to x \}$.
}
\noindent{\textbf{Preuve :}\\
$(\subset)$ Soit $x \in X$ tq $\exists (x_n) \in A$ avec $d(x_n, x) \to 0$.\\
On veut montrer que $x \in \overline{A} \Rightarrow$ pour tout $r > 0, B(x,r) \cap A \neq \emptyset$.\\
Soit $r > 0$. Comme $d(x_n, x) \to 0$, il existe $N \in \mathbb{N}$ tel que $\forall n \geq N, d(x_n, x) < r$.\\
On a $x_n \in A$ donc $x_n \in B(x,r) \cap A \neq \emptyset \forall n \geq N$. Donc $x \in \overline{A}$. $\Box$\\
$(\supset)$ Soit $x \in \overline{A} \Rightarrow$ pour tout $r > 0, B(x,r) \cap A \neq \emptyset$.\\*
Prenons, $n \in \mathbb{N}$, $r = \frac{1}{n+1} > 0$.\\
Alors $B(x, \frac{1}{n+1}) \cap A \neq \emptyset$.\\
Pour chaque $n \in \mathbb{N}$, choisissons $x_n \in B(x, \frac{1}{n+1}) \cap A$.\\
On a alors construit une suite $(x_n)$ qui vérifie $d(x_n, x) < \frac{1}{n+1}$.\\
Ainsi, $x$ est bien la limite d'une suite d'éléments de $A$. $\Box$
}

\theorem{Proposition}{Caractérisation des fermés par des suites}{false}{
    Une partie $F$ de $(X,d)$ est fermée si et seulement si pour toute suite $(x_n)$ d'éléments de $F$ qui converge vers $x \in X$, on a $x \in F$.
}

\noindent{\textbf{Preuve :}\\
On a $\overline{F} = \{ x \in X \mid \exists (x_n) \text{ suite de } F, x_n \to x \}$.\\
Or $\overline{F} = F$. On a donc le résultat.
}

\subsection{Continuité}

\subsubsection{Définitions}

\definition{
    Soit $f \colon (X,d_X) \to (Y,d_Y)$ une application entre deux espaces métriques.\\
    \begin{itemize}
        \item $f$ est continue en $a \in X$ si $\forall \varepsilon > 0, \exists \delta_\varepsilon > 0 \colon x \in B(a, \delta_\varepsilon)$, alors $f(x) \in B(f(a), \varepsilon)$.\\
        \textit{Autrement dit, pour tout $\varepsilon \exists \delta_\varepsilon$ tel que $d_X(x,a) < \delta_\varepsilon \Rightarrow d_Y(f(x), f(a)) < \varepsilon$.}
        \item $f$ est continue sur $X$ si $f$ est continue en tout point de $X$.
    \end{itemize}
}

\remark{Si $f \colon (\mathbb{R}, |\cdot|) \to (\mathbb{R}, |\cdot|)$, on retrouve la définition usuelle de la continuité des fonctions réelles.}

\definition{
    Soit $k \geq 0$ un réel.\\
    On dit que $f \colon (X,d_X) \to (Y,d_Y)$ est k-lipschitzienne si $\forall x,y \in X, d_Y(f(x), f(y)) \leq k d_X(x,y)$.
}

\definition{Si $f \colon (X, d_X) \to (Y, d_Y)$ est une application k-lipschitzienne, alors $f$ est continue sur $X$.}

\noindent{\textbf{Preuve :}\\
Montrons que $f$ est continue.\\
Soit $a \in X$ et soit $\varepsilon > 0$. Posons $\delta_\varepsilon = \frac{\varepsilon}{2k} > 0$.\\
$\forall x \in B(a, \delta_\varepsilon)$, on a $d_Y(f(a),f(x)) \leq kd_X(a,x) \leq k \delta_\varepsilon = \frac{\varepsilon}{2} < \varepsilon$.\\
Ainsi, $f$ est continue.\\
Remarquons de plus que $\delta_\varepsilon$ ne dépend pas de $a$, donc $f$ est uniformément continue sur $X$. (HP)
}

\example{
    Exemple d'une fonction 1-lipschitzienne : \function{f}{(\mathbb{R}^n, \|\cdot\|) \to (\mathbb{R}, |\cdot|)}{x \mapsto \|x\|}.\\
    En effet, $|f(x) - f(y)| = ||x| - |y|| \leq \|x - y\| = d(x,y)$ (inégalité triangulaire renversée).
}

\subsubsection{Caractérisation de la continuité}

\theorem{Proposition}{Caractérisation de la continuité par des suites}{false}{
    Soit $f \colon (X,d_X) \to (Y,d_Y)$ une application entre deux espaces métriques.\\
    Alors $f$ est continue en $a \in X$ si et seulement si pour toute suite $(x_n)$ de $X$ qui converge vers $a$, la suite $(f(x_n))$ converge vers $f(a)$.
}

\noindent{\textbf{Preuve :}\\
$(\Rightarrow)$ Soit $(x_n)$ une suite de $X$ qui converge vers $a$.\\
Soit $\varepsilon > 0$. Comme $f$ est continue en $a$, il existe $\delta_\varepsilon > 0$ tel que $d_X(x,a) < \delta_\varepsilon \Rightarrow d_Y(f(x), f(a)) < \varepsilon$.\\
Comme $x_n \to a$, il existe $N \in \mathbb{N}$ tel que $\forall n \geq N, d_X(x_n, a) < \delta_\varepsilon$.\\
Donc $\forall n \geq N, d_Y(f(x_n), f(a)) < \varepsilon$ pour tout $n \geq N$.\\
Ainsi $\forall \varepsilon > 0, \exists N \in \mathbb{N}, \forall n \geq N, d_Y(f(x_n), f(a)) < \varepsilon$. Donc $f(x_n) \to f(a)$.\\
$(\Leftarrow)$ Par l'absurde, supposons que $f$ n'est pas continue en $a$.\\
C'est-à-dire qu'il existe $\varepsilon_0 > 0, \forall \delta > 0, \exists x \in X$ tel que $d_X(x,a) < \delta$ mais $d_Y(f(x), f(a)) > \varepsilon_0$.\\
$\forall n \in \mathbb{N}$, posons $\delta = \frac{1}{n+1} > 0$.\\
On construit une suite $(x_n)$ de $X$ telle que $d_X(x_n, a) < \frac{1}{n+1}$, $d_Y(f(x_n), f(a)) > \varepsilon_0$.\\
On a $x_n \to a$ mais $f(x_n) \not\to f(a)$ (car $d_Y(f(x_n), f(a)) > \varepsilon_0$ pour tout $n$).\\
Absurde. Donc $f$ est continue en $a$.
}

\theorem{Proposition}{}{false}{
    $f \colon (X,d_X) \to (Y,d_Y)$ est continue sur $X$ si et seulement si pour tout ouvert $U$ de $(Y,d_Y)$, $f^{-1}(U)$ est un ouvert de $(X,d_X)$.
}
\reminder{
    $f^{-1}(U) = \{ x \in X \mid f(x) \in U \}$.
}

\noindent{\textbf{Preuve :}\\
$(\Rightarrow)$ Soit $U$ un ouvert de $(Y,d_Y)$.\\
Montrons que $f^{-1}(U)$ est un ouvert de $(X,d_X)$.\\
Soit $a \in f^{-1}(U)$. Alors $f(a) \in U$.\\
Comme $U$ est un ouvert de $(Y,d_Y)$, il existe $\varepsilon > 0$ tel que $B_Y(f(a), \varepsilon) \subset U$.\\
Or $\exists \delta_\varepsilon > 0$ tel que $x \in B_X(a, \delta_\varepsilon) \Rightarrow f(x) \in B_Y(f(a), \varepsilon)$.\\
Vérifions que $B_X(a, \delta_\varepsilon) \subset f^{-1}(U)$.\\
Soit $x \in B_X(a, \delta_\varepsilon)$.
 Alors $d_X(x,a) < \delta_\varepsilon \Rightarrow  d_Y(f(x),f(a)) < \varepsilon \Rightarrow f(x) \in B_Y(f(a), \varepsilon) \subset U$ $\Rightarrow x \in f^{-1}(U)$.\\
Donc $B_X(a, \delta_\varepsilon) \subset f^{-1}(U)$. Ainsi, $f^{-1}(U)$ est un ouvert de $(X,d_X)$.\\

\noindent$(\Leftarrow)$ Soit $U$ un ouvert de $(Y,d_Y)$.\\
Alors $f^{-1}(U)$ est un ouvert de $(X,d_X)$.\\
Soit $a \in f^{-1}(U)$. Alors $f(a) \in U$.\\
Soit $\varepsilon > 0$ tq $B(f(a), \varepsilon) \subset U$.\\
Montrons que $\exists \delta_\varepsilon > 0$ tel que $d_X(x,a) < \delta_\varepsilon$ et $d_Y(f(x), f(a)) < \varepsilon$.\\
Comme $f^{-1}(U)$ est un ouvert de $(X,d_X)$, il existe $\delta_\varepsilon > 0$ tel que $B_X(a, \delta_\varepsilon) \subset f^{-1}(U)$ qui est ouvert.\\
Alors si $x \in X$ tq $d_X(x,a) < \delta_\varepsilon$, donc $f(x) \in B(f(a), \varepsilon)$.\\
Donc on a $d_Y(f(x), f(a)) < \varepsilon$ car $f(x) \in U$.\\
Ainsi, $f$ est continue en $a$.}

\noindent\textbf{Interlude : Inégalité de Cauchy-Schwarz}

Considérons $<\cdot , \cdot > \colon \mathbb{R}^n \times \mathbb{R}^n \to \mathbb{R}$ définie par $<x,y> = \sum_{i=1}^{n} x_i y_i$.
Cette application est bilinéaire, symétrique, positive et définie ($<x,x> = 0 \Leftrightarrow x = 0$).\\
On observe que la norme euclidienne s'écrit $\|x\|_2 = \sqrt{<x,x>}$.\\\\

On dit que $<\cdot , \cdot >$ est un produit scalaire (forme bilinéaire symétrique définie positive), et ce produit scalaire est relié à la norme 2 (il s'agit du produit scalaire euclidien).

\theorem{Théorème}{Inégalité de Cauchy-Schwarz}{false}{
    $\forall X,Y \in \mathbb{R}^n, |<X,Y>| \leq \|X\|_2 \|Y\|_2$.
}
\noindent {\textbf{Démonstration :}
\begin{itemize}
    \item Fixons $X, Y \in \mathbb{R}^n$ ($\neq 0_{\mathbb{R}^n}$)
    \[
    P(t) = \|X + tY\|_2^2 = \langle X + tY, X + tY \rangle = \|X\|_2^2 + t^2 \|Y\|_2^2 + 2t \langle X, Y \rangle 
    \ge 0 
    = t^2 \|Y\|_2^2 + 2t \langle X, Y \rangle + \|X\|_2^2.
    \]
    $P$ est un polynôme de degré 2 en $t$. $P(t) \ge 0, \forall t \in \mathbb{R}$, son discriminant est $\le 0$.
    \[
    \Delta = 4 \langle X, Y \rangle^2 - 4 \|X\|_2^2 \|Y\|_2^2 = 4(\langle X, Y \rangle^2 - \|X\|_2^2 \|Y\|_2^2).
    \]
    Or, $\Delta \le 0 \iff \langle X, Y \rangle^2 - \|X\|_2^2 \|Y\|_2^2 \le 0 \iff \langle X, Y \rangle^2 \le \|X\|_2^2 \|Y\|_2^2
    \iff |\langle X, Y \rangle| \le \|X\|_2 \|Y\|_2$.
    
    \item De plus, si $\Delta = 0$, $\exists t_0 \in \mathbb{R}$ racine double, $P(t_0) = 0 = \|X + t_0 Y\|_2^2$
    \[
    \implies X + t_0 Y = 0 \implies X \text{ et } Y \text{ sont colinéaires}.
    \]
    Et si $X$ et $Y$ sont colinéaires, alors
    \[
    |\langle X, Y \rangle| = |\langle X, \lambda X \rangle| = |\lambda| \langle X, X \rangle 
    = |\lambda| \|X\|_2^2 = |\lambda| \|X\|_2 \|X\|_2 
    = \|X\|_2 \|\lambda X\|_2 = \|X\|_2 \|Y\|_2.
    \]
\end{itemize}}

\theorem{Corollaire}{Égalité de Cauchy-Schwarz}{false}{
    L'égalité $|<X,Y>| = \|X\|_2 \|Y\|_2$ est vérifiée si et seulement si $X$ et $Y$ sont colinéaires.
}

\ndlr{Démo à reprendre, cf screen Laurent 2}

\subsubsection{Continuité des applications linéaires dans les evn}

\textit{A priori}, les espaces vectoriels ne sont pas forcément de dimension finie.\\
Soit $f \colon (E, \|\cdot\|_E) \to (F, \|\cdot\|_F)$ une application linéaire entre deux $K$-espaces vectoriels normés (evn).\\

\theorem{Proposition}{Caractérisation de la continuité des applications linéaires}{false}{
    Les propriétés suivantes sont équivalentes :
    \begin{enumerate}
        \item $f$ est continue de $(E, \|\cdot\|_E)$ dans $(F, \|\cdot\|_F)$.
        \item $f$ est continue en 0.
        \item $\exists k > 0$ tel que $\forall x \in E, \|f(x)\|_F \leq k \|x\|_E$.
    \end{enumerate}
}

\noindent{\textbf{Démonstration :}

\begin{itemize}
    \item $i) \Rightarrow ii)$ : $f$ continue sur $E \Rightarrow f$ continue en $0$, \quad $f(0_E) = 0_F$
    
    \item $ii) \Rightarrow iii)$ : $f$ continue en $0$ : Soit $\varepsilon > 0$, $\exists \delta > 0$ tel que si $\|x\|_E < \delta \Rightarrow \|f(x)\|_F < \varepsilon$.
    
    Soit $x \in E$. Posons $y = \frac{\delta x}{2\|x\|_E}$. Alors
    \[
        \|f(y)\|_F < \varepsilon \quad \Rightarrow \quad f\left(\frac{\delta x}{2\|x\|_E}\right) < \varepsilon
    \]
    \[
        \Rightarrow \frac{\delta}{2\|x\|_E} \|f(x)\|_F < \varepsilon
    \]
    \[
        \Rightarrow \|f(x)\|_F < \frac{2\varepsilon}{\delta} \|x\|_E
    \]
    
    On trouve $K = \frac{2\varepsilon}{\delta}$, indépendant de $x \in E$, tel que $\forall x \in E, \, \|f(x)\|_F \le K \|x\|_E$.
    
    \item $iii) \Rightarrow i)$ : On part de $\exists K > 0$ tel que $\forall x \in E, \|f(x)\|_F \le K \|x\|_E$, on obtient $\forall x,y \in E, \|f(x-y)\|_F \le K \|x-y\|_E$
    \[
        \Rightarrow \exists K > 0, \, \forall x,y \in E, \, \|f(x) - f(y)\|_F \le K \|x-y\|_E
    \]
    \[
        \Rightarrow f \text{ est } K\text{-Lipschitz} \quad \Rightarrow f \text{ continue.}
    \]
\end{itemize}
}
\subsection{Equivalence de normes}

\textbf{Problème :} Soit $E$ un evn avec une normée notée $N_1 \colon E \in \mathbb{R}^+$.\\
On peut \textit{a priori} mettre d'autres normes sur $E$, disons $N_2 \colon E \in \mathbb{R}^+$.\\
Si $(x_n)$ une suite de $E$ converge pour la norme $N_1$, converge-t-elle aussi pour la norme $N_2$ ?\\

\definition{
    $N_1$ et $N_2$ sont \textbf{équivalentes} si $\exists C,c > 0$ tels que $\forall x \in E, c N_2(x) \leq N_1(x) \leq C N_2(x)$.\\
    On note $N_1 \sim N_2$.
}

\theorem{Proposition}{}{false}{
    \begin{enumerate}
        \item $N_1 \sim N_2 \Leftrightarrow N_2 \sim N_1$.
        \item Si $N_1 \sim N_2$ et $N_2 \sim N_3$, alors $N_1 \sim N_3$.
    \end{enumerate}
}

\example{Voir TD, normes $\|\cdot\|_1, \|\cdot\|_2, \|\cdot\|_{\infty}$ sur $\mathbb{R}^n$.}
\remark{Un des buts du cours est de montrer que sur $\mathbb{R}^n$ (+ généralement pour tout evn), toutes les normes sont équivalentes.}

\theorem{Proposition}{}{false}{
    Soit $(x_n)$ une suite de $E$, où $E$ est muni de $N_1$ et $N_2$ deux normes équivalentes.\\
    Alors $(x_n)$ converge pour la norme $N_1$ si et seulement si $(x_n)$ converge pour la norme $N_2$.
}
\noindent{\textbf{Démonstration :}

\begin{itemize}
    \item $\Rightarrow$ : On suppose $\exists a \in X$ tel que $x_n \xrightarrow{N_1} a \Leftrightarrow N_1(x_n - a) \underset{n \to +\infty}{\longrightarrow} 0.$\\
    Or, $\underline{c N_2(z) \leq N_1(z)} \leq C N_2(z)$\\
    Cette inégalité implique $N_2(x_n - a) \leq \frac{1}{c} \underline{N_1(x_n - a)} \underset{n \to +\infty}{\longrightarrow} 0$\\
    $\Rightarrow N_2(x_n - a) \underset{n \to +\infty}{\longrightarrow} 0$\\
    $\Rightarrow x_n \xrightarrow{N_2} a.$\\
    \item $\Leftarrow$  : Si $x_n \xrightarrow{N_2} a \Rightarrow N_2(x_n - a) \underset{n \to +\infty}{\longrightarrow} 0.$\\
    On utilise $\underline{N_1(\cdot) \leq C N_2(\cdot)}$ ($C > 0$)\\
    $\frac{N_1}{c}(x_n - a) \to 0 \Rightarrow N_1(x_n - a) \underset{n \to +\infty}{\longrightarrow} 0.$\\
\end{itemize}}

\subsection{Norme subordonnée}
Soient $(E, \|\dot\|_E)$ et $(F, \|\dot\|_F)$.\\
Soit $f \colon E \to F$ une application linéaire continue.
On a vu que la continuité d'une application linéaire entre evn se caractérisaient de la façon suivante :\\
$\exists K > 0 \colon \forall x \in E \|f(x)\|_F \leq K\|x\|_E$.\\
Si $x \neq 0$, on peut considérer $\frac{\|f(x)\|_F}{\|x\|_E} \in \mathbb{R}_{+}$\\
Par continuité de $f$, $\forall x \in E\setminus\{0\} \frac{\|f(x)\|_F}{\|x\|_E} \leq K$.

\definition{
    La norme subordonnée de $f$ par rapport à $\|\cdot\|_E$ et $\|\cdot\|_F$ est définie par $|||f||| := \sup_{x\in E\setminus\{0\}} \frac{\|f(x)\|_F}{\|x\|_E}$.
}
\remark{Cette triple barre est bien définie et correspond à la meilleure constante de continuité de $f$.}
\theorem{Proposition}{Espace des applications linéaires continues}{false}{
    Notons $\mathcal{L}_{c}(E,F) = \{ f \colon E \to F \mid f \text{ est continue} \}$.
    Alors $|||\cdot|||$ est une norme sur $\mathcal{L}_{c}(E,F)$ et $(\mathcal{L}_{c}(E,F), |||\cdot|||)$ est un evn.
}

\vocabulary{On dit que $|||\cdot|||$ est la "norme triple".}

\noindent{\textbf{Démonstration :}
\begin{itemize}
    \item Séparation : Si $|||f||| = 0 \Rightarrow \sup_{x \neq 0} \frac{\|f(x)\|_F}{\|x\|_E} = 0 \Rightarrow \forall x \neq 0 \frac{\|f(x)\|_F}{\|x\|_E} = 0 \Rightarrow \forall x \neq 0, \|f(x)\|_F = 0 \Rightarrow f(x) = 0_F$.
    \item Homogénéité : Soit $\lambda \in K$, $f \in \mathcal{L}_c(E,F)$.
    \[
        |||\lambda f||| = \sup_{x \neq 0} \frac{\|\lambda f(x)\|_F}{\|x\|_E} = \sup_{x \neq 0} \frac{|\lambda| \|f(x)\|_F}{\|x\|_E} = |\lambda| \sup_{x \neq 0} \frac{\|f(x)\|_F}{\|x\|_E} = |\lambda| |||f|||
    \]
    \item Inégalité triangulaire : Soient $f,g \in \mathcal{L}_c(E,F)$.
    \[
        |||f + g||| = \sup_{x \neq 0} \frac{\|(f+g)(x)\|_F}{\|x\|_E} \leq \sup_{x \neq 0} \frac{\|f(x)\|_F + \|g(x)\|_F}{\|x\|_E} \leq \sup_{x \neq 0} \frac{\|f(x)\|_F}{\|x\|_E} + \sup_{x \neq 0} \frac{\|g(x)\|_F}{\|x\|_E} = |||f||| + |||g|||
    \]
\end{itemize}
}

\theorem{Proposition}{}{true}{
    Soit $f \in \mathcal{L}_c(E,F)$.
    On a $|||f||| = \sup_{x \neq 0} \frac{\|f(x)\|_F}{\|x\|_E} = \sup_{\|x\|_E \leq 1} \|f(x)\|_F$ = $ \sup_{\|x\|_E = 1} \|f(x)\|_F$.
}

\remark{On montrera qu'en dimension finie, toute application linéaire est continue.}

\subsection{Introduction à la complétude dans les espaces métriques}

\subsubsection{Définitions et premières propriétés}

\definition{
    Soit $(X,d)$ un espace métrique.\\
    Une suite $(x_n)$ de $(X,d)$ est dite de Cauchy si \[\forall \varepsilon > 0, \exists N \in \mathbb{N}, \forall m,n \geq N, d(x_n,x_m) < \varepsilon\]
}

\remark{Si on travaille dans $(X,d)$ = $(E, \|\cdot\|)$ un evn, avec $d_E(x,y) = \|x-y\|$, et si on se donne une autre norme $\|\cdot\|'$ sur $E$ équivalente à $\|\cdot\|$, alors une suite est de Cauchy pour $\|\cdot\|$ si et seulement si elle est de Cauchy pour $\|\cdot\|'$.}
\ndlr{cf. Laurent pour la démonstration de la remarque}

\theorem{Proposition}{}{false}{
    \begin{enumerate}
        \item Soit $(x_n)$ une suite d'éléments de $(X,d)$ qui converge vers $x \in X$.\\
        Alors $(x_n)$ est une suite de Cauchy. 
        \item Une suite de Cauchy a au plus une valeur d'adhérence.
        \item Une suite de Cauchy qui a une valeur d'adhérence converge vers cette valeur d'adhérence.
    \end{enumerate}
}

\noindent{\textbf{Preuve :}
\begin{enumerate}
    \item Soit $(x_n)$ une suite qui converge vers $a \in X$.\\
    Donc $\forall \varepsilon > 0, \exists N_\varepsilon \in \mathbb{N}, \forall n \geq N_\varepsilon, d(x_n,a) < \varepsilon$.\\
    Montrons que $(x_n)$ est de Cauchy. Par convergence de $(x_n)$, on a $\exists N_\varepsilon \in \mathbb{N}, \forall n \geq N_\varepsilon, d(x_n,a) < \frac{\varepsilon}{2}$.\\
    Ainsi, $d(x_n,x_m) \leq d(x_n,a) + d(a,x_m) < \frac{\varepsilon}{2} + \frac{\varepsilon}{2} = \varepsilon$.\\
    Donc on a : $\forall \varepsilon > 0, \exists N_\varepsilon \in \mathbb{N}, \forall m,n \geq N_\varepsilon, d(x_n,x_m) < \varepsilon$.\\
    C'est-à-dire que $(x_n)$ est de Cauchy.
    \item Supposons que $(x_n)$ est une suite de Cauchy qui admet deux valeurs d'adhérence distinctes $a \neq b \in X$.\\
    On a $x_{\varphi(n)} \to a$ et $x_{\psi(n)} \to b$ deux sous-suites de $(x_n)$.\\
    Posons $\varepsilon = d(a,b) > 0$ car $a \neq b$.\\
    Comme $(x_n)$ est de Cauchy, $\exists N_0 \in \mathbb{N}, \forall m,n \geq N_0, d(x_n,x_m) < \frac{\varepsilon}{3}$.\\
    De plus, comme $x_{\varphi(n)} \to a$, $\exists N_1 \in \mathbb{N}, \forall n \geq N_1, d(x_{\varphi(n)}, a) < \frac{\varepsilon}{3}$.\\
    Et comme $x_{\psi(n)} \to b$, $\exists N_2 \in \mathbb{N}, \forall n \geq N_2, d(x_{\psi(n)}, b) < \frac{\varepsilon}{3}$.\\
    Comme $\varphi(n), \psi(n) \to +\infty$, on peut choisir $n$ tel que $\varphi(n), \psi(n) \geq N_0$ avec $\varphi(n) \geq N_1$ et $\psi(n) \geq N_2$.\\
    $\varepsilon = d(a,b) \leq d(a,x_{\varphi(n)}) + d(x_{\varphi(n)}, x_{\psi(n)}) + d(x_{\psi(n)}, b) < \frac{\varepsilon}{3} + \frac{\varepsilon}{3} + \frac{\varepsilon}{3} = \varepsilon$.\\
    Absurde. Donc $a = b$, donc la valeur d'adhérence est unique.
    \item Soit $(x_n)$ une suite de Cauchy et soit $(x_{\varphi(n)})$ une sous-suite de $(x_n)$ qui converge vers $a$.\\
    Soit $\varepsilon > 0$. Comme $(x_n)$ est de Cauchy, $\exists N_0 \in \mathbb{N}, \forall m,n \geq N_0, d(x_n,x_m) < \frac{\varepsilon}{2}$.\\
    De plus, comme $x_{\varphi(n)} \to a$, $\exists N_1 \in \mathbb{N}, \forall n \geq N_1, d(x_{\varphi(n)}, a) < \frac{\varepsilon}{2}$.\\
    On a alors :\\
    $d(x_n,a) \leq d(x_n, x_{\varphi(m)}) + d(x_{\varphi(m)}, a) < \frac{\varepsilon}{2} + \frac{\varepsilon}{2} = \varepsilon$ pour tout $\forall n,m \geq max(N_0, N_1)$ (possible car $\varphi(n) \to +\infty$).\\
    Donc $\forall \varepsilon > 0, \exists N \in \mathbb{N}, \forall n \geq N, d(x_n,a) < \varepsilon$.\\
    C'est-à-dire que $x_n \to a$. $\Box$
\end{enumerate}}

\definition{
    Une partie $A \in (X,d)$ est bornée si $\exists x \in X \exists r > 0$ tel que $A \subset B(x,r)$.
}

\illustration{
    Patatoïdes, une partie A dans un autre espace, avec un $x$ dans cet autre espace, mais pas dans A.
}

\theorem{Proposition}{}{false}{
    Si $(x_n)$ est une suite de Cauchy dans $(X,d)$, alors $(x_n)$ est bornée.
}
\noindent{\textbf{Preuve :}\\
Posons $A = \{ x_0, x_1, x_2, \ldots, x_n \ldots\} \subset X$.\\
On veut montrer que $\exists y \in X, r > 0$ tel que $d(x_n, y) < r, \forall n \in \mathbb{N}$.\\
Pour $\varepsilon = 1 \exists N \in \mathbb{N}, \forall m,n \geq N, d(x_n,x_m) < 1$.\\
On a $\forall n \geq N, d(x_n, y) \leq d(x_n, x_N) + d(x_N, y) < 1 + d(x_N, y) \leq 1 + \max_{0 \leq k \leq N} d(x_k, y) = r$.\\
Ainsi, $\forall n \in \mathbb{N}, d(x_n, y) < r$. Donc $(x_n)$ est bornée. $\Box$
}

\definition{
    Un espace métrique $(X,d)$ est complet si toute suite de Cauchy de $(X,d)$ converge dans $X$.
}

\example{$(\mathbb{R}, |\cdot|)$ est complet.}
\theorem{Théorème}{}{false}{
    $\mathbb{R}^d$ $(d \in \mathbb{N})$ est complet pour $\|\cdot\|_{\infty}$. \textit{(feuille de TD4, exercice 13)}
}

\noindent{\textbf{Preuve :}\\
Soit $(X_n)$ une suite de Cauchy dans $(\mathbb{R}^d, \|\cdot\|_{\infty})$.\\
$\forall \varepsilon > 0, \exists N \in \mathbb{N}, \forall m,n \geq N, \|X_m - X_n\|_{\infty} < \varepsilon$.\\
Écrivons $X_n = \begin{pmatrix} x_{1}^{(n)} \\ x_{2}^{(n)} \\ \vdots \\ x_{d}^{(n)} \end{pmatrix}$ où $(x_{i}^{(n)})$ est une suite de $\mathbb{R}$ pour tout $i \in \{1, \ldots, d\}$.\\
Or $\forall i \in \{1, \ldots, d\}, |x_{i}^{(m)} - x_{i}^{(n)}| \leq \|X_m - X_n\|_{\infty}$.\\
Donc $\forall i \in \{1, \ldots, d\}, (x_{i}^{(n)})$ est une suite de Cauchy dans $(\mathbb{R}, |\cdot|)$.\\
Or $(\mathbb{R}, |\cdot|)$ est complet, donc $\forall i \in \{1, \ldots, d\},  x_i^{(n)} \to l_i \in \mathbb{R}$.\\
Considérons le vecteur $X = \begin{pmatrix} l_1 \\ l_2 \\ \vdots \\ l_d \end{pmatrix} \in \mathbb{R}^d$.\\
Alors $X_n \to X$ dans $(\mathbb{R}^d, \|\cdot\|_{\infty})$.\\
Donc $(\mathbb{R}^d, \|\cdot\|_{\infty})$ est complet. $\Box$
}   

\subsubsection{Complétude et fermeture}

\theorem{Proposition}{}{false}{
    Soit $(X,d)$ un espace métrique.\\
    On considère $A \subset X$, et donc $A$ devient un espace métrique pour la distance $d$ restreinte à $A$.\\
    Si $(A,d)$ est complet, alors $A$ est fermé dans $(X,d)$.
}

\noindent{\textbf{Preuve :}\\
Soit $(x_n)$ une suite de $A$ qui converge vers $l \in X$.\\
Si $(x_n)$ est une suite convergente, alors $(x_n)$ est de Cauchy dans $(A,d)$.\\
Comme $A$ est complet, $x_n \to l \in A$. (unicité de la limite)\\
Donc $A$ est fermé dans $(X,d)$. $\Box$
}

\theorem{Proposition}{}{false}{
    Soient $(X,d)$ un espace métrique et $A \subset X$.\\
    Si $A$ est fermé dans $(X,d)$ alors $(A,d)$ est complet.
}

\noindent{\textbf{Preuve (le prof l'a effacée) :}\\
Soit $(x_n)$ une suite de Cauchy dans $(A,d)$.\\
Mais $(x_n)$ est aussi une suite de $A$ qui est fermé. Alors $(x_n)$ admet une valeur d'adhérence $l \in X$.\\
Or une suite de Cauchy qui admet une valeur d'adhérence converge vers cette valeur d'adhérence.\\
Donc $x_n \to l \in A$.\\
Donc toute suite de Cauchy de $(A,d)$ converge dans $A$.\\
Ainsi, $(A,d)$ est complet. $\Box$
}


\noindent{\textbf{Preuve plus adaptée:}\\
Soit $(x_n)$ une suite de Cauchy dans $(A,d)$.\\
Mais $(x_n)$ est aussi une suite de $X$ qui est complet.\\
Alors $(x_n)$ admet une limite $l \in X$ a priori.\\
Donc $x_n \to l \in A$ et $l\in A$ car $A$ est fermé dans $(X,d)$.\\
Donc toute suite de Cauchy de $(A,d)$ converge dans $A$.\\
Ainsi, $(A,d)$ est complet. $\Box$
}

\remark{Comme dans $\mathbb{R}$ on a la Caractérisation suivante : pour $(X,d)$ complet, alors $(A,d) \subset (X,d)$ est complet si et seulement si $A$ est fermé dans $(X,d)$.}

\subsection{Compacité}

\subsubsection{Définitions et premières propriétés}
\definition{On dit qu'un espace métrique $(X,d)$ est compact si toute suite de $(X,d)$ admet une sous-suite convergente dans $(X,d)$.}

\remark{On sait que si $(x_n)$ est une suite bornée (\textit{i.e.} $\exists a<b \in \mathbb{R}, \forall n \in \mathbb{N}, a \leq x_n \leq b$) dans $(\mathbb{R}, |\cdot|)$, alors par le théorème de Bolzano-Weierstrass, $\exists$ une sous-suite de $(x_n)$ qui converge.\\
Autrement dit, tout segment fermé $[a,b]$ est compact avec la définition. On peut dire qu'on a choisi une définition qui "généralise" Bolzano-Weierstrass.}

\theorem{Proposition}{}{false}{
    $(X,d)$ est compact $\Rightarrow (X,d)$ est complet.
}

\noindent{\textbf{Preuve :}\\
Soit $(x_n)$ une suite de Cauchy dans $(X,d)$.\\
Comme $(X,d)$ est compact, $(x_n)$ admet une sous-suite convergente.\\
Autrement dit, $(x_n)$ est de Cauchy et admet une valeur d'adhérence.\\
Donc $(x_n)$ converge vers cette valeur d'adhérence.\\ (suite de Cauchy qui admet une valeur d'adhérence converge vers cette valeur d'adhérence)\\
Ainsi, $(X,d)$ est complet. $\Box$
}

\theorem{Proposition}{}{false}{
    Soit $(X,d)$ compact. Soit $F \subset X$ (on pense $(F,d)$).\\
    Alors $F$ est fermé $\Leftrightarrow$ $(F,d)$ est compact.
}
\noindent{\textbf{Preuve :}\\
\begin{itemize}
    \item ($\Rightarrow$) On suppose $F$ fermé. Soit $(x_n)$ une suite de $F$.\\
    $(x_n)$ est aussi une suite de $X$ qui est compact.\\
    Il existe donc $(x_{\varphi(n)})$ une sous-suite de $(x_n)$ qui converge vers $l \in X$.\\
    Or $(x_{\varphi(n)})$ est une suite convergente de $F$ qui est fermé.\\
    Donc $l \in F$.\\
    Donc $(x_n)$ suite de $F$ admet une sous-suite convergente dans $F$.\\
    Ainsi, $(F,d)$ est compact. $\Box$
    \item ($\Leftarrow$) On suppose $(F,d)$ compact. Soit $(x_n)$ une suite de $F$ qui converge vers $l \in X$.\\
    Or comme $F$ est compact, $(x_n)$ admet une sous-suite convergente dans $F$.\\
    Comme $(x_n)$ converge vers $l \in X$, cette sous-suite converge aussi vers $l$. (unicité de la limite + une suite convergente a une seule valeur d'adhérence)\\
    Donc $l \in F$.\\
    Ainsi, $F$ est fermé. $\Box$
\end{itemize}
}

\remark{Dans le cas $(A,d) \subset (X,d)$ :
\begin{itemize}
    \item La définition de compacité est : toute suite $(x_n)$ de $A$ admet une sous-suite convergente dans $A$.
    \item La défintiion de la complétude est : toute suite de Cauchy $(x_n)$ de $A$ converge dans $A$.
\end{itemize}
}

\theorem{Proposition}{}{false}{
    Soit $(X,d)$ et $(Y,d')$ deux espaces compacts.\\
    Alors $X \times Y$ est compact.\\\\
    Autrement dit le produit d'espaces compacts est compact.
}

\remark{D'abord, mussions $X \times Y$ d'une distance.\\
Soit $\delta \colon ((x,y), (x',y')) \in (X \times Y)^2 \mapsto d(x,x') + d'(y,y')$.
}
\training{Montrer que $\delta$ est une distance sur $X \times Y$.}
\textit{On aurait pu considérer aussi $\delta_{\infty} \colon ((x,y), (x',y')) \mapsto \max(d(x,x'), d'(y,y'))$.}\\
\carreaux{10}

\noindent{\textbf{Preuve (de la proposition) :}\\
Montrons que $(X \times Y, \delta)$ est compact.\\
Soit $((x_n, y_n))$ une suite de $X \times Y$.\\
Or $(x_n)$ est une suite de $X$ qui est compact.\\
Et $(y_n)$ est une suite de $Y$ qui est compact.\\
Donc par compacité, $(x_n)$ admet une sous-suite convergente $(x_{\varphi(n)})$ vers $a$ dans $X$.\\
Et $(y_n)$ admet une sous-suite convergente $(y_{\psi(n)})$ vers $b$ dans $Y$.\\
Considérons alors la sous-suite $((x_{\varphi(\psi(n))}, y_{\psi(\varphi(n))}))$ de $((x_n, y_n))$.\\
Alors $(x_{\varphi(\psi(n))})$ converge vers $a$ dans $X$ et $(y_{\psi(\varphi(n))})$ converge vers $b$ dans $Y$.\\
Donc $((x_{\varphi(\psi(n))}, y_{\psi(\varphi(n))}))$ converge vers $(a,b)$.\\
Ainsi, $(X \times Y, \delta)$ est compact. $\Box$
}\\

\remark{Plus généralement, le produit fini d'espaces compacts est compact.}

\subsubsection{Fonctions continues sur un compact}

\theorem{Propriété fondamentale}{}{false}{
    Soit $f$ continue de $(X,d)$ (compact) dans $\mathbb{R}$.\\
    Alors $f$ est bornée et atteint ses bornes.\\\\

    Autrement dit, $m = inf_{x \in X} f(x) \leq f(x) \leq sup_{x \in X} f(x) = M$ pour tout $x \in X$, et $\exists a,b \in X$ tels que $f(a) = m$ et $f(b) = M$.
}

\noindent{\textbf{Preuve :}\\
(Pour le $sup$, le $inf$ est similaire)\\
Par définition de $sup$, $\forall \varepsilon > 0, \exists x_\varepsilon \in X$ tel que $sup_{x \in X} f(x) - \varepsilon \leq f(x_\varepsilon) \leq sup_{x \in X} f(x)$.\\
En discrétisant, $\varepsilon = \frac{1}{n+1}$, on obtient une suite $(x_n)$ de $X$ telle que $f(x_n) \to sup_{x \in X} f(x)$.\\
Or $(x_n)$ est une suite du compact $(X,d)$, donc $(x_n)$ admet une sous-suite convergente $(x_{\varphi(n)})$ qui converge vers $b \in X$.\\
Par continuité de $f$, $f(x_{\varphi(n)}) \to f(b)$.\\
Donc $f(b) = sup_{x \in X} f(x) < +\infty$ et $b$ réalise le $sup$. $\Box$
\textit{faire de même pour l'inf}
}

\theorem{Proposition}{}{false}{
    Soit $K \subset (X,d)$ et $K$ compact. Alors $K$ est borné.
}
\noindent{\textbf{Preuve :}\\
Soit $a \in X$. Considérons la fonction $f \colon X \in \mathbb{R}^+, x \mapsto d(x,a)$.\\
$|f(x) - f(y)| = |d(x,a) - d(y,a)| \leq d(x,y)$ par inégalité triangulaire.\\
Donc $f$ est 1-Lipschitz, donc continue.\\
Regardons $f$ restreinte à $K$, qui est compact.\\
Donc par la propriété fondamentale, $f$ est bornée sur $K$ et atteint ses bornes sur $K$.\\
Donc $\exists M > 0, \forall x \in K, d(x,a) \leq M$.\\
Ainsi, $K \subset B(a,M)$, donc $K$ est borné. $\Box$
}

\theorem{Proposition HP}{}{true}{
    Soit $(K,d)$ compact. Soit $f \colon (K,d) \to (Y,d')$ continue.\\
    Alors $f(K)$ est compact dans $(Y,d')$.
}

\noindent{\textbf{Preuve :} \textit{Laissée à l'appréciation du lecteur.}}

\subsubsection{Compacité dans un evn de dimension finie}

\theorem{Théorème}{}{false}{
    Les compacts de $\mathbb{R}^n$ sont les fermés et bornés.\\\\
    Autrement dit, $K \subset \mathbb{R}^n$ est compact $\Leftrightarrow K$ est fermé et borné.
}

\noindent{\textbf{Preuve :}
\begin{itemize}
    \item ($\Rightarrow$) Soit $K$ un compact de $\mathbb{R}^n$.\\
    Alors $K$ est borné (proposition vue plus haut dans le cadre d'espaces métriques).\\
    Or $\mathbb{R}^n$ est complet (avec par exemple la norme $\|\cdot\|_{\infty}$).\\
    Et comme $K$ est un compact de $\mathbb{R}^n$, $K$ est complet donc $K$ est fermé dans $\mathbb{R}^n$ (proposition vue plus haut).
    \item ($\Leftarrow$) Soit $K$ un fermé et borné de $\mathbb{R}^n$.\\
    Comme $K$ est borné, $\exists R > 0$ tel que $K \subset B(0,R)$ pour la norme $\|\cdot\|_{\infty}$.\\
    Ainsi $K \subset B(0,R) \subset [-R,R]^n$ qui est un compact de $\mathbb{R}^n$.\\
    Or $K$ est fermé dans $\mathbb{R}^n$, donc $K$ est fermé dans $[-R,R]^n$ (car $[-R,R]^n$ est fermé dans $\mathbb{R}^n$).\\
    Donc $K$ est un fermé d'un compact, donc $K$ est compact. $\Box$
    \end{itemize}
}
\end{document}