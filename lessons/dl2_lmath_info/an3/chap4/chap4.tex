\documentclass{article}

\usepackage[a4paper, left=1.5cm, right=1.5cm, top=2cm, bottom=2cm]{geometry}

\usepackage{amsmath,amssymb}

\usepackage{../../../../components/components} % <-- ton fichier .sty, avec toutes tes définitions

\usepackage{fancyhdr}


% Configuration des en-têtes et pieds de page
\pagestyle{fancy}
\fancyhf{} % reset tout

\fancyhead[L]{DL2 Math-Info}
\fancyhead[C]{Topologie}
\fancyhead[R]{2025-2026}

\fancyfoot[L]{Ewen Rodrigues de Oliveira}
\fancyfoot[R]{\thepage}

\begin{document}

\docTitle{Chapitre 4 : Topologie des espaces métriques et des espaces vectoriels normés}

\section{Distances et normes}

Soit $X$ un ensemble quelconque (dans la suite supposé non nul).

\definition{
    Une distance $d$ sur $X$ est une application $d\colon X\times X \to \mathbb{R}_+$ qui vérifie les axiomes suivants :
    \begin{enumerate}
        \item $\forall (x,y) \in X^2, d(x,y) = d(y,x)$ (symétrie)
        \item $\forall x,y,z \in X, d(x,z) \leq d(x,y) + d(y,z)$ (inégalité triangulaire)
        \item $\forall (x,y) \in X^2, d(x,y) = 0 \Leftrightarrow x = y$ (séparation)
    \end{enumerate}
}

\vocabulary{
    On appelle \textbf{espace métrique} un couple $(X,d)$ où $X$ est un ensemble et $d$ une distance sur $X$.
}

\example{
    \begin{itemize}
        \item $\mathbb{R}$ muni de la distance $d(x,y) = |x-y|$, où $|\cdot|$ est la valeur absolue.
        \item $\mathbb{C}$ muni de la distance $d(z_1, z_2) = |z_1 - z_2|$, où $|\cdot|$ est le module.
        \item Une autre façon de voir l'exemple 2, on considère l'ensemble $\mathbb{R}^2$ muni de la distance suivante :\\
        Si $A = (x_A, y_A)$ et $B = (x_B, y_B)$ sont deux points de $\mathbb{R}^2$, on définit la distance $d(A,B) = \sqrt{(x_A - x_B)^2 + (y_A - y_B)^2}$.\\
        On appelle cette distance la \textbf{distance euclidienne} sur $\mathbb{R}^2$.
        \item Prenons $X =$ cercle unité muni de la distance $d(A,B) = arccos(cos(\theta_2 - \theta_1))$ où $\theta_1$ et $\theta_2$ sont les arguments des points $A$ et $B$ respectivement.
        \textit{(voir schéma OneNote)}
    \end{itemize}
}

\remark{
    On peut voir l'exemple 1 comme un cas particulier de l'exemple 2, en identifiant $\mathbb{R}$ à l'axe des réels dans le plan complexe.
}

\remark{On peut généraliser l'exemple 3 à $\mathbb{R}^n$ muni de la distance euclidienne définie par :
\[d(A,B) = \sqrt{\sum_{i=1}^{n} (x_{A_i} - x_{B_i})^2}\]
où $A = (x_{A_1}, x_{A_2}, \ldots, x_{A_n})$ et $B = (x_{B_1}, x_{B_2}, \ldots, x_{B_n})$ sont deux points de $\mathbb{R}^n$.}

Il se trouve que les exemples 1, 2 et 3 proviennent d'espaces vectoriels normés, qu'on verra (très) rapidement.

\theorem{Proposition}{Inégalités triangulaires}{false}{
    Soit $(X,d)$ un espace métrique.\\
    On a :
    \begin{enumerate}
        \item $\forall (x,y,z) \in X^3, d(x,z) \leq d(x,y) + d(y,z)$ (inégalité triangulaire) 
        \item $\forall (x,y,z) \in X^3, |d(x,y) - d(x,z)| \leq d(y,z)$ (inégalité triangulaire généralisée)
    \end{enumerate}
}

\noindent{\textbf{Preuve :} 
    \begin{enumerate}
        \item C'est l'axiome 2 de la définition d'une distance.
        \item On a $d(x,y) \leq d(x,z) + d(z,y)$ d'après l'inégalité triangulaire.\\
        Donc $d(x,y) - d(x,z) \leq d(z,y)$ = $d(y,z)$.\\
        De plus, $d(x,z) \leq d(x,y) + d(y,z)$ d'après l'inégalité triangulaire.\\
        Donc $d(x,z) - d(x,y) \leq d(y,z)$.\\
        En combinant les deux, on obtient $|d(x,y) - d(x,z)| \leq d(y,z)$.
    \end{enumerate}
}

\theorem{Construction d'espaces métriques}{Restriction}{true}{
    On part de $(X,d)$ un espace métrique.\\
    Soit $Y \subset X$. Alors $(Y, d|_{Y\times Y})$ est un espace métrique.
    }
\vocabulary{On dit que $d|_{Y\times Y}$ est la \textbf{distance sur $Y$ induite} par la distance sur $X$.}
\remark{Ainsi, tout sous-ensemble d'un espace métrique est muni d'une "structure d'espace métrique" induite : c'est la distance induite.}

\example{$(X, d_X) = (\mathbb{C}, |\cdot|)$ un espace métrique. Alors pour $Y = \mathbb{R} \subset \mathbb{C}$, $d_Y$ n'est autre que la distance usuelle sur $\mathbb{R}$.\\
Idem pour $\mathbb{Q}$.}

\theorem{Construction d'espaces métriques}{Bijection}{true}{
    Soient $(X,d_X)$ et $Y$ un ensemble quelconque et \functionSets{f}{Y \rightarrow X} bijective.\\
    Alors $(Y, d_Y)$ est un espace métrique, où $d_Y(y_1,y_2) = d_X(f(y_1),f(y_2))$ pour tout $y_1,y_2 \in Y$.\\

    Autrement dit, \function{d_Y}{Y \times Y \rightarrow \mathbb{R}_+}{(y_1,y_2) \mapsto d_X(f(y_1),f(y_2))}.
}

\training{Démontrer le théorème ci-dessus. \textit{(vérifier que $d_Y$ satisfait bien les axiomes d'une distance.)}}
\noindent\carreaux{10}

\subsection{Espaces vectoriels normés}

On a plus un ensemble $X$ quelconque mais on considère $E$ un $\mathbb{K}$-espace vectoriel ($\mathbb{K} = \mathbb{R}$ ou $\mathbb{C}$). (a priori pas de dimension finie)

\subsubsection{Définitions}
\definition{
    Une norme $N$ sur $E$ est une application $N\colon E \to \mathbb{R}_+$ souvent notée $\| \cdot \|$ qui vérifie les axiomes suivants :
    \begin{enumerate}
        \item $\forall u \in E, \forall \lambda \in \mathbb{K}, \|\lambda u\| = |\lambda| \cdot \|u\|$ (homogénéité)
        \item $\forall (u,v) \in E^2, \|u+v\| \leq \|u\| + \|v\|$ (inégalité triangulaire)
        \item $\forall u \in E, \|u\| = 0 \Leftrightarrow u = 0_E$ (séparation)
    \end{enumerate}
}
\vocabulary{On dit que le couple $(E, N) = (E, \| \cdot \|)$ est un \textbf{espace vectoriel normé}. Il est commun d'écrire $e.v.n.$.}
\remark{$\forall x \in E, N(-x) = N(x)$ d'après l'axiome 1.}

\theorem{Théorème}{Association norme-distance}{false}{
    Soit $(E, \| \cdot \|)$ un espace vectoriel normé.\\
    Alors c'est en particulier un espace métrique pour la distance définie par $d_N(u,v) = \|v - u\|$.
}

\noindent{\textbf{Preuve :} Il suffit de vérifier les axiomes d'une distance.
    \begin{enumerate}
        \item Symétrie : $d_N(u,v) = \|v-u\| = \|-(u-v)\| = \|u-v\| = d_N(v,u)$.
        \item Inégalité triangulaire : $d_N(u,w) = \|w-u\| = \|(w-v) + (v-u)\| \leq \|w-v\| + \|v-u\| = d_N(u,v) + d_N(v,w)$.
        \item Séparation : $d_N(u,v) = 0 \Leftrightarrow \|v-u\| = 0 \Leftrightarrow v-u = 0_E \Leftrightarrow u=v$.
    \end{enumerate}
}

\remark{Hors programme : espaces euclidiens $\subset$ espaces vectoriels normés $\subset$ espaces métriques.}

\example{
    En fait, les exemples 1, 2 et 3 dans "Espaces métriques" sont des evn : $(\mathbb{R}, |\cdot|)$, $(\mathbb{C}, |\cdot|)$ et $(\mathbb{R}^n, \|\cdot\|_2)$ où $\| (x_1, \ldots, x_n) \|_2 = \sqrt{\sum_{i=1}^{n} x_i^2} = \| \cdot \|_\text{euclidienne}$.
}

\training{
    Définissons $X = (x_1, x_2, \ldots, x_n) \in \mathbb{R}^n \mapsto \|X\|_{\infty} = \max(|x_1|, |x_2|, \ldots, |x_n|)$.\\
    Montrer que $\|\cdot\|_{\infty}$ est une norme sur $\mathbb{R}^n$.\\
    \textbf{Observation :} $\|X\|_{\infty} = |x_{i_0}|$ où $i_0 \in \{1, 2, \ldots, n\}$ réalise le maximum.\\
    \textbf{Homogénéité :} $\forall i \in \{1, \ldots, n\}, |x_i| \leq |x_{i_0}|$\\
    $\forall \lambda \in \mathbb{R},$ $|\lambda| \cdot |x_i| \leq |\lambda| \cdot |x_{i_0}| \forall i \in \{1, \ldots, n\}$\\
    D'où $\forall \lambda \in \mathbb{R}, |\lambda x_i| \leq |\lambda| \cdot |x_{i_0}| \forall i \in \{1, \ldots, n\}$\\
    Or $\lambda X = (\lambda x_1, \ldots, \lambda x_n)$, $i_0$ réalise le maximum de $(|\lambda x_1|, \ldots, |\lambda x_n|)$\\
    Donc $\|\lambda X\|_{\infty} = |\lambda| \cdot |x_{i_0}| = |\lambda| \cdot \|X\|_{\infty}$.\\
    \textbf{Inégalité triangulaire :} Soient $X = (x_1, \ldots, x_n)$ et $Y = (y_1, \ldots, y_n)$ dans $\mathbb{R}^n$.\\
    $\|X+Y\|_{\infty} = \max(|x_1 + y_1|, \ldots, |x_n + y_n|)$.\\
    On a $\forall i \in \{1, \ldots, n\}, |x_i + y_i| \leq |x_i| + |y_i| \leq \|X\|_{\infty} + \|Y\|_{\infty}$.\\
    En prenant le $max$ sur $i$, on obtient $\|X+Y\|_{\infty} \leq \|X\|_{\infty} + \|Y\|_{\infty}$.\\
    \textbf{Séparation :} Soit $X = (x_1, \ldots, x_n) \in \mathbb{R}^n$.\\
    $\|X\|_{\infty} = 0 \Leftrightarrow \max(|x_1|, \ldots, |x_n|) = 0$.\\
    Or $\forall k \in \{1, \ldots, n\}, |x_k| \leq \max(|x_1|, \ldots, |x_n|) = 0 \Rightarrow x_k = 0$ $\Rightarrow X = 0_{\mathbb{R}^n}$.\\
}

\training{
\begin{enumerate}
    \item Montrer que \function{\|\cdot\|_1}{\mathbb{R}^n \rightarrow \mathbb{R}_+}{X \mapsto \sum_{i=1}^{n} |x_i|} est une norme sur $\mathbb{R}^n$.
    \item \textbf{Plus difficile.} Plus généralement, montrer que pour $p \in \mathbb{R}_{+}$, \function{\|\cdot\|_p}{\mathbb{R}^n \rightarrow \mathbb{R}_+}{X \mapsto \left( \sum_{i=1}^{n} |x_i|^p \right)^{1/p}} est une norme sur $\mathbb{R}^n$.
\end{enumerate}    
}

\example{Exemples de normes sur des espaces de dimension infinie :
    \begin{itemize}
        \item Considérons $E = \mathcal{C}([0,1], \mathbb{R})$ l'espace vectoriel des fonctions continues de $[0,1]$ dans $\mathbb{R}$. On définit la norme $\|f\|_{\infty} = \sup_{x \in [0,1]} |f(x)|$ pour $f \in E$. En effet,\\\\
        \textbf{Homogénéité :} Soit $f \in E$ et $\lambda \in \mathbb{R}$.\\
        On a $\|\lambda f\|_{\infty} = \sup_{x \in [0,1]} |\lambda f(x)| = |\lambda| \cdot \sup_{x \in [0,1]} |f(x)| = |\lambda| \cdot \|f\|_{\infty}$.\\
        \textbf{Inégalité triangulaire :} Soient $f,g \in E$.\\
        On a $\|f+g\|_{\infty} = \sup_{x \in [0,1]} |f(x) + g(x)|$.\\
        Or $\forall x \in [0,1], |f(x) + g(x)| \leq |f(x)| + |g(x)| \leq \|f\|_{\infty} + \|g\|_{\infty}$.\\
        En prenant le supremum sur $x$, on obtient $\|f+g\|_{\infty} \leq \|f\|_{\infty} + \|g\|_{\infty}$.\\
        \textbf{Séparation :} Soit $f \in E$.\\
        On a $\|f\|_{\infty} = 0 \Leftrightarrow \sup_{x \in [0,1]} |f(x)| = 0$.\\
        Or $\forall x \in [0,1], |f(x)| \leq \sup_{x \in [0,1]} |f(x)| = 0 \Rightarrow f(x) = 0$ pour tout $x$ dans $[0,1]$ $\Rightarrow f = 0_E$.\\
        \item Considérons $E = \{ f\colon I \to \mathbb{R}, f \in \mathcal{C}^0(I) \mid \int_I |f|(t)dt \text{ converge} \} = L_1$ où $I$ est un intervalle de $\mathbb{R}$. On définit la norme $\|f\|_1 = \int_I |f(t)| dt$ pour $f \in E$.
    \end{itemize}
}

\attention{$\|f\|_{\infty}$ existe car $\sup|f(t)| < +\infty$ pour $f$ continue sur le compact $[0,1]$.}

\cexample{$f \in \mathcal{C}^0(]0,1], \mathbb{R})$ définie par $f(x) = \frac{1}{x}$. Alors $\|f\|_{\infty}$ n'existe pas car $\sup_{x \in ]0,1]} |f(x)| = +\infty$.}

\subsubsection{Propriétés}

\theorem{Propriété}{Vecteurs unitaires}{true}{
    Soit $E$ un $\mathbb{K}$-espace vectoriel non nul, soit $x \in E\setminus \{0_E\}$ et soit $N$ une norme sur $E$.\\
    Alors $\frac{x}{N(x)}$ est un vecteur unitaire (ou vecteur normé), c'est-à-dire $N\left(\frac{x}{N(x)}\right) = 1$. (existe car $N(x) \neq 0$ par séparation)
}

\theorem{Propriété}{}{true} {
    Soit $(E, N)$ un espace vectoriel normé.\\
    Alors $\forall (x_1, x_2, \ldots, x_n) \in E^n$, $\forall (\lambda_1, \lambda_2, \ldots, \lambda_n) \in \mathbb{K}^n$, on a :
    \[N(\sum_{i=1}^{n} \lambda_i x_i) \leq \sum_{i=1}^{n} |\lambda_i| \cdot N(x_i)\]
}

\theorem{Propriété}{Inégalité triangulaire renversée}{false} {
    Soit $(E, N)$ un espace vectoriel normé.\\
    \begin{enumerate}
        \item $\forall (x,y) \in E^2, N(x + y) \leq N(x) + N(y)$
        \item $\forall (x,y) \in E^2, |N(x) - N(y)| \leq N(x - y)$
    \end{enumerate}
}

\noindent{\textbf{Preuve :} 
    \begin{enumerate}
        \item C'est l'axiome 2 de la définition d'une norme.
        \item On a $N(x) = N((x - y) + y) \leq N(x - y) + N(y)$ d'après l'inégalité triangulaire.\\
        Donc $N(x) - N(y) \leq N(x - y)$.\\
        De plus, $N(y) = N((y - x) + x) \leq N(y - x) + N(x)$ d'après l'inégalité triangulaire.\\
        Donc $N(y) - N(x) \leq N(y - x) = N(x - y)$.\\
        En combinant les deux, on obtient $|N(x) - N(y)| \leq N(x - y)$.
    \end{enumerate}
}

\section{Topologie sur les espaces métriques}
\subsection{Ouverts et fermés}
\remark{Un evn étant un espace métrique avec la distance $d(x,y) = N(y - x)$, toutes les notions de topologie vues ici s'appliquent aux evn.}

\definition{Soit $(X,d)$ un espace métrique.\\
    Une boule ouverte de centre $a\in X$ de rayon $r>0$ est $B(a,r) = \{ x \in X \mid d(x,a) = d(a,x) < r \}$.
}

\definition{Soit $(X,d)$ un espace métrique.\\
    Une boule fermée de centre $a\in X$ de rayon $r>0$ est $B_F(a,r) = \{ x \in X \mid d(a,x) \leq r \}$.
}

\definition{Soit $(X,d)$ un espace métrique.\\
    La sphère de centre $a\in X$ de rayon $r>0$ est $S(a,r) = \{ x \in X \mid d(a,x) = r \}$.
}

\example{
    Dans $(\mathbb{R}, |\cdot|)$, $B(1,1) = ]0,2[$
}

\example{
    Dans $\mathbb{R}^2$ avec trois métriques :\\
    $d_1 = \|y-x\|_1$, $d_2 = \|y-x\|_2$ et $d_{\infty} = \|y-x\|_{\infty}$.\\
    Traçons les boules de centre 0 de rayon 1 associées à chaque distance.\\
    $B(0,1) = \{ x\in \mathbb{R}^2 \mid d(x,0) < 1\}$.\\
    On a : $\|x\|_{1,2,\infty} < 1$.\\
    \begin{itemize}
        \item Pour $d_1$ : $\|x\|_1 = |x_1| + |x_2| < 1$. C'est un losange.
        \item Pour $d_2$ : $\|x\|_2 = \sqrt{x_1^2 + x_2^2} < 1$. C'est un disque.
        \item Pour $d_{\infty}$ : $\|x\|_{\infty} = \max(|x_1|, |x_2|) < 1$. C'est un carré.
    \end{itemize}
}

\remark{Les bordures correspondent aux sphères $S(0,1)$ associées à chaque distance. Les boules fermées $B_F(0,1)$ correspondent aux mêmes figures mais en incluant les bordures.}

\definition{
    Une partie $U$ de $X$ est un ouvert de $(X,d)$ si $\forall x \in U, \exists r > 0$ tel que $B(x,r) \subset U$.
}

\example{
    $]0,1[ \subset \mathcal{C}(]0,1[,\mathbb{R})$ est un ouvert : $\forall x \in ]0,1[, B(x, \min(x, 1-x)) \subset ]0,1[$.
}

\definition{
    Une topologie sur $(X,d)$ est l'ensemble des ouverts de $(X,d)$. Autrement dit, $\tau = \{ U \subset X \mid U \text{ est un ouvert de } (X,d) \}$.
}

\theorem{Proposition}{}{false}{
    \begin{enumerate}
        \item Toute boule ouverte de $(X,d)$ est un ouvert de $(X,d)$.
        \item Si $(U_i)_{i \in I}$ est une famille d'ouverts de $(X,d)$, alors $\bigcup_{i \in I} U_i$ est un ouvert de $(X,d)$. (I un ensemble quelconque)
        \item Si $(U_1, U_2, \ldots, U_n)$ est une famille finie d'ouverts de $(X,d)$, alors $\bigcap_{i=1}^{n} U_i$ est un ouvert de $(X,d)$.
    \end{enumerate}
}

\noindent{\textbf{Preuve :} 
    \begin{enumerate}
        \item Soit $B(a,r)$ une boule ouverte de $(X,d)$. Soit $x \in B(a,r)$. On a $d(a,x) < r$. Posons $s = r - d(a,x) > 0$. Montrons que $B(x,s) \subset B(a,r)$.\\
        Soit $y \in B(x,s)$. On a $d(x,y) < s$. D'après l'inégalité triangulaire, on a $d(a,y) \leq d(a,x) + d(x,y) < d(a,x) + s = d(a,x) + (r - d(a,x)) = r$. Donc $y \in B(a,r)$. Ainsi, $B(x,s) \subset B(a,r)$ et donc $B(a,r)$ est un ouvert de $(X,d)$.
        \item Soit $x \in (U_i)_{i \in I}$. Alors $\exists i_0 \in I$ tel que $x \in U_{i_0}$. Comme $U_{i_0}$ est un ouvert de $(X,d)$, il existe $r > 0$ tel que $B(x,r) \subset U_{i_0} \subset \bigcup_{i \in I} U_i$. Donc $\bigcup_{i \in I} U_i$ est un ouvert de $(X,d)$.
        \item Soit $x \in \bigcap_{i=1}^{n} U_i$, où on a écrit $I = \{1, 2, \ldots, n\}, n \in \mathbb{N}^*$.\\
        Alors $\forall i \in \{1, 2, \ldots, n\}, \exists r_i > 0$ tel que $B(x,r_i) \subset U_i$. Posons $r = \min(r_1, r_2, \ldots, r_n) > 0$. Alors $B(x,r) \subset \bigcap U_i$ pour tout $i \in \{1, 2, \ldots, n\}$.\\
        En effet, soit $z \in B(x,r), d(z,x) < r \leq r_i$ donc $z \in B(x,r_i) \subset U_i$ pour tout $i \in \{1, 2, \ldots, n\}$. Donc $z \in \bigcap_{i=1}^{n} U_i$. Ainsi, $\bigcap_{i=1}^{n} U_i$ est un ouvert de $(X,d)$.
    \end{enumerate}
}

\cexample{Si $I$ est infini, la propriété 3 n'est pas vraie en général. Par exemple, dans $(\mathbb{R}, |\cdot|)$, considérons la famille d'ouverts $U_n = ]-1/n, 1/n[$ pour $n \in \mathbb{N}^*$. Alors $\bigcap_{n=1}^{\infty} U_n = \{0\}$ qui n'est pas un ouvert de $(\mathbb{R}, |\cdot|)$.}

\definition{
    Soit $(X,d)$ un espace métrique.\\
    Une partie $F$ de $X$ est un fermé de $(X,d)$ si son complémentaire $X \setminus F$ est un ouvert de $(X,d)$.
}

\theorem{Proposition}{}{false}{
    \begin{enumerate}
        \item Toute boule fermée de $(X,d)$ est un fermé de $(X,d)$.
        \item Si $(F_i)_{i \in I}$ est une famille de fermés de $(X,d)$, alors $\bigcap_{i \in I} F_i$ est un fermé de $(X,d)$. (I un ensemble quelconque)
        \item Si $(F_1, F_2, \ldots, F_n)$ est une famille finie de fermés de $(X,d)$, alors $\bigcup_{i=1}^{n} F_i$ est un fermé de $(X,d)$.
    \end{enumerate}
}

\noindent{\textbf{Preuve :} 
    \begin{enumerate}
        \item Soit $B_F(x,r)$ une boule fermée de $(X,d)$. C'est un fermé $\Leftrightarrow$ $X \setminus B_F(x,r)$ est un ouvert de $(X,d)$.\\
        Soit $y \in X \setminus B_F(x,r)$. On a $d(x,y) > r$. Posons $\varrho = d(y,x) - r > 0$. Montrons que $B(y,\varrho) \subset X \setminus B_F(x,r)$.\\
        Soit $z \in B(y,\varrho)$. On a $d(y,z) < \varrho$.\\
        $d(z,y) < d(y,x) - r \Rightarrow r < d(y,x) - d(z,y) \leq d(z,x)$ (inégalité triangulaire) $\Rightarrow z \in X \setminus B_F(x,r)$.\\
        Ainsi, $B(y,\varrho) \subset X \setminus B_F(x,r)$ et donc $X \setminus B_F(x,r)$ est un ouvert de $(X,d)$.\\
        Donc $B_F(x,r)$ est un fermé de $(X,d)$.
        \item Laissé en exercice au lecteur.
        \item Laissé en exercice au lecteur.
    \end{enumerate}
}

\remark{
    Dans $(\mathbb{R}, |\cdot|)$, tout intervalle fermé est un fermé, et tout intervalle ouvert est un ouvert. ($\mathbb{R}$ est ouvert).\\
    On a : $]a, +\infty[ = \bigcup_{n=1}^{\infty} ]a, a+n[$ est un ouvert.
}

\remark{$X$ et $\emptyset$ sont des ouverts et des fermés de $(X,d)$ (prendre $r$ arbitrairement grand pour $X$ et $r$ quelconque pour $\emptyset$).}

\definition{Soit $\mathcal{V} \subset X$ une partie de l'espace métrique $(X,d)$.\\
    On dit que $\mathcal{V}$ est un voisinage de $a \in X$ si $\exists r > 0$ tel que $B(a,r) \subset \mathcal{V}$.
}

\theorem{Proposition}{}{false}{
    On dit aussi que $U$ est un ouvert de $(X,d)$ si et seulement si $U$ est un voisinage de chacun de ses points.
}

\subsection{Intérieur et adhérence}
Soit $(X,d)$ un espace métrique.
\definition{Soit $A \subset X$ une partie de l'espace métrique $(X,d)$.\\
    On appelle intérieur de $A$ l'ensemble des points $a \in A$ tels que $A$ est un voisinage de $a$. On le note : $\overset{\circ}{A}$.\\
    On a : \[\overset{\circ}{A} = \bigcup \{ U \subset A \mid U \text{ est un ouvert de } (X,d) \}\]
}

\definition{Soit $A \subset X$ une partie de l'espace métrique $(X,d)$.\\
    On appelle adhérence de $A$ l'ensemble des points $x \in X$ tels que $\forall r > 0, B(x,r) \cap A \neq \emptyset$. On le note : $\overline{A}$.\\
    On a : \[\overline{A} = \bigcap _{F \supset A, F \text{ fermé de } (X,d)} F\]
}

\theorem{Proposition}{Lien avec les ouverts}{false}{
    Soit $A \subset X$ une partie de l'espace métrique $(X,d)$.
    \begin{enumerate}
        \item $\overset{\circ}{A}$ est un ouvert contenu dans $A$.
        \item Si $U \subset A$ est un ouvert de $(X,d)$, alors $U \subset \overset{\circ}{A}$.\\
        Autrement dit, $\overset{\circ}{A}$ est le plus grand ouvert contenu dans $A$.
    \end{enumerate}
}

\noindent{\textbf{Preuve :} 
    \begin{enumerate}
        \item C'est une union quelconque d'ouverts indus dans $A$, donc $\overset{\circ}{A}$ est un ouvert. De plus, par définition, $\overset{\circ}{A} \subset A$.
        \item Par définition $\overset{\circ}{A} = \bigcup \{ U \subset A \mid U \text{ est un ouvert de } (X,d) \}$. Donc si $U \subset A$ est un ouvert de $(X,d)$, alors $U$ est dans la famille indexée par l'union, donc $U \subset \overset{\circ}{A}$.
    \end{enumerate}
}

\theorem{Proposition}{Lien avec les fermés}{false}{
    Soit $A \subset X$ une partie de l'espace métrique $(X,d)$.
    \begin{enumerate}
        \item $\overline{A}$ est un fermé contenant $A$.
        \item Si $F \supset A$ est un fermé de $(X,d)$, alors $\overline{A} \subset F$.\\
        Autrement dit, $\overline{A}$ est le plus petit fermé contenant $A$.
    \end{enumerate}
}

\noindent{\textbf{Preuve :} 
    \begin{enumerate}
        \item C'est une intersection quelconque de fermés contenant $A$, donc $\overline{A}$ est un fermé. De plus, par définition, $A \subset \overline{A}$.
        \item Par définition $\overline{A} = \bigcap _{F \supset A, F \text{ fermé de } (X,d)} F$. Donc si $F \supset A$ est un fermé de $(X,d)$, alors $F$ est dans la famille indexée par l'intersection, donc $\overline{A} \subset F$.
    \end{enumerate}
}

\theorem{Proposition}{}{false}{
    \begin{enumerate}
        \item $\overset{\circ}{A} = X \setminus \overline{(X \setminus A)}$
        \item $\overline{A} = X \setminus \overset{\circ}{(X \setminus A)}$
        \item $x \in \overline{A} \Leftrightarrow \forall r > 0, B(x,r) \cap A \neq \emptyset$
    \end{enumerate}
}

\noindent{\textbf{Preuve :} 
    \begin{enumerate}
        \item ($\Rightarrow$) $x \in \overset{\circ}{A}$. Par définition d'un ouvert, $\exists r > 0$ tel que $B(x,r) \subset \overset{\circ}{A} \subset A$.\\
        ($\Leftarrow$) On a $B(x,r) \subset A$ qui est un ouvert.\\
        Donc $B(x,r) \subset \overset{\circ}{A}$ car $\overset{\circ}{A}$ est le plus grand ouvert contenu dans $A$. Donc $x \in \overset{\circ}{A}$, donc $x \in \overset{\circ}{A}$.\\
        \item $\overset{\circ}{A} \Leftrightarrow X\setminus\overset{\circ}{A} = \overline{(X \setminus A)}$.\\
        Or $\overset{\circ}{A} = \bigcup \{ U \subset A \mid U \text{ est un ouvert de } (X,d) \}$.\\
        Donc $X \setminus \overset{\circ}{A} = X \setminus \bigcup \{ U \subset A \mid U \text{ est un ouvert de } (X,d) \} = \bigcap_{F \supset X \setminus A, F \text{ fermé de } (X,d)} F = \overline{(X \setminus A)}$.\\
        Faire de même avec $\overline{A} = X \setminus \overset{\circ}{(X \setminus A)}$.
        \item $x \in \overline{A} \Leftrightarrow x \in X \setminus \overset{\circ}{(X \setminus A)}$ (d'après 2) $\Leftrightarrow x \notin \overset{\circ}{(X \setminus A)}$ $\Leftrightarrow$ pour tout $r > 0$, $B(x,r) \not\subset X \setminus A$ $\Leftrightarrow$ pour tout $r > 0$.
        $\Leftarrow B(x,r) \cap A \neq \emptyset$.
    \end{enumerate}
}

\theorem{Proposition}{}{false}{
    \begin{itemize}
    \item $U$ est ouvert $\Leftrightarrow \overset{\circ}{U} = U$.
    \item $F$ est fermé $\Leftrightarrow \overline{F} = F$.
    \end{itemize}
}

\subsection{Suites dans un espace métrique}

\subsubsection{Définitions}

\definition{
    On dit qu'une suite $(x_n)$, $x_n \in X$ pour tout $n \in \mathbb{N}$, converge si $\exists x \in X$ tel que $d(x_n,x) \to 0$ quand $n \to +\infty$.\\
    Autrement dit, \[\forall \varepsilon > 0, \exists N \in \mathbb{N}, \forall n \geq N, d(x_n,x) < \varepsilon\]
}

\remark{Si $(X,d) = (\mathbb{R}, |\cdot|)$, on retrouve la définition usuelle de la convergence des suites réelles.}

\definition{
    On dit que $x \in X$ est une valeur d'adhérence d'une suite $(x_n)$ si il existe une sous suite qui converge vers $x$.
    i.e. $\exists (n_k)_{k \in \mathbb{N}}$ strictement croissante telle que $d(x_{n_k}, x) \to 0$ quand $k \to +\infty$.
}

\theorem{Proposition}{}{false}{
    Soit $(x_n)$ une suite de $(X,d)$ qui converge vers $x$.\\
    Alors $x$ est la seule valeur d'adhérence de $(x_n)$. En particulier, la limite de $(x_n)$ est unique.
}


\noindent{\textbf{Preuve :}\\
Soit $x$ la limite de $(x_n)$. Suppons qu'il existe une sous-suite $(x_{n_k})$ qui converge vers une autre valeur $y \neq x$.\\
On a $d(x,y) > 0$. Posons $\varepsilon = \frac{d(x,y)}{2} > 0$.\\
$\exists k_1, (\varepsilon, y) \in \mathbb{N}, \forall k \geq k_1, d(x_{n_k}, y) < \varepsilon$. (convergence de la sous-suite)\\
Et comme $x_n \to x$, alors $x_{n_k} \to x$ aussi.\\
Donc $\exists k_2, (\varepsilon, x) \in \mathbb{N}, \forall k \geq k_2, d(x_{n_k}, x) < \varepsilon$.\\
Alors on a pour $k \geq \max(k_1, k_2)$ :\\
$d(x,y) \leq d(x, x_{n_k}) + d(x_{n_k}, y) < \varepsilon + \varepsilon = 2\varepsilon = d(x,y)$, ce qui est absurde. Donc $x$ est la seule valeur d'adhérence de $(x_n)$.
}\\

\noindent{\textbf{Fermés}\\Soit $A$ une partie quelconque de l'espace métrique $(X,d)$.
}

\theorem{Proposition}{Caractérisation de l'adhérence par des suites}{false}{
    $\overline{A} = \{ x \in X \mid \exists (x_n) \text{ suite de } A, x_n \to x \}$.
}
\noindent{\textbf{Preuve :}\\
$(\subset)$ Soit $x \in X$ tq $\exists (x_n) \in A$ avec $d(x_n, x) \to 0$.\\
On veut montrer que $x \in \overline{A} \Rightarrow$ pour tout $r > 0, B(x,r) \cap A \neq \emptyset$.\\
Soit $r > 0$. Comme $d(x_n, x) \to 0$, il existe $N \in \mathbb{N}$ tel que $\forall n \geq N, d(x_n, x) < r$.\\
On a $x_n \in A$ donc $x_n \in B(x,r) \cap A \neq \emptyset \forall n \geq N$. Donc $x \in \overline{A}$. $\Box$\\
$(\supset)$ Soit $x \in \overline{A} \Rightarrow$ pour tout $r > 0, B(x,r) \cap A \neq \emptyset$.\\*
Prenons, $n \in \mathbb{N}$, $r = \frac{1}{n+1} > 0$.\\
Alors $B(x, \frac{1}{n+1}) \cap A \neq \emptyset$.\\
Pour chaque $n \in \mathbb{N}$, choisissons $x_n \in B(x, \frac{1}{n+1}) \cap A$.\\
On a alors construit une suite $(x_n)$ qui vérifie $d(x_n, x) < \frac{1}{n+1}$.\\
Ainsi, $x$ est bien la limite d'une suite d'éléments de $A$. $\Box$
}

\theorem{Proposition}{Caractérisation des fermés par des suites}{false}{
    Une partie $F$ de $(X,d)$ est fermée si et seulement si pour toute suite $(x_n)$ d'éléments de $F$ qui converge vers $x \in X$, on a $x \in F$.
}

\noindent{\textbf{Preuve :}\\
On a $\overline{F} = \{ x \in X \mid \exists (x_n) \text{ suite de } F, x_n \to x \}$.\\
Or $\overline{F} = F$. On a donc le résultat.
}

\subsection{Continuité}

\subsubsection{Définitions}

\definition{
    Soit $f \colon (X,d_X) \to (Y,d_Y)$ une application entre deux espaces métriques.\\
    \begin{itemize}
        \item $f$ est continue en $a \in X$ si $\forall \varepsilon > 0, \exists \delta_\varepsilon > 0 \colon x \in B(a, \delta_\varepsilon)$, alors $f(x) \in B(f(a), \varepsilon)$.\\
        \textit{Autrement dit, pour tout $\varepsilon \exists \delta_\varepsilon$ tel que $d_X(x,a) < \delta_\varepsilon \Rightarrow d_Y(f(x), f(a)) < \varepsilon$.}
        \item $f$ est continue sur $X$ si $f$ est continue en tout point de $X$.
    \end{itemize}
}

\remark{Si $f \colon (\mathbb{R}, |\cdot|) \to (\mathbb{R}, |\cdot|)$, on retrouve la définition usuelle de la continuité des fonctions réelles.}

\definition{
    Soit $k \geq 0$ un réel.\\
    On dit que $f \colon (X,d_X) \to (Y,d_Y)$ est k-lipschitzienne si $\forall x,y \in X, d_Y(f(x), f(y)) \leq k d_X(x,y)$.
}

\definition{Si $f \colon (X, d_X) \to (Y, d_Y)$ est une application k-lipschitzienne, alors $f$ est continue sur $X$.}

\noindent{\textbf{Preuve :}\\
Montrons que $f$ est continue.\\
Soit $a \in X$ et soit $\varepsilon > 0$. Posons $\delta_\varepsilon = \frac{\varepsilon}{2k} > 0$.\\
$\forall x \in B(a, \delta_\varepsilon)$, on a $d_Y(f(a),f(x)) \leq kd_X(a,x) \leq k \delta_\varepsilon = \frac{\varepsilon}{2} < \varepsilon$.\\
Ainsi, $f$ est continue.\\
Remarquons de plus que $\delta_\varepsilon$ ne dépend pas de $a$, donc $f$ est uniformément continue sur $X$. (HP)
}

\example{
    Exemple d'une fonction 1-lipschitzienne : \function{f}{(\mathbb{R}^n, \|\cdot\|) \to (\mathbb{R}, |\cdot|)}{x \mapsto \|x\|}.\\
    En effet, $|f(x) - f(y)| = ||x| - |y|| \leq \|x - y\| = d(x,y)$ (inégalité triangulaire renversée).
}

\subsubsection{Caractérisation de la continuité}

\theorem{Proposition}{Caractérisation de la continuité par des suites}{false}{
    Soit $f \colon (X,d_X) \to (Y,d_Y)$ une application entre deux espaces métriques.\\
    Alors $f$ est continue en $a \in X$ si et seulement si pour toute suite $(x_n)$ de $X$ qui converge vers $a$, la suite $(f(x_n))$ converge vers $f(a)$.
}

\noindent{\textbf{Preuve :}\\
$(\Rightarrow)$ Soit $(x_n)$ une suite de $X$ qui converge vers $a$.\\
Soit $\varepsilon > 0$. Comme $f$ est continue en $a$, il existe $\delta_\varepsilon > 0$ tel que $d_X(x,a) < \delta_\varepsilon \Rightarrow d_Y(f(x), f(a)) < \varepsilon$.\\
Comme $x_n \to a$, il existe $N \in \mathbb{N}$ tel que $\forall n \geq N, d_X(x_n, a) < \delta_\varepsilon$.\\
Donc $\forall n \geq N, d_Y(f(x_n), f(a)) < \varepsilon$ pour tout $n \geq N$.\\
Ainsi $\forall \varepsilon > 0, \exists N \in \mathbb{N}, \forall n \geq N, d_Y(f(x_n), f(a)) < \varepsilon$. Donc $f(x_n) \to f(a)$.\\
$(\Leftarrow)$ Par l'absurde, supposons que $f$ n'est pas continue en $a$.\\
C'est-à-dire qu'il existe $\varepsilon_0 > 0, \forall \delta > 0, \exists x \in X$ tel que $d_X(x,a) < \delta$ mais $d_Y(f(x), f(a)) > \varepsilon_0$.\\
$\forall n \in \mathbb{N}$, posons $\delta = \frac{1}{n+1} > 0$.\\
On construit une suite $(x_n)$ de $X$ telle que $d_X(x_n, a) < \frac{1}{n+1}$, $d_Y(f(x_n), f(a)) > \varepsilon_0$.\\
On a $x_n \to a$ mais $f(x_n) \not\to f(a)$ (car $d_Y(f(x_n), f(a)) > \varepsilon_0$ pour tout $n$).\\
Absurde. Donc $f$ est continue en $a$.
}

\theorem{Proposition}{}{false}{
    $f \colon (X,d_X) \to (Y,d_Y)$ est continue sur $X$ si et seulement si pour tout ouvert $U$ de $(Y,d_Y)$, $f^{-1}(U)$ est un ouvert de $(X,d_X)$.
}
\reminder{
    $f^{-1}(U) = \{ x \in X \mid f(x) \in U \}$.
}

\noindent{\textbf{Preuve :}\\
$(\Rightarrow)$ Soit $U$ un ouvert de $(Y,d_Y)$.\\
Montrons que $f^{-1}(U)$ est un ouvert de $(X,d_X)$.\\
Soit $a \in f^{-1}(U)$. Alors $f(a) \in U$.\\
Comme $U$ est un ouvert de $(Y,d_Y)$, il existe $\varepsilon > 0$ tel que $B_Y(f(a), \varepsilon) \subset U$.\\
Or $\exists \delta_\varepsilon > 0$ tel que $x \in B_X(a, \delta_\varepsilon) \Rightarrow f(x) \in B_Y(f(a), \varepsilon)$.\\
Vérifions que $B_X(a, \delta_\varepsilon) \subset f^{-1}(U)$.\\
Soit $x \in B_X(a, \delta_\varepsilon)$.
 Alors $d_X(x,a) < \delta_\varepsilon \Rightarrow  d_Y(f(x),f(a)) < \varepsilon \Rightarrow f(x) \in B_Y(f(a), \varepsilon) \subset U$ $\Rightarrow x \in f^{-1}(U)$.\\
Donc $B_X(a, \delta_\varepsilon) \subset f^{-1}(U)$. Ainsi, $f^{-1}(U)$ est un ouvert de $(X,d_X)$.\\

\noindent$(\Leftarrow)$ Soit $U$ un ouvert de $(Y,d_Y)$.\\
Alors $f^{-1}(U)$ est un ouvert de $(X,d_X)$.\\
Soit $a \in f^{-1}(U)$. Alors $f(a) \in U$.\\
Soit $\varepsilon > 0$ tq $B(f(a), \varepsilon) \subset U$.\\
Montrons que $\exists \delta_\varepsilon > 0$ tel que $d_X(x,a) < \delta_\varepsilon$ et $d_Y(f(x), f(a)) < \varepsilon$.\\
Comme $f^{-1}(U)$ est un ouvert de $(X,d_X)$, il existe $\delta_\varepsilon > 0$ tel que $B_X(a, \delta_\varepsilon) \subset f^{-1}(U)$ qui est ouvert.\\
Alors si $x \in X$ tq $d_X(x,a) < \delta_\varepsilon$, donc $f(x) \in B(f(a), \varepsilon)$.\\
Donc on a $d_Y(f(x), f(a)) < \varepsilon$ car $f(x) \in U$.\\
Ainsi, $f$ est continue en $a$.}

\noindent\textbf{Interlude : Inégalité de Cauchy-Schwarz}

Considérons $<\cdot , \cdot > \colon \mathbb{R}^n \times \mathbb{R}^n \to \mathbb{R}$ définie par $<x,y> = \sum_{i=1}^{n} x_i y_i$.
Cette application est bilinéaire, symétrique, positive et définie ($<x,x> = 0 \Leftrightarrow x = 0$).\\
On observe que la norme euclidienne s'écrit $\|x\|_2 = \sqrt{<x,x>}$.\\\\

On dit que $<\cdot , \cdot >$ est un produit scalaire (forme bilinéaire symétrique définie positive), et ce produit scalaire est relié à la norme 2 (il s'agit du produit scalaire euclidien).

\theorem{Théorème}{Inégalité de Cauchy-Schwarz}{false}{
    $\forall X,Y \in \mathbb{R}^n, |<X,Y>| \leq \|X\|_2 \|Y\|_2$.
}
\noindent {\textbf{Démonstration :}
\begin{itemize}
    \item Fixons $X, Y \in \mathbb{R}^n$ ($\neq 0_{\mathbb{R}^n}$)
    \[
    P(t) = \|X + tY\|_2^2 = \langle X + tY, X + tY \rangle = \|X\|_2^2 + t^2 \|Y\|_2^2 + 2t \langle X, Y \rangle 
    \ge 0 
    = t^2 \|Y\|_2^2 + 2t \langle X, Y \rangle + \|X\|_2^2.
    \]
    $P$ est un polynôme de degré 2 en $t$. $P(t) \ge 0, \forall t \in \mathbb{R}$, son discriminant est $\le 0$.
    \[
    \Delta = 4 \langle X, Y \rangle^2 - 4 \|X\|_2^2 \|Y\|_2^2 = 4(\langle X, Y \rangle^2 - \|X\|_2^2 \|Y\|_2^2).
    \]
    Or, $\Delta \le 0 \iff \langle X, Y \rangle^2 - \|X\|_2^2 \|Y\|_2^2 \le 0 \iff \langle X, Y \rangle^2 \le \|X\|_2^2 \|Y\|_2^2
    \iff |\langle X, Y \rangle| \le \|X\|_2 \|Y\|_2$.
    
    \item De plus, si $\Delta = 0$, $\exists t_0 \in \mathbb{R}$ racine double, $P(t_0) = 0 = \|X + t_0 Y\|_2^2$
    \[
    \implies X + t_0 Y = 0 \implies X \text{ et } Y \text{ sont colinéaires}.
    \]
    Et si $X$ et $Y$ sont colinéaires, alors
    \[
    |\langle X, Y \rangle| = |\langle X, \lambda X \rangle| = |\lambda| \langle X, X \rangle 
    = |\lambda| \|X\|_2^2 = |\lambda| \|X\|_2 \|X\|_2 
    = \|X\|_2 \|\lambda X\|_2 = \|X\|_2 \|Y\|_2.
    \]
\end{itemize}}

\theorem{Corollaire}{Égalité de Cauchy-Schwarz}{false}{
    L'égalité $|<X,Y>| = \|X\|_2 \|Y\|_2$ est vérifiée si et seulement si $X$ et $Y$ sont colinéaires.
}

\ndlr{Démo à reprendre, cf screen Laurent 2}

\subsubsection{Continuité des applications linéaires dans les evn}

\textit{A priori}, les espaces vectoriels ne sont pas forcément de dimension finie.\\
Soit $f \colon (E, \|\cdot\|_E) \to (F, \|\cdot\|_F)$ une application linéaire entre deux $K$-espaces vectoriels normés (evn).\\

\theorem{Proposition}{Caractérisation de la continuité des applications linéaires}{false}{
    Les propriétés suivantes sont équivalentes :
    \begin{enumerate}
        \item $f$ est continue de $(E, \|\cdot\|_E)$ dans $(F, \|\cdot\|_F)$.
        \item $f$ est continue en 0.
        \item $\exists k > 0$ tel que $\forall x \in E, \|f(x)\|_F \leq k \|x\|_E$.
    \end{enumerate}
}

\noindent{\textbf{Démonstration :}

\begin{itemize}
    \item $i) \Rightarrow ii)$ : $f$ continue sur $E \Rightarrow f$ continue en $0$, \quad $f(0_E) = 0_F$
    
    \item $ii) \Rightarrow iii)$ : $f$ continue en $0$ : Soit $\varepsilon > 0$, $\exists \delta > 0$ tel que si $\|x\|_E < \delta \Rightarrow \|f(x)\|_F < \varepsilon$.
    
    Soit $x \in E$. Posons $y = \frac{\delta x}{2\|x\|_E}$. Alors
    \[
        \|f(y)\|_F < \varepsilon \quad \Rightarrow \quad f\left(\frac{\delta x}{2\|x\|_E}\right) < \varepsilon
    \]
    \[
        \Rightarrow \frac{\delta}{2\|x\|_E} \|f(x)\|_F < \varepsilon
    \]
    \[
        \Rightarrow \|f(x)\|_F < \frac{2\varepsilon}{\delta} \|x\|_E
    \]
    
    On trouve $K = \frac{2\varepsilon}{\delta}$, indépendant de $x \in E$, tel que $\forall x \in E, \, \|f(x)\|_F \le K \|x\|_E$.
    
    \item $iii) \Rightarrow i)$ : On part de $\exists K > 0$ tel que $\forall x \in E, \|f(x)\|_F \le K \|x\|_E$, on obtient $\forall x,y \in E, \|f(x-y)\|_F \le K \|x-y\|_E$
    \[
        \Rightarrow \exists K > 0, \, \forall x,y \in E, \, \|f(x) - f(y)\|_F \le K \|x-y\|_E
    \]
    \[
        \Rightarrow f \text{ est } K\text{-Lipschitz} \quad \Rightarrow f \text{ continue.}
    \]
\end{itemize}
}
\subsection{Equivalence de normes}

\textbf{Problème :} Soit $E$ un evn avec une normée notée $N_1 \colon E \in \mathbb{R}^+$.\\
On peut \textit{a priori} mettre d'autres normes sur $E$, disons $N_2 \colon E \in \mathbb{R}^+$.\\
Si $(x_n)$ une suite de $E$ converge pour la norme $N_1$, converge-t-elle aussi pour la norme $N_2$ ?\\

\definition{
    $N_1$ et $N_2$ sont \textbf{équivalentes} si $\exists C,c > 0$ tels que $\forall x \in E, c N_2(x) \leq N_1(x) \leq C N_2(x)$.\\
    On note $N_1 \sim N_2$.
}

\theorem{Proposition}{}{false}{
    \begin{enumerate}
        \item $N_1 \sim N_2 \Leftrightarrow N_2 \sim N_1$.
        \item Si $N_1 \sim N_2$ et $N_2 \sim N_3$, alors $N_1 \sim N_3$.
    \end{enumerate}
}

\example{Voir TD, normes $\|\cdot\|_1, \|\cdot\|_2, \|\cdot\|_{\infty}$ sur $\mathbb{R}^n$.}
\remark{Un des buts du cours est de montrer que sur $\mathbb{R}^n$ (+ généralement pour tout evn), toutes les normes sont équivalentes.}

\theorem{Proposition}{}{false}{
    Soit $(x_n)$ une suite de $E$, où $E$ est muni de $N_1$ et $N_2$ deux normes équivalentes.\\
    Alors $(x_n)$ converge pour la norme $N_1$ si et seulement si $(x_n)$ converge pour la norme $N_2$.
}
\noindent{\textbf{Démonstration :}

\begin{itemize}
    \item $\Rightarrow$ : On suppose $\exists a \in X$ tel que $x_n \xrightarrow{N_1} a \Leftrightarrow N_1(x_n - a) \underset{n \to +\infty}{\longrightarrow} 0.$\\
    Or, $\underline{c N_2(z) \leq N_1(z)} \leq C N_2(z)$\\
    Cette inégalité implique $N_2(x_n - a) \leq \frac{1}{c} \underline{N_1(x_n - a)} \underset{n \to +\infty}{\longrightarrow} 0$\\
    $\Rightarrow N_2(x_n - a) \underset{n \to +\infty}{\longrightarrow} 0$\\
    $\Rightarrow x_n \xrightarrow{N_2} a.$\\
    \item $\Leftarrow$  : Si $x_n \xrightarrow{N_2} a \Rightarrow N_2(x_n - a) \underset{n \to +\infty}{\longrightarrow} 0.$\\
    On utilise $\underline{N_1(\cdot) \leq C N_2(\cdot)}$ ($C > 0$)\\
    $\frac{N_1}{c}(x_n - a) \to 0 \Rightarrow N_1(x_n - a) \underset{n \to +\infty}{\longrightarrow} 0.$\\
\end{itemize}}

\subsection{Norme subordonnée}
Soient $(E, \|\dot\|_E)$ et $(F, \|\dot\|_F)$.\\
Soit $f \colon E \to F$ une application linéaire continue.
On a vu que la continuité d'une application linéaire entre evn se caractérisaient de la façon suivante :\\
$\exists K > 0 \colon \forall x \in E \|f(x)\|_F \leq K\|x\|_E$.\\
Si $x \neq 0$, on peut considérer $\frac{\|f(x)\|_F}{\|x\|_E} \in \mathbb{R}_{+}$\\
Par continuité de $f$, $\forall x \in E\setminus\{0\} \frac{\|f(x)\|_F}{\|x\|_E} \leq K$.

\definition{
    La norme subordonnée de $f$ par rapport à $\|\cdot\|_E$ et $\|\cdot\|_F$ est définie par $|||f||| := \sup_{x\in E\setminus\{0\}} \frac{\|f(x)\|_F}{\|x\|_E}$.
}
\remark{Cette triple barre est bien définie et correspond à la meilleure constante de continuité de $f$.}
\theorem{Proposition}{Espace des applications linéaires continues}{false}{
    Notons $\mathcal{L}_{c}(E,F) = \{ f \colon E \to F \mid f \text{ est continue} \}$.
    Alors $|||\cdot|||$ est une norme sur $\mathcal{L}_{c}(E,F)$ et $(\mathcal{L}_{c}(E,F), |||\cdot|||)$ est un evn.
}

\vocabulary{On dit que $|||\cdot|||$ est la "norme triple".}

\noindent{\textbf{Démonstration :}
\begin{itemize}
    \item Séparation : Si $|||f||| = 0 \Rightarrow \sup_{x \neq 0} \frac{\|f(x)\|_F}{\|x\|_E} = 0 \Rightarrow \forall x \neq 0 \frac{\|f(x)\|_F}{\|x\|_E} = 0 \Rightarrow \forall x \neq 0, \|f(x)\|_F = 0 \Rightarrow f(x) = 0_F$.
    \item Homogénéité : Soit $\lambda \in K$, $f \in \mathcal{L}_c(E,F)$.
    \[
        |||\lambda f||| = \sup_{x \neq 0} \frac{\|\lambda f(x)\|_F}{\|x\|_E} = \sup_{x \neq 0} \frac{|\lambda| \|f(x)\|_F}{\|x\|_E} = |\lambda| \sup_{x \neq 0} \frac{\|f(x)\|_F}{\|x\|_E} = |\lambda| |||f|||
    \]
    \item Inégalité triangulaire : Soient $f,g \in \mathcal{L}_c(E,F)$.
    \[
        |||f + g||| = \sup_{x \neq 0} \frac{\|(f+g)(x)\|_F}{\|x\|_E} \leq \sup_{x \neq 0} \frac{\|f(x)\|_F + \|g(x)\|_F}{\|x\|_E} \leq \sup_{x \neq 0} \frac{\|f(x)\|_F}{\|x\|_E} + \sup_{x \neq 0} \frac{\|g(x)\|_F}{\|x\|_E} = |||f||| + |||g|||
    \]
\end{itemize}
}

\theorem{Proposition}{}{true}{
    Soit $f \in \mathcal{L}_c(E,F)$.
    On a $|||f||| = \sup_{x \neq 0} \frac{\|f(x)\|_F}{\|x\|_E} = \sup_{\|x\|_E \leq 1} \|f(x)\|_F$ = $ \sup_{\|x\|_E = 1} \|f(x)\|_F$.
}

\remark{On montrera qu'en dimension finie, toute application linéaire est continue.}

\subsection{Introduction à la complétude dans les espaces métriques}

\subsubsection{Définitions et premières propriétés}

\definition{
    Soit $(X,d)$ un espace métrique.\\
    Une suite $(x_n)$ de $(X,d)$ est dite de Cauchy si \[\forall \varepsilon > 0, \exists N \in \mathbb{N}, \forall m,n \geq N, d(x_n,x_m) < \varepsilon\]
}

\remark{Si on travaille dans $(X,d)$ = $(E, \|\cdot\|)$ un evn, avec $d_E(x,y) = \|x-y\|$, et si on se donne une autre norme $\|\cdot\|'$ sur $E$ équivalente à $\|\cdot\|$, alors une suite est de Cauchy pour $\|\cdot\|$ si et seulement si elle est de Cauchy pour $\|\cdot\|'$.}
\ndlr{cf. Laurent pour la démonstration de la remarque}

\theorem{Proposition}{}{false}{
    \begin{itemize}
        \item Soit $(x_n)$ une suite d'éléments de $(X,d)$ qui converge vers $x \in X$.\\
        Alors $(x_n)$ est une suite de Cauchy. 
        \item Une suite de Cauchy a au plus une valeur d'adhérence.
        \item Une suite de Cauchy qui a une valeur d'adhérence converge vers cette valeur d'adhérence.
        
    \end{itemize}
}

\end{document}