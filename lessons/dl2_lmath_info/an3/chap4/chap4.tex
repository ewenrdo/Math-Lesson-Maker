\documentclass{article}

\usepackage[a4paper, left=1.5cm, right=1.5cm, top=2cm, bottom=2cm]{geometry}

\usepackage{amsmath,amssymb}

\usepackage{../../../../components/components} % <-- ton fichier .sty, avec toutes tes définitions

\usepackage{fancyhdr}


% Configuration des en-têtes et pieds de page
\pagestyle{fancy}
\fancyhf{} % reset tout

\fancyhead[L]{DL2 Math-Info}
\fancyhead[C]{Topologie}
\fancyhead[R]{2025-2026}

\fancyfoot[L]{Ewen Rodrigues de Oliveira}
\fancyfoot[R]{\thepage}

\begin{document}

\docTitle{Chapitre 4 : Topologie des espaces métriques et des espaces vectoriels normés}

\section{Distances et normes}

Soit $X$ un ensemble quelconque (dans la suite supposé non nul).

\definition{
    Une distance $d$ sur $X$ est une application $d\colon X\times X \to \mathbb{R}_+$ qui vérifie les axiomes suivants :
    \begin{enumerate}
        \item $\forall (x,y) \in X^2, d(x,y) = d(y,x)$ (symétrie)
        \item $\forall x,y,z \in X, d(x,z) \leq d(x,y) + d(y,z)$ (inégalité triangulaire)
        \item $\forall (x,y) \in X^2, d(x,y) = 0 \Leftrightarrow x = y$ (séparation)
    \end{enumerate}
}

\vocabulary{
    On appelle \textbf{espace métrique} un couple $(X,d)$ où $X$ est un ensemble et $d$ une distance sur $X$.
}

\example{
    \begin{itemize}
        \item $\mathbb{R}$ muni de la distance $d(x,y) = |x-y|$, où $|\cdot|$ est la valeur absolue.
        \item $\mathbb{C}$ muni de la distance $d(z_1, z_2) = |z_1 - z_2|$, où $|\cdot|$ est le module.
        \item Une autre façon de voir l'exemple 2, on considère l'ensemble $\mathbb{R}^2$ muni de la distance suivante :\\
        Si $A = (x_A, y_A)$ et $B = (x_B, y_B)$ sont deux points de $\mathbb{R}^2$, on définit la distance $d(A,B) = \sqrt{(x_A - x_B)^2 + (y_A - y_B)^2}$.\\
        On appelle cette distance la \textbf{distance euclidienne} sur $\mathbb{R}^2$.
        \item Prenons $X =$ cercle unité muni de la distance $d(A,B) = arccos(cos(\theta_2 - \theta_1))$ où $\theta_1$ et $\theta_2$ sont les arguments des points $A$ et $B$ respectivement.
        \textit{(voir schéma OneNote)}
    \end{itemize}
}

\remark{
    On peut voir l'exemple 1 comme un cas particulier de l'exemple 2, en identifiant $\mathbb{R}$ à l'axe des réels dans le plan complexe.
}

\remark{On peut généraliser l'exemple 3 à $\mathbb{R}^n$ muni de la distance euclidienne définie par :
\[d(A,B) = \sqrt{\sum_{i=1}^{n} (x_{A_i} - x_{B_i})^2}\]
où $A = (x_{A_1}, x_{A_2}, \ldots, x_{A_n})$ et $B = (x_{B_1}, x_{B_2}, \ldots, x_{B_n})$ sont deux points de $\mathbb{R}^n$.}

Il se trouve que les exemples 1, 2 et 3 proviennent d'espaces vectoriels normés, qu'on verra (très) rapidement.

\theorem{Proposition}{Inégalités triangulaires}{false}{
    Soit $(X,d)$ un espace métrique.\\
    On a :
    \begin{enumerate}
        \item $\forall (x,y,z) \in X^3, d(x,z) \leq d(x,y) + d(y,z)$ (inégalité triangulaire) 
        \item $\forall (x,y,z) \in X^3, |d(x,y) - d(x,z)| \leq d(y,z)$ (inégalité triangulaire généralisée)
    \end{enumerate}
}

\noindent{\textbf{Preuve :} 
    \begin{enumerate}
        \item C'est l'axiome 2 de la définition d'une distance.
        \item On a $d(x,y) \leq d(x,z) + d(z,y)$ d'après l'inégalité triangulaire.\\
        Donc $d(x,y) - d(x,z) \leq d(z,y)$ = $d(y,z)$.\\
        De plus, $d(x,z) \leq d(x,y) + d(y,z)$ d'après l'inégalité triangulaire.\\
        Donc $d(x,z) - d(x,y) \leq d(y,z)$.\\
        En combinant les deux, on obtient $|d(x,y) - d(x,z)| \leq d(y,z)$.
    \end{enumerate}
}

\theorem{Construction d'espaces métriques}{Restriction}{true}{
    On part de $(X,d)$ un espace métrique.\\
    Soit $Y \subset X$. Alors $(Y, d|_{Y\times Y})$ est un espace métrique.
    }
\vocabulary{On dit que $d|_{Y\times Y}$ est la \textbf{distance sur $Y$ induite} par la distance sur $X$.}
\remark{Ainsi, tout sous-ensemble d'un espace métrique est muni d'une "structure d'espace métrique" induite : c'est la distance induite.}

\example{$(X, d_X) = (\mathbb{C}, |\cdot|)$ un espace métrique. Alors pour $Y = \mathbb{R} \subset \mathbb{C}$, $d_Y$ n'est autre que la distance usuelle sur $\mathbb{R}$.\\
Idem pour $\mathbb{Q}$.}

\theorem{Construction d'espaces métriques}{Bijection}{true}{
    Soient $(X,d_X)$ et $Y$ un ensemble quelconque et \functionSets{f}{Y \rightarrow X} bijective.\\
    Alors $(Y, d_Y)$ est un espace métrique, où $d_Y(y_1,y_2) = d_X(f(y_1),f(y_2))$ pour tout $y_1,y_2 \in Y$.\\

    Autrement dit, \function{d_Y}{Y \times Y \rightarrow \mathbb{R}_+}{(y_1,y_2) \mapsto d_X(f(y_1),f(y_2))}.
}

\training{Démontrer le théorème ci-dessus. \textit{(vérifier que $d_Y$ satisfait bien les axiomes d'une distance.)}}
\noindent\carreaux{10}

\subsection{Espaces vectoriels normés}

On a plus un ensemble $X$ quelconque mais on considère $E$ un $\mathbb{K}$-espace vectoriel ($\mathbb{K} = \mathbb{R}$ ou $\mathbb{C}$). (a priori pas de dimension finie)

\subsubsection{Définitions}
\definition{
    Une norme $N$ sur $E$ est une application $N\colon E \to \mathbb{R}_+$ souvent notée $\| \cdot \|$ qui vérifie les axiomes suivants :
    \begin{enumerate}
        \item $\forall u \in E, \forall \lambda \in \mathbb{K}, \|\lambda u\| = |\lambda| \cdot \|u\|$ (homogénéité)
        \item $\forall (u,v) \in E^2, \|u+v\| \leq \|u\| + \|v\|$ (inégalité triangulaire)
        \item $\forall u \in E, \|u\| = 0 \Leftrightarrow u = 0_E$ (séparation)
    \end{enumerate}
}
\vocabulary{On dit que le couple $(E, N) = (E, \| \cdot \|)$ est un \textbf{espace vectoriel normé}. Il est commun d'écrire $e.v.n.$.}
\remark{$\forall x \in E, N(-x) = N(x)$ d'après l'axiome 1.}

\theorem{Théorème}{Association norme-distance}{false}{
    Soit $(E, \| \cdot \|)$ un espace vectoriel normé.\\
    Alors c'est en particulier un espace métrique pour la distance définie par $d_N(u,v) = \|v - u\|$.
}

\noindent{\textbf{Preuve :} Il suffit de vérifier les axiomes d'une distance.
    \begin{enumerate}
        \item Symétrie : $d_N(u,v) = \|v-u\| = \|-(u-v)\| = \|u-v\| = d_N(v,u)$.
        \item Inégalité triangulaire : $d_N(u,w) = \|w-u\| = \|(w-v) + (v-u)\| \leq \|w-v\| + \|v-u\| = d_N(u,v) + d_N(v,w)$.
        \item Séparation : $d_N(u,v) = 0 \Leftrightarrow \|v-u\| = 0 \Leftrightarrow v-u = 0_E \Leftrightarrow u=v$.
    \end{enumerate}
}

\remark{Hors programme : espaces euclidiens $\subset$ espaces vectoriels normés $\subset$ espaces métriques.}

\example{
    En fait, les exemples 1, 2 et 3 dans "Espaces métriques" sont des evn : $(\mathbb{R}, |\cdot|)$, $(\mathbb{C}, |\cdot|)$ et $(\mathbb{R}^n, \|\cdot\|_2)$ où $\| (x_1, \ldots, x_n) \|_2 = \sqrt{\sum_{i=1}^{n} x_i^2} = \| \cdot \|_\text{euclidienne}$.
}

\training{
    Définissons $X = (x_1, x_2, \ldots, x_n) \in \mathbb{R}^n \mapsto \|X\|_{\infty} = \max(|x_1|, |x_2|, \ldots, |x_n|)$.\\
    Montrer que $\|\cdot\|_{\infty}$ est une norme sur $\mathbb{R}^n$.\\
    \textbf{Observation :} $\|X\|_{\infty} = |x_{i_0}|$ où $i_0 \in \{1, 2, \ldots, n\}$ réalise le maximum.\\
    \textbf{Homogénéité :} $\forall i \in \{1, \ldots, n\}, |x_i| \leq |x_{i_0}|$\\
    $\forall \lambda \in \mathbb{R},$ $|\lambda| \cdot |x_i| \leq |\lambda| \cdot |x_{i_0}| \forall i \in \{1, \ldots, n\}$\\
    D'où $\forall \lambda \in \mathbb{R}, |\lambda x_i| \leq |\lambda| \cdot |x_{i_0}| \forall i \in \{1, \ldots, n\}$\\
    Or $\lambda X = (\lambda x_1, \ldots, \lambda x_n)$, $i_0$ réalise le maximum de $(|\lambda x_1|, \ldots, |\lambda x_n|)$\\
    Donc $\|\lambda X\|_{\infty} = |\lambda| \cdot |x_{i_0}| = |\lambda| \cdot \|X\|_{\infty}$.\\
    \textbf{Inégalité triangulaire :} Soient $X = (x_1, \ldots, x_n)$ et $Y = (y_1, \ldots, y_n)$ dans $\mathbb{R}^n$.\\
    $\|X+Y\|_{\infty} = \max(|x_1 + y_1|, \ldots, |x_n + y_n|)$.\\
    On a $\forall i \in \{1, \ldots, n\}, |x_i + y_i| \leq |x_i| + |y_i| \leq \|X\|_{\infty} + \|Y\|_{\infty}$.\\
    En prenant le $max$ sur $i$, on obtient $\|X+Y\|_{\infty} \leq \|X\|_{\infty} + \|Y\|_{\infty}$.\\
    \textbf{Séparation :} Soit $X = (x_1, \ldots, x_n) \in \mathbb{R}^n$.\\
    $\|X\|_{\infty} = 0 \Leftrightarrow \max(|x_1|, \ldots, |x_n|) = 0$.\\
    Or $\forall k \in \{1, \ldots, n\}, |x_k| \leq \max(|x_1|, \ldots, |x_n|) = 0 \Rightarrow x_k = 0$ $\Rightarrow X = 0_{\mathbb{R}^n}$.\\
}

\training{
\begin{enumerate}
    \item Montrer que \function{\|\cdot\|_1}{\mathbb{R}^n \rightarrow \mathbb{R}_+}{X \mapsto \sum_{i=1}^{n} |x_i|} est une norme sur $\mathbb{R}^n$.
    \item \textbf{Plus difficile.} Plus généralement, montrer que pour $p \in \mathbb{R}_{+}$, \function{\|\cdot\|_p}{\mathbb{R}^n \rightarrow \mathbb{R}_+}{X \mapsto \left( \sum_{i=1}^{n} |x_i|^p \right)^{1/p}} est une norme sur $\mathbb{R}^n$.
\end{enumerate}    
}

\example{Exemples de normes sur des espaces de dimension infinie :
    \begin{itemize}
        \item Considérons $E = \mathcal{C}([0,1], \mathbb{R})$ l'espace vectoriel des fonctions continues de $[0,1]$ dans $\mathbb{R}$. On définit la norme $\|f\|_{\infty} = \sup_{x \in [0,1]} |f(x)|$ pour $f \in E$. En effet,\\\\
        \textbf{Homogénéité :} Soit $f \in E$ et $\lambda \in \mathbb{R}$.\\
        On a $\|\lambda f\|_{\infty} = \sup_{x \in [0,1]} |\lambda f(x)| = |\lambda| \cdot \sup_{x \in [0,1]} |f(x)| = |\lambda| \cdot \|f\|_{\infty}$.\\
        \textbf{Inégalité triangulaire :} Soient $f,g \in E$.\\
        On a $\|f+g\|_{\infty} = \sup_{x \in [0,1]} |f(x) + g(x)|$.\\
        Or $\forall x \in [0,1], |f(x) + g(x)| \leq |f(x)| + |g(x)| \leq \|f\|_{\infty} + \|g\|_{\infty}$.\\
        En prenant le supremum sur $x$, on obtient $\|f+g\|_{\infty} \leq \|f\|_{\infty} + \|g\|_{\infty}$.\\
        \textbf{Séparation :} Soit $f \in E$.\\
        On a $\|f\|_{\infty} = 0 \Leftrightarrow \sup_{x \in [0,1]} |f(x)| = 0$.\\
        Or $\forall x \in [0,1], |f(x)| \leq \sup_{x \in [0,1]} |f(x)| = 0 \Rightarrow f(x) = 0$ pour tout $x$ dans $[0,1]$ $\Rightarrow f = 0_E$.\\
        \item Considérons $E = \{ f\colon I \to \mathbb{R}, f \in \mathcal{C}^0(I) \mid \int_I |f|(t)dt \text{ converge} \} = L_1$ où $I$ est un intervalle de $\mathbb{R}$. On définit la norme $\|f\|_1 = \int_I |f(t)| dt$ pour $f \in E$.
    \end{itemize}
}

\attention{$\|f\|_{\infty}$ existe car $\sup|f(t)| < +\infty$ pour $f$ continue sur le compact $[0,1]$.}

\cexample{$f \in \mathcal{C}^0(]0,1], \mathbb{R})$ définie par $f(x) = \frac{1}{x}$. Alors $\|f\|_{\infty}$ n'existe pas car $\sup_{x \in ]0,1]} |f(x)| = +\infty$.}

\subsubsection{Propriétés}

\theorem{Propriété}{Vecteurs unitaires}{true}{
    Soit $E$ un $\mathbb{K}$-espace vectoriel non nul, soit $x \in E\setminus \{0_E\}$ et soit $N$ une norme sur $E$.\\
    Alors $\frac{x}{N(x)}$ est un vecteur unitaire (ou vecteur normé), c'est-à-dire $N\left(\frac{x}{N(x)}\right) = 1$. (existe car $N(x) \neq 0$ par séparation)
}

\end{document}