\documentclass{article}

\usepackage[a4paper, left=1.5cm, right=1.5cm, top=2cm, bottom=2cm]{geometry}

\usepackage{amsmath,amssymb}

\usepackage{../../../../components/components} % <-- ton fichier .sty, avec toutes tes définitions

\usepackage{fancyhdr}


% Configuration des en-têtes et pieds de page
\pagestyle{fancy}
\fancyhf{} % reset tout

\fancyhead[L]{DL2 Math-Info}
\fancyhead[C]{Topologie}
\fancyhead[R]{2025-2026}

\fancyfoot[L]{Ewen Rodrigues de Oliveira}
\fancyfoot[R]{\thepage}

\begin{document}

\docTitle{Chapitre 4 : Topologie des espaces métriques et des espaces vectoriels normés}

\section{Distances et normes}

Soit $X$ un ensemble quelconque (dans la suite supposé non nul).

\definition{
    Une distance $d$ sur $X$ est une application $d\colon X\times X \to \mathbb{R}_+$ qui vérifie les axiomes suivants :
    \begin{enumerate}
        \item $\forall (x,y) \in X^2, d(x,y) = d(y,x)$ (symétrie)
        \item $\forall x,y,z \in X, d(x,z) \leq d(x,y) + d(y,z)$ (inégalité triangulaire)
        \item $\forall (x,y) \in X^2, d(x,y) = 0 \Leftrightarrow x = y$ (séparation)
    \end{enumerate}
}

\vocabulary{
    On appelle \textbf{espace métrique} un couple $(X,d)$ où $X$ est un ensemble et $d$ une distance sur $X$.
}

\example{
    \begin{itemize}
        \item $\mathbb{R}$ muni de la distance $d(x,y) = |x-y|$, où $|\cdot|$ est la valeur absolue.
        \item $\mathbb{C}$ muni de la distance $d(z_1, z_2) = |z_1 - z_2|$, où $|\cdot|$ est le module.
        \item Une autre façon de voir l'exemple 2, on considère l'ensemble $\mathbb{R}^2$ muni de la distance suivante :\\
        Si $A = (x_A, y_A)$ et $B = (x_B, y_B)$ sont deux points de $\mathbb{R}^2$, on définit la distance $d(A,B) = \sqrt{(x_A - x_B)^2 + (y_A - y_B)^2}$.\\
        On appelle cette distance la \textbf{distance euclidienne} sur $\mathbb{R}^2$.
        \item Prenons $X =$ cercle unité muni de la distance $d(A,B) = arccos(cos(\theta_2 - \theta_1))$ où $\theta_1$ et $\theta_2$ sont les arguments des points $A$ et $B$ respectivement.
        \textit{(voir schéma OneNote)}
    \end{itemize}
}

\remark{
    On peut voir l'exemple 1 comme un cas particulier de l'exemple 2, en identifiant $\mathbb{R}$ à l'axe des réels dans le plan complexe.
}

\remark{On peut généraliser l'exemple 3 à $\mathbb{R}^n$ muni de la distance euclidienne définie par :
\[d(A,B) = \sqrt{\sum_{i=1}^{n} (x_{A_i} - x_{B_i})^2}\]
où $A = (x_{A_1}, x_{A_2}, \ldots, x_{A_n})$ et $B = (x_{B_1}, x_{B_2}, \ldots, x_{B_n})$ sont deux points de $\mathbb{R}^n$.}

\end{document}