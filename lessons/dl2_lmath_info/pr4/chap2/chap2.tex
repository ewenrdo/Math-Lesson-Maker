\documentclass{article}

\usepackage[a4paper, left=1.5cm, right=1.5cm, top=2cm, bottom=2cm]{geometry}

\usepackage{../../../../components/components}

\usepackage{fancyhdr}


% Configuration des en-têtes et pieds de page
\pagestyle{fancy}
\fancyhf{} % reset tout

\fancyhead[L]{DL2 Math-Info PR4}
\fancyhead[C]{Probabilités}
\fancyhead[R]{2025-2026}

\fancyfoot[L]{Ewen Rodrigues de Oliveira}
\fancyfoot[R]{\thepage}

\begin{document}

\docTitle{Chapitre 2 : Espaces probabilisés}

\section{Dénombrabilité}
\definition{
    On dit que $E$ est \textbf{infini dénombrable} s'il existe une bijection entre $E$ en $\mathbb{N}$.
}

\definition{
    $E$ est \textbf{dénombrale} s'il est fini ou infini dénombrable. 
}

\theorem{Proposition}{}{false}{
    Si $E$ et $F$ sont dénombrables, alors $E \times F$ est dénombrable.
}

\noindent{\textbf{Démonstration :}\\
Supposons $E$ et $F$ dénombrales, donc supposons que $E = \mathbb{N}$ et $F = \mathbb{N}$.
}

\theorem{Proposition}{}{false}{
    Soit $(E_i)_{i \in J}$ où $J \subset \mathbb{N}$ et $E_i$ est dénombrable pour chaque $i \in J$.\\
    Alors $\bigcup_{i \in J} E_i$ est dénombrable.
}

\section{Espaces probabilisés dénombrables}
\subsection{Généralités}
\definition{
    Soit $\Omega$ un ensemble dénombrable.\\
    Une \textbf{probabilité} sur $\Omega$ est une application $\mathbb{P} : \mathcal{P}(\Omega) \to [0, 1]$ telle que :
    \begin{enumerate}
        \item $\mathbb{P}(\emptyset) = 0$ et $\mathbb{P}(\Omega) = 1$
        \item Soit $(E_n)_{n \in \mathbb{N}}$ tq $E_i \cap E_j = \emptyset$ une suite finie ou dénombrable de parties de $\Omega$ alors $\mathbb{P}(\bigcup_{n} E_n) = \sum_{n} \mathbb{P}(E_n)$. ("$\sigma$-additivité")
    \end{enumerate}
}

\remark{$\Omega$ est aussi appelé "univers". Les parties de $\Omega$ sont appelées "événements".}
\vocabulary{$\omega \in \Omega$ est appelé un "résultat-élémentaire".}
\vocabulary{$E \in \mathcal{P}(\Omega)$ est appelé un "événement".}

\remark{
    $E = \{ \omega_1, \omega_2, \ldots , \omega_n, \ldots \} = \bigcup_{i} \{ \omega_i \}$\\
    $\mathbb{P}(E) = \mathbb{P}(\bigcup_{i} \{ \omega_i \}) = \sum_{i} \mathbb{P}(\{ \omega_i \})$
}

\theorem{Proposition}{}{true}{
    Soient $\Omega$ et $\mathbb{P}$ sur $\Omega$ une probabilité sur $\Omega$.\\
    Soit $f : \Omega \to \mathbb{R}$ définie par $f(\omega) = \mathbb{P}(\{\omega\})$ pour tout $\omega \in \Omega$.\\
    Alors : 
    \begin{enumerate}
        \item $0 \leq f(\omega) \leq 1$.
        \item $\sum_{\omega \in \Omega} f(\omega) = 1$.
    \end{enumerate}
}

\theorem{Proposition}{}{true}{
    Soit $\Omega$ un ensemble dénombrable alors et $f : \Omega \to \mathbb{R}$ tel que :
    \begin{enumerate}
        \item $f(\omega) \geq 0$
        \item $\sum_{\omega \in \Omega} f(\omega) = 1$
    \end{enumerate}
    Alors $\mathbb{P}_f(E) : \mathcal{P}(\Omega) \to \mathbb{R}$ définie par $\mathbb{P}_f(E) = \sum_{\omega \in E} f(\omega)$ est une probabilité sur $\Omega$.
}

\noindent{\textbf{Démonstration :}\\
\begin{enumerate}
    \item $\mathbb{P}_f(E) = [0,1]$\\
    $\mathbb{P}_f(E) \geq 0$ car $f(\omega) \geq 0$.\\
    $\mathbb{P}_f(E) = \sum_{\omega \in E} f(\omega) \leq \sum_{\omega \in \Omega} f(\omega) = 1$.\\
    \item $\sigma$-additivité : Soit $(E_n)_{n \in \mathbb{N}}$ où $E_n \cap E_m = \emptyset$ pour $n \neq m$ et $E \in \mathcal{P}(\Omega)$.\\
    $\mathbb{P}_f(\bigcup_{n} E_n) = \sum_{\omega \in \bigcup_{n} E_n} f(\omega) = \sum_{n} \sum_{\omega \in E_n} f(\omega) = \sum_{n} \mathbb{P}_f(E_n)$.    
\end{enumerate}
}

\definition{Soit $\Omega$ un ensemble dénombrable et $\mathbb{P}$ une probabilité sur $\Omega$, $(\Omega, \mathbb{P})$ est un \textbf{espace de probabilité}.}

\definition{
    Soit $\Omega$ un ensemble fini alors la probabilité $\mathbb{P}(\{\omega\}) = \frac{1}{|\Omega|}$ pour tout $\omega \in \Omega$ est appelée \textbf{probabilité uniforme} sur $\Omega$.
}

\example{
    Soit $\Omega=\{1, 2, ..., 6\}=\{(\omega_1, \omega_2): \omega_i\in \{1, ..., 6\}\}$\\
    On a $|\Omega|=36$, $\mathbb{P}_{uniforme}(\{\omega\})=\frac{1}{36}$.\\
    $A=\{$résultat du 1e dé est 4$\}=\{(4, \omega):\omega\in\{1, ..., 6\}\}$. 
    On a $|A|=6$ et $\mathbb{P}(A)=\frac{6}{36}=\frac{1}{6}$.\\
    $B=\{$la somme est égale à 4$\}=\{(1, 3), (3, 1), (2, 2)\}$.
    On a donc $|B|=3$ et donc $\mathbb{P}(B)=\frac{3}{36}.$
} 

\subsection{Propriétés élémentaires}

\theorem{Proposition}{}{false}{
    Soit $A_1, A_2 \in \mathcal{P}(\Omega)$ alors :
    \[
        \mathbb{P}\left(A_1 \cup A_2\right) = \mathbb{P}(A_1) + \mathbb{P}(A_2) - \mathbb{P}(A_1 \cap A_2)
    \]
}

\noindent{\textbf{Démonstration :}\\
Patatoïde \textit{(diagramme de Venn)}.\\\\
On a $\mathbb{P}(A_1) = \mathbb{P}(A_1 \cap A_2^c) + \mathbb{P}(A_1 \cap A_2)$ car $A_1 = (A_1 \cap A_2^c) \cup (A_1 \cap A_2)$.\\\\
De même, $\mathbb{P}(A_2) = \mathbb{P}(A_2 \cap A_1^c) + \mathbb{P}(A_1 \cap A_2)$.\\\\
Donc $\mathbb{P}(A_1) + \mathbb{P}(A_2) = \mathbb{P}(A_1 \cap A_2^c) + \mathbb{P}(A_2 \cap A_1^c) + 2\mathbb{P}(A_1 \cap A_2)$.\\\\
Or, $\mathbb{P}(A_1 \cup A_2) = \mathbb{P}(A_1 \cap A_2^c) + \mathbb{P}(A_2 \cap A_1^c) + \mathbb{P}(A_1 \cap A_2)$.\\\\
D'où : $\mathbb{P}(A_1) + \mathbb{P}(A_2) = \mathbb{P}(A_1 \cup A_2) + \mathbb{P}(A_1 \cap A_2) \Rightarrow \mathbb{P}(A_1 \cup A_2) = \mathbb{P}(A_1) + \mathbb{P}(A_2) - \mathbb{P}(A_1 \cap A_2)$. $\Box$
}

\remark{On a toujours que $\mathbb{P}(A^c) = 1 - \mathbb{P}(A)$ car $\mathbb{P}(A) + \mathbb{P}(A^c) = \mathbb{P}(\Omega) = 1$.}

\theorem{Proposition}{Principe d'inclusion-exclusion}{true}{
    Soit $E_1, E_2, \ldots , E_n \in \mathcal{P}(\Omega)$ alors :
    \begin{align*}
        \mathbb{P}\left(\bigcup_{k=1}^{n} E_k\right) = \sum_{k=1}^n (-1)^{k+1} \left( \sum_{1 \leq i_1 < i_2 < \ldots < i_k \leq n} \mathbb{P}\left(E_{i_1} \cap E_{i_2} \cap \ldots \cap E_{i_k}\right) \right)
    \end{align*}
}

\training{Dans le cas $n = 3$, exprimer la formule précédente.}
On a : $\mathbb{P}(E_1 \cup E_2 \cup E_3) = \mathbb{P}(E_1) + \mathbb{P}(E_2) + \mathbb{P}(E_3) - \mathbb{P}(E_1 \cap E_2) - \mathbb{P}(E_1 \cap E_3) - \mathbb{P}(E_2 \cap E_3) + \mathbb{P}(E_1 \cap E_2 \cap E_3)$.

\theorem{Proposition}{}{false}{
    Si $A \subset B \in \mathcal{P}(\Omega)$ alors $\mathbb{P}(A) \leq \mathbb{P}(B)$.
}
\noindent{\textbf{Démonstration :}\\
$B = A \cup (B \cap A^c)$ avec $A \cap (B \cap A^c) = \emptyset$.\\
Donc, $\mathbb{P}(B) = \mathbb{P}(A) + \mathbb{P}(B \cap A^c) \geq \mathbb{P}(A)$. $\Box$
}

\theorem{Proposition}{}{false}{
    Soit $(E_i)_{i \in \mathbb{N}}$ une suite finie ou infinie dénombrable d'événements de $\Omega$.\\
    On a $\mathbb{P}(\bigcup_{i=1}^{n} E_i) \leq \sum_{i=1}^{n} \mathbb{P}(E_i)$.
}
\noindent{ \textbf{Démonstration : } Par récurrence
    \begin{itemize}
        \item Si n = 2, c'est évident.
        \item Supposons que la propriété est vrai au rang $n$. On montre qu'elle est aussi vraie pour $n+1$\\
            $\mathbb{P}(\bigcup_{i=1}^{n+1} E_i)=\mathbb{P}(\bigcup_{i=1}^n E_i \cup E_{n+1})=\mathbb{P}(\bigcup_{i=1}^n E_i)+\mathbb{P}(E_{i+1})-\mathbb{P}(\bigcup_{i=1}^n E_i \cap E_{N+1})$\\
            $\le P(\bigcup_{i=1}^n E_i) + P(E_{n+1})\le \sum_{i=1}^n \mathbb{P}(E_i) + \mathbb{P}(E_{n+1})$ Par récurrence.
    \end{itemize}
}

\definition{
    Soit $(E_i)_{i \in \mathbb{N}}$ une suite de $\mathcal{P}(\Omega)$, telle que $E_i \subset E_{i+1}$ pour tout $i \in \mathbb{N}$.\\
    La \textbf{limite croissante} de la suite $(E_i)$ est l'événement $E = \bigcup_{i=0}^{\infty} E_i$.
}

\theorem{Proposition}{}{true}{
    $\mathbb{P}(E_{\infty}) = \lim_{n \to \infty} \mathbb{P}(E_n)$.
}

\noindent{\textbf{Démonstration :}\\
$F_0 = E_0$ et $F_i = E_i \setminus E_{i-1}$ pour $i \geq 1$.\\
Alors, $E_n = \bigcup_{i=0}^{n} F_i$, et $F_i \cap F_j = \emptyset$ pour $i \neq j$.\\
Donc, $\mathbb{P}(E_n) = \sum_{i=0}^{n} \mathbb{P}(F_i)$.\\
Or, $E_{\infty} = \bigcup_{i=0}^{\infty} F_i$.\\
Donc, $\mathbb{P}(E_{\infty}) = \sum_{i=0}^{\infty} \mathbb{P}(F_i) = \lim_{n \to \infty} \sum_{i=0}^{n} \mathbb{P}(F_i) = \lim_{n \to \infty} \mathbb{P}(E_n)$. $\Box$
}

\example{
    $(\mathbb{N}^{*}, \mathbb{P})$ avec $\mathbb{P}(\{k\}) = \frac{1}{2^{k}}$ la loi géométrique de paramètre $\frac{1}{2}$.\\
    $E_i = \{2, 4, \ldots , 2i\}$ alors $E_i \subset E_{i+1}$ et $E_{\infty} = (2\mathbb{N})^{*}$.\\
    $A_i = \{1,2, \ldots , i\}$ alors $A_i \subset A_{i+1}$ et $A_{\infty} = \mathbb{N}^{*}$.\\
}

\newpage
\tableofcontents

\end{document}