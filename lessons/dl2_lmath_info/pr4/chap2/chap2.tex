\documentclass{article}

\usepackage[a4paper, left=1.5cm, right=1.5cm, top=2cm, bottom=2cm]{geometry}

\usepackage{../../../../components/components}

\usepackage{fancyhdr}


% Configuration des en-têtes et pieds de page
\pagestyle{fancy}
\fancyhf{} % reset tout

\fancyhead[L]{DL2 Math-Info PR4}
\fancyhead[C]{Probabilités}
\fancyhead[R]{2025-2026}

\fancyfoot[L]{Ewen Rodrigues de Oliveira}
\fancyfoot[R]{\thepage}

\begin{document}

\docTitle{Chapitre 2 : Espaces de probabilités}

\definition{
    On dit que $E$ est \textbf{infini dénombrable} s'il existe une bijection entre $E$ en $\mathbb{N}$.
}

\definition{
    $E$ est \textbf{dénombrale} s'il est fini ou infini dénombrable. 
}

\theorem{Proposition}{}{false}{
    Si $E$ et $F$ sont dénombrables, alors $E \times F$ est dénombrable.
}

\noindent{\textbf{Démonstration :}\\
Supposons $E$ et $F$ dénombrales, donc supposons que $E = \mathbb{N}$ et $F = \mathbb{N}$.
}

\theorem{Proposition}{}{false}{
    Soit $(E_i)_{i \in J}$ où $J \subset \mathbb{N}$ et $E_i$ est dénombrable pour chaque $i \in J$.\\
    Alors $\bigcup_{i \in J} E_i$ est dénombrable.
}

\definition{
    Soit $\Omega$ un ensemble dénombrable.\\
    Une \textbf{probabilité} sur $\Omega$ est une application $\mathbb{P} : \mathcal{P}(\Omega) \to [0, 1]$ telle que :
    \begin{enumerate}
        \item $\mathbb{P}(\emptyset) = 0$ et $\mathbb{P}(\Omega) = 1$
        \item Soit $(E_n)_{n \in \mathbb{N}}$ tq $E_i \cap E_j = \emptyset$ une suite finie ou dénombrable de parties de $\Omega$ alors $\mathbb{P}(\bigcup_{n} E_n) = \sum_{n} \mathbb{P}(E_n)$.
    \end{enumerate}
}

\remark{$\Omega$ est aussi appelé "univers". Les parties de $\Omega$ sont appelées "événements".}
\vocabulary{$\omega \in \Omega$ est appelé un "résultat-élémentaire".}
\vocabulary{$E \in \mathcal{P}(\Omega)$ est appelé un "événement".}

\remark{
    $E = \{ \omega_1, \omega_2, \ldots , \omega_n, \ldots \} = \bigcup_{i} \{ \omega_i \}$\\
    $\mathbb{P}(E) = \mathbb{P}(\bigcup_{i} \{ \omega_i \}) = \sum_{i} \mathbb{P}(\{ \omega_i \})$
}

\theorem{Proposition}{}{true}{
    Soient $\Omega$ et $\mathbb{P}$ sur $\Omega$ une probabilité sur $\Omega$.\\
    Soit $f : \Omega \to \mathbb{R}$ définie par $f(\omega) = \mathbb{P}(\{\omega\})$ pour tout $\omega \in \Omega$.\\
    Alors : 
    \begin{enumerate}
        \item $0 \leq f(\omega) \leq 1$.
        \item $\sum_{\omega \in \Omega} f(\omega) = 1$.
    \end{enumerate}
}

\theorem{Proposition}{}{true}{
    Soit $\Omega$ un ensemble dénombrable alors et $f : \Omega \to \mathbb{R}$ tel que :
    \begin{enumerate}
        \item $f(\omega) \geq 0$
        \item $\sum_{\omega \in \Omega} f(\omega) = 1$
    \end{enumerate}
    Alors $\mathbb{P}_f(E) : \mathcal{P}(\Omega) \to \mathbb{R}$ définie par $\mathbb{P}_f(E) = \sum_{\omega \in E} f(\omega)$ est une probabilité sur $\Omega$.
}

\noindent{\textbf{Démonstration :}\\
\begin{enumerate}
    \item $\mathbb{P}_f(E) = [0,1]$\\
    $\mathbb{P}_f(E) \geq 0$ car $f(\omega) \geq 0$.\\
    $\mathbb{P}_f(E) = \sum_{\omega \in E} f(\omega) \leq \sum_{\omega \in \Omega} f(\omega) = 1$.\\
    \item $\sigma$-additivité : Soit $(E_n)_{n \in \mathbb{N}}$ où $E_n \cap E_m = \emptyset$ pour $n \neq m$ et $E \in \mathcal{P}(\Omega)$.\\
    $\mathbb{P}_f(\bigcup_{n} E_n) = \sum_{\omega \in \bigcup_{n} E_n} f(\omega) = \sum_{n} \sum_{\omega \in E_n} f(\omega) = \sum_{n} \mathbb{P}_f(E_n)$.    
\end{enumerate}
}

\definition{Soit $\Omega$ un ensemble dénombrable et $\mathbb{P}$ une probabilité sur $\Omega$, $(\Omega, \mathbb{P})$ est un \textbf{espace de probabilité}.}

\definition{
    Soit $\Omega$ un ensemble fini alors la probabilité $\mathbb{P}(\{\omega\}) = \frac{1}{|\Omega|}$ pour tout $\omega \in \Omega$ est appelée \textbf{probabilité uniforme} sur $\Omega$.
}

\ndlr{Exemple manquant à revoir avec Laurent.}

\end{document}