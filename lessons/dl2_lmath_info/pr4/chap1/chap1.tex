\documentclass{article}

\usepackage[a4paper, left=1.5cm, right=1.5cm, top=2cm, bottom=2cm]{geometry}

\usepackage{../../../../components/components}

\usepackage{fancyhdr}


% Configuration des en-têtes et pieds de page
\pagestyle{fancy}
\fancyhf{} % reset tout

\fancyhead[L]{DL2 Math-Info PR4}
\fancyhead[C]{Dénombrement}
\fancyhead[R]{2025-2026}

\fancyfoot[L]{Ewen Rodrigues de Oliveira}
\fancyfoot[R]{\thepage}

\begin{document}

\docTitle{Chapitre 1 : Dénombrement}

\section{Cardinal d'un ensemble}

\definition{
    Soit $E$ un ensemble.    
    On dit que $E$ est \textbf{fini} s'il existe un entier naturel $n \in \mathbb{N}$ et une bijection entre $E$ et l'ensemble $\{1, 2, \ldots, n\}$.\\
    Si $n = 0$, on dit que $E$ est vide et on note $E = \emptyset$.\\\\
    On appelle $n$ le \textbf{cardinal} de $E$ et on le note $Card(E)$.
}

\theorem{Lemme}{Égalité des cardinaux}{false}{
    Soient $A$ et $B$ deux ensembles finis non vides. Alors, on a $Card(A) = Card(B)$ si et seulement si il existe une bijection entre $A$ et $B$.
}
\noindent{\textbf{Démonstration :}\\
Soit $n \geq 1$ tq $Card(A) = n$.\\
\textit{On a au tableau le dessin d'une bijection entre $A$ et $S_n = \{1, 2, \ldots, n\}$.}\\
$\Rightarrow \exists f \colon A \to S_n$ bijective.\\
Si $Card(B) = n$, alors $\exists g \colon B \to S_n$ bijective.\\
Donc, $g^{-1} \colon S_n \to B$ est bijective.\\
Considérons $g^{-1} \circ f \colon A \to B$. C'est une bijection entre $A$ et $B$ par composition de bijections.\\
Donc, $Card(A) = Card(B)$.\\\\
}

\training{Montrer que si $A$ et $B$ sont en bijection, alors $Card(A) = Card(B)$.}
\noindent\carreaux{10}

\definition{
    Soient $E$ et $F$ deux ensembles non vides.
    Alors on appelle le \textbf{produit cartésien} de $E$ et $F$ et on note $E \times F := \{(x,y) \mid x \in E, y \in F\}$.
}
\example{
    Soient $E = \{1, 2\}$ et $F = \{5,6,7\}$.\\
    Alors, $E \times F = \{(1,5), (1,6), (1,7), (2,5), (2,6), (2,7)\}$.
}

\theorem{Proposition}{Principe multiplicatif}{false}{
    Soient $A$ et $B$ deux ensembles finis non vides. Alors, $A \times B$ est fini et on a :
    \[
        Card(A \times B) = Card(A) \times Card(B)
    \].\\
    En règle générale, si $A_1, A_2, \ldots, A_k$ sont des ensembles finis non vides, alors :
    \[
        Card(A_1 \times A_2 \times \ldots \times A_k) = \prod_{i=1}^{k} Card(A_i)
    \]
}

\noindent{\textbf{Démonstration :}\\
On démontre cette proposition pour le cas $k=2$.\\
Soient $A$ et $B$ deux ensembles finis non vides.\\
Soient $n = Card(A)$ et $m = Card(B)$.\\

On pose $A = \{1, 2, \ldots, n\}$ et $B = \{1, 2, \ldots, m\}$. (ça revient au même par le lemme précédent)\\
On peut dresser le tableau suivant pour représenter $A \times B$ :\\
\begin{center}
\begin{tabular}{c|c|c|c|c}
    & 1 & 2 & $\ldots$ & n \\
    \hline
    1 & (1,1) & (2,1) & $\ldots$ & (n,1) \\
    \hline
    2 & (1,2) & (2,2) & $\ldots$ & (n,2) \\
    \hline
    $\vdots$ & $\vdots$ & $\vdots$ & $\ddots$ & $\vdots$ \\
    \hline
    m & (1,m) & (2,m) & $\ldots$ & (n,m) \\
\end{tabular}
\end{center}
On remarque que le tableau contient $n$ colonnes et $m$ lignes.\\
Donc, le tableau contient $n \times m$ cases.\\
Chaque case correspond à un élément de $A \times B$.\\\\
On pose donc $f \colon A \times B \to \{1, 2, \ldots, n \times m\}$ l'application définie par $f((i,j)) = i + (j-1) \times n$ (l'idée est de numéroter les cases de gauche à droite et de haut en bas).\\
On vérifie facilement que $f$ est une bijection (strictement croissante et bien définie).\\
Donc, $Card(A \times B) = n \times m = Card(A) \times Card(B)$.\\
}

\definition{
    Soit $E$ un ensemble non vide et $n \geq 1$ un entier naturel.\\
    Les éléments de $E^n := \underbrace{E \times E \times \ldots \times E}_{n \text{ fois}}$ sont appelés des \textbf{n-uplets} d'éléments de $E$, et on les note $(\omega_1, \omega_2, \ldots, \omega_n)$ avec $\omega_i \in E$ pour tout $i \in \{1, 2, \ldots, n\}$.
}

\theorem{Proposition}{Cardinal des n-uplets}{false}{
    Soient $n,M \geq 1$ et soit $E$ un ensemble tel que $Card(E) = M$.\\
    On dit que le nombre de $n$-uplets d'éléments de $E$ est égal à $M^n$, c'est-à-dire :
    \[
        Card(E^n) = M^n
    \]
}

\example{L'ensemble des dates d'anniversaires des 8 étudiants du premier rang est décrit par $(\omega_1, \omega_2, \ldots, \omega_8)$ où chaque $\omega_i$ est une date parmi les 366 possibles \textit{(années bissextiles)}.\\
Le nombre de configurations possibles pour les dates d'anniversaires des 8 étudiants est donc $366^8$.
}

\definition{
    Soit $E$ un ensemble de cardinal $M \geq 1$ et soit $0 \leq n \leq M$.\\
    Un \textbf{arrangement} de $n$ éléments de $E$ est un n-uplet $(\omega_1, \omega_2, \ldots, \omega_n)$ d'éléments de $E$ tels que tous les $\omega_i$ sont distincts.\\
    On parle de $n$-uplets d'éléments distincts de $E$.
}

\vocabulary{Quand $n = M$, on parle de \textbf{permutations} d'éléments de $E$ \textit{(on parle de $A_M^M$)}.}

\example{Si on reprend l'exemple des dates d'anniversaires, un arrangement de 3 éléments parmi les 8 étudiants du premier rang pourrait être $(\omega_1, \omega_3, \omega_5)$ où les dates d'anniversaires des étudiants 1, 3 et 5 sont distinctes.}

\theorem{Proposition}{Cardinal des arrangements}{false}{
    Soient $E$ un ensemble de cardinal $M \geq 1$ et soit $0 \leq n \leq M$.\\
    Le nombre d'arrangements de $n$ éléments de $E$ est donné par la formule :
    \[
        A_M^n = \frac{M!}{(M-n)!}
    \]
    Dans le cas d'une permutation, on a $A_M^M = M!$ \textit{(application de la formule avec $n = M$)}.
}

\noindent{\textbf{Démonstration :}\\
Soit $E$ un ensemble de cardinal $M \geq 1$ et soit $0 \leq n \leq M$.\\
On cherche à compter le nombre d'arrangements de $n$ éléments de $E$.\\

Soit $(\omega_1, \omega_2, \ldots, \omega_n)$ un arrangement de $n$ éléments de $E$.\\
- Pour choisir $\omega_1$, on a $M$ possibilités.\\
- Pour choisir $\omega_2$, on a $M-1$ possibilités (car $\omega_2$ doit être distinct de $\omega_1$).\\
$\vdots$\\
- Pour choisir $\omega_n$, on a $M-n+1$ possibilités (car $\omega_n$ doit être distinct de tous les $\omega_i$ précédents).\\
Par le principe multiplicatif, le nombre total d'arrangements de $n$ éléments de $E$ est donc :
\[
    A_M^n = M \times (M-1) \times (M-2) \times \ldots \times (M-n+1) = \frac{M!}{(M-n)!}
\]
}

\reminder{On rappelle que $k! = \prod_{i=1}^{k} i$ pour tout $k \geq 1$ et $0! = 1$.}

\example{Si on reprend l'exemple des dates d'anniversaires, le nombre d'arrangements de 3 éléments parmi les 8 étudiants du premier rang est donné par :
\[
    A_8^3 = \frac{8!}{(8-3)!} = \frac{8!}{5!} = 8 \times 7 \times 6 = 336
\]
}

\theorem{Proposition}{Dénombrement d'applications}{false}{
    Soient $A$ et $B$ deux ensembles finis tels que $Card(A) = n$ et $Card(B) = M$ (avec $n, M \geq 1$).\\
    Alors :
    \begin{enumerate}
        \item L'ensemble $\mathcal{F}(A,B)$ des applications de $A$ dans $B$ est fini et on a : $Card(\mathcal{F}(A,B)) = M^n$.
        \item Si $n \leq M$, le nombre de fonctions injectives de $A$ dans $B$ est donné par : $A_M^n$.
        \item Si $n = M$, le nombre de fonctions bijectives de $A$ dans $B$ est donné par : $A_M^M = M!$.
    \end{enumerate}
}

\noindent{\textbf{Démonstration :}\\
Soient $A$ et $B$ deux ensembles finis tels que $Card(A) = n$ et $Card(B) = M$ (avec $n, M \geq 1$).\\
\begin{enumerate}
    \item Soit $f \in \mathcal{F}(A,B)$.\\
    On peut représenter $f$ par un n-uplet $(f(a_1), f(a_2), \ldots, f(a_n))$ d'éléments de $B$ où $a_i \in A$ pour tout $i \in \{1, 2, \ldots, n\}$.\\
    Chaque $f(a_i)$ peut prendre n'importe quelle valeur dans $B$, donc on a $M$ possibilités pour chaque $f(a_i)$.\\
    Par le principe multiplicatif, le nombre total de fonctions de $A$ dans $B$ est donc : $M^n$. \textit{Réciproquement, chaque n-uplet d'éléments de $B$ correspond à une fonction de $A$ dans $B$.}
    
    \item Toute fonction injective correspond à un arrangement de $n$ éléments de $B$ (car les images des éléments de $A$ doivent être distinctes).\\
    \item \textit{Laissée en exercice.}
\end{enumerate}}

\reminder{Soient $X, Y$ deux ensembles. Soit $f : X \to Y$.
    $\forall A \subset Y$ $f^{-1}(A) = \{x \in X \mid f(x) \in A\}$ est l'\textbf{image réciproque} de $A$ par $f$.
}
\remark{$f^{-1}(A)$ peut être $\emptyset$, même si $A \neq \emptyset$.}
\theorem{Proposition}{Lemme des bergers}{false}{
    Soit $X$ et $Y$ des ensembles finis et $f \colon X \to Y$ une application.\\
    On suppose que $Card(f^{-1}(\{ y \})) = k$ où $k \in \mathbb{N}^*$ pour tout $y \in f(X)$.\\\\
    Alors $Card(Y) = \frac{Card(X)}{k}$.
}
\example{C'est l'idée du berger qui compte ses moutons. Si chaque mouton a 4 pattes, en comptant le nombre de pattes total et en divisant par 4, on obtient le nombre de moutons.}

\noindent{\textbf{Démonstration :}\\
    $X_i = f^{-1}(\{y_i\})$.\\
    $X = \bigcup_{y_i \in f(X)} X_i$ et les $X_i$ sont disjoints.\\
    Donc $Card(X) = \sum_{i = 1}^n Card(X_i)$.\\
    En effet, si $z_1 \in f^{-1}(\{y_1\}) \Rightarrow z_1 \notin f^{-1}(\{y_j\})$ pour $j \neq 1$.\\
    Donc $X_i \cap X_j = \emptyset$ pour $i \neq j$.\\
    $\forall x \in X$, $\exists i : f(x) = y_i$ par définition de l'image.\\
    Donc $X = \bigcup_{i=1}^{n} X_i$.\\
    Or $Card(X_i) = K$ donc $Card(X) = \sum_{i=1}^{n} k = n \times k$ où $n = Card(f(X))$.\\
    Donc $n = \frac{Card(X)}{k}$.\\
}

\example{
    Prenons le mot "Ensemble". On a 8 lettres, dont 3 "e" identiques.\\
    Le nombre de façons de réarranger les lettres du mot "Ensemble" est donc : $\frac{8!}{3!} = 6720$.
}

\definition{
    Soient $E$ un ensemble fini de cardinal $N$ et $0 \leq n \leq N$.\\
    Un sous-ensemble à $n$ éléments de $E$ est appelé une \textbf{combinaison} de $n$ éléments de $E$.\\\\
    Elle est donnée $(\omega_1, \omega_2, \ldots, \omega_n)$ où les $\omega_i$ sont distincts et non ordonnés.
}

\theorem{Proposition}{}{false}{
    Le nombre total des combinaisons à $n$ éléments de $E$ est donné par :
    \[
        C^n_N = \frac{N!}{n!(N-n)!} = \frac{1}{n!} A^n_N = \binom{N}{n}
    \]
}

\noindent{\textbf{Démonstration :}\\
    $X = \{ \text{arrangements de n éléments de E} \}$.\\
    $Y = \{ \text{combinaisons de n éléments de E} \}$.\\
    Soit $f \colon X \to Y$ l'application telle que $f((\omega_1, \omega_2, \ldots, \omega_n)) = \{\omega_1, \omega_2, \ldots, \omega_n\}$.\\
    Donc $Card(f^{-1}(\{\omega_1, \omega_2, \ldots, \omega_n\})) = n!$ car il y a $n!$ façons d'ordonner les éléments d'une combinaison.\\
    Donc $Card(Y) = \frac{Card(X)}{n!} = \frac{A^n_N}{n!} = \frac{N!}{n!(N-n)!}$.
}

\theorem{Proposition}{}{false}{
    \begin{enumerate}
        \item Soient $A$ et $B$ tq $A \cap B = \emptyset, A, B$ finis.\\
        $\Rightarrow Card(A \cup B) = Card(A) + Card(B)$.
        \item Soient $A, B$ finis.\\
        $\Rightarrow Card(A \cup B) = Card(A) + Card(B) - Card(A \cap B)$.
    \end{enumerate}
}

\noindent{\textbf{Démonstration :}\\
    \begin{enumerate}
        \item Exercice : si $n = Card(A)$ et $m = Card(B)$.\\
        Idée montrer $A \cup B$ est en bijection avec $\{1, \ldots, n, \ldots, n+m\}$.
        \item cf. Laurent.
    \end{enumerate}
}

\theorem{Formulaire}{Cas d'utilisation des formules de dénombrement}{false}{
    \begin{center}
    \includegraphics[width=1\textwidth]{./images/combinatoire_formules.png}
    \captionof{figure}{Arbre décisionnel illustrant les cas d'utilisation des formules de dénombrement}
\end{center}
}

\end{document}