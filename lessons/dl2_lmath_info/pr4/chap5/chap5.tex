\documentclass{article}

\usepackage[a4paper, left=1.5cm, right=1.5cm, top=2cm, bottom=2cm]{geometry}

\usepackage{../../../../components/components}

\usepackage{fancyhdr}


% Configuration des en-têtes et pieds de page
\pagestyle{fancy}
\fancyhf{} % reset tout

\fancyhead[L]{DL2 Math-Info PR4}
\fancyhead[C]{Probabilités}
\fancyhead[R]{2025-2026}

\fancyfoot[L]{Ewen Rodrigues de Oliveira}
\fancyfoot[R]{\thepage}

\begin{document}

\docTitle{Chapitre 5 : Espérance}

\section{Définition et exemples}

\reminder{Rappel sur les séries. Soit $(a_n)_{n\in\mathbb{N}}$ une suite de nombres réels.\\
Si les $a_n$ sont positifs, alors :
\begin{itemize}
    \item la série $\sum_{n \geq 1} a_n$ est définie positive
    \item l'ordre dans lequel on somme les termes n'affecte pas la somme ni la finitude
    \item Si on partitionnel $\mathbb{N}^*$ en une collection dénombrable $(I_p)_{p\in\mathbb{N}}$ d'ensembles infinis, alors $\sum_{n \geq 1} a_n = \sum_{p \geq 1} \sum_{n \in I_p} a_n$
\end{itemize}
}

On suppose à présent que les $a_n$ sont de signes quelconques. 
Si $\sum_{n \ge 1} |a_n| < +\infty$, alors :

\begin{enumerate}
    \item la série $\sum_{n \ge 1} a_n$ converge (dans $]-\infty, +\infty[$) ;
    
    \item l’ordre dans lequel l’on somme les $a_n$ n’influe pas sur la finitude ni sur la valeur de la série ;
    
    \item si l’on partitionne $\mathbb{N}^*$ en une collection dénombrable $(I_p)_{p \ge 1}$ de parties, alors
    \[
    \sum_{n \ge 1} a_n 
    = 
    \sum_{p \ge 1} \; \sum_{n \in I_p} a_n .
    \]
\end{enumerate}

\end{document}