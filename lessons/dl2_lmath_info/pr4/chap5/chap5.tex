\documentclass{article}

\usepackage[a4paper, left=1.5cm, right=1.5cm, top=2cm, bottom=2cm]{geometry}

\usepackage{../../../../components/components}

\usepackage{fancyhdr}


% Configuration des en-têtes et pieds de page
\pagestyle{fancy}
\fancyhf{} % reset tout

\fancyhead[L]{DL2 Math-Info PR4}
\fancyhead[C]{Probabilités}
\fancyhead[R]{2025-2026}

\fancyfoot[L]{Ewen Rodrigues de Oliveira}
\fancyfoot[R]{\thepage}

\begin{document}

\docTitle{Chapitre 5 : Espérance}

\section{Définition et exemples}

\reminder{Rappel sur les séries. Soit $(a_n)_{n\in\mathbb{N}}$ une suite de nombres réels.\\
Si les $a_n$ sont positifs, alors :
\begin{itemize}
    \item la série $\sum_{n \geq 1} a_n$ est définie positive
    \item l'ordre dans lequel on somme les termes n'affecte pas la somme ni la finitude
    \item Si on partitionnel $\mathbb{N}^*$ en une collection dénombrable $(I_p)_{p\in\mathbb{N}}$ d'ensembles infinis, alors $\sum_{n \geq 1} a_n = \sum_{p \geq 1} \sum_{n \in I_p} a_n$
\end{itemize}
}

On suppose à présent que les $a_n$ sont de signes quelconques. 
Si $\sum_{n \ge 1} |a_n| < +\infty$ (série absolument convergente), alors :

\begin{enumerate}
    \item la série $\sum_{n \ge 1} a_n$ converge (dans $]-\infty, +\infty[$) ;
    
    \item l'ordre dans lequel l'on somme les $a_n$ n'influe pas sur la finitude ni sur la valeur de la série ;
    
    \item si l'on partitionne $\mathbb{N}^*$ en une collection dénombrable $(I_p)_{p \ge 1}$ de parties, alors
    \[
    \sum_{n \ge 1} a_n 
    = 
    \sum_{p \ge 1} \; \sum_{n \in I_p} a_n .
    \]
\end{enumerate}

\definition{
    Soit $X$ une variable aléatoire discrète définie
sur un espace probabilisé ($\Omega$, $\mathbb{P}$) et à valeurs dans un ensemble (dénombrable) $E \subset \mathbb{R}_+$.
Soit $\mu_X$ la loi de $X$. L'espérance de X est définie comme la quantité :
\[\mathbb{E}[X] = \sum_{x \in E} x \cdot \mu_X(\{x\}) \in [0, +\infty[\]
}

\definition{
 Soit $X$ une variable aléatoire discrète
définie sur un espace probabilisé ($\Omega$, $\mathbb{P}$) et à valeurs dans un ensemble (dénombrable)
$E \subset \mathbb{R}$. Soit $\mu_X$ la loi de $X$. On dit que $X$ est intégrable si l'espérance de $|X| = \{ |x| : x \in E \}$ est finie, et
dans ce cas on définit l'espérance de $X$ comme la quantité :
\[\mathbb{E}[X] := \sum_{x \in E} x \cdot \mu_X(\{x\}) \in ]-\infty, +\infty[ .\]
}

\vocabulary{On dira que $X$ admet une espérance si l'on est dans le cadre de l'une des deux définitions précédentes.}

\[
\begin{array}{c|c|c}
    & \text{Admet une espérance} & \text{Intégrable} \\
    \hline
    X \text{ positive et } \mathbb{E}[X] < +\infty & \text{oui (finie)} & \text{oui} \\
    X \text{ positive et } \mathbb{E}[X] = +\infty & \text{oui (infinie)} & \text{non} \\
    X \text{ quelconque et } \mathbb{E}[|X|] < +\infty & \text{oui (finie)} & \text{oui} \\
    X \text{ quelconque et } \mathbb{E}[|X|] = +\infty & \text{non} & \text{non}
\end{array}
\]
\remark{Dans le cas particulier où l’ensemble $E$ est fini, $X$ admet toujours une espérance et $X$ est même toujours intégrable.}

\example{
    On pose $\Omega = \{1, \ldots, 6\}$, $\mathbb{P}$ la probabilité uniforme et $X(\omega) = \omega$ la variable qui associe à chaque lancer de dé le nombre sur la face supérieure. $X$ prend un nombre fini de valeurs, donc $X$ est intégrable, et est à valeurs positives, donc $X$ admet une espérance. On trouve $\mathbb{E}[X] = \frac{1}{6} \cdot (1 + 2 + 3 + 4 + 5 + 6) = \frac{7}{2}$.
}

\vocabulary{En vocabulaire sur les séries, on dit que $X$ admet une espérance si la série $\sum_{x \in E} x \cdot \mu_X(\{x\})$ est absolument convergente.}

\section{Formule de transfert et propriétés}

\theorem{Théorème}{Formule de transfert}{false}{
    Soit $X$ une variable aléatoire sur un espace probabilisé $(\Omega, \mathbb{P})$ à valeurs dans un ensemble dénombrable $E$.\\
    Soit $g \colon E \to F$ une fonction, avec $F \subset \mathbb{R}$ un ensemble dénombrable.
    \begin{enumerate}
        \item Si $g$ est à valeurs positives, alors :
        \[ \mathbb{E}[g(X)] = \sum_{x \in E} g(x) \cdot \mu_X(\{x\}) .\]
        \item Si $g$ est de signe quelconque, alors : la variable aléatoire $g(X)$ est intégrable si et seulement si la série $\sum_{x \in E} |g(x)| \cdot \mu_X(\{x\})$ est finie, et dans ce cas :
        \[ \mathbb{E}[g(X)] = \sum_{x \in E} g(x) \cdot \mu_X(\{x\}) .\]
    \end{enumerate}
}

\theorem{Corollaire}{Expression de l'espérance à partir de la loi de $X$}{false}{
    Soit $X$ une variable aléatoire réelle discrète dans un ensemble $E$.
    \begin{enumerate}
        \item Si $X$ est à valeurs positives, alors :
        \[ \mathbb{E}[X] = \sum_{\omega \in \Omega} X(\omega) \cdot \mathbb{P}(\{\omega\}) .\]
        \item Si $X$ est de signe quelconque, alors : $X$ est intégrable si et seulement si la série $\sum_{\omega \in \Omega} |X(\omega)| \cdot \mathbb{P}(\{\omega\})$ est finie, et dans ce cas :
        \[ \mathbb{E}[X] = \sum_{\omega \in \Omega} X(\omega) \cdot \mathbb{P}(\{\omega\}) .\]
    \end{enumerate}
}

On a donc vu que l'espérance d'une variable aléatoire discrète est une série de la forme $\sum_{n \ge 1} a_n$ avec $a_n = x \cdot \mu_X(\{x\})$ ou $a_n = X(\omega) \cdot \mathbb{P}(\{\omega\})$ qui converge absolument. On peut alors appliquer les propriétés des séries absolument convergentes pour obtenir les propriétés suivantes de l'espérance.
\theorem{Propriété}{}{false}{
    \begin{enumerate}
        \item Linéarité : pour toutes variables aléatoires intégrables $X, Y$ et tous réels $a, b$, la variable aléatoire $aX + bY$ est intégrable et $\mathbb{E}[aX + bY] = a \mathbb{E}[X] + b \mathbb{E}[Y]$.
        \item Monotonie : pour toutes variables aléatoires $X, Y$ admettant une espérance, si $X(\omega) \leq Y(\omega)$ pour tout $\omega \in \Omega$, alors $\mathbb{E}[X] \leq \mathbb{E}[Y]$.
        \item Valeur absolue : $|\mathbb{E}[X]| \leq \mathbb{E}[|X|]$ pour toute variable aléatoire $X$ admettant une espérance.
    \end{enumerate}
}

\theorem{Proposition}{}{false}{
    Soit $X$ une variable aléatoire discrète. Supposons $X(\omega) \geq 0$ pour tout $\omega \in \Omega$ (ie. $X$ est à valeurs positives).\\
    Alors $E[X] = 0 \Leftrightarrow \mathbb{P}(X = 0) = 1$.\\\\

    Attention, si $X$ est de signe quelconque, $\Rightarrow$ n'est plus vérifié.
}

\theorem{Proposition}{Variable aléatoire entière}{false}{
    Soit $X$ une variable aléatoire à valeurs dans $\mathbb{N}$. Alors :
    \[ \mathbb{E}[X] = \sum_{k=1}^{+\infty} \mathbb{P}(X \geq k) .\]
}

\end{document}