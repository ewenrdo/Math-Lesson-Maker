\documentclass{article}

\usepackage[a4paper, left=1.5cm, right=1.5cm, top=2cm, bottom=2cm]{geometry}

\usepackage{../../../../components/components}

\usepackage{fancyhdr}


% Configuration des en-têtes et pieds de page
\pagestyle{fancy}
\fancyhf{} % reset tout

\fancyhead[L]{DL2 Math-Info PR4}
\fancyhead[C]{Probabilités}
\fancyhead[R]{2025-2026}

\fancyfoot[L]{Ewen Rodrigues de Oliveira}
\fancyfoot[R]{\thepage}

\begin{document}

\docTitle{Chapitre 4 : Variables aléatoires}

\section{Définitions}

\reminder{Un rappel a été proposé sur l'image réciproque d'un ensemble par une application.}

\definition{
    Soit $(\Omega, \mathbb{P})$ un espace de probabilité et soit $E$ un ensemble dénombrable.\\
    Une \textbf{variable aléatoire à valeurs dans $E$} est une application de $\Omega$ dans $E$.\\\\
    \textit{De manière générique, on la note $X : \Omega \to E$}
}
\vocabulary{On écrira souvent \textit{v.a.r.} pour \textit{variable aléatoire réelle}, ie. une variable aléatoire à valeurs dans $\mathbb{R}$.}

\definition{
    Soit $(\Omega, \mathbb{P})$ un espace de probabilité et soit $X : \Omega \to E$ une variable aléatoire à valeurs dans $E$.\\
    La \textbf{loi de $X$} est une probabilité définie sur $E$ par :
    \[
        \mu_X(A) = \mathbb{P}(X^{-1}(A)) := \mathbb{P}(\{\omega \in \Omega : X(\omega) \in A\})
    \] où $A \in \mathcal{P}(E)$
}

\training{Vérifier en montrant les axiomes de la probabilité que $\mu_X$ est bien une probabilité sur $E$.}
\noindent\carreaux{7}

\example{
    On lance deux dés. $\Omega = \{1, 2, 3, 4, 5, 6\}^2 = \{ (\omega_1, \omega_2) : \omega_i \in \{1, 2, 3, 4, 5, 6\} \}$ et $\mathbb{P}$ est la probabilité uniforme.\\
    On pose $S : \Omega \to \{2, 3, \dots, 12\}$ définie par $S((\omega_1, \omega_2)) = \omega_1 + \omega_2$.\\
    $\mu_S({2}) = \mathbb{P}(S^{-1}(\{2\})) = \mathbb{P}(\{(1, 1)\}) = \frac{1}{36} = \mathbb{P}(S=2)$\\
    $\mu_S({3}) = \mathbb{P}(S^{-1}(\{3\})) = \mathbb{P}(\{(1, 2), (2, 1)\}) = \frac{2}{36} = \mathbb{P}(S=3)$\\
    $\mu_S({4}) = \mathbb{P}(S^{-1}(\{4\})) = \mathbb{P}(\{(1, 3), (2, 2), (3, 1)\}) = \frac{3}{36} = \mathbb{P}(S=4)$\\
    $\dots$
}

\vocabulary{On a les abréviations et/ou notations suivantes :
\begin{itemize}
    \item $\mu_X(A) = \mathbb{P}(X^{-1}(A)) = \mathbb{P}(X \in A) = \mathbb{P}(\{ \omega \in \Omega : X(\omega) \in A \})$
    \item $\mathbb{P}(X = k) = \mathbb{P}(X^{-1}(\{k\})) = \mu_X(\{k\})$
\end{itemize}
}

\example{
    On va ici illustrer la loi de Bernouilli.\\
    La loi de Bernouilli est donnée par $\mu(\{0\}) = 1-p$ et $\mu(\{1\}) = p$ où $p \in [0, 1]$.\\
    Une \textbf{variable de Bernouilli} de paramètre $p$ est une variable aléatoire définie sur $(\Omega, \mathbb{P}) \; X : \Omega \to \{0, 1\}$ telle que $\mu_X$ est la loi de Bernouilli de paramètre $p$.\\
    On la note $Ber(p)$
}

\definition{
    Soit $E$ un ensemble fini.\\
    La \textbf{loi uniforme sur $E$} est définie par $\mu(\{x\}) = \frac{1}{|E|}$ pour tout $x \in E$.
    On la note $\mathcal{U}(E)$
}

\newpage
\tableofcontents

\end{document}