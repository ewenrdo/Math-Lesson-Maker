\documentclass{article}

\usepackage[a4paper, left=1.5cm, right=1.5cm, top=2cm, bottom=2cm]{geometry}

\usepackage{../../../../components/components}

\usepackage{fancyhdr}


% Configuration des en-têtes et pieds de page
\pagestyle{fancy}
\fancyhf{} % reset tout

\fancyhead[L]{DL2 Math-Info PR4}
\fancyhead[C]{Préambule}
\fancyhead[R]{2025-2026}

\fancyfoot[L]{Ewen Rodrigues de Oliveira}
\fancyfoot[R]{\thepage}

\begin{document}

\docTitle{Préambule du cours}

\section{Généralités}

\noindent Cours de Raphaël Lefevere, joignable à \textit{lefevere@lpsm.paris}.\\
On étudiera les probabilités discrètes, c'est-à-dire des probabilités sur des ensembles finis ou dénombrables.\\

\attention{Le polycopié sur Moodle n'est pas final. Un chapitre sur les chaînes de Markov sera ajouté ultérieurement.}

\section{Modalités de Contrôle Continu}

Trois contrôles continus organisés :
\begin{enumerate}
    \item CC1 : Semaine du 16 février 2026
    \item CC2 : Semaine du 23 mars 2026
    \item CC3 : Semaine du 4 mai 2026
\end{enumerate}
Il s'agit d'interrogations dont le but est d'aider les étudiants.\\\\
Un examen final (E) est organisé.

On a : 
\[NF = \max\left(E, \frac{E+I}{2}\right)\] où \[I = \frac{\max(E, CC1) + \max(E, CC2) + \max(E, CC3)}{3}\].

\section{Plan du cours}
\begin{enumerate}
    \item Dénombrement (c'est un outil, pas un but en soi)
    \item Espaces probabilisés
    \item Conditionnement et indépendance
    \item Variables aléatoires 
    \item Espérance (et variance)
    \item Vecteurs aléatoires
    \item Quelques outils (inégalités, ...)
    \item Chaînes de Markov
\end{enumerate}
\end{document}