\documentclass{article}

\usepackage[a4paper, left=1.5cm, right=1.5cm, top=2cm, bottom=2cm]{geometry}
\usepackage{../../../components/components} % <-- ton fichier .sty, avec toutes tes définitions

\usepackage{fancyhdr}

% Configuration des en-têtes et pieds de page
\pagestyle{fancy}
\fancyhf{} % reset tout

\fancyhead[L]{DL Math-Info}
\fancyhead[C]{Réduction des endomorphismes}
\fancyhead[R]{2025-2026}

\fancyfoot[L]{Ewen Rodrigues de Oliveira}
\fancyfoot[R]{\thepage}

\begin{document}

\docTitle{Chapitre 1 : Réduction des endomorphismes et matrices diagonalisables}
Dans toute ce chapitre, on notera \( E \) un \(\mathbb{K}\)-espace vectoriel, avec \(\mathbb{K} \in \left\{ \mathbb{R}, \mathbb{C} \right\} \).

\section{Sous-espaces stables par un endomorphisme}

\definition{Soit $f \in \mathcal{L}(E)$, et soit $F \subset E$ un sous-espace vectoriel.

On dit que $F$ est \textbf{stable} par $f$ si $f(F) \subset F$., $i.e.$ $\forall x \in F, f(x) \in F$.
}

\remark{
\begin{itemize}
    \item $0_E$ est stable par $f$.
    \item $E$ est stable par $f$.
    \item $\ker(f)$ et $Im(f)$ sont stables par $f$.
\end{itemize}
}
 
\textbf{Preuve:} \\
\carreaux{20}


\section{Éléments propres d'un endomorphisme}
\vocabulary{On appelle \textbf{élément propre} de $f$ ses valeurs propres, vecteurs propres et sous-espaces propres.}

\subsection{Valeur propre} 

\definition{Soit $f \in \mathcal{L}(E)$, et soit $\lambda \in \mathbb{K}$.\\
$\lambda$ est une \textbf{valeur propre} de $f$ si $\exists x \in E \setminus \left\{ 0_E \right\}$ : $f(x) = \lambda x$.\\
On dit que $x$ est un \textbf{vecteur propre} de $f$ associé à $\lambda$.\\\\
L'ensemble des valeurs propres de $f$ est noté $Sp(f)$ et appelé \textbf{spectre} de $f$.
}

\subsection{Sous-espace propre}
\definition{Soit $f \in \mathcal{L}(E)$, et soit $\lambda \in \mathbb{K}$.\\
On note $E_\lambda(f)$ le \textbf{sous-espace propre} de $f$ associé à $\lambda$, défini par :
\[E_\lambda(f) = \ker(f - \lambda Id_E) = \left\{ x \in E : f(x) = \lambda x \right\}.\]\\
$E_\lambda(f)$ est un sous-espace vectoriel de $E$.}

\remark{
    $E_\lambda(f)$ est l'ensemble des vecteurs propres de $f$ associés à $\lambda$.
}

\remark{
    \begin{itemize}
        \item $\lambda \in Sp(f) \iff f$ n'est pas injective $\iff \det(E_\lambda(f)) \neq 0$.
        \item $0_E \in Sp(f) \iff \det(f) = 0$.
    \end{itemize}
}

\textbf{Preuve:} \\
\carreaux{22}

\section{Polynôme caractéristique}
\definition{On appelle \textbf{polynôme caractéristique} d'une matrice $A \in M_n(\mathbb{K})$ le polynôme défini par :
\[
\chi_A(X) = \det(A - X Id_n).
\]
}

\example{
    Soit $A = \begin{pmatrix}
        5 & 0 & 4 \\
        4 & 1 & 0 \\
        -8 & 0 & -7
    \end{pmatrix}$. Déterminons son polynôme caractéristique.\\
    $\chi_A(X) = \det(A - X Id_3) = \det\left(\begin{pmatrix}
        5 - X & 0 & 4 \\
        4 & 1 - X & 0 \\
        -8 & 0 & -7 - X
    \end{pmatrix}\right)$\\\\
    En développant le déterminant suivant la deuxième colonne, on obtient :
    \[
    \chi_A(X) = (1 - X) \det\left(\begin{pmatrix}
        5 - X & 4 \\
        -8 & -7 - X
    \end{pmatrix}\right) = (1 - X) \left((5 - X)(-7 - X) + 32\right).
    \]\\
    En factorisant ce polynôme, on trouve :
    \[
    \chi_A(X) = -(X-1)^2(X+3)
    \]
}

\remark{
    $\lambda \in Sp(A) \iff \det(A - \lambda Id_E) = 0 \iff \chi_A(\lambda) = 0$.\\
    Donc : $Sp(A) = \left\{ \lambda \in \mathbb{K} : \chi_A(\lambda) = 0 \right\}$.
}
\textbf{Preuve:} \\
\carreaux{7}


\example{
    (suite de l'exemple précédent)\\
    On a trouvé le polynôme caractéristique de $A$ : $\chi_A(X) = -(X-1)^2(X+3)$.\\
    Donc les valeurs propres de $A$ sont $\lambda_1 = 1$ (de multiplicité 2) et $\lambda_2 = -3$.\\
    Alors, on a : $Sp(A) = \left\{ 1, -3 \right\}$.
}

\remark {
    $A \in M_2(\mathbb{K}) \implies Sp(A) = \chi_A(X) = X^2 - Tr(A)X + \det(A)$.
}
\textbf{Preuve:} \\
\carreaux{7}

\section{Matrice diagonalisable}
\definition{Soit $A \in M_n(\mathbb{K})$. On dit que $A$ est \textbf{diagonalisable} si $A$ est \textbf{semblable} à une matrice diagonale.\\\\
Autrement dit, $A$ est diagonalisable si $A = P D P^{-1}$, où $D$ est une matrice diagonale et $P$ est une matrice inversible.}

\remark{
    \begin{itemize}
        \item Soient $\lambda_1, \ldots, \lambda_k$ les valeurs propres de $A$, et soient $E_{\lambda_i}(A)$ les sous-espaces propres associés à $\lambda_i$.\\
        $A$ diagonalisable $\iff \dim(E_{\lambda_i}(A)) =$ ordre de multiplicité de $\lambda_i$.\\
        $e.g.$ $\lambda_i$ est une valeur propre simple, alors $\dim(E_{\lambda_i}(A)) = 1$.
        \item $A$ diagonalisable $\iff E = \bigoplus_{i=1}^k E_{\lambda_i}(A)$. ($i.e. E$ est la somme directe des sous-espaces propres associés aux valeurs propres de $A$)
    \end{itemize}
}

\example{
    (suite de l'exemple précédent)\\
    On a trouvé que $Sp(A) = \left\{ 1, -3 \right\}$ et $\chi_A(X) = -(X-1)^2(X+3)$.\\
Alors, A est diagonalisable si : \[ 
        \begin{cases}
            \dim(E_1(A)) = 2 \\
            \dim(E_{-3}(A)) = 1
        \end{cases}
        \]\\\\
On a aussi que A est diagonalisable si $\mathbb{R}^3 = E_1(A) \oplus E_{-3}(A)$.      
}
\end{document}
